\startexenum

In Exercises \ref{compfactpolyfirst} - \ref{compfactpolylast}, find all of the zeros of the polynomial then completely factor it over the real numbers and completely factor it over the complex numbers.

\begin{shortexenum}[$g(t) = 3t^{3} - 13t^{2} + 43t - 13$]
\item $f(x) = x^{2} - 4x + 13$ \label{compfactpolyfirst}
\item $f(x) = x^2 - 2x + 5$
\item $p(z) = 3z^{2} + 2z + 10$
\item $p(z) = z^3-2z^2+9z-18$
\item $g(t) = t^{3} + 6t^{2} + 6t + 5$
\item $g(t) = 3t^{3} - 13t^{2} + 43t - 13$
\item $f(x) = x^3 + 3x^2 + 4x + 12$
\item $f(x) = 4x^3-6x^2-8x+15$
\item  $p(z) = z^3 + 7z^2+9z-2$
\item  $p(z) = 9z^3+2z+1$
\item $g(t) = 4t^{4} - 4t^{3} + 13t^{2} - 12t + 3$
\item $g(t) = 2t^4-7t^3+14t^2-15t+6$
\item  $f(x) = x^4+x^3+7x^2+9x-18$
\item  $f(x) = 6x^4+17x^3-55x^2+16x+12$
\item  $p(z) = -3z^4-8z^3-12z^2-12z-5$
\item  $p(z) = 8z^4+50z^3+43z^2+2z-4$
\item $g(t) = t^4+9t^2+20$
\item $g(t) = t^4 + 5t^2 - 24$
\item  $f(x) = x^5 - x^4+7x^3-7x^2+12x-12$
\item $f(x) = x^6-64$
\item $f(x) = x^{4} - 2x^{3} + 27x^{2} - 2x + 26$ (Hint: $x = i$ is one of the zeros.)
\item  $p(z) = 2z^4+5z^3+13z^2+7z+5$ (Hint:  $z = -1+2i$ is a zero.) \label{compfactpolylast}
\end{shortexenum}

In Exercises \ref{buildcomppolyfirst} - \ref{buildcompolylast}, use Theorem \ref{complexfactorization} to create a polynomial function with real number coefficients which has all of the desired characteristics.  You may leave the polynomial in factored form. 

\begin{exenum}

\item  \label{buildcomppolyfirst}

\begin{itemize}

\item The zeros of $f$ are $c = \pm 2$ and $c = \pm 1$.
\item The leading term of $f(x)$ is $117x^4$.

\end{itemize}

\item

\begin{itemize}

\item The zeros of $p$ are $c=1$ and $c = 3$.
\item $c=3$ is a zero of multiplicity 2.
\item The leading term of $p(z)$ is $-5z^3$.

\end{itemize}

\item

\begin{itemize}

\item The solutions to $g(t) = 0$ are $t = \pm 3$ and $t=6$.
\item The leading term of $g(t)$ is $7t^4$.
\item The point $(-3,0)$ is a local minimum on the graph of $y=g(t)$.

\end{itemize}

\item

\begin{itemize}

\item The solutions to $f(x) =0$ are $x = \pm 3$, $x=-2$, and $x=4$.
\item The leading term of $f(x)$ is $-x^5$.
\item The point $(-2, 0)$ is a local maximum on the graph of $y=f(x)$.

\end{itemize}

\item 

\begin{itemize}

\item $p$ is degree 4.
\item as $z \rightarrow \infty$, $p(z) \rightarrow -\infty$.
\item $p$ has exactly three $z$-intercepts:  $(-6,0)$, $(1,0)$ and $(117,0)$.
\item  The graph of $y=p(z)$ crosses through the $z$-axis at $(1,0)$.

\end{itemize}

\item

\begin{itemize}

\item The zeros of $g$ are $c=\pm 1$ and $c = \pm i$.
\item The leading term of $g(t)$ is $42t^4$.

\end{itemize}

\item

\begin{itemize}

\item $c=2i$ is a zero.
\item the point $(-1,0)$ is a local minimum on the graph of $y=f(x)$.
\item the leading term of $f(x)$ is $117x^4$.

\end{itemize}

\item

\begin{itemize}

\item The solutions to $p(z) = 0$ are $z = \pm 2$ and $z=\pm 7i$.
\item The leading term of $p(z)$ is $-3z^5$.
\item The point $(2,0)$ is a local maximum on the graph of $y=p(z)$.

\end{itemize}

\item

\begin{itemize}

\item $g$ is degree $5$.
\item $t=6$, $t = i$ and $t = 1-3i$ are zeros of $g$.
\item as $t \rightarrow -\infty$, $g(t) \rightarrow \infty$.

\end{itemize}

\item \label{buildcompolylast}

\begin{itemize}

\item The leading term of $f(x)$ is $-2x^3$.
\item $c=2i$ is a zero.
\item $f(0) = -16$.

\end{itemize}

\end{exenum}

In Exercises \ref{polyfromgraphfirst} - \ref{polyfromgraphlast}, find a possible formula for the polynomial function given its graph.  You may leave the polynomial in factored form. 

\begin{multicols}{2}
\begin{enumerate}
\setcounter{enumi}{\value{HW}}

\item  \label{polyfromgraphfirst} $y=f(x)$.  %$f(x) = x(x+6)(x-6)

\begin{mfpic}[10]{-7}{7}{-6}{6}
\axes
\tlabel[cc](7,-0.5){\scriptsize $x$}
\tlabel[cc](0.5,6){\scriptsize $y$}
\tlabel[cc](-4.5, 0.75){\scriptsize $(-6,0)$}
\tlabel[cc](5, 0.75){\scriptsize $(6,0)$}
\tlabel[cc](1, 0.75){\scriptsize $(0,0)$}
\tlabel[cc](-3, 5){\scriptsize $(-3,81)$}
\point[4pt]{(-6,0),(0,0),(6,0), (-3,4.05) }
\xmarks{-6 step 1 until 6}
\tiny
\tlpointsep{4pt}
\axislabels {x}{{$-6 \hspace{6pt}$} -6, {$-5 \hspace{6pt}$} -5, {$-4 \hspace{6pt}$} -4, {$-3 \hspace{6pt}$} -3, {$-2 \hspace{6pt}$} -2, {$-1 \hspace{6pt}$} -1, {$1$} 1, {$2$} 2, {$3$} 3, {$4$} 4, {$5$} 5, {$6$} 6}
\normalsize
\penwd{1.25pt}
\arrow \reverse \arrow \function{-7,7,0.1}{((x**3) - 36*x)/20}
\end{mfpic}

\vfill

\columnbreak

\item $y=g(t)$  %$g(t) = t(t+2)^3$


\begin{mfpic}[20][20]{-3}{3}{-2}{5}
\axes
\tlabel[cc](3,-0.5){\scriptsize $t$}
\tlabel[cc](0.25,5){\scriptsize $y$}
\tlabel[cc](-1.75, 0.3){\scriptsize $(-2,0)$}
\tlabel[cc](0.5, 0.3){\scriptsize $(0,0)$}
\tlabel[cc](-2, -1){\scriptsize $(-1,-1)$}
\point[4pt]{(-2,0), (0,0), (-1,-1)}
\xmarks{-2,-1, 1, 2}
\tiny
\tlpointsep{4pt}
\axislabels {x}{{$-2 \hspace{6pt}$} -2, {$-1 \hspace{6pt}$} -1, {$1$} 1, {$2$} 2}
\normalsize
\penwd{1.25pt}
\arrow \reverse \arrow \function{-3,0.3,0.1}{x*((x + 2)**3)}

\end{mfpic}


\setcounter{HW}{\value{enumi}}
\end{enumerate}
\end{multicols}


\begin{multicols}{2}
\begin{enumerate}
\setcounter{enumi}{\value{HW}}

\item  $y = p(z)$  %$p(z) = -2(z+1)(z-2)^2$

\begin{mfpic}[20][10]{-3}{3}{-4}{4}
\axes
\tlabel[cc](3,-0.5){\scriptsize $z$}
\tlabel[cc](0.25,4){\scriptsize $y$}
\tlabel[cc](-2, 0.75){\scriptsize $(-1,0)$}
\tlabel[cc](2, 0.75){\scriptsize $(2,0)$}
\tlabel[cc](0.75, -2.25){\scriptsize $(0, -8)$}
\point[4pt]{(2,0), (-1,0), (0, -1.6)}
\xmarks{-2,-1,1,2}
\tiny
\tlpointsep{4pt}
\axislabels {x}{{$-2 \hspace{6pt}$} -2, {$-1 \hspace{6pt}$} -1, {$1$} 1, {$2$} 2}
\normalsize
\penwd{1.25pt}
\arrow \reverse \arrow \function{-1.70,3.45,0.1}{(-0.4)*((x-2)**2)*(x+1)}
\end{mfpic}

\vfill

\columnbreak

\item $y = f(x)$  %$f(x) = 4\left(x+ \frac{1}{2}\right)^2 (x-3)$

\begin{mfpic}[20][10]{-2}{4}{-4}{4}
\axes
\tlabel[cc](4,-0.5){\scriptsize $x$}
\tlabel[cc](0.25,4){\scriptsize $y$}
\tlabel[cc](-0.75, 0.75){\scriptsize $\left(-\frac{1}{2},0 \right)$}
\tlabel[cc](2.5, 0.75){\scriptsize $(3,0)$}
\tlabel[cc](3.25, -3.125){\scriptsize $(2,-25)$}
\point[4pt]{(-0.5,0), (3,0), (2, -3.125)}
\xmarks{-1,1,2,3}
\tiny
\tlpointsep{4pt}
\axislabels {x}{ {$1$} 1, {$2$} 2, {$3$} 3}
\normalsize
\penwd{1.25pt}
\arrow \reverse \arrow \function{-1.5,3.3,0.1}{(0.5)*((x+0.5)**2)*(x-3)}

\end{mfpic}

\setcounter{HW}{\value{enumi}}
\end{enumerate}
\end{multicols}


\begin{multicols}{2}
\begin{enumerate}
\setcounter{enumi}{\value{HW}}

\item $y = F(s)$  %$F(s)  =-s(s+2)^2$

\begin{mfpic}[20][10]{-3}{3}{-5}{5}
\axes
\tlabel[cc](3,-0.5){\scriptsize $s$}
\tlabel[cc](0.25,5){\scriptsize $y$}
\tlabel[cc](-2, -0.75){\scriptsize $(-2,0)$}
\tlabel[cc](0.5, 0.75){\scriptsize $(0,0)$}
\tlabel[cc](-1.25, 1.75){\scriptsize $(-1,1)$}
\point[4pt]{(-2,0), (0,0), (-1,1)}
\xmarks{-2,-1,1,2}
\tiny
\tlpointsep{4pt}
\axislabels {x}{ {$1$} 1, {$2$} 2}
\normalsize
\penwd{1.25pt}
\arrow \reverse \arrow \function{-3,0.65,0.1}{0-x*((x + 2)**2)}

\end{mfpic}

\vfill

\columnbreak

\item \label{polyfromgraphlast} $y = G(t)$  %$G(t) = t^3(t+2)^2$

\begin{mfpic}[20][10]{-3}{3}{-5}{5}
\axes
\tlabel[cc](-2, 0.75){\scriptsize $(-2,0)$}
\tlabel[cc](0.5, -0.75){\scriptsize $(0,0)$}
\tlabel[cc](-1, -2){\scriptsize $(-1,-1)$}
\tlabel[cc](3,-0.5){\scriptsize $t$}
\tlabel[cc](0.25,5){\scriptsize $y$}
\point[4pt]{(-2,0), (0,0), (-1,-1)}
\xmarks{-2,-1,1,2}
\tiny
\tlpointsep{4pt}
\axislabels {x}{  {$2$} 2}
\normalsize
\penwd{1.25pt}
\arrow \reverse \arrow \function{-2.45,0.85,0.1}{(x**3)*((x + 2)**2)}

\end{mfpic}


\setcounter{HW}{\value{enumi}}
\end{enumerate}
\end{multicols}

\begin{enumerate}
\setcounter{enumi}{\value{HW}}

\item  \label{cmpgeoalgexfirst}  With help from your classmates, choose several nonzero complex numbers $z$, find their complex conjugates $\overline{z}$.  Plot each pair $z$ and $\overline{z}$ in the Complex Plane.  What appears to be the relationship between these numbers geometrically?  State and prove a general result.

\item    With help from your classmates, choose several nonzero complex numbers $z$ and  find $-z$.  Plot each pair $z$ and $-z$ in the Complex Plane.  What appears to be the relationship between these numbers geometrically?  State and prove a general result.

\item  With help from your classmates, choose several different complex numbers $z$ and find the product of $i$ and $z$,  $iz$.  Plot each pair of $z$ and $iz$ in the Complex Plane.  In each case, show the line containing the origin and the point corresponding to $z$ is perpendicular\footnote{See Theorem \ref{parallelperpendicularslopetheorem} in Section \ref{AppLines} of you need a refresher on how to do this.} to the line containing the origin and the point corresponding to $iz$.  Show this result holds in general for every nonzero complex number.

\item  \label{cmpgeoalgexlast} Given a complex number $z = a+bi$, we define the \index{modulus ! of a complex number}\textbf{modulus} of $z$, $|z|$, by $|z| = \sqrt{a^2+b^2}$.  With help from your classmates, calculate $|z|$ for several different complex numbers, $z$.  What does $z$ measure geometrically?  Show that if $x$ is a real number, then the modulus of $x$ is the same as the absolute value of $x$, and comment how all this relates to Definition \ref{absvaldistdefn} in Section \ref{AppAbsValEqIneq}.


\setcounter{HW}{\value{enumi}}
\end{enumerate}


\begin{enumerate}
\setcounter{enumi}{\value{HW}}

\item \label{zbarexercise} Let $z$ and $w$ be arbitrary complex numbers.  Show that  $\overline{z} \, \overline{w}  = \overline{zw}$ and $\overline{\overline{z}} = z$.

\setcounter{HW}{\value{enumi}}
\end{enumerate}

\clearpage

\subsection{Answers}

\startexenum

\begin{exenum}

\item $f(x) = x^2-4x+13 = (x-(2+3i)) (x-(2-3i))$ \\
Zeros: $x = 2 \pm 3i$ 
\item $f(x) = x^2 - 2x + 5 = (x-(1+2i))(x-(1-2i))$ \\ 
Zeros:  $x = 1 \pm 2i$

\item $p(z) = 3z^2 + 2z +10 = 3\left(z-\left(-\frac{1}{3} + \frac{\sqrt{29}}{3} i\right) \right) \left(z-\left(-\frac{1}{3} - \frac{\sqrt{29}}{3} i\right) \right)$
Zeros:  $z = -\frac{1}{3} \pm \frac{\sqrt{29}}{3} i$
\item $p(z) = z^3-2z^2+9z-18 = (z-2) \left(z^2+9\right) = (z-2)(z-3i)(z+3i)$\\
Zeros:  $z=2, \pm 3i$

\item $g(t) = t^{3} + 6t^{2} + 6t + 5 = (t + 5)(t^{2} + t + 1) = (t + 5) \left( t - \left( -\frac{1}{2} + \frac{\sqrt{3}}{2}i \right) \right) \left( t - \left(-\frac{1}{2} - \frac{\sqrt{3}}{2}i \right) \right)$ \\
Zeros: $t = -5, \;  t = -\frac{1}{2} \pm \frac{\sqrt{3}}{2}i $
\item $g(t) = 3t^{3} - 13t^{2} + 43t - 13 = (3t - 1)(t^{2} - 4t + 13) = (3t - 1)(t - (2 + 3i))(t - (2 - 3i))$\\
Zeros: $t = \frac{1}{3}, \; t = 2 \pm 3i$

\item $f(x) = x^3 + 3x^2 + 4x + 12 = (x+3) \left(x^2 + 4 \right) = (x+3)(x+2i)(x-2i)$ \\
Zeros:  $x = -3, \; \pm 2i$
\item $f(x) = 4x^3-6x^2-8x+15 = \left(x + \frac{3}{2} \right) \left(4x^2-12x+10\right) \\
 \phantom{f(x)} = 4 \left(x + \frac{3}{2} \right) \left(x - \left( \frac{3}{2} + \frac{1}{2}i  \right) \right) \left(x - \left( \frac{3}{2} - \frac{1}{2}i  \right) \right)$\\
Zeros:  $x = - \frac{3}{2}, \; x = \frac{3}{2} \pm \frac{1}{2}i$


\item  $p(z) = z^3 + 7z^2+9z-2 = (z+2) \left(z - \left( -\frac{5}{2}+\frac{\sqrt{29}}{2}\right) \right) \left(z - \left( -\frac{5}{2}-\frac{\sqrt{29}}{2}\right) \right)$ \\
Zeros:  $z = -2, \; z = -\frac{5}{2} \pm \frac{\sqrt{29}}{2}$
\item  $p(z) = 9z^3+2z+1 = \left(z + \frac{1}{3}\right) \left(9z^2 - 3z + 3\right) \\
\phantom{f(z)}= 9\left(z + \frac{1}{3}\right) \left(z - \left(\frac{1}{6} + \frac{\sqrt{11}}{6} i \right) \right) \left(z - \left(\frac{1}{6} - \frac{\sqrt{11}}{6} i \right) \right)$\\
Zeros:  $z = -\frac{1}{3}, \; z = \frac{1}{6} \pm \frac{\sqrt{11}}{6} i$

\item $g(t) = 4t^{4} - 4t^{3} + 13t^{2} - 12t + 3 = \left(t - \frac{1}{2}\right)^{2}\left(4t^{2} + 12\right) = 4\left(t - \frac{1}{2}\right)^{2}(t + i\sqrt{3})(t - i\sqrt{3})$\\
Zeros: $t = \frac{1}{2}, \; t = \pm \sqrt{3}i$
\item $g(t) = 2t^4-7t^3+14t^2-15t+6 = (t-1)^2 \left(2t^2 - 3t + 6\right)  \\
\phantom{f(t)} = 2 (t-1)^2 \left( t - \left( \frac{3}{4} +  \frac{\sqrt{39}}{4} i \right) \right)  \left( t - \left( \frac{3}{4} -  \frac{\sqrt{39}}{4} i \right) \right) $ \\
Zeros: $t = 1, \; t = \frac{3}{4}  \pm  \frac{\sqrt{39}}{4} i$

\item  $f(x) = x^4+x^3+7x^2+9x-18 = (x+2)(x-1)\left(x^2+9\right) = (x+2)(x-1)(x+3i)(x-3i)$\\
Zeros:  $x = -2, \; 1, \; \pm 3i$
\item  $f(x) = 6x^4+17x^3-55x^2+16x+12 = 6 \left(x + \frac{1}{3} \right) \left(x - \frac{3}{2} \right) \left(x - \left( -2 + 2 \sqrt{2}\right)\right) \left(x - \left( -2 - 2 \sqrt{2}\right)\right)$ \\
Zeros:  $x = -\frac{1}{3}, \; x = \frac{3}{2}, \; x = -2 \pm 2 \sqrt{2}$

\item  $p(z) = -3z^4-8z^3-12z^2-12z-5 = (z+1)^2 \left(-3z^2-2z-5\right) \\
\phantom{f(z)}= -3(z+1)^2\left(z - \left( -\frac{1}{3}+\frac{\sqrt{14}}{3} i\right) \right) \left(z - \left( -\frac{1}{3}-\frac{\sqrt{14}}{3} i\right) \right)$ \\
Zeros:  $z = -1, \; z = -\frac{1}{3} \pm \frac{\sqrt{14}}{3} i$
\item  $p(z) = 8z^4+50z^3+43z^2+2z-4 = 8\left(z + \frac{1}{2}\right) \left(z - \frac{1}{4}\right)(z - (-3 + \sqrt{5}))(z - (-3 - \sqrt{5}))$ \\
Zeros:  $z = -\frac{1}{2}, \; \frac{1}{4}, \; z = -3 \pm \sqrt{5}$

\item  $g(t) = t^4+9t^2+20 = \left(t^2+4\right) \left(t^2+5\right) = (t-2i)(t+2i)\left(t - i \sqrt{5}\right)\left(t + i \sqrt{5}\right)$\\
Zeros:  $t = \pm 2i, \pm i \sqrt{5}$
\item  $g(t) = t^4+5t^2-24 = \left(t^2-3 \right) \left(t^2+8\right) = (t-\sqrt{3})(t+\sqrt{3})\left(t - 2i \sqrt{2}\right)\left(t + 2i \sqrt{2}\right)$\\
Zeros:  $t = \pm \sqrt{3}, \pm 2i \sqrt{2}$

\item  $f(x) = x^5 - x^4+7x^3-7x^2+12x-12 = (x-1) \left(x^2 + 3\right) \left(x^2 + 4 \right) \\
\phantom{f(x)} = (x-1)(x - i \sqrt{3})(x + i \sqrt{3})(x-2i)(x+2i)$ \\
Zeros:  $x = 1, \;  \pm  \sqrt{3}i,  \; \pm 2i$
\item $f(x) = x^6 - 64 = (x-2)(x+2)\left(x^2+2x+4\right)\left(x^2-2x+4\right) \\
      \phantom{f(x)} = (x-2)(x+2)\left( x - \left( -1+i\sqrt{3} \right) \right)\left( x - \left( -1-i\sqrt{3} \right) \right)\left( x - \left( 1+i\sqrt{3} \right) \right)\left( x - \left( 1-i\sqrt{3} \right) \right)$ \\
Zeros:  $x = \pm 2$, $x = -1 \pm i\sqrt{3}$, $x = 1 \pm i\sqrt{3}$


\item $f(x) = x^{4} - 2x^{3} + 27x^{2} - 2x + 26 = (x^{2} - 2x + 26)(x^{2} + 1) = (x - (1 + 5i))(x - (1 - 5i))(x + i)(x - i)$\\ 
Zeros: $x = 1 \pm 5i, \; x = \pm i$
\item  $p(z) = 2z^4+5z^3+13z^2+7z+5 = \left(z^2+2z+5\right) \left(2z^2+z+1\right)  \\ \phantom{f(z)} = 2 (z-(-1+2i))(z-(-1-2i))\left(z - \left(-\frac{1}{4} + i \frac{\sqrt{7}}{4}\right) \right)\left(z - \left(-\frac{1}{4} - i \frac{\sqrt{7}}{4}\right) \right) $\\
Zeros:  $z = -1 \pm 2i, -\frac{1}{4} \pm i \frac{\sqrt{7}}{4}$

\item $f(x) = 117(x+2)(x-2)(x+1)(x-1)$

\item $p(z)= -5(z-1)(z-3)^2$

\item  $g(t) = 7(t+3)^2(t-3)(t-6)$

\item $f(x) = -(x + 2)^{2}(x - 3)(x + 3)(x - 4)$

\item $p(z) = a(z+6)^2(z-1)(z-117)$ where $a$ can be any real number as long as $a<0$

\item $g(t) = 42(t-1)(t+1)(t-i)(t+i)$

\item $f(x) = 117(x+1)^2(x-2i)(x+2i)$

\item  $p(z) = -3(z-2)^2(z+2)(z-7i)(z+7i)$

\item $g(t) = a(t-6)(t-i)(t+i)(t-(1-3i))(t-(1+3i))$ where $a$ is any real number,  $a < 0$

\item $f(x) = -2(x-2i)(x+2i)(x+2)$

\end{exenum}

\begin{shortexenum}[$p(z) = -2(z+1)(z-2)^2$]
\item $f(x) = x(x+6)(x-6)$
\item $g(t) = t(t+2)^3$
\item $p(z) = -2(z+1)(z-2)^2$
\item $f(x) = 4\left(x+ \frac{1}{2}\right)^2 (x-3)$
\item $F(s)  =-s(s+2)^2$
\item $G(t) = t^3(t+2)^2$
\end{shortexenum}

\begin{exenum}

\item If $z = a+bi$, then $z$ corresponds to the point $(a,b)$ in the $xy$-plane.  Hence, $\overline{z} = \overline{a+bi} = a-bi$ corresponds to the point $(a,-b)$.  Hence, the points corresponding to $z$ and $\overline{z}$ are reflections about the $x$-axis.

\item    If $z = a+bi$, then $z$ corresponds to the point $(a,b)$ in the $xy$-plane.  Hence, $-z =-(a+bi) = -a-bi$ corresponds to the point $(-a,-b)$.  Hence, the points corresponding to $z$ and $-z$ are reflections through the origin.

\item  If $z = a+bi$, then $z$ corresponds to the point $(a,b)$ in the $xy$-plane.  Writing out the product $iz$, we get: $iz =i(a+bi) = ia+bi^2 = ia - b = -b+ia$.  Hence, $iz$ corresponds to the point $(-b,a)$.  If $z \neq 0$, then neither $a$ nor $b$ is $0$ (do you see why?) Hence, the slope of the line containing $(0,0)$ and $(a,b)$ is $\frac{b}{a}$ and the slope of the line containing $(0,0)$ and $(-b,a)$ is $-\frac{a}{b}$.  Per Theorem \ref{parallelperpendicularslopetheorem}, since the slopes of these lines are negative reciprocals, the lines themselves are perpendicular.\footnote{We'll be able to show in Section \ref{PolarComplex}  that, more precisely, multiplication by $i$ rotates the complex number counter-clockwise by $90^{\circ}$.}

\item $|z| = \sqrt{a^2+b^2}$ measures the distance from the origin to the point $(a,b)$.  Hence, $|z|$ measures the distance from $z$ to $0$ in the Complex Plane.  This is exactly how $|x|$ is defined in Definition \ref{absvaldistdefn} in Section \ref{AppAbsValEqIneq}.  In that section, however,  the only part of the Complex Plane under discussion is the real number line.

\end{exenum}


