The authors really wanted the Trigonometry portion of Precalculus, Episode IV to start with the definitions of the circular functions so one purpose of this Geometry Review Appendix is to find a home for the material that is prerequisite to those definitions.  Another reason for this Appendix is to further support a ``co-requisite'' approach to teaching a Precalculus\footnote{Remember how we define ``Precalculus'' - to us, Precalculus = College Algebra + College Trigonometry without formal limits.  In order to fully support a ``co-requisite'' approach to a class that has Trigonometry in it, we felt it necessary to provide some material to assist students who have gaps in their Geometry background. The careful reader will note that all of this material was in the main body of our third edition so it can be included nearly seemlessly into a regular Trigonometry class.} class.  As is the case with the Algebra Review Appendix, this chapter is not designed for students who have never seen this material before.  In fact, our treatment of Geometry is even more brief than that of Algebra because we assume a student who is taking a stand alone college-level Trigonometry class is already proficient in College Algebra, and those learning the Trigonometry portion of a full Precalculus class have ostensibly survived the College Algebra portion.  Thus we review only some very basic concepts covered in a typical high school Geometry course.  Where appropriate, we have referenced specific sections of the main body of the Precalculus text in an effort to assist faculty who would like to assign the Appendix as ``just in time'' review reading to their students.  This Appendix contains two sections which are briefly described below:

Section \ref{AppAngles} (Angles in Degrees) is a brief review of some of the terminology and concepts from a typical high school Geometry course.  Radian measure is deferred until Chapter \ref{FoundationsofTrigonometry}.

Section \ref{AppRightTrig} (Basic Right Triangle Trigonometry) defines the trigonometric functions in the context of a right triangle using angles measured in degrees.  Basic applications are discussed and a proof of the Pythagorean Theorem is given but trigonometric identities are deferred until Chapter \ref{FoundationsofTrigonometry}.

\clearpage

\section{Angles in Degrees}

\mfpicnumber{1}

\opengraphsfile{AppAngles}

\setcounter{footnote}{0}

\label{AppAngles}

This section serves as a review of the concept of `angle' and the use of the degree system to measure angles.  Recall that a   \index{ray ! definition of} \textbf{ray} is usually described as a `half-line' and can be thought of as a line segment in which one of the two endpoints is pushed off infinitely distant from the other, as pictured below.  The point from which the ray originates is called the \index{ray ! initial point} \textbf{initial point} of the ray.

\begin{center}

\begin{mfpic}[15]{-1}{7}{-1}{3}
\penwd{1.25pt}
\arrow \polyline{(0,0), (7,2)}
\point[4pt]{(0,0)}
\tlabel[cc](-0.25,-0.5){\scriptsize $P$}
\tcaption{A ray with initial point $P$.}
\end{mfpic}

\end{center}

When two rays share a common initial point they form an \index{angle ! definition} \textbf{angle} and the common initial point is called the \index{angle ! vertex}\index{vertex ! of an angle}\textbf{vertex} of the angle.  Two  examples of what are commonly thought of as angles are

\begin{multicols}{2}

\begin{center}    

\begin{mfpic}[15]{-5}{5}{-3}{3}
\penwd{1.25pt}
\arrow \reverse \arrow \polyline{(-5,2), (0,0), (5,2)}
\point[4pt]{(0,0)}
\tlabel[cc](-0.1,-0.5){\scriptsize $P$}
\tcaption{An angle with vertex $P$.}
\end{mfpic}  

\end{center}

\columnbreak

\begin{center}    

\begin{mfpic}[15]{-5}{7}{-3}{3}
\penwd{1.25pt}
\arrow \reverse \arrow \polyline{(7,2), (0,0), (7,-2)}
\point[4pt]{(0,0)}
\tlabel[cc](-0.25,-0.5){\scriptsize $Q$}
\tcaption{An angle with vertex $Q$.}
\end{mfpic} 

\end{center}

\end{multicols}

However, the two figures below also depict angles - albeit these are, in some sense, extreme cases.  In the first case, the two rays are directly opposite each other forming what is known as a \index{angle ! straight}\index{straight angle}\textbf{straight angle}; in the second, the rays are identical so the `angle' is indistinguishable from the ray itself.

\begin{multicols}{2}
    
\begin{center}
\begin{mfpic}[15]{-5}{5}{-3}{3}
\penwd{1.25pt}
\arrow \reverse \arrow \polyline{(-5,0), (5,0)}
\point[4pt]{(0,0)}
\tlabel[cc](-0.1,-0.5){\scriptsize $P$}
\tcaption{A straight angle.}
\end{mfpic}  
\end{center}

\columnbreak

\begin{center}
\begin{mfpic}[15]{-5}{7}{-3}{3}
\penwd{1.25pt}
\arrow  \polyline{(0,0), (7,-2)}
\point[4pt]{(0,0)}
\tlabel[cc](-0.25,-0.5){\scriptsize $Q$}
\end{mfpic}
\end{center}

\end{multicols}

The \index{angle ! measurement}\index{measure of an angle}\textbf{measure of an angle} is a number which indicates the amount of rotation that separates the rays of the angle.  There is one immediate problem with this, as pictured below. 

\begin{multicols}{2}
    
\begin{center}
\begin{mfpic}[15]{-5}{5}{-3}{3}
\arrow \reverse \arrow \arc[c]{(0,0), (2.5,-0.9), 40}
\penwd{1.25pt}
\arrow \reverse \arrow \polyline{(5,2), (0,0), (5,-2)}
\point[4pt]{(0,0)}
\drawcolor{white} \arc[c]{(0,0), (2.4,-1.1), -310}
\end{mfpic}  
\end{center}

\columnbreak 

\begin{center}
\begin{mfpic}[15]{-5}{5}{-3}{3}
\arrow \reverse \arrow \arc[c]{(0,0), (2.4,-1.1), -310}
\penwd{1.25pt}
\arrow \reverse \arrow \polyline{(5,2), (0,0), (5,-2)}
\point[4pt]{(0,0)}
\end{mfpic}
\end{center}

\end{multicols}

Which amount of rotation are we attempting to quantify?  What we have just discovered is that we have at least two angles described by this diagram.\footnote{The phrase `at least' will be justified in short order.}  Clearly these two angles have different measures because one appears to represent a larger rotation than the other, so we must label them differently.  In this book, we use lower case Greek letters such as $\alpha$ (alpha),   $\beta$ (beta),  $\gamma$ (gamma) and $\theta$ (theta) to label angles.  So, for instance, we have

\[ \begin{mfpic}[15]{-5}{5}{-3}{3}

\arrow \reverse \arrow \arc[c]{(0,0), (2.5,-0.9), 40}
\tlabel[cc](3,0){\scriptsize{$\alpha$}}
\arrow \reverse \arrow \arc[c]{(0,0), (2.4,-1.1), -310}
\tlabel[cc](-3,0){\scriptsize{$\beta$}}
\penwd{1.25pt}
\arrow \reverse \arrow \polyline{(5,2), (0,0), (5,-2)}
\point[3pt]{(0,0)}
\end{mfpic}  \]

One system to measure angles is \index{angle ! degree}\index{degree measure}\textbf{degree measure}.  Quantities measured in degrees are denoted by the symbol `$^{\circ}$.'  One complete revolution as shown below is $360^{\circ}$, and parts of a revolution are measured proportionately.\footnote{The choice of `$360$' is most often attributed to the \href{http://en.wikipedia.org/wiki/Degree_(angle)}{\underline{Babylonians}}.}  Thus half of a revolution (a straight angle) measures $\frac{1}{2} \left(360^{\circ}\right) = 180^{\circ}$, a quarter of a revolution (a \index{right angle}\index{angle ! right}\textbf{right angle}) measures $\frac{1}{4} \left(360^{\circ}\right) = 90^{\circ}$ and so on.

\[ \begin{array}{ccc}
    
\begin{mfpic}[10]{-5}{5}{-3}{3}
\arrow \reverse \arrow \arc[c]{(0,0), (2.5,0.1), 355}
\penwd{1.25pt}
\arrow  \polyline{(0,0), (5,0)}
\point[4pt]{(0,0)}
\tcaption{{\small One revolution $\leftrightarrow 360^{\circ}$}}
\end{mfpic} 

&

\begin{mfpic}[10]{-5}{5}{-3}{3}
\arrow \reverse \arrow \arc[c]{(0,0), (2.5,0.1), 175}
\penwd{1.25pt}
\arrow \reverse \arrow  \polyline{(-5,0), (5,0)}
\drawcolor{white} \arc[c]{(0,0), (2.5,-0.1), -175}
\point[4pt]{(0,0)}
\tcaption{$180^{\circ}$}
\end{mfpic} 

&

\begin{mfpic}[10]{-5}{5}{-3}{5}
\arrow \reverse \arrow \arc[c]{(0,0), (2.5,0.1), 85}
\polyline{(0,0.5), (0.5,0.5), (0.5,0)}
\penwd{1.25pt}
\arrow \reverse \arrow  \polyline{(0,5), (0,0),  (5,0)}
\drawcolor{white} \arc[c]{(0,0), (2.5,-0.1), -265}
\point[4pt]{(0,0)}
\tcaption{$90^{\circ}$}
\end{mfpic} 

\end{array}\]

Note that in the above figure,  we have used the small square `$\! \! \! \! \! \! \qed$' to denote a right angle, as is commonplace in Geometry.  Recall that if an angle measures strictly between $0^{\circ}$ and $90^{\circ}$ it is called an \index{acute angle}\index{angle ! acute}\textbf{acute angle} and if it measures strictly between $90^{\circ}$ and $180^{\circ}$ it is called an \index{obtuse angle}\index{angle ! obtuse}\textbf{obtuse angle}. It is important to note that, theoretically, we can know the measure of any angle as long as we know the proportion it represents of entire revolution.\footnote{This is how a protractor is graded.}  For instance, the measure of an angle which represents a rotation of $\frac{2}{3}$ of a revolution would measure $\frac{2}{3} \left(360^{\circ}\right) = 240^{\circ}$,  the measure of an angle which constitutes only $\frac{1}{12}$ of a revolution measures $\frac{1}{12} \left(360^{\circ}\right) = 30^{\circ}$ and an angle which indicates no rotation at all is measured as $0^{\circ}$.

\[ \begin{array}{ccc}

\begin{mfpic}[10]{-5}{5}{-5}{5}
\arrow \reverse \arrow \arc[c]{(0,0), (2.5,0.1), 235}
\penwd{1.25pt}
\arrow \reverse \arrow \polyline{(-2.5,-4.33), (0,0), (5,0)}
\point[4pt]{(0,0)}
\tcaption{$240^{\circ}$}
\end{mfpic} 

&

\begin{mfpic}[10]{-5}{5}{-5}{5}
\drawcolor{white}
\arc[c]{(0,0), (2.5,0.1), 235}
\drawcolor{black}
\arrow \reverse \arrow \arc[c]{(0,0), (2.5,0.1), 25}
\penwd{1.25pt}\drawcolor{white}
\polyline{(-2.5,-4.33), (0,0), (5,0)}
\drawcolor{black}
\arrow \reverse \arrow  \polyline{(4.33, 2.5), (0,0), (5,0)}
\point[4pt]{(0,0)}

\tcaption{$30^{\circ}$}
\end{mfpic} 

&

\begin{mfpic}[10]{-5}{5}{-5}{5}
\drawcolor{white}
\arc[c]{(0,0), (2.5,0.1), 235}
\penwd{1.25pt}
\polyline{(-2.5,-4.33), (0,0), (5,0)}
\drawcolor{black}
\arrow \polyline{(0,0), (5,0)}
\point[3pt]{(0,0)}

\tcaption{$0^{\circ}$}
\end{mfpic} 

\\  \end{array} \]

Using our definition of degree measure, we have that $1^{\circ}$ represents the measure of an angle which constitutes $\frac{1}{360}$ of a revolution.  Even though it may be hard to draw, it is nonetheless not difficult to imagine an angle with measure smaller than $1^{\circ}$.  There are two ways to subdivide degrees.  The first, and most familiar, is \index{decimal degrees}\index{angle ! decimal degrees}\textbf{decimal degrees}.  For example, an angle with a measure of $30.5^{\circ}$ would represent a rotation halfway between $30^{\circ}$ and $31^{\circ}$, or equivalently, $\frac{30.5}{360} = \frac{61}{720}$ of a full rotation.  This can be taken to the limit using Calculus so that measures like $\sqrt{2}^{\, \circ}$ make sense.\footnote{Awesome math pun aside, this is the same idea behind defining irrational exponents in Section \ref{PowerFunctions}.}  The second way to divide degrees is the \index{angle ! DMS}\index{DMS}\textbf{Degree - Minute - Second} (\textbf{DMS}) system.  In this system, one degree is divided equally into sixty minutes, and in turn, each minute is divided equally into sixty seconds.\footnote{Does this kind of system seem familiar?}  In symbols, we write $1^{\circ} = 60'$ and $1' = 60''$, from which it follows that  $1^{\circ} = 3600''$.  To convert a measure of $42.125^{\circ}$ to the DMS system, we start by noting that $42.125^{\circ} = 42^{\circ} + 0.125^{\circ}$. Converting the partial amount of degrees to minutes, we find $0.125^{\circ} \left( \frac{60'}{1^{\circ}} \right) = 7.5' = 7' + 0.5'$. Converting the partial amount of minutes to seconds gives  $0.5' \left(\frac{60''}{1'} \right) = 30''$.  Putting it all together yields 

\[ \begin{array}{rcl}

42.125^{\circ} & = &  42^{\circ} + 0.125^{\circ} \\
               & = & 42^{\circ} + 7.5' \\
               & = & 42^{\circ} + 7' + 0.5' \\
               & = & 42^{\circ} + 7' + 30'' \\
               & = & 42^{\circ} 7' 30'' \\ \end{array} \]
      
On the other hand, to convert $117^{\circ}15'45''$ to decimal degrees, we first compute $15' \left(\frac{1^{\circ}}{60'}\right) = \frac{1}{4}^{\circ}$ and $45'' \left(\frac{1^{\circ}}{3600''}\right) = \frac{1}{80}^{\circ}$. Then we find

\[ \begin{array}{rcl}

 117^{\circ}15'45'' & = & 117^{\circ} + 15' + 45'' \\ [5pt]
                    & = & 117^{\circ} + \frac{1}{4}^{\circ} + \frac{1}{80}^{\circ} \\ [5pt]
                    & = & \frac{9381}{80}^{\circ} \\ [5pt]
                    & = &  117.2625^{\circ} \\ \end{array} \]

Recall that two acute angles are called \index{complementary angles}\index{angle ! complementary}\textbf{complementary angles} if their measures add to $90^{\circ}$.  Two angles, either a pair of right angles or one acute angle and one obtuse angle, are called \index{supplementary angles}\index{angle ! supplementary}\textbf{supplementary angles} if their measures add to $180^{\circ}$. In the diagram below,  the angles $\alpha$ and $\beta$ are supplementary angles while the pair $\gamma$ and $\theta$ are complementary angles. 

\[ \begin{array}{cc}

\begin{mfpic}[15]{-5}{5}{-5}{5}

\arrow \reverse \arrow \arc[c]{(0,0), (2.5,0.1), 25}
\arrow \reverse \arrow \arc[c]{(0,0), (-2.5,0.1), -145}
\penwd{1.25pt}
\arrow \reverse \arrow  \polyline{(-5,0), (5,0)}
\arrow \polyline{(0,0),  (4.33, 2.5)}
\point[4pt]{(0,0)}
\tlabel[cc](3,0.75){\scriptsize{$\alpha$}}
\tlabel[cc](-0.5,3){\scriptsize{$\beta$}}
\tcaption{Supplementary Angles}
\end{mfpic} 

&

\hspace{0.5in}

\begin{mfpic}[15]{-5}{6}{-5}{5}
\arrow \reverse \arrow \arc[c]{(0,0), (2.5,0.1), 25}
\arrow \reverse \arrow \arc[c]{(0,0), (0.1,2.5), -55}
\penwd{1.25pt}
\arrow \reverse \arrow  \polyline{(0,5), (0,0), (6,0)}
\arrow \polyline{(0,0),  (4.33, 2.5)}
\point[4pt]{(0,0)}
\tlabel[cc](3,0.75){\scriptsize{$\gamma$}}
\tlabel[cc](1.5,2.5){\scriptsize{$\theta$}}
\tcaption{Complementary Angles}
\end{mfpic} 

\\  \end{array} \]

In practice, the distinction between the angle itself and its measure is blurred so that the sentence `$\alpha$ is an angle measuring $42^{\circ}$' is often abbreviated as `$\alpha = 42^{\circ}$.'  It is now time for an example.

\begin{ex} \label{degreeex}  Let $\alpha = 111.371^{\circ}$  and $\beta = 37^{\circ}28'17''$.

\begin{enumerate}

\item  Convert $\alpha$ to the DMS system.  Round your answer to the nearest second.

\item  Convert $\beta$ to decimal degrees.  Round your answer to the nearest thousandth of a degree.

\item  Sketch $\alpha$ and $\beta$.

\item  Find a supplementary angle for $\alpha$.

\item  Find a complementary angle for $\beta$.

\end{enumerate}

{\bf Solution.}

\begin{enumerate}

\item  To convert $\alpha$ to the DMS system, we start with $111.371^{\circ} = 111^{\circ}+ 0.371^{\circ}$.  Next we convert $0.371^{\circ} \left(\frac{60'}{1^{\circ}}\right) = 22.26'$.  Writing $22.26' = 22'+ 0.26'$, we convert $0.26' \left( \frac{60''}{1'} \right) = 15.6''$.  Hence,

\[ \begin{array}{rcl}

111.371^{\circ} & = & 111^{\circ} + 0.371^{\circ} \\
                & = & 111^{\circ} + 22.26' \\
                & = & 111^{\circ} + 22' + 0.26' \\
                & = & 111^{\circ} + 22' + 15.6'' \\
                & = & 111^{\circ}22'15.6'' \\ \end{array} \]

Rounding to seconds, we obtain $\alpha \approx 111^{\circ}22'16''$.

\item  To convert $\beta$ to decimal degrees, we convert $28' \left(\frac{1^{\circ}}{60'}\right) = \frac{7}{15}^{\, \circ}$ and $17''\left(\frac{1^{\circ}}{3600'}\right) = \frac{17}{3600}^{\, \circ}$.  Putting it all together, we have

\[ \begin{array}{rcl}

 37^{\circ}28'17'' & = & 37^{\circ} + 28' + 17'' \\ [5pt]
                   & = & 37^{\circ} +  \frac{7}{15}^{\, \circ} + \frac{17}{3600}^{\, \circ} \\ [5pt]
                   & = & \frac{134897}{3600}^{\circ} \\ [5pt]
                   & \approx & 37.471^{\circ} \\ \end{array} \]

\item  To sketch $\alpha$, we first note that $90^{\circ} < \alpha < 180^{\circ}$.  Dividing this range in half, we get $90^{\circ} < \alpha < 135^{\circ}$, and once more, we have $90^{\circ} < \alpha < 112.5^{\circ}$.  This gives us a pretty good estimate for $\alpha$, as shown below.\footnote{If this process seems hauntingly familiar, it should. Compare this method to the Bisection Method introduced in Section \ref{IVTaninequalities}.}  Proceeding similarly for $\beta$, we find $0^{\circ} < \beta < 90^{\circ}$, then $0^{\circ} < \beta < 45^{\circ}$, $22.5^{\circ} < \beta < 45^{\circ}$, and lastly, $33.75^{\circ} < \beta < 45^{\circ}$.  

\[ \begin{array}{cc}

\begin{mfpic}[15]{-5}{5}{-5}{5}
\dotted \polyline{ (-5,0), (0,0), (0,5)}
\dotted \polyline{ (-3.5355,3.5355), (0,0)}
\dotted \polyline{ (-1.9134,4.6194), (0,0)}
\arrow \reverse \arrow \arc[c]{(0,0), (2.5,0.1), 107}
\penwd{1.25pt}
\arrow \reverse \arrow  \polyline{(-1.822, 4.656), (0,0), (5,0)}
\point[4pt]{(0,0)}
\tcaption{Angle $\alpha$}
\end{mfpic} 

&

\hspace{0.5in}

\begin{mfpic}[15]{-5}{5}{-5}{5}
\dotted \polyline{ (0,5), (0,0), (5,0)}
\dotted \polyline{ (3.5355,3.5355), (0,0)}
\dotted \polyline{ (4.6194,1.9134), (0,0)}
\dotted \polyline{ (4.1573,2.7778), (0,0)}
\arrow \reverse \arrow \arc[c]{(0,0), (2.5,0.1), 33}
\penwd{1.25pt}
\arrow \reverse \arrow  \polyline{(3.9683, 3.0417), (0,0), (5,0)}
\point[4pt]{(0,0)}
\tcaption{Angle $\beta$}
\end{mfpic}  \\ \end{array} \]

\item  To find a supplementary angle for $\alpha$, we seek an angle $\theta$ so that $\alpha + \theta = 180^{\circ}$.  We get $\theta = 180^{\circ} - \alpha =  180^{\circ} - 111.371^{\circ} = 68.629^{\circ}$.

\item  To find a complementary  angle for $\beta$, we seek an angle $\gamma$ so that $\beta + \gamma = 90^{\circ}$.  We get $\gamma = 90^{\circ} - \beta =  90^{\circ} - 37^{\circ}28'17''$.  While we could reach for the calculator to obtain an approximate answer, we choose instead to do a bit of sexagesimal\footnote{Like `latus rectum,' this is also a real math term.} arithmetic.  We first rewrite  $90^{\circ} = 90^{\circ} 0' 0'' =  89^{\circ}60' 0'' =  89^{\circ}59'60''$. In essence, we are `borrowing' $1^{\circ} = 60'$ from the degree place,  and then borrowing $1' = 60''$ from the minutes place.\footnote{This is the exact same kind of `borrowing' you used to do in Elementary School when trying to find $300 - 125$. Back then, you were working in a base ten system;  here, it is base sixty.} This yields, $\gamma = 90^{\circ} - 37^{\circ}28'17'' = 89^{\circ}59'60'' - 37^{\circ}28'17'' = 52^{\circ}31'43''$.  \qed

\end{enumerate}

\end{ex} 

Up to this point, we have discussed only angles which measure between $0^{\circ}$ and $360^{\circ}$, inclusive.  Ultimately, we want to use the arsenal of Algebra which we have stockpiled in Chapters \ref{IntroductiontoFunctions} through \ref{SequencesandtheBinomialTheorem} to not only solve geometric problems involving angles, but also to extend their applicability to other real-world phenomena.  A first step in this direction is to extend our notion of `angle' from merely measuring an extent of rotation to quantities which indicate an amount of rotation along with a \textbf{direction}.  To that end, we introduce the concept of an \index{angle ! oriented}\index{oriented angle}\textbf{oriented angle}.  As its name suggests, in an oriented angle, the direction of the rotation is important.  We imagine the angle being swept out starting from an \index{angle ! initial side}\index{initial side of an angle}\textbf{initial side} and ending at a \index{angle ! terminal side}\index{terminal side of an angle}\textbf{terminal side}, as shown below.  When the rotation is counter-clockwise\footnote{`widdershins'} from initial side to terminal side, we say that the angle is \index{angle ! positive}\index{positive angle}\textbf{positive}; when the rotation is clockwise, we say that the angle is \index{angle ! negative}\index{negative angle}\textbf{negative}.

\[ \begin{array}{cc}

\begin{mfpic}[15]{-5}{5}{-5}{5}
\arrow \arc[c]{(0,0), (2.5,0.1), 40}
\penwd{1.25pt}
\arrow \reverse \arrow  \polyline{(3.5355, 3.5355), (0,0), (5,0)}
\point[4pt]{(0,0)}
\tlabel[cc](2, -0.5){\tiny Initial Side}
\tlabel[cc](1.5,2){\tiny \rotatebox{45}{Terminal Side}}
\end{mfpic}

&

\hspace{.5in}

\begin{mfpic}[15]{-5}{5}{-5}{5}
\arrow \arc[c]{(0,0), (2.5,-0.1), -40}
\penwd{1.25pt}
\arrow \reverse \arrow  \polyline{(3.5355, -3.5355), (0,0), (5,0)}
\point[4pt]{(0,0)}
\tlabel[cc](2, 0.5){\tiny Initial Side}
\tlabel[cc](1.5,-2){\tiny \rotatebox{-45}{Terminal Side}}
\end{mfpic} \\ 

\text{A positive angle, $45^{\circ}$} & \hspace{.5in} \text{A negative angle, $-45^{\circ}$}

\end{array} \]

At this point, we also extend our allowable rotations to include angles which encompass more than one revolution.  For example, to sketch an angle with measure $450^{\circ}$ we start with an initial side, rotate counter-clockwise one complete revolution (to take care of the `first' $360^{\circ}$) then continue with an additional $90^{\circ}$ counter-clockwise rotation, as seen below.

\begin{center}

\begin{mfpic}[15]{-5}{5}{-3}{5}
\arrow \parafcn{0,445,5}{(t+200)*dir(t)/200} 
\tcaption{$450^{\circ}$}
\penwd{1.25pt}
\arrow \reverse \arrow  \polyline{(0,5), (0,0),  (5,0)}
\point[4pt]{(0,0)}
\end{mfpic} 

\end{center}

To further connect angles with the Algebra which has come before, we shall often overlay an angle diagram on the coordinate plane.  An angle is said to be in \index{angle ! standard position}\index{standard position of an angle}\textbf{standard position} if its vertex is the origin and its initial side coincides with the positive horizontal (usually labeled as the $x$-) axis.  Angles in standard position are classified according to where their terminal side lies.  For instance, an angle in standard position whose terminal side lies in Quadrant I is called a `Quadrant I angle'.  If the terminal side of an angle lies on one of the coordinate axes, it is called a \index{angle ! quadrantal}\index{quadrantal angle}\textbf{quadrantal angle}.  Two angles in standard position are called \index{angle ! coterminal}\index{coterminal angle}\textbf{coterminal} if they share the same terminal side.\footnote{Note that by being in standard position they automatically share the same initial side which is the positive $x$-axis.}  In the figure below, $\alpha = 120^{\circ}$ and $\beta = -240^{\circ}$ are two coterminal Quadrant II angles drawn in standard position.    Note that $\alpha = \beta + 360^{\circ}$, or equivalently, $\beta = \alpha - 360^{\circ}$. We leave it as an exercise to the reader to verify that coterminal angles always differ by a multiple of $360^{\circ}$.\footnote{It is worth noting that all of the pathologies of Analytic Trigonometry result from this innocuous fact.} More precisely, if $\alpha$ and $\beta$ are coterminal angles, then $\beta = \alpha + 360^{\circ} \cdot k$ where $k$ is an integer.\footnote{Recall that this means $k = 0, \pm 1, \pm 2, \ldots$.}

\begin{center}

\begin{mfpic}[15]{-5}{5}{-5}{5}
\drawcolor[gray]{0.7}
\axes
\xmarks{-4,-3,-2,-1,1,2,3,4}
\ymarks{-4,-3,-2,-1,1,2,3,4}
\tlabel(5,-0.5){\scriptsize $x$}
\tlabel(0.25,4.75){\scriptsize $y$}
\tlabel(2,2){\scriptsize $\alpha = 120^{\circ}$}
\tlabel(-5,-2){\scriptsize $\beta = -240^{\circ}$}
%\drawcolor[rgb]{0.33,0.33,0.33}
\drawcolor{black}
\arrow \arc[c]{(0,0), (2.5,0.1), 115}
\arrow \arc[c]{(0,0), (2.5,-0.1), -235}
\penwd{1.25pt}
\arrow \reverse \arrow \polyline{(-2.5, 4.3301), (0,0), (5,0)}
\point[4pt]{(0,0)}
\tlpointsep{5pt}
\scriptsize
\axislabels {x}{{$-4 \hspace{7pt}$} -4, {$-3 \hspace{7pt}$} -3, {$-2 \hspace{7pt}$} -2, {$-1 \hspace{7pt}$} -1, {$1$} 1, {$2$} 2, {$3$} 3, {$4$} 4}
\axislabels {y}{{$-1$} -1, {$-2$} -2, {$-3$} -3, {$-4$} -4, {$1$} 1, {$2$} 2, {$3$} 3, {$4$} 4}
\normalsize
\end{mfpic}

Two coterminal angles, $\alpha = 120^{\circ}$ and $\beta = -240^{\circ}$, in standard position.

\end{center}

\begin{ex}  \label{orientedcoterminaldegree} Graph each of the (oriented) angles below in standard position and classify them according to where their terminal side lies. Find three coterminal angles, at least one of which is positive and one of which is negative.

\begin{tasks}(4)

\task  $\alpha = 60^{\circ}$

\task  $\beta = -225^{\circ}$

\task  $\gamma = 540^{\circ}$

\task  $\phi = -750^{\circ}$

\end{tasks}

{\bf Solution.}  

\begin{enumerate}

\item  To graph $\alpha = 60^{\circ}$, we draw an angle with its initial side on the positive $x$-axis and rotate counter-clockwise $\frac{60^{\circ}}{360^{\circ}} = \frac{1}{6}$ of a revolution.  We see that $\alpha$ is a Quadrant I angle.  To find angles which are coterminal, we look for angles $\theta$ of the form $\theta = \alpha + 360^{\circ} \cdot k$, for some integer $k$.  When $k = 1$, we get $\theta =  60^{\circ} + 360^{\circ} = 420^{\circ}$.   Substituting $k = -1$ gives $\theta = 60^{\circ} - 360^{\circ} = -300^{\circ}$.  Finally, if we let $k = 2$, we get $\theta =  60^{\circ} + 720^{\circ} = 780^{\circ}$.  

\item  Since $\beta = - 225^{\circ}$ is negative, we start at the positive $x$-axis and rotate \textit{clockwise} $\frac{225^{\circ}}{360^{\circ}} = \frac{5}{8}$ of a revolution. We see that $\beta$ is a Quadrant II angle.  To find coterminal angles, we proceed as before and compute $\theta = -225^{\circ} + 360^{\circ} \cdot k$ for integer values of $k$.  We find $135^{\circ}$, $-585^{\circ}$ and $495^{\circ}$ are all coterminal with $-225^{\circ}$.   

\begin{center}

\begin{mfpic}[15]{-5}{5}{-5}{5}
\drawcolor[gray]{0.7}
\axes
\xmarks{-4,-3,-2,-1,1,2,3,4}
\ymarks{-4,-3,-2,-1,1,2,3,4}
\tlabel(5,-0.5){\scriptsize $x$}
\tlabel(0.25,4.75){\scriptsize $y$}
\tlabel(2.5,1){\scriptsize $\alpha = 60^{\circ}$}
\drawcolor{black}
\arrow \arc[c]{(0,0), (2.5,0.1), 55}
\penwd{1.25pt}
\arrow \reverse \arrow \polyline{(2.5, 4.3301), (0,0), (5,0)}
\point[4pt]{(0,0)}

\tlpointsep{5pt}
\scriptsize
\axislabels {x}{{$-4 \hspace{7pt}$} -4, {$-3 \hspace{7pt}$} -3, {$-2 \hspace{7pt}$} -2, {$-1 \hspace{7pt}$} -1, {$1$} 1, {$2$} 2, {$3$} 3, {$4$} 4}
\axislabels {y}{{$-1$} -1, {$-2$} -2, {$-3$} -3, {$-4$} -4, {$1$} 1, {$2$} 2, {$3$} 3, {$4$} 4}
\normalsize
\end{mfpic}

$\alpha = 60^{\circ}$ in standard position. 

\bigskip

\begin{mfpic}[15]{-5}{5}{-5}{5}
\drawcolor[gray]{0.7}
\axes
\xmarks{-4,-3,-2,-1,1,2,3,4}
\ymarks{-4,-3,-2,-1,1,2,3,4}
\tlabel(5,-0.5){\scriptsize $x$}
\tlabel(0.25,4.75){\scriptsize $y$}
\tlabel(-4.5,-2.5){\scriptsize $\beta = -225^{\circ}$}
%\drawcolor[rgb]{0.33,0.33,0.33}
\drawcolor{black}
\arrow \arc[c]{(0,0), (2.5,-0.1), -220}
\penwd{1.25pt}
\arrow \reverse \arrow \polyline{(-3.801, 3.801), (0,0), (5,0)}
\point[4pt]{(0,0)}
\tlpointsep{5pt}
\scriptsize
\axislabels {x}{{$-4 \hspace{7pt}$} -4, {$-3 \hspace{7pt}$} -3, {$-2 \hspace{7pt}$} -2, {$-1 \hspace{7pt}$} -1, {$1$} 1, {$2$} 2, {$3$} 3, {$4$} 4}
\axislabels {y}{{$-1$} -1, {$-2$} -2, {$-3$} -3, {$-4$} -4, {$1$} 1, {$2$} 2, {$3$} 3, {$4$} 4}
\normalsize
\end{mfpic} 

$\beta = -225^{\circ}$ in standard position.

\end{center}

\item Since $\gamma = 540^{\circ}$ is positive, we rotate counter-clockwise from the positive $x$-axis.  One full revolution accounts for $360^{\circ}$, with $180^{\circ}$, or $\frac{1}{2}$ of a revolution remaining.  Since the terminal side of $\gamma$ lies on the negative $x$-axis, $\gamma$ is a quadrantal angle.  All angles coterminal with $\gamma$ are of the form $\theta = 540^{\circ} + 360^{\circ} \cdot k$, where $k$ is an integer.  Working through the arithmetic, we find three such angles: $180^{\circ}$, $-180^{\circ}$ and $900^{\circ}$.

\item  The Greek letter $\phi$ is pronounced `fee' or `fie' and since $\phi$ is negative, we begin our rotation clockwise from the positive $x$-axis.  Two full revolutions account for $720^{\circ}$, with just $30^{\circ}$ or $\frac{1}{12}$ of a revolution to go. We find that $\phi$ is a Quadrant IV angle. To find coterminal angles, we compute $\theta = -750^{\circ} +   360^{\circ} \cdot k$ for a few integers $k$ and obtain $-390^{\circ}$, $-30^{\circ}$ and $330^{\circ}$.

\begin{center}

\begin{mfpic}[15]{-5}{5}{-5}{5}
\drawcolor[gray]{0.7}
\axes
\xmarks{-4,-3,-2,-1,1,2,3,4}
\ymarks{-4,-3,-2,-1,1,2,3,4}
\tlabel(5,-0.5){\scriptsize $x$}
\tlabel(0.25,4.75){\scriptsize $y$}
\tlabel(-4.5,2.5){\scriptsize $\gamma = 540^{\circ}$}
\drawcolor{black}
\arrow \parafcn{0,535,5}{(t+200)*dir(t)/200} 
\penwd{1.25pt}
\arrow \reverse \arrow \polyline{(-5,0), (0,0), (5,0)}
\point[4pt]{(0,0)}

\tlpointsep{5pt}
\scriptsize
\axislabels {x}{{$-4 \hspace{7pt}$} -4, {$-3 \hspace{7pt}$} -3, {$-2 \hspace{7pt}$} -2, {$-1 \hspace{7pt}$} -1, {$1$} 1, {$2$} 2, {$3$} 3, {$4$} 4}
\axislabels {y}{{$-1$} -1, {$-2$} -2, {$-3$} -3, {$-4$} -4, {$1$} 1, {$2$} 2, {$3$} 3, {$4$} 4}
\normalsize
\end{mfpic}

$\gamma = 540^{\circ}$ in standard position.

\bigskip

\begin{mfpic}[15]{-5}{5}{-5}{5}
\drawcolor[gray]{0.7}
\axes
\xmarks{-4,-3,-2,-1,1,2,3,4}
\ymarks{-4,-3,-2,-1,1,2,3,4}
\tlabel(5,-0.5){\scriptsize $x$}
\tlabel(0.25,4.75){\scriptsize $y$}
\tlabel(0.5,-2.5){\scriptsize $\phi = -750^{\circ}$}
\drawcolor{black}
\arrow \parafcn{0,745,5}{(t+100)*dir(0-t)/400}
\penwd{1.25pt}
\arrow \reverse \arrow \polyline{(4.3301, -2.5), (0,0), (5,0)}
\point[4pt]{(0,0)}

\tlpointsep{5pt}
\scriptsize
\axislabels {x}{{$-4 \hspace{7pt}$} -4, {$-3 \hspace{7pt}$} -3, {$-2 \hspace{7pt}$} -2, {$-1 \hspace{7pt}$} -1, {$1$} 1, {$2$} 2, {$3$} 3, {$4$} 4}
\axislabels {y}{{$-1$} -1, {$-2$} -2, {$-3$} -3, {$-4$} -4, {$1$} 1, {$2$} 2, {$3$} 3, {$4$} 4}
\normalsize
\end{mfpic} 

$\phi = -750^{\circ}$ in standard position.

\end{center}

\end{enumerate}
\qed

\end{ex}

Note that since there are infinitely many integers, any given angle has infinitely many coterminal angles, and the reader is encouraged to plot the few sets of coterminal angles found in Example \ref{orientedcoterminaldegree} to see this.  

\smallskip

As we'll see in Section \ref{AppRightTrig} and throughout Chapter \ref{GeometricApplicationsofTrigonometry}, degree measure is very popular for many applications involving geometry and modeling physical forces.  In Section \ref{RadianMeasure}, we'll introduce a different method of measuring angles, \textbf{radian measure}, which is tied directly to arc length and is useful in other applications involving circular motion and periodic phenomenon. 

\newpage

\subsection{Exercises}

\label{ExercisesforAppAngles}

In Exercises \ref{dmsfirst} - \ref{dmslast}, convert the angles into the DMS system.  Round each of your answers to the nearest second.

\begin{tasks}(4)

\task $63.75^{\circ}$ \label{dmsfirst}
\task $200.325^{\circ}$
\task $-317.06^{\circ}$
\task $179.999^{\circ}$ \label{dmslast}

\end{tasks}

In Exercises \ref{decimaldegfirst} - \ref{decimaldeglast}, convert the angles into decimal degrees.  Round each of your answers to three decimal places.

\begin{tasks}[resume](4)

\task $125^{\circ} 50'$ \label{decimaldegfirst}
\task $-32^{\circ} 10' 12''$
\task $502^{\circ} 35'$
\task $237^{\circ} 58' 43''$ \label{decimaldeglast}

\end{tasks}

In Exercises \ref{orientedanglefirst} - \ref{orientedanglelast}, graph the oriented angle in standard position. Classify each angle according to where its terminal side lies and then give two coterminal angles, one of which is positive and the other negative.

\begin{tasks}(4)

\task  $30^{\circ}$  \label{orientedanglefirst}

\task  $120^{\circ}$

\task  $225^{\circ}$

\task $330^{\circ}$ 
\task  $-30^{\circ}$

\task $-135^{\circ}$ 

\task $-240^{\circ}$

\task $-270^{\circ}$

\task $405^{\circ}$  

\task $840^{\circ}$ 

\task $-510^{\circ}$

\task $-900^{\circ}$

\label{orientedanglelast}

\task! With help from your classmates, explain why if $(x,y)$ is a point on the terminal side of an angle $\alpha$ in standard position, then so is $(r\,x, r\,y)$ for any number $r > 0$.  What happens if $r < 0$?

\end{tasks}

\clearpage

\subsection{Answers}

\begin{tasks}(2)

\task $63^{\circ} 45'$
\task $200^{\circ} 19' 30''$
\task $-317^{\circ} 3' 36''$
\task $179^{\circ} 59' 56''$

\task $125.833^{\circ}$
\task $-32.17^{\circ}$
\task $502.583^{\circ}$
\task $237.979^{\circ}$

\task $30^{\circ}$ is a Quadrant I angle\\
coterminal with $390^{\circ}$ and $-330^{\circ}$

\medskip

\begin{mfpic}[12]{-5}{5}{-5}{5}
\drawcolor[gray]{0.7}
\axes
\xmarks{-4,-3,-2,-1,1,2,3,4}
\ymarks{-4,-3,-2,-1,1,2,3,4}
\tlabel(5,-0.5){\scriptsize $x$}
\tlabel(0.25,4.75){\scriptsize $y$}
%\drawcolor[rgb]{0.33,0.33,0.33}
\drawcolor{black}
\arrow \arc[c]{(0,0), (2.5,0.1), 25}
\penwd{1.25pt}
\arrow \reverse \arrow \polyline{(4.3301, 2.5), (0,0), (5,0)}
\point[4pt]{(0,0)}

\tlpointsep{5pt}
\scriptsize
\axislabels {x}{{$-4 \hspace{7pt}$} -4, {$-3 \hspace{7pt}$} -3, {$-2 \hspace{7pt}$} -2, {$-1 \hspace{7pt}$} -1, {$1$} 1, {$2$} 2, {$3$} 3, {$4$} 4}
\axislabels {y}{{$-1$} -1, {$-2$} -2, {$-3$} -3, {$-4$} -4, {$1$} 1, {$2$} 2, {$3$} 3, {$4$} 4}
\normalsize
\end{mfpic}

\task $120^{\circ}$ is a Quadrant II angle\\
coterminal with $480^{\circ}$ and $-240^{\circ}$

\medskip

\begin{mfpic}[12]{-5}{5}{-5}{5}
\drawcolor[gray]{0.7}
\axes
\xmarks{-4,-3,-2,-1,1,2,3,4}
\ymarks{-4,-3,-2,-1,1,2,3,4}
\tlabel(5,-0.5){\scriptsize $x$}
\tlabel(0.25,4.75){\scriptsize $y$}
%\drawcolor[rgb]{0.33,0.33,0.33}
\drawcolor{black}
\arrow \arc[c]{(0,0), (2.5,0.1), 115}
\penwd{1.25pt}
\arrow \reverse \arrow \polyline{(-2.5, 4.3301), (0,0), (5,0)}
\point[4pt]{(0,0)}
\tlpointsep{5pt}
\scriptsize
\axislabels {x}{{$-4 \hspace{7pt}$} -4, {$-3 \hspace{7pt}$} -3, {$-2 \hspace{7pt}$} -2, {$-1 \hspace{7pt}$} -1, {$1$} 1, {$2$} 2, {$3$} 3, {$4$} 4}
\axislabels {y}{{$-1$} -1, {$-2$} -2, {$-3$} -3, {$-4$} -4,  {$2$} 2, {$3$} 3, {$4$} 4}
\normalsize
\end{mfpic}


\task $225^{\circ}$ is a Quadrant III angle\\
coterminal with $585^{\circ}$ and $-135^{\circ}$

\medskip

\begin{mfpic}[12]{-5}{5}{-5}{5}
\drawcolor[gray]{0.7}
\axes
\xmarks{-4,-3,-2,-1,1,2,3,4}
\ymarks{-4,-3,-2,-1,1,2,3,4}
\tlabel(5,-0.5){\scriptsize $x$}
\tlabel(0.25,4.75){\scriptsize $y$}
%\drawcolor[rgb]{0.33,0.33,0.33}
\drawcolor{black}
\arrow \arc[c]{(0,0), (2.5,0.1), 220}
\penwd{1.25pt}
\arrow \reverse \arrow \polyline{(-3.5355, -3.5355), (0,0), (5,0)}
\point[4pt]{(0,0)}
\tlpointsep{5pt}
\scriptsize
\axislabels {x}{{$-4 \hspace{7pt}$} -4, {$-3 \hspace{7pt}$} -3, {$-2 \hspace{7pt}$} -2, {$-1 \hspace{7pt}$} -1, {$1$} 1, {$2$} 2, {$3$} 3, {$4$} 4}
\axislabels {y}{{$-1$} -1, {$-2$} -2, {$-3$} -3, {$-4$} -4, {$1$} 1, {$2$} 2, {$3$} 3, {$4$} 4}
\normalsize
\end{mfpic}

\task $330^{\circ}$ is a Quadrant IV angle\\
coterminal with $690^{\circ}$ and $-30^{\circ}$

\medskip

\begin{mfpic}[12]{-5}{5}{-5}{5}
\drawcolor[gray]{0.7}
\axes
\xmarks{-4,-3,-2,-1,1,2,3,4}
\ymarks{-4,-3,-2,-1,1,2,3,4}
\tlabel(5,-0.5){\scriptsize $x$}
\tlabel(0.25,4.75){\scriptsize $y$}
%\drawcolor[rgb]{0.33,0.33,0.33}
\drawcolor{black}
\arrow \arc[c]{(0,0), (2.5,0.1), 325}
\penwd{1.25pt}
\arrow \reverse \arrow \polyline{(4.3301, -2.5), (0,0), (5,0)}
\point[4pt]{(0,0)}
\tlpointsep{5pt}
\scriptsize
\axislabels {x}{{$-4 \hspace{7pt}$} -4, {$-3 \hspace{7pt}$} -3, {$-2 \hspace{7pt}$} -2, {$-1 \hspace{7pt}$} -1,  {$2$} 2, {$3$} 3, {$4$} 4}
\axislabels {y}{{$-1$} -1, {$-2$} -2, {$-3$} -3, {$-4$} -4, {$1$} 1, {$2$} 2, {$3$} 3, {$4$} 4}
\normalsize
\end{mfpic}

\task $-30^{\circ}$ is a Quadrant IV angle\\
coterminal with $330^{\circ}$ and $-390^{\circ}$

\medskip

\begin{mfpic}[12]{-5}{5}{-5}{5}
\drawcolor[gray]{0.7}
\axes
\xmarks{-4,-3,-2,-1,1,2,3,4}
\ymarks{-4,-3,-2,-1,1,2,3,4}
\tlabel(5,-0.5){\scriptsize $x$}
\tlabel(0.25,4.75){\scriptsize $y$}
%\drawcolor[rgb]{0.33,0.33,0.33}
\drawcolor{black}
\arrow \arc[c]{(0,0), (2.5, -0.1), -25}
\penwd{1.25pt}
\arrow \reverse \arrow \polyline{(4.3301, -2.5), (0,0), (5,0)}
\point[4pt]{(0,0)}

\tlpointsep{5pt}
\scriptsize
\axislabels {x}{{$-4 \hspace{7pt}$} -4, {$-3 \hspace{7pt}$} -3, {$-2 \hspace{7pt}$} -2, {$-1 \hspace{7pt}$} -1, {$2$} 2, {$3$} 3, {$4$} 4}
\axislabels {y}{{$-1$} -1, {$-2$} -2, {$-3$} -3, {$-4$} -4, {$1$} 1, {$2$} 2, {$3$} 3, {$4$} 4}
\normalsize
\end{mfpic}

\task $-135^{\circ}$ is a Quadrant III angle\\
coterminal with $225^{\circ}$ and $-495^{\circ}$

\medskip

\begin{mfpic}[12]{-5}{5}{-5}{5}
\drawcolor[gray]{0.7}
\axes
\xmarks{-4,-3,-2,-1,1,2,3,4}
\ymarks{-4,-3,-2,-1,1,2,3,4}
\tlabel(5,-0.5){\scriptsize $x$}
\tlabel(0.25,4.75){\scriptsize $y$}
%\drawcolor[rgb]{0.33,0.33,0.33}
\drawcolor{black}
\arrow \arc[c]{(0,0), (2.5, -0.1), -130}
\penwd{1.25pt}
\arrow \reverse \arrow \polyline{(-3.5355, -3.5355), (0,0), (5,0)}
\point[4pt]{(0,0)}

\tlpointsep{5pt}
\scriptsize
\axislabels {x}{{$-4 \hspace{7pt}$} -4, {$-3 \hspace{7pt}$} -3, {$-2 \hspace{7pt}$} -2, {$-1 \hspace{7pt}$} -1, {$1$} 1, {$2$} 2, {$3$} 3, {$4$} 4}
\axislabels {y}{{$-2$} -2, {$-3$} -3, {$-4$} -4, {$1$} 1, {$2$} 2, {$3$} 3, {$4$} 4}
\normalsize
\end{mfpic}

\task $-240^{\circ}$ is a Quadrant II angle\\
coterminal with $120^{\circ}$ and $-600^{\circ}$

\medskip

\begin{mfpic}[12]{-5}{5}{-5}{5}
\drawcolor[gray]{0.7}
\axes
\xmarks{-4,-3,-2,-1,1,2,3,4}
\ymarks{-4,-3,-2,-1,1,2,3,4}
\tlabel(5,-0.5){\scriptsize $x$}
\tlabel(0.25,4.75){\scriptsize $y$}
%\drawcolor[rgb]{0.33,0.33,0.33}
\drawcolor{black}
\arrow \arc[c]{(0,0), (2.5, -0.1), -235}
\penwd{1.25pt}
\arrow \reverse \arrow \polyline{(-2.5, 4.3301), (0,0), (5,0)}
\point[4pt]{(0,0)}
\tlpointsep{5pt}
\scriptsize
\axislabels {x}{{$-4 \hspace{7pt}$} -4, {$-3 \hspace{7pt}$} -3, {$-2 \hspace{7pt}$} -2, {$-1 \hspace{7pt}$} -1, {$1$} 1, {$2$} 2, {$3$} 3, {$4$} 4}
\axislabels {y}{{$-1$} -1, {$-2$} -2, {$-3$} -3, {$-4$} -4,  {$2$} 2, {$3$} 3, {$4$} 4}
\normalsize
\end{mfpic}

\task $-270^{\circ}$ is a quadrantal angle \\
coterminal with $90^{\circ}$ and $-630^{\circ}$

\begin{mfpic}[12]{-5}{5}{-5}{5}
\drawcolor[gray]{0.7}
\axes
\xmarks{-4,-3,-2,-1,1,2,3,4}
\ymarks{-4,-3,-2,-1,1,2,3,4}
\tlabel(5,-0.5){\scriptsize $x$}
\tlabel(0.25,4.75){\scriptsize $y$}
%\drawcolor[rgb]{0.33,0.33,0.33}
\drawcolor{black}
\arrow \arc[c]{(0,0), (2.5, -0.1), -265}
\penwd{1.25pt}
\arrow \reverse \arrow \polyline{(0, 5), (0,0), (5,0)}
\point[4pt]{(0,0)}

\tlpointsep{5pt}
\scriptsize
\axislabels {x}{{$-4 \hspace{7pt}$} -4, {$-3 \hspace{7pt}$} -3, {$-2 \hspace{7pt}$} -2, {$-1 \hspace{7pt}$} -1, {$1$} 1, {$2$} 2, {$3$} 3, {$4$} 4}
\axislabels {y}{{$-1$} -1, {$-2$} -2, {$-3$} -3, {$-4$} -4, {$1$} 1, {$2$} 2, {$3$} 3, {$4$} 4}
\normalsize
\end{mfpic}

\task $405^{\circ}$ is a Quadrant I angle\\
coterminal with $45^{\circ}$ and $-315^{\circ}$

\medskip

\begin{mfpic}[12]{-5}{5}{-5}{5}
\drawcolor[gray]{0.7}
\axes
\xmarks{-4,-3,-2,-1,1,2,3,4}
\ymarks{-4,-3,-2,-1,1,2,3,4}
\tlabel(5,-0.5){\scriptsize $x$}
\tlabel(0.25,4.75){\scriptsize $y$}
%\drawcolor[rgb]{0.33,0.33,0.33}
\drawcolor{black}

\arrow \parafcn{0,400,5}{(t+100)*dir(t)/400}
\penwd{1.25pt}
\arrow \reverse \arrow \polyline{(3.5355,3.5355), (0,0), (5,0)}
\point[4pt]{(0,0)}

\tlpointsep{5pt}
\scriptsize
\axislabels {x}{{$-4 \hspace{7pt}$} -4, {$-3 \hspace{7pt}$} -3, {$-2 \hspace{7pt}$} -2, {$-1 \hspace{7pt}$} -1, {$1$} 1, {$2$} 2, {$3$} 3, {$4$} 4}
\axislabels {y}{{$-1$} -1, {$-2$} -2, {$-3$} -3, {$-4$} -4, {$1$} 1, {$2$} 2, {$3$} 3, {$4$} 4}
\normalsize
\end{mfpic} 

\task $840^{\circ}$ is a Quadrant II angle\\
coterminal with $120^{\circ}$ and $-240^{\circ}$

\medskip

\begin{mfpic}[12]{-5}{5}{-5}{5}
\drawcolor[gray]{0.7}
\axes
\xmarks{-4,-3,-2,-1,1,2,3,4}
\ymarks{-4,-3,-2,-1,1,2,3,4}
\tlabel(5,-0.5){\scriptsize $x$}
\tlabel(0.25,4.75){\scriptsize $y$}
%\drawcolor[rgb]{0.33,0.33,0.33}
\drawcolor{black}
\arrow \parafcn{0,835,5}{(t+100)*dir(t)/400}
\penwd{1.25pt}
\arrow \reverse \arrow \polyline{(-2.5, 4.3301), (0,0), (5,0)}
\point[4pt]{(0,0)}

\tlpointsep{5pt}
\scriptsize
\axislabels {x}{{$-4 \hspace{7pt}$} -4, {$-3 \hspace{7pt}$} -3, {$-2 \hspace{7pt}$} -2, {$-1 \hspace{7pt}$} -1, {$1$} 1, {$2$} 2, {$3$} 3, {$4$} 4}
\axislabels {y}{{$-1$} -1, {$-2$} -2, {$-3$} -3, {$-4$} -4,  {$2$} 2, {$3$} 3, {$4$} 4}
\normalsize
\end{mfpic} 


\task $-510^{\circ}$ is a Quadrant III angle\\
coterminal with $-150^{\circ}$ and $210^{\circ}$

\medskip

\begin{mfpic}[12]{-5}{5}{-5}{5}
\drawcolor[gray]{0.7}
\axes
\xmarks{-4,-3,-2,-1,1,2,3,4}
\ymarks{-4,-3,-2,-1,1,2,3,4}
\tlabel(5,-0.5){\scriptsize $x$}
\tlabel(0.25,4.75){\scriptsize $y$}
%\drawcolor[rgb]{0.33,0.33,0.33}
\drawcolor{black}
\arrow \parafcn{0,505,5}{(t+100)*dir(0-t)/400}
\penwd{1.25pt}
\arrow \reverse \arrow \polyline{(-4.3301, -2.5), (0,0), (5,0)}
\point[4pt]{(0,0)}
\tlpointsep{5pt}
\scriptsize
\axislabels {x}{{$-4 \hspace{7pt}$} -4, {$-3 \hspace{7pt}$} -3, {$-2 \hspace{7pt}$} -2,  {$1$} 1, {$2$} 2, {$3$} 3, {$4$} 4}
\axislabels {y}{{$-1$} -1, {$-2$} -2, {$-3$} -3, {$-4$} -4,  {$2$} 2, {$3$} 3, {$4$} 4}
\normalsize
\end{mfpic}

\task $-900^{\circ}$  is a quadrantal angle \\
coterminal with $-180^{\circ}$ and $180^{\circ}$

\medskip

\begin{mfpic}[12]{-5}{5}{-5}{5}
\drawcolor[gray]{0.7}
\axes
\xmarks{-4,-3,-2,-1,1,2,3,4}
\ymarks{-4,-3,-2,-1,1,2,3,4}
\tlabel(5,-0.5){\scriptsize $x$}
\tlabel(0.25,4.75){\scriptsize $y$}
%\drawcolor[rgb]{0.33,0.33,0.33}
\drawcolor{black}
\arrow \parafcn{0,895,5}{(t+100)*dir(0-t)/400}
\penwd{1.25pt}
\arrow \reverse \arrow \polyline{(-5, 0), (0,0), (5,0)}
\point[4pt]{(0,0)}

\tlpointsep{5pt}
\scriptsize
\axislabels {x}{{$-4 \hspace{7pt}$} -4, {$-3 \hspace{7pt}$} -3, {$-2 \hspace{7pt}$} -2, {$-1 \hspace{7pt}$} -1, {$1$} 1, {$2$} 2, {$3$} 3, {$4$} 4}
\axislabels {y}{{$-1$} -1, {$-2$} -2, {$-3$} -3, {$-4$} -4, {$1$} 1, {$2$} 2, {$3$} 3, {$4$} 4}
\normalsize
\end{mfpic}

\end{tasks}












\closegraphsfile

\clearpage

\section{Right Triangle Trigonometry}

\mfpicnumber{1}

\opengraphsfile{AppRightTrig}

\setcounter{footnote}{0}

\label{AppRightTrig}

The word `trigonometry' literally means `measuring triangles,'  so naturally most students' first introduction to trigonometry focuses on triangles.   This section focuses on \index{triangle ! right}\index{right triangle}  \textbf{right triangles}, triangles in which one angle measures $90^{\circ}$.  Consider the right triangle below, where, as usual, the  small square `$\! \! \! \! \! \! \qed$' denotes the  right angle, the  labels `$a$,' `$b$,' and `$c$'  denote the lengths of the sides of the triangle, and $\alpha$ and $\beta$ represent the (measure of) the non-right angles.  As you may recall, the side opposite the right angle is called the  \index{hypotenuse} \textbf{hypotenuse} of the right triangle.  Also note that since the sum of the measures of all angles in a triangle must add to $180^{\circ}$, we have $\alpha + \beta + 90^{\circ}= 180^{\circ}$, or $\alpha + \beta = 90^{\circ}$.  Said differently, the non-right angles in a right triangle are \textit{complements}.


\begin{center}

\begin{mfpic}[15]{-5}{5}{-5}{5}
\tlabel(0,-0.75){$a$}
\tlabel(4.75,2.25){$b$}
\tlabel(0,3){$c$}
\polyline{(3.93, 0), (3.93, 0.4), (4.33, 0.4)}
\arrow \reverse \arrow \shiftpath{(-4.330,0)} \parafcn{5, 25, 5}{2.5*dir(t)}
\arrow \reverse \arrow \shiftpath{(4.330,5)}  \parafcn{215, 265, 5}{1.5*dir(t)}
\tlabel(-1.5,0.5){$\beta$}
\tlabel(3,3){$\alpha$}
\penwd{1.25pt}
\polyline{(-4.330,0), (4.330,0), (4.330,5), (-4.330,0)}
\end{mfpic}

\end{center}

We now state and prove the most famous result about right triangles:  \index{Pythagorean Theorem}\index{Theorem ! Pythagorean} \textbf{The Pythagorean Theorem}.

\begin{tcolorbox}
    
\begin{thm} \label{PythagoreanTheorem} (\textbf{The Pythagorean Theorem}) The square of the length of the hypotenuse of a right triangle is equal to the sums of the squares of the other two sides.  More specifically, if $c$ is the length of the hypotenuse of a right triangle and $a$ and $b$ are the lengths of the other two sides, then $a^2 + b^2 = c^2$.

\end{thm}

\end{tcolorbox}

There are several proofs of the Pythagorean Theorem,\footnote{Including one by Mentor, Ohio native \href{http://www.maa.org/press/periodicals/convergence/mathematical-treasure-james-a-garfields-proof-of-the-pythagorean-theorem}{\underline{President James Garfield}}.} but the one we choose to reproduce here showcases a nice interplay between algebra and geometry.  Consider taking four copies of the right triangle below on the left and arranging them as seen below on the right.


\begin{center}

\begin{multicols}{2}

\begin{mfpic}[22.5]{-1}{5}{-1}{5}

\arrow \reverse \arrow \shiftpath{(3,0)} \parafcn{130, 175, 5}{dir(t)}
\arrow \reverse \arrow \shiftpath{(0,4)} \parafcn{275, 300, 5}{1.5*dir(t)}
\polyline{(0, 0), (0.4, 0), (0.4, 0.4), (0, 0.4), (0,0)}

\penwd{1.25pt}
\polyline{(0,0), (3,0), (0,4), (0,0)}

\tlabel[cc](1.5,-0.5){$a$}
\tlabel[cc](-0.5,2){$b$}
\tlabel[cc](2,2){$c$}
\tlabel[cc](1.5,0.5){$\beta$}
\tlabel[cc](0.5,2){$\alpha$}
\end{mfpic}


\begin{mfpic}[20]{-1}{8}{-1}{8}

\arrow \reverse \arrow \shiftpath{(3,0)} \parafcn{130, 175, 5}{dir(t)}
\arrow \reverse \arrow \shiftpath{(7,3)} \parafcn{220, 265, 5}{dir(t)}
\arrow \reverse \arrow \shiftpath{(0,4)} \parafcn{40, 85, 5}{dir(t)}
\arrow \reverse \arrow \shiftpath{(4,7)} \parafcn{310, 355, 5}{dir(t)}

\arrow \reverse \arrow \shiftpath{(3,0)} \parafcn{5, 30, 5}{1.5*dir(t)}
\arrow \reverse \arrow \shiftpath{(7,3)} \parafcn{95, 120, 5}{1.5*dir(t)}
\arrow \reverse \arrow \shiftpath{(0,4)} \parafcn{275, 300, 5}{1.5*dir(t)}
\arrow \reverse \arrow \shiftpath{(4,7)} \parafcn{185, 210, 5}{1.5*dir(t)}



\polyline{(0, 0), (0.4, 0), (0.4, 0.4), (0, 0.4), (0,0)}
\polyline{(6.6, 0), (7, 0), (7, 0.4), (6.6, 0.4), (6.6,0)}
\polyline{(6.6, 6.6), (7, 6.6), (7, 7), (6.6, 7), (6.6, 6.6)}
\polyline{(0, 6.6), (0.4, 6.6), (0.4, 7), (0, 7), (0,6.6)}
\penwd{1.25pt}
\polyline{(0,0), (7,0), (7,7), (0,7), (0,0)}
\fillcolor[gray]{.7}
\gfill \polygon{(0,4), (3,0), (7,3), (4,7), (0,4)}
\polyline{(0,4), (3,0), (7,3), (4,7), (0,4)}
\tlabel[cc](1.5,-0.5){$a$}
\tlabel[cc](-0.5,5.5){$a$}
\tlabel[cc](5.5,7.5){$a$}
\tlabel[cc](7.5,1.5){$a$}
\tlabel[cc](5,-0.5){$b$}
\tlabel[cc](-0.5,2){$b$}
\tlabel[cc](2,7.5){$b$}
\tlabel[cc](7.5,5){$b$}

\tlabel[cc](2, 5){$c$}
\tlabel[cc](5,5){$c$}
\tlabel[cc](2,2){$c$}
\tlabel[cc](5,2){$c$}


\tlabel[cc](1.5,0.5){$\beta$}
\tlabel[cc](0.5,5.5){$\beta$}
\tlabel[cc](5.5,6.5){$\beta$}
\tlabel[cc](6.5,1.5){$\beta$}

\tlabel[cc](0.5,2){$\alpha$}
\tlabel[cc](2,6.5){$\alpha$}
\tlabel[cc](6.5,5){$\alpha$}
\tlabel[cc](5,0.5){$\alpha$}

\end{mfpic}

\end{multicols}

\end{center}

It should be clear that we have produced a large square with a side length of $(a+b)$. What is also true, but may not be obvious,  is that the shaded quadrilateral is also a square.   We can readily see the shaded quadrilateral has equal sides of length $c$.  Moreover, since $\alpha + \beta = 90^{\circ}$, we get the interior angles of the shaded quadrilateral are each $90^{\circ}$.   Hence,  the shaded quadrilateral is indeed a square.

We finish the proof by computing the area of the of the  large square in two ways.  First, we square the length of its side: $(a+b)^2$.  Next, we add up the areas of the four triangles, each having area $\frac{1}{2} ab$ along with the area of the shaded square, $c^2$.  Equating these to expressions gives: $(a+b)^2 = 4 \left( \frac{1}{2} ab\right)+c^2$.  Since $(a+b)^2 = a^2+2ab+b^2$ and $4 \left( \frac{1}{2} ab\right)  = 2ab$, we have $a^2+2ab+b^2 = 2ab + c^2$ or $a^2+b^2 = c^2$, as required.

It should be noted that the converse of the Pythagorean Theorem is also true.  That is if $a$, $b$, and $c$ are the lengths of sides of a triangle and $a^2+b^2 = c^2$, then $c$ the triangle is a right triangle.\footnote{We will prove this in Section \ref{TheLawofCosines} by generalizing the Pythagorean Theorem to a formula that works for \textit{all} triangles.}

A list of integers $(a,b,c)$  which satisfy the relationship $a^2+b^2 = c^2$ is called a  \index{Pythagorean triple}\textbf{Pythagorean Triple}.  Some of the more common triples are: $(3,4,5)$,  $(5,12,13)$, $(7,24,25)$, and $(8,15,17)$.   We leave it to the reader to verify these integers satisfy the equation $a^2+b^2 = c^2$ and suggest committing these triples to memory.

Next, we set about defining characteristic ratios associated with acute angles.  Given any acute angle $\theta$, we can imagine $\theta$ being an interior angle of a right triangle as seen below.   

\phantomsection
\label{righttranglediagram}

\begin{center}

\begin{mfpic}[15]{-5}{5}{-5}{5}
\arrow \reverse \arrow \shiftpath{(-4.330,0)} \parafcn{5, 25, 5}{3*dir(t)}
\tlabel(-1, 0.6){$\theta$}
\tlabel(0,-0.75){$a$}
\tlabel(4.75,2.25){$b$}
\tlabel(0,3){$c$}
\polyline{(3.93, 0), (3.93, 0.4), (4.33, 0.4)}
\penwd{1.25pt}
\polyline{(-4.330,0), (4.330,0), (4.330,5), (-4.330,0)}
\end{mfpic}

\end{center}

Focusing on the arrangement of the sides of the triangle with respect to the angle $\theta$, we make the following definitions:  the side with length $a$  is called the side of the triangle which is \index{side ! adjacent}\index{adjacent side} \textbf{adjacent} to  $\theta$ and the side with length $b$ is called the side of the triangle \index{side ! opposite}\index{opposite side}\textbf{opposite} $\theta$. As usual, the side labeled `$c$' (the side opposite the right angle) is the hypotenuse.  Using this diagram, we  define three important \index{ratios ! trigonometric}\index{trigonometric ratios} \textbf{trigonometric ratios} of $\theta$.

\begin{tcolorbox}
    
\begin{defn} \label{righttrianglesinecosinetangent}    Suppose $\theta$ is an acute angle residing in a right triangle as depicted above.

\begin{itemize}

\item  The \index{sine ! right triangle} \textbf{sine} of $\theta$, denoted $\sin(\theta)$ is defined by the ratio: $\sin(\theta) = \dfrac{b}{c}$, or $\dfrac{\text{`length of opposite'}}{\text{`length of hypotenuse'}}$.

\item  The \index{cosine ! right triangle} \textbf{cosine} of $\theta$, denoted $\cos(\theta)$ is defined by the ratio: $\cos(\theta) = \dfrac{a}{c}$, or $\dfrac{\text{`length of adjacent'}}{\text{`length of hypotenuse'}}$.

\item  The \index{tangent ! right triangle} \textbf{tangent} of $\theta$, denoted $\tan(\theta)$ is defined by the ratio: $\tan(\theta) = \dfrac{b}{a}$, or $\dfrac{\text{`length of opposite'}}{\text{`length of adjacent'}}$.\

\end{itemize}

\smallskip

\end{defn}

\end{tcolorbox}

\smallskip

For example, consider the angle $\theta$  indicated in the triangle below on the left.  Using Definition \ref{righttrianglesinecosinetangent},  we get $\sin(\theta) = \frac{4}{5}$, $\cos(\theta) = \frac{3}{5}$, and $\tan(\theta) = \frac{4}{3}$.   One may well wonder if these trigonometric ratios we've found for $\theta$ change if the triangle containing $\theta$ changes.  For example, if we scale all the sides of the triangle below on the left by a factor of $2$, we produce the  \index{similar triangle}\index{triangle ! similar} \textbf{similar triangle} below in the middle.\footnote{That is, a triangle with the same `shape' - that is, the same angles.}  Using this triangle to compute our ratios for $\theta$, we find $\sin(\theta) = \frac{8}{10} = \frac{4}{5}$, $\cos(\theta) = \frac{6}{10} = \frac{3}{5}$, and $\tan(\theta) = \frac{8}{6}  = \frac{4}{3}$.  Note that the scaling factor, here $2$, is common to all sides of the triangle, and, hence, cancels from the numerator and denominator when simplifying each of the ratios.  

\begin{center}

\begin{multicols}{3}

\begin{mfpic}[22.5]{-1}{5}{-1}{5}

\arrow \reverse \arrow  \parafcn{5, 50, 5}{dir(t)}

\polyline{(2.6, 0), (3, 0), (3, 0.4), (2.6, 0.4), (2.6,0)}

\penwd{1.25pt}
\polyline{(0,0), (3,0), (3,4), (0,0)}

\tlabel[cc](1.5,-0.5){$3$}
\tlabel[cc](3.5,2){$4$}
\tlabel[cc](1,2){$5$}
\tlabel[cc](1.5,0.5){$\theta$}
\end{mfpic}

\begin{mfpic}[22.5]{-1}{5}{-1}{5}

\arrow \reverse \arrow  \parafcn{5, 50, 5}{dir(t)}

\polyline{(2.6, 0), (3, 0), (3, 0.4), (2.6, 0.4), (2.6,0)}

\penwd{1.25pt}
\polyline{(0,0), (3,0), (3,4), (0,0)}

\tlabel[cc](1.5,-0.5){$6$}
\tlabel[cc](3.5,2){$8$}
\tlabel[cc](1,2){$10$}
\tlabel[cc](1.5,0.5){$\theta$}
\end{mfpic}

\begin{mfpic}[22.5]{-1}{5}{-1}{5}

\arrow \reverse \arrow  \parafcn{5, 50, 5}{dir(t)}

\polyline{(2.6, 0), (3, 0), (3, 0.4), (2.6, 0.4), (2.6,0)}

\penwd{1.25pt}
\polyline{(0,0), (3,0), (3,4), (0,0)}

\tlabel[cc](1.5,-0.5){$3r$}
\tlabel[cc](3.5,2){$4r$}
\tlabel[cc](1,2){$5r$}
\tlabel[cc](1.5,0.5){$\theta$}
\end{mfpic}


\end{multicols}


\end{center}

In general, thanks to the  \href{https://en.wikipedia.org/wiki/AA_postulate}{\underline{Angle Angle Similarity Postulate}},  any two \textit{right} triangles which contain our angle $\theta$ are similar which means there is a positive constant $r$ so that the sides of the triangle are $3r$, $4r$, and $5r$ as seen above on the right.  Hence, regardless of the right triangle in which we choose to imagine $\theta$,  $\sin(\theta) = \frac{4r}{5r} = \frac{4}{5}$, $\cos(\theta) = \frac{3r}{5r} = \frac{3}{5}$, and $\tan(\theta) = \frac{4r}{3r}  = \frac{4}{3}$.  Generalizing this same argument to any acute angle $\theta$ assures us that the ratios as described in Definition \ref{righttrianglesinecosinetangent} are independent of the triangle we use.

Our next objective is to determine the values of $\sin(\theta)$, $\cos(\theta)$, and $\tan(\theta)$ for some of the more commonly used angles.  We begin with $45^{\circ}$ as shown in \autoref{fig:trigratios45}.  In a right triangle, if one of the non-right angles measures $45^{\circ}$, then the other measures $45^{\circ}$ as well.  It follows that the two legs of the triangle must be congruent.  Since we may choose any right triangle containing a $45^{\circ}$ angle for our computations, we choose the length of one (hence both) of the legs to be $1$.  The Pythagorean Theorem gives the hypotenuse is:  $c^2 = 1^2+1^2 = 2$, so $c = \sqrt{2}$. (We take only the positive square root here since $c$ represents the length of the hypotenuse here, so, necessarily $c>0$.)  From this, we obtain the values shown by the side, and suggest committing them to memory. 

\begin{figure}
    
\begin{multicols}{2}

\begin{center}
\begin{mfpic}[18]{-5}{5}{-5}{5}
\arrow \reverse \arrow \shiftpath{(-2.5,0)} \parafcn{5, 35, 5}{1.5*dir(t)}
\arrow \reverse \arrow \shiftpath{(2.5,5)}  \parafcn{230, 265, 5}{1.5*dir(t)}
\tlabel(-0.8, 0.4){$45^{\circ}$}
\tlabel(1.4,2.8){$45^{\circ}$}
\tlabel(0,-0.5){$1$}
\tlabel(2.75,2.25){$1$}
\tlabel[cc](-1,3){$c=\sqrt{2}$}
\polyline{(2.1, 0), (2.1, 0.4), (2.5, 0.4)}
\penwd{1.25pt}
\polyline{(-2.5,0), (2.5,0), (2.5,5), (-2.5,0)}
\end{mfpic} 
\end{center}

\begin{itemize}

\item  $\sin\left(45^{\circ}\right) = \dfrac{1}{\sqrt{2}} = \dfrac{\sqrt{2}}{2}$

\item  $\cos\left(45^{\circ}\right) = \dfrac{1}{\sqrt{2}} =\dfrac{\sqrt{2}}{2}$


\item  $\tan\left(45^{\circ}\right) = \dfrac{1}{1} = 1$

\end{itemize}

\end{multicols}

\caption{Trigonometric ratios for $45^{\circ}$}
\label{fig:trigratios45}
\end{figure}

Note that we have `rationalized' here to avoid the irrational number $\sqrt{2}$ appearing in the denominator.  This is a common convention in trigonometry, and we will adhere to it unless extremely inconvenient.  

Next, we investigate $60^{\circ}$ and $30^{\circ}$ angles.  Consider the equilateral triangle in \autoref{fig:trigratio60}  each of whose sides measures $2$ units.  Each of its interior angles is necessarily $60^{\circ}$, so if we drop an altitude, we produce two $30^{\circ} - 60^{\circ} - 90^{\circ}$ triangles each having a base measuring $1$ unit and a hypotenuse of $2$ units.  Using the Pythagorean Theorem, we can find the height, $h$ of these triangles: $1^2+h^2 = 2^2$ so $h^2 = 3$ or $h = \sqrt{3}$.  Using these, we can find the values of the trigonometric ratios for both $60^{\circ}$ and $30^{\circ}$.  Again, we recommend committing these values to memory.

\begin{figure}

\begin{multicols}{2}

\begin{mfpic}[15]{-5}{5}{-5}{5}
\arrow \reverse \arrow \shiftpath{(-5,-4.330)} \parafcn{5, 55, 5}{1.5*dir(t)}
\arrow \reverse \arrow \shiftpath{(0,4.330)}  \parafcn{245, 265, 5}{2.5*dir(t)}
\tlabel(-3.4,-3.75){$60^{\circ}$}
\tlabel(-1.4,0.75){$30^{\circ}$}
\tlabel(-2.5,-5.25){$1$}
\tlabel(2.5,-5.25){$1$}
\tlabel(0.25,-1){$h=\sqrt{3}$}
\tlabel[cc](-3.25,0){$2$}
\tlabel[cc](3.25,0){$2$}
\polyline{(-0.4, -4.330), (-0.4,-3.930), (0, -3.930)}
\polyline{(0.4, -4.330), (0.4,-3.930), (0, -3.930)}
\penwd{1.25pt}
\polyline{(-5,-4.330), (0,-4.330), (0,4.330), (-5,-4.330)}
\polyline{(0,4.330), (5,-4.330), (0,-4.330)}
\end{mfpic}

\columnbreak

\begin{itemize}

\item  $\sin\left(60^{\circ}\right) = \dfrac{\sqrt{3}}{2}$

\item  $\cos\left(60^{\circ}\right) = \dfrac{1}{2}$

\item  $\tan\left(60^{\circ}\right) = \dfrac{\sqrt{3}}{1} = \sqrt{3}$

\item  $\sin\left(30^{\circ}\right) = \dfrac{1}{2}$

\item  $\cos\left(30^{\circ}\right) = \dfrac{\sqrt{3}}{2}$

\item  $\tan\left(30^{\circ}\right) = \dfrac{1}{\sqrt{3}} = \dfrac{\sqrt{3}}{3}$

\end{itemize}

\end{multicols}
\caption{Trigonometric ratios for $60^{\circ}$ and $30^{\circ}$}
\label{fig:trigratio60}
\end{figure}

Since $30^{\circ}$ and $60^{\circ}$ are complements, the side \textit{adjacent} to the $60^{\circ}$ angle is the side \textit{opposite} the $30^{\circ}$ and the side \textit{opposite}  the $60^{\circ}$ angle is the side \textit{adjacent} to the $30^{\circ}$ .  This sort of `swapping' is true of all complementary angles and will be generalized in Section \ref{MoreTrigonometricIdentities}, Theorem \ref{cofunctionidentities}.

\smallskip

Note that the values of the trigonometric ratios we have derived for $30^{\circ}$, $45^{\circ}$, and $60^{\circ}$ angles are the \textit{exact} values of these ratios. For these angles, we can conveniently express the exact values of their sines, cosines, and tangents  resorting, at worst, to using square roots.    The reader may well wonder if, for instance, we can express the exact value of, say, $\sin\left(42^{\circ}\right)$ in terms of radicals.  The answer in this case is `yes'  (see \href{https://math.la.asu.edu/~surgent/mat170/Exact_Trig_Values.pdf}{\underline{here}}), but, in general, we will not take the time to pursue such representations.\footnote{We will do a little of this in Section \ref{MoreTrigonometricIdentities}.}  Hence, if a problem requests an `exact' answer involving $\sin\left(42^{\circ}\right)$, we will leave it written as `$\sin\left(42^{\circ}\right)$'  and use a calculator to produce a suitable approximation as the situation warrants.

Our first example requires the concept of an `angle of inclination.'  The \index{angle ! of inclination} angle of inclination (or \index{angle ! of elevation} angle of elevation) of an object refers to the angle whose initial side is some kind of base-line (say, the ground), and whose terminal side is the line-of-sight to an object above the base-line.  Schematically:
\phantomsection
\label{angleofelevation}

\begin{center}

\begin{mfpic}[18]{-5}{5}{-5}{5}
\polyline{(-4.330,0), (5,0)}
\dashed \polyline{(-4.330,0), (4.330,5)}
\arrow \shiftpath{(-4.330,0)} \parafcn{5, 25, 5}{3*dir(t)}
\tlabel(-1, 0.6){$\theta$}
\tlabel[cc](0,-1){`base line'}
\plotsymbol[3pt]{Asterisk}{(4.330,5)}
\tlabel(4.5,4.5){object}
\end{mfpic} 

\smallskip

The angle of inclination from the base line to the object is $\theta$
\end{center}

\begin{ex} \label{righttriangleex1} $~$

\begin{enumerate}

\item  The angle of inclination from a point on the ground 30 feet away to the top of Lakeland's Armington Clocktower\footnote{Named in honor of Raymond Q. Armington, Lakeland's Clocktower has been a part of campus since 1972.} is  $60^{\circ}$.  Find the height of the Clocktower to the nearest foot.

\item  The Americans with Disabilities Act (ADA) stipulates the incline on an accessibility ramp be $5^{\circ}$.  If a ramp is to be built so that it replaces stairs that measure 21 inches tall, how long does the ramp need to be?  Round your answer to the nearest inch.

\item  In order to determine the height of a California Redwood tree, two sightings from the ground, one 200 feet directly behind the other, are made.  If the angles of inclination were $45^{\circ}$ and $30^{\circ}$, respectively, how tall is the tree to the nearest foot?

\end{enumerate}

{\bf Solution.}

\begin{enumerate}

\item  We can represent the problem situation using a right triangle as shown below on the left.  If we let $h$ denote the height of the tower, then we have $\tan\left(60^{\circ}\right) = \frac{h}{30}$.  From this we get an exact answer of $h = 30 \tan\left(60^{\circ}\right) = 30 \sqrt{3}$ feet.  Using a calculator, we get the approximation  $51.96$ which, when rounded to the nearest foot, gives us our answer of $52$ feet.

\item  We diagram the situation below on the left using $\ell$ to represent the unknown length of the ramp.  We have $\sin\left(5^{\circ} \right)= \frac{21}{\ell}$ so that $\ell = \frac{21}{\sin\left(5^{\circ} \right)} \approx 240.95$ inches.  Hence, the ramp is $241$ inches long.

\begin{center}

\begin{multicols}{2}

\begin{mfpic}[10]{-5}{5}{-5}{5}
\arrow \shiftpath{(0,-4.330)} \parafcn{5, 55, 5}{1.5*dir(t)}
\tlabel(1.6,-3.75){$60^{\circ}$}
\tlabel(2,-5.5){$30$ ft.}
\tlabel(5.25,0){$h$ ft.}
\polyline{(4.6, -4.330), (4.6,-3.930), (5, -3.930)}
\penwd{1.25pt}
\polyline{(0,-4.330), (5,-4.330), (5,4.330), (0,-4.330)}

\end{mfpic}

\begin{mfpic}[10]{-5}{5}{-5}{5}
\arrow \shiftpath{(-5,-4.330)} \parafcn{5, 25, 5}{3*dir(t)}
\tlabel(-1.5,-3.75){$5^{\circ}$}
\tlabel(5.25,-1.83){$21$ in.}
\tlabel(0,-1){$\ell$ in.}
\polyline{(4.6, -4.330), (4.6,-3.930), (5, -3.930)}
\penwd{1.25pt}
\polyline{(-5,-4.330), (5, -4.330), (5, 0.667), (-5, -4.330)}

\end{mfpic}


\end{multicols}

\begin{multicols}{2}
Finding the height of the Clocktower

Finding the length of an accessibility ramp.

\end{multicols}

\end{center}

\item  Sketching the problem situation below, we find ourselves with two unknowns: the height $h$ of the tree and the distance $x$ from the base of the tree to the first observation point. 

\begin{center}

\begin{mfpic}[18]{-7}{5}{-5}{5}
\arrow \shiftpath{(-2.5,0)} \parafcn{5, 35, 5}{1.5*dir(t)}
\tlabel(-0.8, 0.4){$45^{\circ}$}
\tlabel(-4, 0.4){$30^{\circ}$}
\arrow \shiftpath{(-6,0)} \parafcn{5, 25, 5}{1.75*dir(t)}
\tlabel(-5,-1){$200$ ft.}
\tlabel(-1,-1){$x$ ft.}
\tlabel(2.75,2.25){$h$ ft.}
\polyline{(2.1, 0), (2.1, 0.4), (2.5, 0.4)}
\penwd{1.25pt}
\polyline{(-2.5,0), (2.5,0), (2.5,5), (-2.5,0)}
\polyline{(-6,0), (2.5,0), (2.5,5), (-6,0)}
\point[4pt]{(-2.5,0), (-6,0)}
\end{mfpic} 

Finding the height of a California Redwood
\end{center}


Luckily, we have two right triangles to help us find each unknown, as shown below. From the triangle below on the left, we get $\tan\left(45^{\circ}\right) = \frac{h}{x}$.  From the triangle below on the right, we see  $\tan\left(30^{\circ}\right) = \frac{h}{x+200}$.  


\begin{center}

\begin{multicols}{2}

\begin{mfpic}[18]{-7}{5}{-5}{5}
\arrow \shiftpath{(-2.5,0)} \parafcn{5, 35, 5}{1.5*dir(t)}
\tlabel(-0.8, 0.4){$45^{\circ}$}
\tlabel(-1,-1){$x$ ft.}
\tlabel(2.75,2.25){$h$ ft.}
\polyline{(2.1, 0), (2.1, 0.4), (2.5, 0.4)}
\penwd{1.25pt}
\polyline{(-2.5,0), (2.5,0), (2.5,5), (-2.5,0)}
\end{mfpic} 

\begin{mfpic}[15]{-7}{5}{-5}{5}
\tlabel(-4, 0.4){$30^{\circ}$}
\arrow \shiftpath{(-6,0)} \parafcn{5, 25, 5}{1.75*dir(t)}
\tlabel(-3,-1){$x+200$ ft.}
\tlabel(2.75,2.25){$h$ ft.}
\polyline{(2.1, 0), (2.1, 0.4), (2.5, 0.4)}
\penwd{1.25pt}
\polyline{(-6,0), (2.5,0), (2.5,5), (-6,0)}
\end{mfpic} 

\end{multicols}

\end{center}


Since $\tan\left(45^{\circ}\right) = 1$, the first equation gives $\frac{h}{x} = 1$, or $x = h$.  Substituting this into the second equation gives $\frac{h}{h+200} = \tan\left(30^{\circ}\right) = \frac{\sqrt{3}}{3}$.  Clearing fractions,  we get $3h = (h+200) \sqrt{3}$.  The result is a linear equation for $h$, so we  expand the right hand side and gather all the terms involving $h$ to one side.

\[ \begin{array}{rcl}

3h & = & (h+200)\sqrt{3} \\ [5pt]
3h & = & h \sqrt{3} + 200 \sqrt{3} \\ [5pt]
3h - h \sqrt{3} & = & 200 \sqrt{3} \\ [5pt]
(3-\sqrt{3}) h & = & 200 \sqrt{3} \\ [5pt]
h & = & \dfrac{200\sqrt{3}}{3-\sqrt{3}} \approx 273.20 \\ \end{array} \] 


Hence, the tree is approximately $273$ feet tall.  \qed

\end{enumerate}

\end{ex} 

There are three more trigonometric ratios which are commonly used and they are defined in the same manner the ratios in Definition \ref{righttrianglesinecosinetangent} are defined.  They are listed below.

\begin{tcolorbox}
    
\begin{defn} \label{righttriangletherest}

Suppose $\theta$ is an acute angle residing in a right triangle as depicted on page \pageref{righttranglediagram}.

\begin{itemize}

\item  The \index{cosecant ! right triangle} \textbf{cosecant} of $\theta$, denoted $\csc(\theta)$ is defined by the ratio: $\csc(\theta) = \dfrac{c}{b}$ , or $\dfrac{\text{`length of hypotenuse'}}{\text{`length of opposite'}}$.

\item  The \index{secant ! right triangle} \textbf{secant} of $\theta$, denoted $\sec(\theta)$ is defined by the ratio: $\sec(\theta) = \dfrac{c}{a} $ , or $\dfrac{\text{`length of hypotenuse'}}{\text{`length of adjacent'}}$.

\item  The \index{cotangent ! right triangle} \textbf{cotangent} of $\theta$, denoted $\cot(\theta)$ is defined by the ratio: $\cot(\theta) = \dfrac{a}{b} $ , or $\dfrac{\text{`length of adjacent'}}{\text{`length of opposite'}}$.

\end{itemize}

\end{defn}

\end{tcolorbox}

We practice these definitions in the following example.

\begin{ex} \label{righttriangleex2}  Suppose $\theta$ is an acute angle with $\cot(\theta) = 3$.  Find the values of the remaining five trigonometric ratios:  $\sin(\theta)$, $\cos(\theta)$, $\tan(\theta)$, $\csc(\theta)$, and $\sec(\theta)$.

{\bf Solution.}  We are given $\cot(\theta) = 3 $.  So, to proceed, we construct a right triangle in which the length of the side adjacent to $\theta$ and the length of the side opposite of $\theta$ has a ratio of $3 = \frac{3}{1}$.  Note there are infinitely many such right triangles - we have produced two below for reference.  We will focus our attention on the triangle below on the left and encourage the reader to work through the details using the triangle below on the right to verify the choice of triangle doesn't matter.

\begin{multicols}{2}

\begin{mfpic}[20]{-1}{14}{-1}{5}
\tlabel(2.25, 0.25){$\theta$}
\arrow \reverse \arrow \parafcn{3, 17, 5}{2*dir(t)}
\tlabel(3,-0.5){$3$}
\tlabel(1.25,1.5){$c = \sqrt{10}$}
\tlabel(6.5,1){$1$}
\polyline{(5.6, 0), (5.6, 0.4), (6, 0.4)}
\penwd{1.25pt}
\polyline{(0,0), (6,0), (6,2), (0,0)}
\end{mfpic} 

\begin{mfpic}[20]{-1}{14}{-1}{5}
\tlabel(2.25, 0.25){$\theta$}
\arrow \reverse \arrow \parafcn{3, 17, 5}{2*dir(t)}
\tlabel(3,-0.5){$6$}
\tlabel(6.5,1){$2$}
\polyline{(5.6, 0), (5.6, 0.4), (6, 0.4)}
\penwd{1.25pt}
\polyline{(0,0), (6,0), (6,2), (0,0)}
\end{mfpic} 


\end{multicols}

From the diagram, we see immediately $\tan(\theta) = \frac{1}{3}$, but in order to determine the remaining four trigonometric ratios, we need to first find the value of the hypotenuse. The Pythagorean Theorem gives $1^2 + 3^2 = c^2$ so $c^2 = 10$ or $c = \sqrt{10}$.   Rationalizing denominators, we find $\sin(\theta) = \frac{1}{\sqrt{10}} = \frac{\sqrt{10}}{10}$, $\cos(\theta) = \frac{3}{\sqrt{10}} = \frac{3\sqrt{10}}{10}$, $\csc(\theta) = \frac{\sqrt{10}}{1} = \sqrt{10}$ and $\sec(\theta) = \frac{\sqrt{10}}{3}$.  \qed


\end{ex}

While we learned all about the trigonometric ratios of $\theta$ in Example \ref{righttriangleex2}, the identity of $\theta$ remains unknown.  Since $\sin(\theta) = \frac{\sqrt{10}}{10} \approx 0.316 $ is decidedly less than $\sin\left(30^{\circ}\right) = \frac{1}{2} = 0.5$, it stands to reason that $\theta < 30^{\circ}$. It turns out the calculator can provide for us a decimal approximation of $\theta$ by way of the `$\sin^{-1}(x)$' function.  Here, the `$-1$' exponent denotes an inverse function (see Section \ref{InverseFunctions}) does \textbf{not} mean reciprocal.\footnote{That is, $\sin^{-1}(x) \neq \frac{1}{\sin(x)} $.  That being said, $(\sin(x))^{-1} = \frac{1}{\sin(x)} = \csc(x)$.}  That is, $\sin^{-1}(x)$ (read `sine-inverse of $x$') gives an angle whose sine is $x$.  Hence, we may write $\theta = \sin^{-1}\left(\frac{\sqrt{10}}{10}\right) \approx 18.43^{\circ}$.  The functions $\cos^{-1}(x)$ and $\tan^{-1}(x)$ work similarly.  Indeed, \[ \theta = \sin^{-1}\left(\frac{\sqrt{10}}{10}\right)  = \cos^{-1}\left(\frac{3 \sqrt{10}}{10}\right)  = \tan^{-1} \left( \frac{1}{3} \right),\] and the reader is encouraged to use a calculator to verify these statements.

\smallskip

Please note there is \textbf{much} more to these inverse functions than the `angle finder' description use here.\footnote{See Section \ref{TheInverseTrigonometricFunctions} for all of the pedantic details.}  That being said, we finish this section showcasing a use for the $\tan^{-1}(x)$ function below.

\begin{ex}\footnote{The authors would like to thank Dan Stitz for this problem and associated graphics.} \label{roofpitchex}  The roof on the house below has a  `$6/12$ pitch'.  This means that when viewed from the side, the roof line has a rise of 6 feet over a run of 12 feet.  Find the angle of inclination from the bottom of the roof to the top of the roof. Round your answer to the  nearest hundredth of a degree.

\begin{center}

\begin{tabular}{cc}

\includegraphics[width=2in]{./AppRightTrigGraphics/AppRightTrig01.jpg} &
\includegraphics[width=1.454in]{./AppRightTrigGraphics/AppRightTrig02.jpg}  \\ 
Front View &  Side View \\

\end{tabular} 

\end{center}

{\bf Solution.} If we divide the side view of the house down the middle, we find that the roof line forms the hypotenuse of a right triangle with legs of length $6$ feet and $12$ feet as depicted below.  


\begin{center}

\begin{mfpic}[13]{0}{13.25}{-1}{6}
\polyline{(11.25,0), (11.25,0.75), (12,0.75)}
\arrow \reverse \arrow \polyline{(0,-1),(12,-1)}
\gclear \tlabelrect[cc](6,-1){$12$ feet}
\arrow \reverse \arrow \polyline{(13.25,0),(13.25,6)}
\gclear \tlabelrect[cc](13.25,3){$6$ feet}
\arrow \reverse \arrow \parafcn{3, 19, 5}{2.75*dir(t)}
\tlabel[cc](3.25,0.65){$\theta$}
\penwd{1.25pt}
\polyline{(0,0), (12,0), (12,6), (0,0)}
\end{mfpic}


\end{center}

The angle of inclination, $\theta$, satisfies $\tan(\theta) = \frac{6}{12} = \frac{1}{2}$.  Hence,  $\theta = \tan^{-1}\left( \frac{1}{2}\right)  \approx 26.56^{\circ}$. \qed 

\end{ex}


\clearpage

\subsection{Exercises}

\label{ExercisesforAppRightTrig}
In Exercises \ref{trianglecircfirst} - \ref{trianglecirclast},  find the requested quantities.

\begin{tasks}(2)
\task Find $\theta$, $a$, and $c$.  \label{trianglecircfirst}

 \begin{mfpic}[15]{-5}{5}{-5}{5}

\arrow \reverse \arrow \shiftpath{(-4.330,0)} \parafcn{5, 25, 5}{3*dir(t)}
\arrow \reverse \arrow \shiftpath{(4.330,5)}  \parafcn{215, 265, 5}{1.5*dir(t)}
\tlabel(-1.25, 0.6){$\theta$}
\tlabel(0,-0.75){$9$}
\tlabel(4.75,2.25){$a$}
\tlabel(-0.5,3){$c$}
\tlabel(2.75,2.85){$60^{\circ}$}
\polyline{(3.93, 0), (3.93, 0.4), (4.33, 0.4)}
\penwd{1.25pt}
\polyline{(-4.330,0), (4.330,0), (4.330,5), (-4.330,0)}
\end{mfpic}

\task  Find $\alpha$, $b$, and $c$.

\begin{mfpic}[15]{-1}{5}{-1}{7}
\arrow \reverse \arrow \parafcn{60, 87, 5}{1.75*dir(t)}
\arrow \reverse \arrow \shiftpath{(4.357,6.709)}  \parafcn{185, 232, 5}{1.5*dir(t)}
\tlabel(0.25, 2){$34^{\circ}$}
\tlabel(2.5,3){$c$}
\tlabel(2,7){$b$}
\tlabel(-0.85,4){$12$}
\tlabel(2.25,5.75){$\alpha$}
\polyline{(0,6.304), (0.4, 6.304),  (0.4, 6.704)}
\penwd{1.25pt}
\polyline{(0,0), (0,6.709), (4.357, 6.709), (0,0)}
\end{mfpic}

\task  Find $\theta$, $a$, and $c$.

\begin{mfpic}[18]{-5}{5}{-5}{5}
\arrow \reverse \arrow \shiftpath{(2.5,0)} \parafcn{140, 175, 5}{1.5*dir(t)}
\arrow \reverse \arrow \shiftpath{(-2.5,5)}  \parafcn{275, 310, 5}{1.5*dir(t)}
\tlabel(-2, 2.75){$47^{\circ}$}
\tlabel(-0.5,-0.75){$6$}
\tlabel(-3.25,2.25){$a$}
\tlabel(0,3){$c$}
\tlabel(0.5,0.5){$\theta$}
\polyline{(-2.5, 0.4), (-2.1, 0.4), (-2.1, 0)}
\penwd{1.25pt}
\polyline{(-2.5, 0), (2.5,0), (-2.5,5), (-2.5,0)}
\end{mfpic}

\task Find $\beta$, $b$, and $c$.  \label{trianglecirclast}

\begin{mfpic}[18]{-6}{1}{-1}{9}
\arrow \reverse \arrow \parafcn{95, 127, 5}{1.75*dir(t)}
\arrow \reverse \arrow \shiftpath{(-5.402,6)}  \parafcn{317, 355, 5}{1.5*dir(t)}
\tlabel(-3.75, 5){$\beta$}
\tlabel(0.5,3){$2.5$}
\tlabel(-2.6,6.25){$b$}
\tlabel(-3.25,2.5){$c$}
\tlabel(-1.2,2){$50^{\circ}$}
\polyline{(0,5.6), (-0.4, 5.6),  (-0.4, 6)}
\penwd{1.25pt}
\polyline{(0,0), (0,6), (-5.402, 6), (0,0)}
\end{mfpic} 

\end{tasks}

In Exercises \ref{moretrianglecircfirst} - \ref{moretrianglecirclast}, answer the following questions assuming  $\theta$ is an angle in a right triangle.

\begin{tasks}[resume](1)
\task  If $\theta = 30^{\circ}$ and the side opposite $\theta$ has length $4$, how long is the side adjacent to $\theta$? \label{moretrianglecircfirst}

\task  If $\theta = 15^{\circ}$ and the hypotenuse has length $10$, how long is the side opposite $\theta$?

\task  If $\theta = 87^{\circ}$ and the side adjacent to $\theta$ has length $2$, how long is the side opposite $\theta$?

\task  If $\theta = 38.2^{\circ}$ and the side opposite $\theta$ has lengh $14$, how long is the hypoteneuse?

\task  If $\theta = 2.05^{\circ}$ and the hypotenuse has length $3.98$, how long is the side adjacent to $\theta$?

\task  If $\theta = 42^{\circ}$ and the side adjacent to $\theta$ has length $31$, how long is the side opposite $\theta$? \label{moretrianglecirclast}

\end{tasks}

In Exercises \ref{trianglesidesfirst} - \ref{trianglesideslast}, find the two acute angles in the right triangle whose sides have the given lengths.  Express your answers using degree measure rounded to two decimal places.

\begin{tasks}[resume](2)
\task 3, 4 and 5 \label{trianglesidesfirst}

\task 5, 12 and 13

\task 336, 527 and 625 \label{trianglesideslast}

\end{tasks}

In Exercises \ref{findothercircfirstapprighttrig} - \ref{findothercirclastapprighttrig}, $\theta$ is an acute angle.  Use the given trigonometric ratio to find the exact values of the remaining trigonometric ratios of $\theta$.  Find a decimal approximation to $\theta$, rounded to two decimal places.

\begin{tasks}[resume](3)
\task $\sin(\theta) = \dfrac{3}{5}$  \label{findothercircfirstapprighttrig}
\task $\tan(\theta) = \dfrac{12}{5}$
\task $\csc(\theta) = \dfrac{25}{24}$

\task $\sec(\theta) = 7$  \vphantom{$\dfrac{10}{\sqrt{91}}$}
\task $\csc(\theta) = \dfrac{10\sqrt{91}}{91}$ 
\task $\cot(\theta) = 23$ 

\task  $\tan(\theta) = 2$  \vphantom{$\sqrt{5}$}
\task  $\sec(\theta) = 4$  \vphantom{$\sqrt{5}$}
\task $\cot(\theta) = \sqrt{5}$ 
\task  $\cos(\theta) = \dfrac{1}{3}$ 

\task  $\cot(\theta) = 2$ \vphantom{$ \dfrac{1}{3}$}

\task  $\csc(\theta) = 5$ \vphantom{$ \dfrac{1}{3}$}

\task  $\tan(\theta) = \sqrt{10}$ 
\task  $\sec(\theta) = 2\sqrt{5}$ 
\task  $\cos(\theta) = 0.4$  \vphantom{$\sqrt{10}$}  \label{findothercirclastapprighttrig}

\end{tasks}

\begin{tasks}[resume](1)
\task A tree standing vertically on level ground casts a 120 foot long shadow.  The angle of elevation from the end of the shadow to the top of the tree is $21.4^{\circ}$.  Find the height of the tree to the nearest foot.  With the help of your classmates, research the term \emph{umbra versa} and see what it has to do with the shadow in this problem.

\task The broadcast tower for radio station WSAZ (Home of ``Algebra in the Morning with Carl and Jeff'') has two enormous flashing red lights on it: one at the very top and one a few feet below the top.  From a point 5000 feet away from the base of the tower on level ground the angle of elevation to the top light is $7.970^{\circ}$ and to the second light is $7.125^{\circ}$.  Find the distance between the lights to the nearest foot.

\task On page \pageref{angleofelevation} we defined the angle of inclination (also known as the angle of elevation) and in this exercise we introduce a related angle - \index{angle ! of depression} the angle of depression (also known as \index{angle ! of declination} the angle of declination).  The angle of depression of an object refers to the angle whose initial side is a horizontal line above the object and whose terminal side is the line-of-sight to the object below the horizontal.  This is represented schematically below.
\label{angleofdepression}

\begin{center}

\begin{mfpic}[15]{-5}{5}{-5}{5}
\polyline{(-5,5), (4.330,5)}
\point[3pt]{(4.330,5)}
\dashed \polyline{(-4.330,0), (4.330,5)}
\reverse \arrow \shiftpath{(4.330,5)} \parafcn{185, 205, 5}{3*dir(t)}
\tlabel(0.75, 4){$\theta$}
\tlabel[cc](-1,5.5){horizontal}
\tlabel[cc](5.25,4.5){observer}
\plotsymbol[3pt]{Asterisk}{(-4.330,0)}
\tlabel(-5.0,-0.75){object}
\end{mfpic} 


The angle of depression from the horizontal to the object is $\theta$

\end{center}

\begin{enumerate}[label=(\alph*)]

\item Show that if the horizontal is above and parallel to level ground then the angle of depression (from observer to object) and the angle of inclination (from object to observer) will be congruent because they are alternate interior angles.

\item \label{sasquatchfire} From a firetower 200 feet above level ground in the Sasquatch National Forest, a ranger spots a fire off in the distance.  The angle of depression to the fire is $2.5^{\circ}$.  How far away from the base of the tower is the fire?

\item  The ranger in part \ref{sasquatchfire} sees a Sasquatch running directly from the fire towards the firetower.  The ranger takes two sightings.  At the first sighting, the angle of depression from the tower to the Sasquatch is $6^{\circ}$.  The second sighting, taken just 10 seconds later, gives the the angle of depression as $6.5^{\circ}$.  How far did the Saquatch travel in those 10 seconds?  Round your answer to the nearest foot.  How fast is it running in miles per hour? Round your answer to the nearest mile per hour.  If the Sasquatch keeps up this pace, how long will it take for the Sasquatch to reach the firetower from his location at the second sighting?  Round your answer to the nearest minute.

\end{enumerate}

\task  When I stand 30 feet away from a tree at home, the angle of elevation to the top of the tree is $50^{\circ}$ and the angle of depression to the base of the tree is $10^{\circ}$.  What is the height of the tree?  Round your answer to the nearest foot.

\task From the observation deck of the lighthouse at Sasquatch Point 50 feet above the surface of Lake Ippizuti, a lifeguard spots a boat out on the lake sailing directly toward the lighthouse.  The first sighting had an angle of depression of $8.2^{\circ}$ and the second sighting had an angle of depression of $25.9^{\circ}$.  How far had the boat traveled between the sightings?

\task A guy wire 1000 feet long is attached to the top of a tower.  When pulled taut it makes a $43^{\circ}$ angle with the ground.  How tall is the tower?  How far away from the base of the tower does the wire hit the ground?

\task A guy wire 1000 feet long is attached to the top of a tower.  When pulled taut it touches level ground 360 feet from the base of the tower.  What angle does the wire make with the ground?  Express your answer using degree measure rounded to one decimal place.

\task At Cliffs of Insanity Point, The Great Sasquatch Canyon is 7117 feet deep.  From that point, a fire is seen at a location known to be 10 miles away from the base of the sheer canyon wall.  What angle of depression is made by the line of sight from the canyon edge to the fire?  Express your answer using degree measure rounded to one decimal place.

\task Shelving is being built at the Utility Muffin Research Library which is to be 14 inches deep.  An 18-inch rod will be attached to the wall and the underside of the shelf at its edge away from the wall, forming a right triangle under the shelf to support it.  What angle, to the nearest degree, will the rod make with the wall?

\task A parasailor is being pulled by a boat on Lake Ippizuti.  The cable is 300 feet long and the parasailor is 100 feet above the surface of the water.  What is the angle of elevation from the boat to the parasailor?  Express your answer using degree measure rounded to one decimal place.

\task  A tag-and-release program to study the Sasquatch population of the eponymous Sasquatch National Park is begun.  From a 200 foot tall tower, a ranger spots a Sasquatch lumbering through the wilderness directly towards the tower.  Let $\theta$ denote the angle of depression from the top of the tower to a point on the ground.  If the range of the rifle with a tranquilizer dart is 300 feet, find the smallest value of $\theta$ for which the corresponding point on the ground is in range of the rifle.  Round your answer to the nearest hundreth of a degree.

\task  The rule of thumb for safe ladder use states that the length of the ladder should be at least four times as long as the distance from the base of the ladder to the wall. Assuming the ladder is resting against a wall which is `plumb' (that is, makes a $90^{\circ}$ angle with the ground), determine the acute angle the ladder makes with the ground, rounded to the nearest tenth of a degree.

\end{tasks}

As you may have already noticed in working through the exercises, since the six trigonometric ratios are all defined in terms of the three sides of a right triangle, there are several relationships between them.  In Exercises \ref{rtidentityfirst} - \ref{rtidentitylast}, use the diagram on page \pageref{righttranglediagram} along with Definitions \ref{righttrianglesinecosinetangent} and \ref{righttriangletherest} to show the following relationships hold for all acute angles.\footnote{These are called trigonometric \textit{identities} and will be studied in greater detail in Section \ref{FundamentalTrigonometricIdentities}.}

\begin{tasks}[resume](3)
\task  $\tan(\theta) = \dfrac{\sin(\theta)}{\cos(\theta)}$ \label{rtidentityfirst}

\task  $\csc(\theta) = \dfrac{1}{\sin(\theta)}$

\task  $\sec(\theta) = \dfrac{1}{\cos(\theta)}$

\end{tasks}

For Exercises \ref{rtcofunctionfirst} - \ref{rrcofunctionlast}, it may be helpful to recall that $90^{\circ} - \theta$ is the measure of the `other' acute angle in the right triangle besides $\theta$.


\begin{tasks}[resume](2)
\task  $\cos(\theta) = \sin\left( 90^{\circ} - \theta \right) $ \label{rtcofunctionfirst} \label{cofunctionforeshadowing}

\task  $\csc(\theta) = \sec\left( 90^{\circ} - \theta \right) $

\task  $\cot(\theta) = \tan\left( 90^{\circ} - \theta \right) $ \label{rrcofunctionlast}

\end{tasks}

For Exercises \ref{rtpythexercisefirst} - \ref{rtpythexerciselast}, it may be helpful to remember that $a^2+b^2 = c^2$:

\begin{tasks}[resume](2)
\task  $(\cos(\theta))^2 + (\sin(\theta))^2 = 1$ \label{rtpythexercisefirst}

\task  $1 + (\tan(\theta))^2 = (\sec(\theta))^2$

\task  $ 1 + (\cot(\theta))^2 = (\csc(\theta))^2$ \label{rtpythexerciselast} \label{rtidentitylast}

\end{tasks}

\clearpage

\subsection{Answers}

\begin{enumerate}

\item  $\theta = 30^{\circ}$, $a = 3\sqrt{3}$, $c = \sqrt{108} = 6\sqrt{3}$

\item  $\alpha = 56^{\circ}$, $b = 12 \tan(34^{\circ}) =  8.094$, $c = 12\sec(34^{\circ}) = \dfrac{12}{\cos(34^{\circ})} \approx 14.475$

\item  $\theta = 43^{\circ}$, $a = 6\cot(47^{\circ}) = \dfrac{6}{\tan(47^{\circ})} \approx 5.595$, $c = 6\csc(47^{\circ}) = \dfrac{6}{\sin(47^{\circ})} \approx 8.204$

\item  $\beta = 40^{\circ}$, $b = 2.5 \tan(50^{\circ}) \approx 2.979$, $c = 2.5\sec(50^{\circ}) = \dfrac{2.5}{\cos(50^{\circ})} \approx 3.889$

\setcounter{HW}{\value{enumi}}

\end{enumerate}

\begin{enumerate}

\setcounter{enumi}{\value{HW}}

\item  The side adjacent to $\theta$ has length $4\sqrt{3} \approx 6.928$

\item  The side opposite $\theta$ has length $10 \sin(15^{\circ}) \approx 2.588$

\item  The side opposite $\theta$ is $2\tan(87^{\circ}) \approx 38.162$

\item  The hypoteneuse has length $14 \csc(38.2^{\circ}) = \dfrac{14}{\sin(38.2^{\circ})} \approx 22.639$

\item  The side adjacent to $\theta$ has length $3.98 \cos(2.05^{\circ}) \approx 3.977$

\item  The side opposite $\theta$ has length $31\tan(42^{\circ}) \approx 27.912$

\setcounter{HW}{\value{enumi}}

\end{enumerate}

\begin{multicols}{3}

\begin{enumerate}

\setcounter{enumi}{\value{HW}}

\item $36.87^{\circ}$ and $53.13^{\circ}$
\item $22.62^{\circ}$ and $67.38^{\circ}$
\item $32.52^{\circ}$ and $57.48^{\circ}$

\setcounter{HW}{\value{enumi}}

\end{enumerate}

\end{multicols}


\begin{enumerate}

\setcounter{enumi}{\value{HW}}

\item $\sin(\theta) = \frac{3}{5}, \cos(\theta) = \frac{4}{5}, \tan(\theta) = \frac{3}{4}, \csc(\theta) = \frac{5}{3}, \sec(\theta) = \frac{5}{4}, \cot(\theta) = \frac{4}{3}$, $\theta \approx 36.87^{\circ}$

\item $\sin(\theta) = \frac{12}{13}, \cos(\theta) = \frac{5}{13}, \tan(\theta) = \frac{12}{5}, \csc(\theta) = \frac{13}{12}, \sec(\theta) = \frac{13}{5}, \cot(\theta) = \frac{5}{12}$, $\theta \approx 67.38^{\circ}$

\item $\sin(\theta) = \frac{24}{25}, \cos(\theta) = \frac{7}{25}, \tan(\theta) = \frac{24}{7}, \csc(\theta) = \frac{25}{24}, \sec(\theta) = \frac{25}{7}, \cot(\theta) = \frac{7}{24}$, $\theta \approx 73.74^{\circ}$

\item $\sin(\theta) = \frac{4\sqrt{3}}{7}, \cos(\theta) = \frac{1}{7}, \tan(\theta) = 4\sqrt{3}, \csc(\theta) = \frac{7\sqrt{3}}{12}, \sec(\theta) = 7, \cot(\theta) = \frac{\sqrt{3}}{12}$,  $\theta \approx 81.79^{\circ}$

\item $\sin(\theta) = \frac{\sqrt{91}}{10}, \cos(\theta) = \frac{3}{10}, \tan(\theta) = \frac{\sqrt{91}}{3}, \csc(\theta) = \frac{10\sqrt{91}}{91}, \sec(\theta) = \frac{10}{3}, \cot(\theta) = \frac{3\sqrt{91}}{91}$, $\theta \approx 72.54^{\circ}$

\item $\sin(\theta) = \frac{\sqrt{530}}{530}, \cos(\theta) = \frac{23\sqrt{530}}{530}, \tan(\theta) = \frac{1}{23}, \csc(\theta) = \sqrt{530}, \sec(\theta) = \frac{\sqrt{530}}{23}, \cot(\theta) = 23$, $\theta \approx 2.49^{\circ}$

\item $\sin(\theta) = \frac{2\sqrt{5}}{5}, \cos(\theta) = \frac{\sqrt{5}}{5}, \tan(\theta) = 2, \csc(\theta) = \frac{\sqrt{5}}{2}, \sec(\theta) = \sqrt{5}, \cot(\theta) = \frac{1}{2}$, $\theta \approx 63.43^{\circ}$

\item  $\sin(\theta) = \frac{\sqrt{15}}{4}, \cos(\theta) = \frac{1}{4}, \tan(\theta) = \sqrt{15}, \csc(\theta) = \frac{4\sqrt{15}}{15}, \sec(\theta) = 4, \cot(\theta) = \frac{\sqrt{15}}{15}$, $\theta \approx 75.52^{\circ}$

\item $\sin(\theta) = \frac{\sqrt{6}}{6}, \cos(\theta) = \frac{\sqrt{30}}{6}, \tan(\theta) = \frac{\sqrt{5}}{5}, \csc(\theta) = \sqrt{6}, \sec(\theta) = \frac{\sqrt{30}}{5}, \cot(\theta) = \sqrt{5}$, $\theta \approx 24.09^{\circ}$

\item $\sin(\theta) = \frac{2\sqrt{2}}{3}, \cos(\theta) = \frac{1}{3}, \tan(\theta) = 2\sqrt{2}, \csc(\theta) = \frac{3\sqrt{2}}{4}, \sec(\theta) = 3, \cot(\theta) = \frac{\sqrt{2}}{4}$, $\theta \approx 70.53^{\circ}$

\item $\sin(\theta) = \frac{\sqrt{5}}{5}, \cos(\theta) = \frac{2\sqrt{5}}{5}, \tan(\theta) = \frac{1}{2}, \csc(\theta) = \sqrt{5}, \sec(\theta) = \frac{\sqrt{5}}{2}, \cot(\theta) = 2$, $\theta \approx 26.57^{\circ}$

\item $\sin(\theta) = \frac{1}{5}, \cos(\theta) = \frac{2\sqrt{6}}{5}, \tan(\theta) = \frac{\sqrt{6}}{12}, \csc(\theta) = 5, \sec(\theta) = \frac{5\sqrt{6}}{12}, \cot(\theta) = 2\sqrt{6}$, $\theta \approx 11.54^{\circ}$

\item $\sin(\theta) = \frac{\sqrt{110}}{11}, \cos(\theta) = \frac{\sqrt{11}}{11}, \tan(\theta) = \sqrt{10}, \csc(\theta) = \frac{\sqrt{110}}{10}, \sec(\theta) = \sqrt{11}, \cot(\theta) = \frac{\sqrt{10}}{10}$, $\theta \approx 72.45^{\circ}$

\item $\sin(\theta) = \frac{\sqrt{95}}{10}, \cos(\theta) = \frac{\sqrt{5}}{10}, \tan(\theta) = \sqrt{19}, \csc(\theta) = \frac{2\sqrt{95}}{19}, \sec(\theta) = 2\sqrt{5}, \cot(\theta) = \frac{\sqrt{19}}{19}$, $\theta \approx 77.08^{\circ}$

\item $\sin(\theta) = \frac{\sqrt{21}}{5}, \cos(\theta) = \frac{2}{5}, \tan(\theta) = \frac{\sqrt{21}}{2}, \csc(\theta) = \frac{5\sqrt{21}}{21}, \sec(\theta) = \frac{5}{2}, \cot(\theta) = \frac{2\sqrt{21}}{21}$, $\theta \approx  66.42^{\circ}$

\setcounter{HW}{\value{enumi}}

\end{enumerate}

\begin{enumerate}

\setcounter{enumi}{\value{HW}}

\item The tree is about 47 feet tall.

\item The lights are about 75 feet apart.

\setcounter{HW}{\value{enumi}}

\end{enumerate}

\begin{enumerate}

\setcounter{enumi}{\value{HW}}

\item \begin{enumerate}

\addtocounter{enumii}{1}

\item The fire is about 4581 feet from the base of the tower.

\item  The Sasquatch ran $200\cot(6^{\circ}) - 200\cot(6.5^{\circ}) \approx 147$ feet in those 10 seconds. This translates to $\approx 10$ miles per hour.  At the scene of the second sighting, the Sasquatch was $\approx 1755$ feet from the tower, which means, if it keeps up this pace, it will reach the tower in about $2$ minutes.

\end{enumerate}

\setcounter{HW}{\value{enumi}}

\end{enumerate}

\begin{enumerate}

\setcounter{enumi}{\value{HW}}

\item  The tree is about 41 feet tall.

\item The boat has traveled about 244 feet.

\item  The tower is about 682 feet tall. The guy wire hits the ground about  731 feet away from the base of the tower.

\setcounter{HW}{\value{enumi}}

\end{enumerate}

\begin{multicols}{3}

\begin{enumerate}

\setcounter{enumi}{\value{HW}}

\item $68.9^{\circ}$

\item $7.7^{\circ}$

\item $51^{\circ}$

\setcounter{HW}{\value{enumi}}

\end{enumerate}

\end{multicols}

\begin{multicols}{3}

\begin{enumerate}

\setcounter{enumi}{\value{HW}}

\item $19.5^{\circ}$

\item  $41.81^{\circ}$

\item  $75.5^{\circ}$.

\setcounter{HW}{\value{enumi}}

\end{enumerate}

\end{multicols}

\closegraphsfile

\newpage