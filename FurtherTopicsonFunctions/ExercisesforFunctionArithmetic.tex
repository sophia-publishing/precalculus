\startexenum

\label{ExercisesforFunctionArithmetic}

\begin{exenum}
\mexinstr{%
In Exercises \ref{basicarithonefirst} - \ref{basicarithonelast}, use the pair of functions $f$ and $g$ to find the following values if they exist.

\begin{itemize}
\item  $(f+g)(2)$ 
\item  $(f-g)(-1)$
\item  $(g-f)(1)$
\item  $(fg)\left(\frac{1}{2}\right)$
\item  $\left(\frac{f}{g}\right)(0)$
\item  $\left(\frac{g}{f}\right)\left(-2\right)$
\end{itemize}
}

\item  $f(x) = 3x+1$ and  $g(t) = 4-t$ \label{basicarithonefirst}
\item  $f(x) = x^2$ and $g(t) = -2t+1$
\item  $f(x) = x^2 - x$ and  $g(t) = 12-t^2$
\item  $f(x) = 2x^3$ and $g(t) = -t^2-2t-3$
\item  $f(x) = \sqrt{x+3}$ and  $g(t) = 2t-1$
\item  $f(x) = \sqrt{4-x}$ and $g(t) = \sqrt{t+2}$
\item  $f(x) = 2x$ and  $g(t) = \dfrac{1}{2t+1}$
\item  $f(x) = x^2$ and $g(t) = \dfrac{3}{2t-3}$
\item  $f(x) = x^2$ and  $g(t) = \dfrac{1}{t^2}$
\item  $f(x) = x^2+1$ and $g(t) = \dfrac{1}{t^2+1}$ \label{basicarithonelast}

\iexinstr{%
Exercises \ref{arithfromgraphfirst} - \ref{arithfromgraphlast} refer to the functions $f$ and $g$ whose graphs are given in \autoref{fig:yeqfxexeleventotwenty} and \autoref{fig:yeqgtexeleventotwenty} respectively. 
}

\begin{mfigure}

\begin{mfpic}[12]{-5}{5}{-5}{5}
\tlabel[cc](-3,0.5){\small $\left( -2, 0 \right)$}
\tlabel[cc](2.5,0.5){\small $\left(2, 0 \right)$}
\tlabel[cc](4,-3.5){\small $\left( 4, -3 \right)$}
\tlabel[cc](-4,-3.5){\small $\left(-4, -3 \right)$}
\tlabel[cc](1,3.5){\small $\left(0, 3 \right)$}
\axes
\tlabel[cc](5,-0.5){\scriptsize $x$}
\tlabel[cc](0.5,5){\scriptsize $y$}
\xmarks{-4,-3,-2,-1,1,2,3,4}
\ymarks{-4,-3,-2,-1,1,2,3,4}
\tlpointsep{5pt}
\scriptsize
\axislabels {x}{{$-4 \hspace{7pt}$} -4, {$-3 \hspace{7pt}$} -3, {$-2 \hspace{7pt}$} -2, {$-1 \hspace{7pt}$} -1, {$1$} 1, {$2$} 2, {$3$} 3, {$4$} 4}
\axislabels {y}{{$-4$} -4, {$-3$} -3, {$-2$} -2, {$-1$} -1, {$1$} 1, {$2$} 2, {$4$} 4}
\normalsize
\point[4pt]{(-2,0), (2,0), (4,-3), (-4,-3), (0,3)}
\penwd{1.25pt}
\function{-4,4,.1}{3*cos(3.14159265*x/4)}
\end{mfpic}

\caption{$y = f(x)$}
\label{fig:yeqfxexeleventotwenty}
\end{mfigure}

\begin{mfigure}

\begin{mfpic}[12]{-5}{5}{-5}{5}
\tlabel[cc](-4,-2.5){\small $\left( -4, -2 \right)$}
\tlabel[cc](-2,0.5){\small $\left(-1, 0 \right)$}
\tlabel[cc](-1,2){\small $\left( 0,2 \right)$}
\axes
\tlabel[cc](5,-0.5){\scriptsize $t$}
\tlabel[cc](0.5,5){\scriptsize $y$}
\xmarks{-4,-3,-2,-1,1,2,3,4}
\ymarks{-4,-3,-2,-1,1,2,3,4}
\tlpointsep{5pt}
\scriptsize
\axislabels {x}{{$-4 \hspace{7pt}$} -4, {$-3 \hspace{7pt}$} -3, {$-2 \hspace{7pt}$} -2, {$-1 \hspace{7pt}$} -1, {$1$} 1, {$2$} 2, {$3$} 3, {$4$} 4}
\axislabels {y}{{$-4$} -4, {$-3$} -3, {$-2$} -2, {$-1$} -1, {$1$} 1, {$4$} 4, {$3$} 3}
\normalsize
\point[4pt]{(-4,-2), (-1,0), (0,2)}
\penwd{1.25pt}
\polyline{(-4,-2), (-1,0), (0,2)}
\arrow \polyline{(0,2), (5,2)}
\tcaption{}
\end{mfpic}

\caption{$y = g(t)$}
\label{fig:yeqgtexeleventotwenty}
\end{mfigure}

\item $(f + g)(-4)$ \label{arithfromgraphfirst}
\item $(f + g)(0)$
\item $(f- g)(4)$

\item $(fg)(-4)$ 
\item $(fg)(-2)$
\item $(fg)(4)$

\item $\left(\dfrac{f}{g}\right)(0)$
\item $\left(\dfrac{f}{g}\right)(2)$
\item $\left(\dfrac{g}{f}\right)(-1)$ 

\item Find the domains of $f+g$, $f-g$,  $fg$, $\dfrac{f}{g}$ and $\dfrac{g}{f}$.  \label{arithfromgraphlast}

\mexinstr{%
In Exercises \ref{reformarithfirst} - \ref{reformarithlast}, let $f$ be the function defined by \begin{multline*}
      f = \{(-3, 4), (-2, 2),\\
      (-1, 0), (0, 1),
       (1, 3), \\(2, 4),
       (3, -1)\}
\end{multline*} and let $g$ be the function defined by \begin{multline*}
      g = \{(-3, -2), (-2, 0), \\
      (-1, -4), (0, 0), 
      (1, -3), \\(2, 1),
      (3, 2)\}
\end{multline*} Compute the indicated value if it exists.
}

\item $(f + g)(-3)$ \label{reformarithfirst}
\item $(f - g)(2)$
\item $(fg)(-1)$

\item $(g + f)(1)$
\item $(g - f)(3)$
\item $(gf)(-3)$

\item $\left(\frac{f}{g}\right)(-2)$
\item $\left(\frac{f}{g}\right)(-1)$
\item $\left(\frac{f}{g}\right)(2)$

\item $\left(\frac{g}{f}\right)(-1)$
\item $\left(\frac{g}{f}\right)(3)$
\item $\left(\frac{g}{f}\right)(-3)$ \label{reformarithlast}

\mexinstr{%
In Exercises \ref{basicarithtwofirst} - \ref{basicarithtwolast}, use the pair of functions $f$ and $g$ to find the domain of the indicated function then find and simplify an expression for it.

\begin{itemize}

\item  $(f+g)(x)$
\item  $(f-g)(x)$
\item  $(fg)(x)$
\item  $\left(\frac{f}{g}\right)(x)$

\end{itemize}
}

\item $f(x) = 2x+1$ and $g(x) = x-2$ \label{basicarithtwofirst}
\item $f(x) = 1-4x$ and $g(x) = 2x-1$
\item $f(x) = x^2$ and $g(x) = 3x-1$
\item $f(x) = x^2-x$ and $g(x) = 7x$
\item $f(x) = x^2-4$ and $g(x) = 3x+6$
\item $f(x) = -x^2+x+6$ and $g(x) = x^2-9$
\item $f(x) = \dfrac{x}{2}$ and $g(x) = \dfrac{2}{x}$
\item $f(x) =x-1$ and $g(x) = \dfrac{1}{x-1}$
\item $f(x) = x$ and $g(x) = \sqrt{x+1}$
\item $f(x) =\sqrt{x-5}$ and $g(x) = f(x) = \sqrt{x-5}$ \label{basicarithtwolast}

\mexinstr{%
In Exercises \ref{decomposebasicfirst} - \ref{decomposebasiclast}, write the given function as a nontrivial decomposition of functions as directed.
}

\item  For $p(z) = 4z-z^3$, find functions $f$ and $g$ so that $p=f-g$. \label{decomposebasicfirst}
\item  For $p(z) = 4z-z^3$, find functions $f$ and $g$ so that $p=f+g$.
\item  For $g(t) = 3t|2t-1|$, find functions $f$ and $h$  so that $g = fh$.
\item  For $r(x) = \dfrac{3-x}{x+1}$, find functions $f$ and $g$ so $r = \dfrac{f}{g}$.
\item  For $r(x) = \dfrac{3-x}{x+1}$, find functions $f$ and $g$ so $r = fg$. \label{decomposebasiclast}

\item    Can $f(x) = x$ be decomposed as $f = g-h$ where $g(x) = x+\dfrac{1}{x}$ and $h(x) = \dfrac{1}{x}$?

\item   Discuss with your classmates how to phrase the quantities revenue and profit in Definition \ref{revenueprofitdefns} terms of function arithmetic as defined in Definition \ref{functionarithmeticdefn}.
 
\mexinstr{%
In Exercises \ref{diffquotexerfirsta} - \ref{diffquotexerlasta}, find and simplify the difference quotients:

\begin{itemize}
\item  $\dfrac{f(2+h) - f(2)}{h}$
\item  $\dfrac{f(x+h) - f(x)}{h}$
\end{itemize}
}

\item $f(x) = 2x - 5$ \label{diffquotexerfirsta}
\item $f(x) = -3x + 5$

\item $f(x) = 6$
\item $f(x) = 3x^2 - x$

\item $f(x) = -x^2 + 2x - 1$
\item  $f(x) = 4x^2$ 

\item  $f(x) = x-x^2$ 
\item $f(x) = x^{3} + 1$

\item $f(x) = mx + b\;$ where $m \neq 0$
\item $f(x) = ax^{2} + bx + c\;$ where $a \neq 0$  \label{diffquotexerlasta}

\mexinstr{%
In Exercises \ref{diffquotexerfirstb} - \ref{diffquotexerlastb}, find and simplify the difference quotients:

\begin{itemize}
\item  $\dfrac{f(-1+\Delta x) - f(-1)}{\Delta x}$
\item  $\dfrac{f(x+\Delta x) - f(x)}{\Delta x}$
\end{itemize}
}

\item $f(x) = \dfrac{2}{x}$  \label{diffquotexerfirstb}
\item $f(x) = \dfrac{3}{1-x}$

\item  $f(x) = \dfrac{1}{x^2}$
\item  $f(x) = \dfrac{2}{x+5}$

\item $f(x) = \dfrac{1}{4x-3}$ 
\item $f(x) = \dfrac{3x}{x+2}$ 

\item $f(x) = \dfrac{x}{x - 9}$
\item $f(x) = \dfrac{x^2}{2x+1}$  \label{diffquotexerlastb}

\mexinstr{%
In Exercises \ref{diffquotexerfirstc} - \ref{diffquotexerlastc}, find and simplify the difference quotients:

\begin{itemize}
\item  $\dfrac{g(\Delta t) - g(0)}{\Delta t}$
\item  $\dfrac{g(t+\Delta t) - g(t)}{\Delta t}$
\end{itemize}
}

\item  $g(t) = \sqrt{9-t}$  \label{diffquotexerfirstc}
\item  $g(t) = \sqrt{2t+1}$
\item  $g(t) = \sqrt{-4t+5}$
\item  $g(t) = \sqrt{4-t}$
\item  $g(t) = \sqrt{at+b}$, where $a \neq 0$.
\item  $g(t) = t \sqrt{t}$ 
\item  $g(t) = \sqrt[3]{t}$.  \textbf{HINT:}  $(a-b)\left(a^2+ab+b^2\right) = a^3 - b^3$  \label{diffquotexerlastc}

\item \label{posnegdecompexercise}  In this exercise, we explore decomposing a function into its positive and negative parts.  Given a function $f$, we define the \index{positive part of a function}\textbf{positive part} of $f$, denoted $f_{+}$ and \index{negative part of a function}\textbf{negative part} of $f$, denoted $f_{-}$ by:

\[ f_{+}(x) = \dfrac{f(x) + |f(x)|}{2}, \qquad \text{and} \qquad f_{-}(x) = \dfrac{f(x) - |f(x)|}{2}. \]

\begin{enumerate}

\item Using a graphing utility, graph each of the functions $f$ below along with $f_{+}$ and $f_{-}$.

\begin{shortitemize}
\item  $f(x) = x-3$
\item  $f(x) = x^2-x-6$
\item  $f(x) = 4x-x^3$
\end{shortitemize}

Why is $f_{+}$ called the `positive part' of $f$ and $f_{-}$ called the `negative part' of $f$?

\item Show that $f = f_{+} + f_{-}$.

\item Use Definition \ref{absolutevaluepiecewise} to rewrite the expressions for $f_{+}(x)$ and $f_{-}(x)$ as piecewise defined functions.

\end{enumerate}  

\end{exenum}

\clearpage

\subsection{Answers}

\startexenum

\begin{exenum}

\item For  $f(x) = 3x+1$ and $g(x) = 4-x$

\begin{shortitemize}[MMMMMMMMMM]
\item  $(f+g)(2) = 9$
\item  $(f-g)(-1) = -7$
\item  $(g-f)(1) = -1$
\item  $(fg)\left(\frac{1}{2}\right) = \frac{35}{4}$
\item  $\left(\frac{f}{g}\right)(0) = \frac{1}{4}$
\item  $\left(\frac{g}{f}\right)\left(-2\right) = -\frac{6}{5}$
\end{shortitemize}

\item For  $f(x) = x^2$ and $g(x) = -2x+1$

\begin{shortitemize}[MMMMMMMMMM]
\item  $(f+g)(2) = 1$
\item  $(f-g)(-1) = -2$
\item  $(g-f)(1) = -2$
\item  $(fg)\left(\frac{1}{2}\right) = 0$
\item  $\left(\frac{f}{g}\right)(0) = 0$
\item  $\left(\frac{g}{f}\right)\left(-2\right) = \frac{5}{4}$
\end{shortitemize}

\item For  $f(x) = x^2 - x$ and  $g(x) = 12-x^2$

\begin{shortitemize}[MMMMMMMMMM]
\item  $(f+g)(2) = 10$
\item  $(f-g)(-1) = -9$
\item  $(g-f)(1) = 11$
\item  $(fg)\left(\frac{1}{2}\right) = -\frac{47}{16}$
\item  $\left(\frac{f}{g}\right)(0) = 0$
\item  $\left(\frac{g}{f}\right)\left(-2\right) = \frac{4}{3}$
\end{shortitemize}

\item For $f(x) = 2x^3$ and  $g(x) = -x^2-2x-3$

\begin{itemize}
\item  $(f+g)(2) = 5$
\item  $(f-g)(-1) = 0$
\item  $(g-f)(1) = -8$
\item  $(fg)\left(\frac{1}{2}\right) = -\frac{17}{16}$
\item  $\left(\frac{f}{g}\right)(0) = 0$
\item  $\left(\frac{g}{f}\right)\left(-2\right) = \frac{3}{16}$
\end{itemize}

\item For $f(x) = \sqrt{x+3}$ and  $g(x) = 2x-1$

\begin{itemize}
\item  $(f+g)(2) = 3+\sqrt{5}$
\item  $(f-g)(-1) = 3+\sqrt{2}$
\item  $(g-f)(1) = -1$
\item  $(fg)\left(\frac{1}{2}\right) = 0$
\item  $\left(\frac{f}{g}\right)(0) = -\sqrt{3}$
\item  $\left(\frac{g}{f}\right)\left(-2\right) = -5$
\end{itemize}

\item For $f(x) = \sqrt{4-x}$ and $g(x) = \sqrt{x+2}$

\begin{shortitemize}[MMMMM]
\item  $(f+g)(2) = 2+\sqrt{2}$
\item  $(f-g)(-1) = -1+\sqrt{5}$
\item  $(g-f)(1) = 0$
\item  $(fg)\left(\frac{1}{2}\right) = \frac{\sqrt{35}}{2}$
\item  $\left(\frac{f}{g}\right)(0) = \sqrt{2}$
\item  $\left(\frac{g}{f}\right)\left(-2\right) = 0$
\end{shortitemize}

\item For  $f(x) = 2x$ and  $g(x) = \frac{1}{2x+1}$

\begin{shortitemize}[MMMMMMMMMMM]
\item  $(f+g)(2) = \frac{21}{5}$
\item  $(f-g)(-1) = -1$
\item  $(g-f)(1) = -\frac{5}{3}$
\item  $(fg)\left(\frac{1}{2}\right) = \frac{1}{2}$
\item  $\left(\frac{f}{g}\right)(0) = 0$
\item  $\left(\frac{g}{f}\right)\left(-2\right) = \frac{1}{12}$
\end{shortitemize}

\item For  $f(x) = x^2$ and $g(x) = \frac{3}{2x-3}$

\begin{shortitemize}[MMMMMMMMMMM]
\item  $(f+g)(2) = 7$
\item  $(f-g)(-1) = \frac{8}{5}$
\item  $(g-f)(1) = -4$
\item  $(fg)\left(\frac{1}{2}\right) = -\frac{3}{8}$
\item  $\left(\frac{f}{g}\right)(0) = 0$
\item  $\left(\frac{g}{f}\right)\left(-2\right) = -\frac{3}{28}$
\end{shortitemize}

\item For  $f(x) = x^2$ and $g(x) = \frac{1}{x^2}$

\begin{shortitemize}[MMMMMMMMMMMM]
\item  $(f+g)(2) =\frac{17}{4}$
\item  $(f-g)(-1) = 0$
\item  $(g-f)(1) = 0$
\item  $(fg)\left(\frac{1}{2}\right) =1$
\item  $\left(\frac{f}{g}\right)(0)$ is undefined.
\item  $\left(\frac{g}{f}\right)\left(-2\right) = \frac{1}{16}$
\end{shortitemize}

\item For  $f(x) = x^2+1$ and $g(x) = \frac{1}{x^2+1}$

\begin{itemize}

\item  $(f+g)(2) =\frac{26}{5}$
\item  $(f-g)(-1) = \frac{3}{2}$
\item  $(g-f)(1) = -\frac{3}{2}$

\item  $(fg)\left(\frac{1}{2}\right) =1$
\item  $\left(\frac{f}{g}\right)(0) = 1$
\item  $\left(\frac{g}{f}\right)\left(-2\right) = \frac{1}{25}$

\end{itemize}

\item $(f + g)(-4) = -5$   
\item $(f + g)(0) = 5$
\item $(f-g)(4) = -5$

\item $(fg)(-4) = 6$ 
\item $(fg)(-2) = 0$
\item $(fg)(4) = -6$

\item $\left(\dfrac{f}{g}\right)(0) = \dfrac{3}{2}$
\item $\left(\dfrac{f}{g}\right)(2) =  0$
\item $\left(\dfrac{g}{f}\right)(-1) = 0$ 

\item The domains of $f+g$, $f-g$ and $fg$ are all $[-4,4]$.  The domain of $\frac{f}{g}$ is $[-4, -1) \cup (-1,4]$ and the domain of $\frac{g}{f}$ is $[-4, -2) \cup (-2,2) \cup (2, 4]$.

\item $(f + g)(-3) = 2$
\item $(f - g)(2) = 3$
\item $(fg)(-1) = 0$

\item $(g + f)(1) = 0$
\item $(g - f)(3) = 3$
\item $(gf)(-3) = -8$

\item $\left(\frac{f}{g}\right)(-2)$ does not exist
\item $\left(\frac{f}{g}\right)(-1) = 0$
\item $\left(\frac{f}{g}\right)(2) = 4$
\item $\left(\frac{g}{f}\right)(-1)$ does not exist
\item $\left(\frac{g}{f}\right)(3) = -2$ 
\item $\left(\frac{g}{f}\right)(-3) = -\frac{1}{2}$ 

\item For $f(x) = 2x+1$ and $g(x) = x-2$

\begin{itemize}
\item $(f+g)(x) = 3x-1$ \\
      Domain: $(-\infty, \infty)$
\item $(f-g)(x) = x+3$ \\
      Domain:  $(-\infty, \infty)$
\item $(fg)(x) = 2x^2-3x-2$ \\
      Domain: $(-\infty, \infty)$
\item $\left(\frac{f}{g}\right)(x) = \frac{2x+1}{x-2}$ \\
      Domain:  $(-\infty, 2) \cup (2, \infty)$
\end{itemize}

\item For $f(x) = 1-4x$ and $g(x) = 2x-1$

\begin{itemize}
\item $(f+g)(x) = -2x$ \\
      Domain: $(-\infty, \infty)$
\item $(f-g)(x) = 2-6x$ \\
      Domain:  $(-\infty, \infty)$
\item $(fg)(x) = -8x^2+6x-1$ \\
      Domain: $(-\infty, \infty)$
\item $\left(\frac{f}{g}\right)(x) = \frac{1-4x}{2x-1}$ \\
      Domain:  $\left(-\infty, \frac{1}{2} \right) \cup \left(\frac{1}{2}, \infty \right)$
\end{itemize}

\item For $f(x) = x^2$ and $g(x) = 3x-1$

\begin{itemize}
\item $(f+g)(x) = x^2+3x-1$ \\
      Domain: $(-\infty, \infty)$
\item $(f-g)(x) = x^2-3x+1$ \\
      Domain:  $(-\infty, \infty)$
\item $(fg)(x) = 3x^3-x^2$ \\
      Domain: $(-\infty, \infty)$
\item $\left(\frac{f}{g}\right)(x) = \frac{x^2}{3x-1}$ \\
      Domain:  $\left(-\infty, \frac{1}{3} \right) \cup \left(\frac{1}{3}, \infty \right)$
\end{itemize}

\item For $f(x) = x^2-x$ and $g(x) = 7x$

\begin{itemize}
\item $(f+g)(x) = x^2+6x$ \\
      Domain: $(-\infty, \infty)$
\item $(f-g)(x) = x^2-8x$ \\
      Domain:  $(-\infty, \infty)$
\item $(fg)(x) = 7x^3-7x^2$ \\
      Domain: $(-\infty, \infty)$
\item $\left(\frac{f}{g}\right)(x) = \frac{x-1}{7}$ \\
      Domain:  $\left(-\infty, 0 \right) \cup \left(0, \infty \right)$
\end{itemize}


\item For $f(x) = x^2-4$ and $g(x) = 3x+6$

\begin{itemize}
\item $(f+g)(x) = x^2+3x+2$ \\
      Domain: $(-\infty, \infty)$
\item $(f-g)(x) = x^2-3x-10$ \\
      Domain:  $(-\infty, \infty)$
\item $(fg)(x) = 3x^3+6x^2-12x-24$ \\
      Domain: $(-\infty, \infty)$
\item $\left(\frac{f}{g}\right)(x) = \frac{x-2}{3}$ \\
      Domain:  $\left(-\infty, -2 \right) \cup \left(-2, \infty \right)$
\end{itemize}

\item For $f(x) = -x^2+x+6$ and $g(x) = x^2-9$

\begin{itemize}
\item $(f+g)(x) = x-3$ \\
      Domain: $(-\infty, \infty)$
\item $(f-g)(x) = -2x^2+x+15$ \\
      Domain:  $(-\infty, \infty)$
\item $(fg)(x) = -x^4+x^3+15x^2-9x-54$ \\
      Domain: $(-\infty, \infty)$
\item $\left(\frac{f}{g}\right)(x) = -\frac{x+2}{x+3}$ \\
      Domain:  $\left(-\infty, -3 \right) \cup \left(-3, 3 \right) \cup (3, \infty)$
\end{itemize}


\item For  $f(x) = \frac{x}{2}$ and $g(x) = \frac{2}{x}$

\begin{itemize}
\item $(f+g)(x) = \frac{x^2+4}{2x}$ \\
      Domain: $(-\infty, 0) \cup (0, \infty)$
\item $(f-g)(x) = \frac{x^2-4}{2x}$ \\
      Domain:  $(-\infty,0) \cup (0, \infty)$
\item $(fg)(x) = 1$ \\
      Domain: $(-\infty,0) \cup (0, \infty)$
\item $\left(\frac{f}{g}\right)(x) = \frac{x^2}{4}$ \\
      Domain: $(-\infty,0) \cup (0, \infty)$
\end{itemize}

\item For   $f(x) =x-1$ and $g(x) = \frac{1}{x-1}$

\begin{itemize}
\item $(f+g)(x) = \frac{x^2-2x+2}{x-1}$ \\
      Domain: $(-\infty, 1) \cup (1, \infty)$
\item $(f-g)(x) = \frac{x^2-2x}{x-1}$ \\
      Domain:  $(-\infty,1) \cup (1, \infty)$
\item $(fg)(x) = 1$ \\
      Domain: $(-\infty,1) \cup (1, \infty)$
\item $\left(\frac{f}{g}\right)(x) =x^2-2x+1$ \\
      Domain: $(-\infty,1) \cup (1, \infty)$
\end{itemize}

\item For   $f(x) =x$ and $g(x) = \sqrt{x+1}$

\begin{itemize}
\item $(f+g)(x) = x+\sqrt{x+1}$ \\
      Domain: $[-1,\infty)$
\item $(f-g)(x) = x-\sqrt{x+1}$ \\
       Domain: $[-1,\infty)$
\item $(fg)(x) = x\sqrt{x+1}$ \\
       Domain: $[-1,\infty)$
\item $\left(\frac{f}{g}\right)(x) =\frac{x}{\sqrt{x+1}}$ \\
       Domain: $(-1,\infty)$
\end{itemize}

\item For   $f(x) = \sqrt{x-5}$ and $g(x) = f(x) = \sqrt{x-5}$

\begin{itemize}
\item $(f+g)(x) = 2\sqrt{x-5}$ \\
      Domain: $[5,\infty)$
\item $(f-g)(x) =0$ \\
       Domain: $[5,\infty)$
\item $(fg)(x) =x-5$ \\
       Domain: $[5,\infty)$
\item $\left(\frac{f}{g}\right)(x) =1$ \\
       Domain: $(5,\infty)$
\end{itemize}

\item One solution is $f(z) = 4z$ and $g(z) = z^3$. 
\item One solution is $f(z) = 4z$ and $g(z) = - z^3$. 
\item One solution is  $f(t) = 3t$ and $h(t) = |2t-1|$ 
\item One solution is $f(x) = 3-x$ and $g(x) = x+1$.  
\item  One solution is $f(x) = 3-x$ and $g(x) = (x+1)^{-1}$.  
\item No.  The equivalence does not hold when $x = 0$.

\addtocounter{enumi}{1}

\item $2$, $2$.
\item $-3$, $-3$.

\item $0$, $0$
\item $3h+11$, $6x+3h-1$

\item $-h-2$, $-2x-h+2$
\item  $4h+16$, $8x+4h$

\item $-h-3$, $-2x-h+1$
\item $h^2+6h+12$, $3x^{2} + 3xh + h^{2}$

\item $m$, $m$
\item $ah + 4a+b$, $2ax + ah + b$


\item  $\dfrac{2}{\Delta x-1}$, $\dfrac{-2}{x(x+\Delta x)}$
\item $\dfrac{-3}{2(\Delta x - 2)}$, $\dfrac{3}{(x+\Delta x-1)(x-1)}$


\item  $\dfrac{2-\Delta x}{(\Delta x - 1)^2}$, $\dfrac{-(2x+\Delta x)}{x^2(x+\Delta x)^2}$
\item  $\dfrac{-1}{2(\Delta x+4)}$, $\dfrac{-2}{(x+5)(x+\Delta x+5)}$

\item $\dfrac{4}{7(4 \Delta x - 7)}$, $\dfrac{-4}{(4x-3)(4x+4\Delta x-3)}$
\item $\dfrac{6}{\Delta x + 1}$, $\dfrac{6}{(x+2)(x+\Delta x+2)}$

\item $\dfrac{9}{10(\Delta x - 10)}$, $\dfrac{-9}{(x - 9)(x + \Delta x - 9)}$
\item $\dfrac{\Delta x}{2 \Delta x - 1}$, $\dfrac{2x^2+2x\Delta x+2x+\Delta x}{(2x+1)(2x+2\Delta x+1)}$

\item $\dfrac{-1}{\sqrt{9-\Delta t} +3}$,   $\dfrac{-1}{\sqrt{9-t-\Delta t} + \sqrt{9-t}}$
\item $\dfrac{2}{\sqrt{2\Delta t+1} + 1}$, $\dfrac{2}{\sqrt{2t+2\Delta t+1} + \sqrt{2t+1}}$

\item $\dfrac{-4}{\sqrt{5-4\Delta t} + \sqrt{5}}$, $\dfrac{-4}{\sqrt{-4t-4\Delta t+5} + \sqrt{-4t+5}}$

\item $\dfrac{-1}{\sqrt{4-\Delta t} + 2}$, $\dfrac{-1}{\sqrt{4-t-\Delta t} + \sqrt{4-t}}$

\item $\dfrac{a}{\sqrt{a\Delta t+b} + \sqrt{b}}$, $\dfrac{a}{\sqrt{at+a\Delta t+b} + \sqrt{at+b}}$

\item   $(\Delta t)^{\frac{1}{2}} $, $\dfrac{3t^2+3t\Delta t+(\Delta t)^2}{(t+\Delta t)^{3/2} + t^{3/2}} $

\item $\dfrac{1}{(\Delta t)^{2/3}}$,  $\dfrac{1}{(t+\Delta t)^{2/3} + (t+\Delta t)^{1/3} t^{1/3} + t^{2/3}}$

\item  \begin{enumerate}  

\addtocounter{enumii}{1}

\item $(f_{+} + f_{-})(x) =  f_{+}(x) + f_{-}(x) = \dfrac{f(x) + |f(x)|}{2} + \dfrac{f(x) - |f(x)|}{2} = \dfrac{2f(x)}{2} = f(x)$.

\item
\begin{align*}
f_{+}(x)  &=  \begin{mycases} 
    0 &  \text{if $f(x) < 0$} \\
      f(x) & \text{if $f(x)  \geq 0$} \\
   \end{mycases},\\
f_{-}(x)  &=  \begin{mycases} 
    f(x) &  \text{if $f(x) < 0$} \\
      0 & \text{if $f(x)  \geq 0$} \\
   \end{mycases}
\end{align*}

\end{enumerate}

\end{exenum}
