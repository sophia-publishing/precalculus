\mfpicnumber{1}

\opengraphsfile{FunctionArithmetic}

\setcounter{footnote}{0}

\label{FunctionArithmetic}

As we mentioned in Section \ref{GraphsofFunctions}, in this chapter, we are studying functions in a more abstract and general setting.  In this section, we begin our study of what can be considered as the \textit{algebra of functions}  by defining \textit{function arithmetic.}  

Given two real numbers,  we have four primary arithmetic operations available to us:  addition, subtraction, multiplication, and division (provided we don't divide by $0$.)  Since the functions we study in this text have ranges which are sets of real numbers, it makes sense we can extend these arithmetic notions to functions. 

For example,  to add two functions means  we add their outputs;  to subtract two functions, we subtract their outputs, and so on and so forth.  More formally, given two functions $f$ and $g$, we \textit{define} a new function $f+g$ whose rule is determined by adding the outputs of $f$ and $g$.  That is $(f+g)(x) = f(x) + g(x)$.  While  this looks suspiciously like some kind of distributive property, it is nothing of the sort.  The `$+$' sign in the expression `$f+g$' is part of the \textit{name} of the function we are defining,\footnote{We could have just as easily called this new function $S(x)$ for `sum' of $f$ and $g$ and defined $S$ by $S(x) = f(x) + g(x)$.} whereas the plus sign `$+$' sign in the expression $f(x) + g(x)$ represents real number addition: we are adding the output from $f$, $f(x)$ with the output from $g$, $g(x)$ to determine the output from the sum function, $(f+g)(x)$.

 Of course, in order to define $(f+g)(x)$ by the formula $(f+g)(x) = f(x) + g(x)$, both $f(x)$ and $g(x)$ need to be defined in the first place; that is, $x$ must be in the domain of $f$ \textit{and} the domain of $g$.  You'll recall\footnote{see Section \ref{AppSetTheory}.} this means $x$ must be in the \textit{intersection} of the domains of $f$ and $g$.   We define the following.
 
\begin{tcolorbox}

\begin{defn}  \label{functionarithmeticdefn}  \index{function ! arithmetic} Suppose $f$ and $g$ are functions and $x$ is in both the domain of $f$ and the domain of $g$.
\begin{itemize}

\item  The \index{function ! sum} \textbf{sum} of $f$ and $g$, denoted $f+g$, is the function defined by the formula \[(f+g)(x) = f(x) + g(x)\]

\item  The \index{function ! difference} \textbf{difference} of $f$ and $g$, denoted $f-g$, is the function defined by the formula \[(f-g)(x) = f(x) - g(x)\]

\item  The \index{function ! product} \textbf{product} of $f$ and $g$, denoted $fg$, is the function defined by the formula \[(fg)(x) = f(x)g(x)\]

\item  The \index{function ! quotient} \textbf{quotient} of $f$ and $g$, denoted $\dfrac{f}{g}$, is the function defined by the formula \[\left(\dfrac{f}{g}\right)(x) = \dfrac{f(x)}{g(x)},\] provided $g(x) \neq 0$.

\end{itemize}

\end{defn}

\end{tcolorbox}

We put these definitions to work for us in the next example.

\begin{ex}  \label{funcarithex} Consider the following functions:

\begin{shortitemize}
\item  $f(x) = 6x^2 - 2x$ \vphantom{$g(t) = 3-\dfrac{1}{t}$, $t > 0$.}
\item $g(t) = 3-\dfrac{1}{t}$, $t > 0$
\item  $h = \{ (-3,2), (-2,0.4), (0,\sqrt{2}), (3, -6) \}$  
\item  $s$ whose graph is given in \autoref{fig:yeqst}.
\end{shortitemize}

\begin{figure}
\begin{center}

\begin{mfpic}[15]{-2}{5}{-3}{4}
\axes
\tlabel[cc](5,-0.5){\scriptsize $t$}
\tlabel[cc](0.5,4){\scriptsize $y$}
\xmarks{-1, 0, 1, 2, 3, 4}
\ymarks{-2, -1, 0, 1, 2, 3}
\tcaption{\scriptsize $y = s(t)$}
\tlpointsep{4pt}
\scriptsize
\tlabel[cc](1, -2.5){$(0,-2)$}
\tlabel[cc](2, 0.5){$(1,0)$}
\tlabel[cc](2, 2.5){$(2,2)$}
\tlabel[cc](-2, -2.5){$(-2, -2)$}
\axislabels {x}{{$-1 \hspace{7pt}$} -1,{$1$} 1, {$2$} 2, {$3$} 3, {$4$} 4}
\axislabels {y}{{$2$} 2,{$-1$} -1,{$1$} 1, {$3$} 3}
\normalsize
\penwd{1.25pt}
 \arrow \polyline{(-2, -2), (0, -2), (2, 2), (5, 2)}
\point[4pt]{(-2,-2), (0,-2), (1,0), (2,2)}
\end{mfpic}

\caption{}
\label{fig:yeqst}
\end{center}
\end{figure}

\begin{enumerate}

\item  \label{funcarithfindvaluesex} Find and simplify the following function values:

\begin{shortenumerate}[MMMMMMMMMMMMM]
\item  $(f+g)(1)$ \vphantom{ $\left( \dfrac{s}{h} \right)(0)$}
\item $(s-f)(-1)$ \vphantom{ $\left( \dfrac{s}{h} \right)(0)$}
\item $(fg)(2)$ \vphantom{ $\left( \dfrac{s}{h} \right)(0)$}
\item  $\left( \dfrac{s}{h} \right)(0)$
\item \label{threefunctionsfirstex} $((s+g)+h)(3)$  \vphantom{$\left(\dfrac{f+g}{s}\right)(3)$}
\item $(s+(g+h))(3)$  \vphantom{$\left(\dfrac{f+g}{s}\right)(3)$}
\item $\left(\dfrac{f+h}{s}\right)(3)$
\item  \label{threefunctionslastex} $(f(g-h))(-2)$ \vphantom{$\left(\dfrac{f+g}{s}\right)(3)$}
\end{shortenumerate}

\item  Find the domain of each of the following functions:

\begin{multicols}{2}

\begin{enumerate}

\item $hg$

\item  $\dfrac{f}{s}$

\end{enumerate}

\end{multicols}

\item \label{arithexpressionex} Find expressions for the functions below.  State the domain for each.

\begin{multicols}{2}

\begin{enumerate}

\item   \label{proddomainex} $(fg)(x)$

\item  \label{quotdomainex}  $\left(\dfrac{g}{f}\right)(t)$

\end{enumerate}

\end{multicols}

\end{enumerate}

{\bf Solution.}  

\begin{enumerate}

\item \begin{enumerate}

\item  By definition, $(f+g)(1) = f(1) + g(1)$.   We find $f(1) = 6(1)^2-2(1) = 4$ and $g(1) = 3 - \frac{1}{1} = 2$. So we get  $(f+g)(1) = 4+2 = 6$.

\item  To find $(s-f)(-1) = s(-1) - f(-1)$, we need both $s(-1)$ and $f(-1)$.  To get $s(-1)$, we look to the graph of $y = s(t)$ and look for the $y$-coordinate of the point on the graph with the $t$-coordinate of $-1$.  While not labeled directly, we infer the point $(-1,-2)$ is on the graph which means $s(-1) = -2$. For $f(-1)$, we compute: $f(-1) = 6(-1)^2-2(-1) = 8$.  Putting it all together, we get $(s-f)(-1) = (-2) -(8) = -10$.

\item Since $(fg)(2) = f(2)g(2)$, we first  compute $f(2)$ and $g(2)$. We find $f(2) = 6(2)^2-2(2) = 20$ and $g(2) = 2 + \frac{1}{2} = \frac{5}{2}$, so $(fg)(2) = f(2) g(2) = (20)\left(\frac{5}{2}\right) = 50$.

\item By definition, $\left( \frac{s}{h} \right)(0) = \frac{s(0)}{h(0)}$.  Since $(0, -2)$ is on the graph of $y=s(t)$, so we know $s(0)  = -2$.  Likewise, the ordered pair $(0, \sqrt{2}) \in h$, so $h(0) = \sqrt{2}$.  We get  $\left( \frac{s}{h} \right)(0) = \frac{s(0)}{h(0)} = \frac{-2}{\sqrt{2}} = -\sqrt{2}$.

\item  The expression $((s+g)+h)(3)$ involves \textit{three} functions.  Fortunately, they are grouped so that we can apply Definition \ref{functionarithmeticdefn} by first considering the sum of the two functions $(s+g)$ and $h$, then to the sum of the two functions $s$ and $g$: $((s+g)+h)(3) = (s+g)(3)+h(3) = (s(3)+g(3))+h(3)$.  To get $s(3)$, we look to the graph of $y = s(t)$.  We infer the point $(3,2)$ is on the graph of $s$, so $s(3) = 2$.  We compute $g(3) = 3-\frac{1}{3} = \frac{8}{3}$.  To find $h(3)$, we note $(3, -6) \in h$, so $h(3) = -6$.  Hence, $((s+g)+h)(3) = (s+g)(3)+h(3) = (s(3)+g(3))+h(3) = \left(2+\frac{8}{3}\right) + (-6) = -\frac{4}{3}$.

\item The expression   $(s+(g+h))(3)$ is very similar to the previous problem, $((s+g)+h)(3)$ except that the $g$ and $h$ are grouped together here instead of the $s$ and $g$.  We proceed as above applying Definition \ref{functionarithmeticdefn} twice and find $(s+(g+h))(3) = s(3) + (g+h)(3) = s(3)+(g(3)+h(3))$.  Substituting the values for $s(3)$, $g(3)$ and $h(3)$, we get   $(s+(g+h))(3) = 2 + \left(\frac{8}{3} + (-6)\right)  = -\frac{4}{3}$, which, not surprisingly, matches our answer to the previous problem.

\item  Once again, we find the expression  $\left(\frac{f+h}{s}\right)(3)$ has more than two functions involved.  As with all fractions, we treat `$-$' as a grouping symbol and interpret  $\left(\frac{f+h}{s}\right)(3) = \frac{(f+h)(3)}{s(3)} = \frac{f(3)+h(3)}{s(3)}$.  We compute $f(3) = 6(3)^2-2(3) = 48$ and have $h(3) = -6$ and $s(3) = 2$ from above.  Hence, $\left(\frac{f+h}{s}\right)(3) =  \frac{f(3)+h(3)}{s(3)} = \frac{48+(-6)}{2} = 21$.

\item We need to need to exercise caution in parsing $(f(g-h))(-2)$.  In this context, $f$, $g$, and $h$ are all functions, so we interpret $(f(g-h))$ as the function and $-2$ as the argument. We view the function $f(g-h)$ as the product of $f$ and the function $g-h$.  Hence, $(f(g-h))(-2) = f(-2) [(g-h)(-2)] = f(-2) [g(-2) - h(-2)]$.  We compute $f(-2) = 6(-2)^2-2(-2) = 28$, and $g(-2)= 3 - \frac{1}{-2} = 3 + \frac{1}{2} = \frac{7}{2} = 3.5$.  Since $(-2, 0.4) \in h$, $h(-2) = 0.4$.  Putting this altogether, we get $(f(g-h))(-2) = f(-2) [(g-h)(-2)] = f(-2) [g(-2) - h(-2)] = 28(3.5-0.4) = 28(3.1) = 86.8$.

\end{enumerate}

\item \begin{enumerate}

\item To find the domain of $hg$, we need to find the real numbers in both the domain of $h$ and the domain of $g$.  The domain of $h$ is $\{ -3, -2, 0, 3 \}$ and the domain of $g$ is $\{ t \in \mathbb{R} \, | \, t > 0 \}$ so the only real number in common here is $3$.  Hence, the domain of $hg$ is $\{ 3\}$, which may be small, but it's better than nothing.\footnote{Since $(hg)(3) = h(3)g(3) = (-6)\left(\frac{8}{3}\right) = -16$, we can write $hg = \{ (3,-16) \}$.}

\item To find the domain of $\frac{f}{s}$, we first note the domain of $f$ is all real numbers, but that the domain of $s$, based on the graph, is just $[-2, \infty)$.  Moreover, $s(t) = 0$ when $t=1$, so we must exclude this value from the domain of $\frac{f}{s}$.  Hence, we are left with $[-2, 1) \cup (1, \infty)$.

\end{enumerate}

\item  

\begin{enumerate}

\item  By definition, $(fg)(x) = f(x)g(x)$.  We are given $f(x) = 6x^2-2x$ and $g(t) = 3 - \frac{1}{t}$ so $g(x) = 3 - \frac{1}{x}$.  Hence, 

\begin{align*}
(fg)(x) & = f(x)g(x) \\
           & = \left( 6x^2-2x \right) \left( 3 - \dfrac{1}{x} \right) \\
           & = 18x^2 - 6x^2 \left(\dfrac{1}{x}\right) - 2x(3) + 2x \left(\dfrac{1}{x}\right)  \tag{distribute} \\
           & = 18x^2 - 6x - 6x + 2 \\
           & = 18x^2 - 12x + 2
\end{align*}
           
To find the domain of $fg$, we note the domain of $f$ is all real numbers, $(-\infty, \infty)$ whereas the domain of $g$ is restricted to $\{ t \in \mathbb{R} \, | \, t > 0 \} = (0, \infty)$.  Hence, the domain of $fg$ is likewise restricted to $(0, \infty)$.  Note if we relied solely on the \textbf{simplified formula} for $(fg)(x) = 18x^2 - 12x + 2$, we would have obtained the \textit{incorrect} answer for the domains of $fg$.  

\item To find an expression for $\left(\frac{g}{f}\right)(t) = \frac{f(t)}{g(t)}$ we first note $f(t) = 6t^2-2t$ and $g(t) = 3 - \frac{1}{t}$.  Hence:
\begin{align*}
	\left( \dfrac{g}{f}\right)(t) &= \dfrac{g(t)}{f(t)} \\
	& = \dfrac{3-\dfrac{1}{t}\vphantom{\left(\dfrac{1}{t}\right)}}{6t^2 - 2t} &= \dfrac{3-\dfrac{1}{t}\vphantom{\left(\dfrac{1}{t}\right)}}{6t^2 - 2t} \cdot \dfrac{t}{t} \tag{simplify compound fractions} \\
	& = \dfrac{\left(3-\dfrac{1}{t}\right) t}{\left(6t^2 - 2t\right)t} &=  \dfrac{3t-1}{\left(6t^2 - 2t\right)t} \\
	& = \dfrac{3t-1}{2t^2(3t-1)} &=  \dfrac{\cancelto{1}{(3t-1)}}{2t^2\cancel{(3t-1)}} \tag{factor and cancel} \\
& = \dfrac{1}{2t^2} \\
\end{align*}

Hence, $\left(\frac{g}{f} \right)(t) = \frac{1}{2t^2} = \frac{1}{2} t^{-2}$.  To find the domain of $\frac{g}{f}$, a real number must be both in the domain of $g$, $(0, \infty)$, and the domain of $f$, $(-\infty, \infty)$ so we start with the set $(0, \infty)$.  Additionally, we require $f(t) \neq 0$.  Solving $f(t) = 0$ amounts to solving $6t^2-2t = 0$ or $2t(3t-1) = 0$.  We find $t = 0$ or $t = \frac{1}{3}$ which means we need to exclude these values from the domain.  Hence, our final answer for the domain of $\frac{g}{f}$ is $\left(0, \frac{1}{3} \right) \cup \left(\frac{1}{3}, \infty \right)$.  Note that, once again,  using the  \textit{simplified formula} for $\left(\frac{g}{f}\right)(t)$ to determine the domain of $\frac{g}{f}$, would have produced erroneous results. \qed

\end{enumerate}

\end{enumerate}

\end{ex}
 
 A few remarks are in order.  First, in number \ref{funcarithfindvaluesex} parts \ref{threefunctionsfirstex} through \ref{threefunctionslastex}, we first encountered combinations of \textit{three} functions despite Definition \ref{functionarithmeticdefn} only addressing combinations of \textit{two} functions at a time.  It turns out that function arithmetic inherits many of the same properties of real number arithmetic. For example, we showed above that  $((s+g)+h)(3) = (s+(g+h))(3)$.  In general, given any three functions $f$, $g$, and $h$, $(f+g)+h = f +(g+h)$ that is, function addition is \textit{assocative}.  To see this, choose an element $x$ common to the domains of $f$, $g$, and $h$.  Then 
 
\begin{align*}
((f+g)+h)(x) & = (f+g)(x)+h(x) \tag{definiton of $((f+g)+h)(x)$} \\
& = (f(x)+g(x))+h(x) \tag{definition of $(f+g)(x)$} \\
& = f(x) + (g(x)+h(x)) \tag{associative property of real number addition} \\
& = f(x) + (g+h)(x) \tag{definition of $(g+h)(x)$} \\
& = (f+(g+h))(x) \tag{definition of $(f+(g+h))(x)$} \\
\end{align*}

The key step  to the argument is that  $(f(x)+g(x))+h(x)   =  f(x) + (g(x)+h(x))$ which is true courtesy of the associative property of real number addition.  And just like with real number addition, because function addition is associative, we may write $f+g+h$ instead of $(f+g)+h$ or $f+(g+h)$ even though, when it comes down to computations, we can only add two things together at a time.\footnote{Addition is a `binary' operation - meaning it is defined only on two objects at once.  Even though we write $1+2+3 = 6$,  mentally, we add just two of  numbers together at any given time to get our answer: for example, $1+2+3 = (1+2)+3 = 3+3 = 6$.}

For completeness, we summarize the properties of function arithmetic in the theorem below.  The proofs of the properties all follow along the same lines as the proof of the associative property and are left to the reader.  We investigate some additional properties in the exercises.

\begin{tcolorbox}

\begin{thm}  \label{functionarithmeticprops}  Suppose $f$, $g$ and $h$ are functions.

\begin{itemize}

\item   \textbf{Commutative Law of Addition:} $f+g =g+f$

\item  \textbf{Associative Law of Addition:} $(f+g) + h = f+(g+h)$

\item  \textbf{Additive Identity:}  The function $Z(x) = 0$ satisfies:  $f+Z = Z+f = f$ for all functions $f$.

\item  \textbf{Additive Inverse:}  The function $F(x) = -f(x)$ for all $x$ in the domain of $f$ satisfies:  \[ f+F=F+f=Z.\] 

\item  \textbf{Commutative Law of Multiplication:} $fg = gf$

\item  \textbf{Associative Law of Multiplication:} $(fg)h =f(gh)$

\item  \textbf{Multiplicative Identity:}  The function $I(x) = 1$ satisfies:  $fI = If = f$ for all functions $f$.

\item  \textbf{Multiplicative Inverse:}  If $f(x) \neq 0$ for all $x$ in the domain of $f$, then $F(x) = \dfrac{1}{f(x)}$ satisfies: \[ fF=Ff = I\]

\item  \textbf{Distributive Law of Multiplication over Addition:}  $f(g+h) = fg+fh$


\end{itemize}

\end{thm}

\end{tcolorbox}

In the next example, we decompose given functions into sums, differences, products and/or quotients of other functions.  Note that there are infinitely many different ways to do this, including some trivial ones.  For example, suppose we were instructed to decompose $f(x) = x+2$ into a sum or difference of functions.  We could write $f = g+h$ where $g(x) = x$ and $h(x) = 2$ or we could choose $g(x) = 2x+3$ and $h(x) = -x-1$.  More simply, we could write $f = g+h$ where $g(x) = x+2$ and $h(x) = 0$.  We'll call this last decomposition a `trivial' decomposition.  Likewise, if we ask for a decomposition of $f(x) = 2x$ as a product, a nontrivial solution would be $f = gh$ where $g(x) = 2$ and $h(x) = x$ whereas a trivial solution would be $g(x) = 2x$ and $h(x) = 1$.  In general, non-trivial solutions to decomposition problems avoid using the additive identity, $0$,  for sums and differences and the multiplicative identity, $1$, for products and quotients.

\begin{ex} \label{funcarithdecompex}  

\begin{enumerate}

\item  For  $f(x) = x^2 - 2x$, find functions $g$, $h$ and $k$ to decompose $f$ nontrivially as:

\begin{shortenumerate}
\item  $f = g-h$
\item  $f = g+h$
\item  $f = gh$
\item  $f = g(h-k)$
\end{shortenumerate}

\item  For $F(t) = \dfrac{2t+1}{\sqrt{t^2-1}}$, find functions $G$, $H$ and $K$ to decompost $F$ nontrivially as:

\begin{shortenumerate}
\item  $F = \dfrac{G}{H}$ \vphantom{$F =  \dfrac{G+H}{K}$}
\item  $F = GH$ \vphantom{$F =  \dfrac{G+H}{K}$}
\item  $F = G+H$ \vphantom{$F =  \dfrac{G+H}{K}$}
\item  $F =  \dfrac{G+H}{K}$ 
\end{shortenumerate}

\end{enumerate}

{\bf Solution.}

\begin{enumerate}

\item 

\begin{enumerate}

\item To decompose $f = g-h$, we need functions $g$ and $h$ so $f(x) = (g-h)(x) = g(x) - h(x)$.  Given $f(x) = x^2 - 2x$,  one option is to let $g(x) = x^2$ and $h(x) = 2x$.  To check, we find $(g-h)(x) = g(x) - h(x) = x^2-2x = f(x)$ as required.  In addition to checking the formulas match up, we also need to check domains.  There isn't much work here since the domains of $g$ and $h$ are all real numbers which combine to give the domain of $f$ which is all real numbers.

\item  In order to write $f = g+h$, we need $f(x) = (g+h)(x) = g(x) + h(x)$.  One way to accomplish this is to write $f(x) = x^2 - 2x = x^2+(-2x)$ and identify $g(x) = x^2$ and $h(x)  = -2x$.  To check, $(g+h)(x) = g(x) + h(x) = x^2 - 2x = f(x)$.  Again, the domains for both $g$ and $h$ are all real numbers which combine to give $f$ its domain of all real numbers.

\item  To write $f = gh$, we require $f(x) = (gh)(x) = g(x) h(x)$.  In other words, we need to factor $f(x)$.  We find $f(x) = x^2-2x = x(x-2)$, so one choice is to select $g(x) = x$ and $h(x) = x-2$.  Then $(gh)(x) = g(x)h(x) = x(x-2) = x^2-2x = f(x)$, as required.  As above, the domains of $g$ and $h$ are all real numbers which combine to give $f$ the correct domain of $(-\infty, \infty)$.

\item  We need to be careful here interpreting the equation $f = g(h-k)$.  What we have is an equality of \textit{functions} so the parentheses here \textit{do not} represent function notation here, but, rather function \textit{multiplication}.  The way to parse $g(h-k)$, then, is the function $g$ \textit{times} the function $h-k$. Hence, we seek functions $g$, $h$, and $k$ so that $f(x) = [g(h-k)](x) = g(x) [(h-k)(x)] = g(x) (h(x) - k(x))$.    From the previous example, we know we can rewrite $f(x) = x(x-2)$, so one option is to set $g(x) = h(x) = x$ and $k(x) = 2$ so that $[g(h-k)](x)  =  g(x) [(h-k)(x)]  =  g(x) (h(x) - k(x)) = x(x-2) = x^2-2x = f(x)$, as required.  As above, the domain of all constituent functions is $(-\infty, \infty)$ which matches the domain of $f$.


\end{enumerate}

\item  \begin{enumerate}

\item To write $F = \frac{G}{H}$, we need $G(t)$ and $H(t)$ so $F(t)= \left(\frac{G}{H}\right)(t) = \frac{G(t)}{H(t)}$. We choose $G(t) = 2t+1$ and $H(t) = \sqrt{t^2-1}$.  Sure enough, $ \left(\frac{G}{H}\right)(t) = \frac{G(t)}{H(t)} = \frac{2t+1}{ \sqrt{t^2-1}} = F(t)$ as required.  When it comes to the domain of $F$, owing to the square root, we require $t^2-1 \geq 0$.  Since we have a denominator  as well, we require $\sqrt{t^2-1} \neq 0$. The former requirement is the same restriction on $H$, and the latter requirement comes from Definition \ref{functionarithmeticdefn}. Starting with the domain of $G$, all real numbers, and working through the details, we arrive at the correct domain of $F$, $(-\infty, -1) \cup (1, \infty)$.

\item  Next, we are asked to find functions $G$ and $H$ so $F (t) = (GH)(t)  = G(t) H(t)$.  This means we need to rewrite the expression for $F(t)$ as a product. One way to do this is to convert radical notation to exponent notation:  \[ F(t) = \dfrac{2t+1}{\sqrt{t^2-1}} = \dfrac{2t+1}{\left(t^2-1 \right)^{\frac{1}{2}}} = (2t+1) \left(t^2-1\right)^{-\frac{1}{2}}. \] Choosing $G(t) = 2t+1$ and $H(t) = \left(t^2-1\right)^{-\frac{1}{2}}$, we see $(GH)(t)  = G(t) H(t) = (2t+1) \left(t^2-1\right)^{-\frac{1}{2}}$ as required.  The domain restrictions on $F$ stem from the presence of the square root in the denominator - both are addressed when finding the domain of $H$.  Hence, we obtain the correct domain of $F$.

\item  To express $F$ as a sum of functions $G$ and $H$, we could rewrite \[ F(t) =  \dfrac{2t+1}{\sqrt{t^2-1}} = \dfrac{2t}{\sqrt{t^2-1}} + \dfrac{1}{\sqrt{t^2-1}},\] so that $G(t) =  \frac{2t}{\sqrt{t^2-1}}$ and $H(t) =  \frac{1}{\sqrt{t^2-1}}$.  Indeed, $(G+H)(t) = G(t)+H(t) =  \frac{2t}{\sqrt{t^2-1}} +  \frac{1}{\sqrt{t^2-1}} = \frac{2t+1}{\sqrt{t^2-1}} = F(t)$, as required.  Moreover, the domain restrictions for $F$ are the same for both $G$ and $H$, so we get agreement on the domain, as required.

\item Last, but not least, to write $F = \frac{G+H}{K}$, we require $F(t) =\left(\frac{G+H}{K}\right)(t) = \frac{(G+H)(t)}{K(t)} = \frac{G(t)+H(t)}{K(t)}$.  Identifying $G(t) = 2t$, $H(t) = 1$, and $K(t) = \sqrt{t^2-1}$, we get \[\left(\dfrac{G+H}{K}\right)(t) = \dfrac{(G+H)(t)}{K(t)} = \dfrac{G(t) + H(t)}{K(t)} = \dfrac{2t+1}{\sqrt{t^2-1}} = F(t).\]  Concerning domains, the domain of both $G$ and $H$ are all real numbers, but the domain of $K$ is restricted to $t^2-1 \geq 0$.  Coupled with the restriction stated in Definition \ref{functionarithmeticdefn} that $K(t) \neq 0$, we recover the domain of $F$, $(-\infty, -1) \cup (1, \infty)$. \qed

\end{enumerate}

\end{enumerate}

\end{ex}

\subsection{Difference Quotients}
\label{differencequotients}

Recall  in Section \ref{AverageRateofChange} the concept of the average rate of change of a function over the interval $[a,b]$  is the slope between the two points $(a, f(a))$ and $(b, f(b))$ and is given by \[ \dfrac{\Delta[f(x)]}{\Delta x} = \dfrac{f(b)-f(a)}{b-a}.\]
Consider a function $f$ defined over an interval containing $x$ and $x+h$ where $h \neq 0$. The average rate of change of $f$ over the interval $[x,x+h]$ is thus given by the formula:\footnote{assuming $h>0$;  otherwise, we  the interval is $[x+h, x]$.  We get the same formula for the difference quotient either way.}

\[ \dfrac{\Delta[f(x)]}{\Delta x} = \dfrac{f(x+h)-f(x)}{h}, \quad h \neq 0.\]

The above is an example of what is traditionally called the  \textbf{difference quotient}  or  \textbf{Newton quotient} of $f$, since it is the \textit{quotient} of two \textit{differences}, namely $\Delta[f(x)]$ and $\Delta x$. Another formula for the difference quotient sticks keeps with the notation $\Delta x$ instead of $h$:


\[ \dfrac{\Delta[f(x)]}{\Delta x} = \dfrac{f(x+\Delta x)-f(x)}{\Delta x}, \quad \Delta x \neq 0.\]


It is important to understand that in this formulation of the difference quotient, the variables `$x$' and `$\Delta x$' are distinct - that is they do not combine as like terms. 

In Section \ref{IntroRational},   the average rate of change of  position function $s$ can be interpreted as the average velocity (see Definition  \ref{averagevelocitydefn}.)  We can likewise re-cast this definition.  After relabeling $t = t_{0}+ \Delta t$, we get

\[ \overline{v}(\Delta t) = \dfrac{\Delta [s(t)]}{\Delta t} = \dfrac{s(t_{0} + \Delta t) - s(t_{0})}{\Delta t}, \quad \Delta t \neq 0, \]

which measures the average velocity between time $t_{0}$ and time $t_{0} + \Delta t$ as a function of $\Delta t$.

Note that, regardless of which form the difference quotient takes, when $h$,  $\Delta x$, or  $\Delta t$ is $0$, the difference quotient returns the indeterminant form `$\frac{0}{0}$.' As we've seen with rational functions in Section \ref{IntroRational}, when this happens, we can reduce the fraction to lowest terms to see if we have a vertical asymptote or hole in the graph.  With this in mind,  when we speak of `simplifying the difference quotient,' we mean to manipulate the expression until the factor of `$h$' or `$\Delta x$' cancels out from the denominator. 

Our next example invites us to simplify three difference quotients, each cast slightly differently.  In each case, the bulk of the work involves Intermediate Algebra. We refer the reader to Sections \ref{AppRatExpEqus} and \ref{AppRadEqus} for additional review, if needed.

\begin{ex}  \label{differencequotientex} Find and simplify the indicated difference quotients for the following functions:

\begin{enumerate}

\item  For $f(x) = x^2-x-2$, find and simplify:   

\begin{multicols}{2}

\begin{enumerate}

\item  $\dfrac{f(3+h)-f(3)}{h}$ 

\item  $\dfrac{f(x+h)-f(x)}{h}$.

\end{enumerate}

\end{multicols}

\item  For $g(x) = \dfrac{3}{2x+1}$, find and simplify:

\begin{multicols}{2}

\begin{enumerate}

\item  $\dfrac{g(\Delta x)-g(0)}{\Delta x}$ 
 
\item  $\dfrac{g(x+\Delta x)-g(x)}{\Delta x}$.

\end{enumerate}

\end{multicols}


\item  $r(t) = \sqrt{t}$,  find and simplify:  


\begin{multicols}{2}

\begin{enumerate}

\item $\dfrac{r(9+\Delta t)-r(9)}{\Delta t}$ 

 \item $\dfrac{r(t+\Delta t)- r(t)}{\Delta t}$.
 
 \end{enumerate}

\end{multicols}


\end{enumerate}

{\bf Solution.}
 
\begin{enumerate}

\item \begin{enumerate} \item For our first difference quotient, we find $f(3+h)$ by substituting the quantity $(3+h)$ in for $x$: 

\[ \begin{array}{rclr}  
  f(3+h) & = & (3+h)^2 - (3+h) -2 & \\ 
  & = & 9 + 6h+h^2 - 3 - h -2 & \\
 & = & 4 + 5h + h^2 & \\
 \end{array} \]

Since $f(3) = (3)^2-3-2 = 4$, we get for the difference quotient:

\begin{align*}
\dfrac{f(3+h)-f(3)}{h} & = \dfrac{(4 + 5h + h^2) -4}{h} \\
                                & = \dfrac{5h+h^2}{h} \\
                                & = \dfrac{h(5+h)}{h} \tag{factor} \\
                                & = \dfrac{\cancel{h}(5+h)}{\cancel{h}} \tag{cancel} \\
                                & = 5+h
\end{align*}				


\item For the second difference quotient, we first find $f(x+h)$, we replace every occurrence of $x$ in the formula $f(x) = x^2-x-2$ with the quantity $(x+h)$ to get \[ \begin{array}{rclr}  
 
 f(x+h) & = & (x+h)^2 - (x+h) -2 & \\ [8pt]
 & = & x^2 + 2xh + h^2 - x - h - 2.
 \end{array} \]

So the difference quotient is

\begin{align*}
\dfrac{f(x+h)-f(x)}{h} & = \dfrac{\left(x^2+2xh+h^2-x-h-2 \right)-\left(x^{2}-x-2 \right)}{h} \\
& = \dfrac{x^2+2xh+h^2-x-h-2-x^2+x+2}{h} \\
& = \dfrac{2xh+h^2-h}{h} \\ 
	& = \dfrac{h \left(2x+h-1\right)}{h} \tag{factor} \\
	& = \dfrac{\cancel{h} \left(2x+h-1\right)}{\cancel{h}} \tag{cancel} \\
& = 2x+h-1
\end{align*} 

Note if we substitute $x=3$ into this expression, we obtain $5+h$ which agrees with our answer from the first difference quotient.

\end{enumerate}

\item 

\begin{enumerate}

\item Rewriting $\Delta x = 0 + \Delta x$, we see the first expression really is a difference quotient:

\[ \dfrac{g(\Delta x)-g(0)}{\Delta x} = \dfrac{g(0+\Delta x)-g(0)}{\Delta x}.\]

Since $g(\Delta x) = \frac{3}{2 \Delta x + 1}$ and $g(0) = \frac{3}{2(0)+1} = 3$, our difference quotient is:

\begin{longtable}{rclr}  

$\dfrac{g(0+\Delta x)-g(0)}{\Delta x}$ & = & $\dfrac{\dfrac{3}{2\Delta x+1}-3}{\Delta x}$ & \\ [10pt]
& = &  $\dfrac{\dfrac{3}{2\Delta x+1}-3}{\Delta x} \cdot \dfrac{(2\Delta x+1)}{(2\Delta x+1)}$ & \\ [10pt]
& = &  $\dfrac{3-3(2\Delta x+1)}{\Delta x(2\Delta x+1)}$  & \\ [10pt]
& = &  $\dfrac{3 - 6 \Delta x - 3}{\Delta x(2\Delta x+1)}$  & \\ [10pt]
& = &  $\dfrac{-6\Delta x}{\Delta x(2\Delta x+1)}$  & \\ [10pt]
& = &  $\dfrac{-6\cancel{\Delta x}}{\cancel{\Delta x}(2\Delta x+1)}$  & \\ [10pt]
& = &  $\dfrac{-6}{2\Delta x+1}$.  & \\ 

\end{longtable}

\item For our next difference quotient, we first find $g(x+\Delta x)$ by replacing every occurrence of $x$ in the formula for $g(x)$ with the quantity $(x+\Delta x)$:

 \[ \begin{array}{rclr}  
 g(x+\Delta x) & = & \dfrac{3}{2(x+\Delta x)+1} & \\
 & = & \dfrac{3}{2x+2\Delta x+1}.
 \end{array} \]

Hence,

\begin{longtable}{rclr}  

$\dfrac{g(x+\Delta x)-g(x)}{\Delta x}$ & = & $\dfrac{\dfrac{3}{2x+2\Delta x+1}-\dfrac{3}{2x+1}}{\Delta x}$ & \\ [10pt]
& = &  $\dfrac{\dfrac{3}{2x+2\Delta x+1}-\dfrac{3}{2x+1}}{\Delta x} \cdot \dfrac{(2x+2\Delta x+1)(2x+1)}{(2x+2\Delta x+1)(2x+1)}$ & \\ [10pt]
& = &  $\dfrac{3(2x+1)-3(2x+2\Delta x+1)}{\Delta x(2x+2\Delta x+1)(2x+1)}$  & \\ [10pt]
& = &  $\dfrac{6x+3-6x-6\Delta x-3}{\Delta x(2x+2\Delta x+1)(2x+1)}$  & \\ [10pt]
& = &  $\dfrac{-6\Delta x}{\Delta x(2x+2\Delta x+1)(2x+1)}$  & \\ [10pt]
& = &  $\dfrac{-6\cancel{\Delta x}}{\cancel{\Delta x}(2x+2\Delta x+1)(2x+1)}$  & \\ [10pt]
& = &  $\dfrac{-6}{(2x+2\Delta x+1)(2x+1)}$.  & \\ 

\end{longtable}

Since we have managed to cancel the factor `$\Delta x$' from the denominator, we are done.  Substituting $x=0$ into our final expression gives $\frac{-6}{2 \Delta x +1}$ thus checking our previous answer.

\end{enumerate}

\item 

\begin{enumerate}


\item We start with $r(9+\Delta t) = \sqrt{9+\Delta t}$ and $r(9) = \sqrt{9} = 3$ and get:

\[ \dfrac{r(9+\Delta t)-r(9)}{\Delta t} = \dfrac{\sqrt{9+\Delta t} - 3}{\Delta t}.\]


In order to cancel the factor `$\Delta t$' from the \textit{denominator}, we set about rationalizing the \textit{numerator} by multiplying both numerator and denominator by the conjugate of the numerator, $\sqrt{9+\Delta t} - 3$:

\begin{align*}
\dfrac{r(9+\Delta t) - r(9)}{\Delta t} & = \dfrac{\sqrt{9+\Delta t} - 3}{\Delta t} \\
& = \dfrac{\left(\sqrt{9+\Delta t} - 3 \right)}{\Delta t} \cdot \dfrac{\left(\sqrt{9+\Delta t} + 3\right)}{\left(\sqrt{9+\Delta t} + 3\right)} \tag{Multiply by the conjugate.} \\
& = \dfrac{\left(\sqrt{9+\Delta t}\right)^2 -(3)^2}{\Delta t\left(\sqrt{9+\Delta t} + 3\right)} \tag{Difference of Squares.}\\
& = \dfrac{(9+\Delta t) - 9}{\Delta t\left(\sqrt{9+\Delta t} + 3\right)} \\
& = \dfrac{\Delta t}{\Delta t\left(\sqrt{9+\Delta t} + 3\right)} \\
& = \dfrac{\cancelto{1}{\Delta t}}{\cancel{\Delta t}\left(\sqrt{9+\Delta t} + 3\right)} \\
& = \dfrac{1}{\sqrt{9+\Delta t} +3} \\ 
\end{align*}

\item As one might expect, we use this same strategy to simplify our final  different quotient. We have:

\begin{align*}
\dfrac{r(t+\Delta t) - r(t)}{\Delta t} & = \dfrac{\sqrt{t+\Delta t} - \sqrt{t}}{\Delta t} \\
& = \dfrac{\left(\sqrt{t+\Delta t} - \sqrt{t}\right)}{\Delta t} \cdot \dfrac{\left(\sqrt{t+\Delta t} + \sqrt{t}\right)}{\left(\sqrt{t+\Delta t} + \sqrt{t}\right)} \tag{Multiply by the conjugate.} \\
& = \dfrac{\left(\sqrt{t+\Delta t}\right)^2 - \left(\sqrt{t}\right)^2}{\Delta t\left(\sqrt{t+\Delta t} + \sqrt{t}\right)} \tag{Difference of Squares.}\\
& = \dfrac{(t+\Delta t) - t}{\Delta t\left(\sqrt{t+\Delta t} + \sqrt{t}\right)} \\
& = \dfrac{\Delta t}{\Delta t\left(\sqrt{t+\Delta t} + \sqrt{t}\right)} \\
& = \dfrac{\cancelto{1}{\Delta t}}{\cancel{\Delta t}\left(\sqrt{t+\Delta t} + \sqrt{t}\right)} \\
& = \dfrac{1}{\sqrt{t+\Delta t} + \sqrt{t}} \\ 
\end{align*}

Since we have canceled the original `$\Delta t$' factor from the denominator, we are done.  Setting $t=9$ in this expression, we get $\frac{1}{\sqrt{9+\Delta t} +3}$ which agrees with our previous answer. \qed

\end{enumerate}

\end{enumerate}

\end{ex}

We close this section revisiting the situation in Example \ref{averagevelocityrocketex}. 


\begin{ex} \label{averagevelocityrocketexreprise} Let $s(t) = -5t^2+100t$, $0 \leq t \leq 20$ give the height of a model rocket above the Moon's surface, in feet,  $t$ seconds after liftoff.  

\begin{enumerate}

\item  Find, and simplify:  $\overline{v}(\Delta t)  = \dfrac{s(15+ \Delta t) - s(15)}{\Delta t}$, for $\Delta t \neq 0$.

\item  Find and interpret $\overline{v}(-1)$.

\item  Graph $y = \overline{v}(t)$.

\item  Describe the behavior of $\overline{v}$ as $\Delta t \rightarrow 0$ and interpret.

\end{enumerate}

\newpage

{\bf Solution.}

\begin{enumerate}

\item  To find $\overline{v}(\Delta t)$, we first find $s(15+\Delta t)$: 

\[ \begin{array}{rclr}  
  s(15+\Delta t) & = & -5(15+\Delta t)^2 + 100(15+\Delta t) & \\ 
  & = & -5(225+30 \Delta t + (\Delta t)^2) + 1500 + 100 \Delta t& \\
 & = & -5(\Delta t)^2 -50 \Delta t +375 & \\
 \end{array} \]

Since $s(15) = -5(15)^2 + 100(15) = 375$, we get:

\begin{align*}
\overline{v}(\Delta t)& = \dfrac{s(15+ \Delta t) - s(15)}{\Delta t} \\
& = \dfrac{(-5(\Delta t)^2-50 \Delta t + 375) - 375}{\Delta t} \\
& = \dfrac{\Delta t (-5 \Delta t - 50)}{\Delta t} \\
& = \dfrac{\cancel{\Delta} t (-5 \Delta t - 50)}{\cancel{\Delta t}} \\
& = -5 \Delta t - 50 \tag{$\Delta t \neq 0$} \\
\end{align*}

In addition to the restriction $\Delta t \neq 0$, we also know the domain of $s$ is $0 \leq t \leq 20$.  Hence, we also require  $0 \leq 15 + \Delta t \leq 20$ or  $-15 \leq \Delta \leq 5$.  Our final answer is $\overline{v}(\Delta t) = -5 \Delta t - 50$, for $\Delta t \in [-15, 0) \cup (0, 5]$

\item  We find  $\overline{v}(-1) = -5(-1) - 50 = -45$.  This means the average velocity over between time $t=15+(-1) = 14$ seconds and $t=15$ seconds is $-45$ feet per second.  This indicates the rocket is, on average, heading \textit{downwards} at a rate of $45$ feet per second.

\item  The graph of $y =  -5 \Delta t - 50$ is a line with slope $-5$ and $y$-intercept $(0, -50)$.  However, since the domain of $\overline{v}$ is $[-15, 0) \cup (0, 5]$, we the graph of $\overline{v}$ is a line \textit{segment} from $(-15, 25)$ to $(5, -75)$ with a hole at $(0, -50)$. See \autoref{fig:yeqvbardeltat}

\begin{figure}
\begin{center}

\begin{mfpic}[15]{-4}{2}{-8}{3}
\axes
\axismarks{x}{-4, -3, -2, -1,1}
\axismarks{y}{-7 step 1 until 2}
\scriptsize
\tlabel[cc](2, -0.5){$\Delta t$}
\tlabel[cc](0.5, 3){$y$}
\tlabel[cc](-4.5, 2.5){$(-15, 25)$}
\tlabel[cc](2.25, -7.5){$(5,-75)$}
\tlabel[cc](1.25, -5){$(0,-50)$}
\normalsize
\penwd{1.25pt}
\polyline{(-3,  2.5), (1, -7.5)}
\point[4pt]{(-3, 2.5), (1, -7.5)}
\pointfillfalse
\point[4pt]{(0, -5)}
\tcaption{}
\end{mfpic}

\caption{$y=\overline{v}(\Delta t)$}
\label{fig:yeqvbardeltat}
\end{center}
\end{figure}

\item  As $\Delta t \rightarrow 0$, $\overline{v}(\Delta t) \rightarrow -50$ meaning as we approach $t=15$, the velocity of the rocket approaches $-50$ feet per second.  Recall from Example \ref{averagevelocityrocketex} that this is the so-called \textit{instantaneous velocity} of the rocket \textit{at} $t=15$ seconds.  That is, $15$ seconds after lift-off, the rocket is heading back towards the surface of the moon at a rate of $15$ feet per second. \qed

\end{enumerate}

\end{ex}

The reader is invited to compare Example \ref{averagevelocityrocketex} in Section \ref{IntroRational} with Exercise \ref{averagevelocityrocketexreprise} above. We obtain the \text{exact same} information because we are asking the \textit{exact same} questions - they are just framed differently.


\clearpage

\subsection{Exercises}

\startexenum

\label{ExercisesforFunctionArithmetic}

In Exercises \ref{basicarithonefirst} - \ref{basicarithonelast}, use the pair of functions $f$ and $g$ to find the following values if they exist.

\begin{multicols}{3}
\begin{itemize}

\item  $(f+g)(2)$ 
\item  $(f-g)(-1)$
\item  $(g-f)(1)$

\end{itemize}
\end{multicols}

\begin{multicols}{3}
\begin{itemize}

\item  $(fg)\left(\frac{1}{2}\right)$
\item  $\left(\frac{f}{g}\right)(0)$
\item  $\left(\frac{g}{f}\right)\left(-2\right)$

\end{itemize}
\end{multicols}


\begin{exenum}
\item  $f(x) = 3x+1$ and  $g(t) = 4-t$ \label{basicarithonefirst}
\item  $f(x) = x^2$ and $g(t) = -2t+1$
\item  $f(x) = x^2 - x$ and  $g(t) = 12-t^2$
\item  $f(x) = 2x^3$ and $g(t) = -t^2-2t-3$
\item  $f(x) = \sqrt{x+3}$ and  $g(t) = 2t-1$
\item  $f(x) = \sqrt{4-x}$ and $g(t) = \sqrt{t+2}$
\item  $f(x) = 2x$ and  $g(t) = \dfrac{1}{2t+1}$
\item  $f(x) = x^2$ and $g(t) = \dfrac{3}{2t-3}$
\item  $f(x) = x^2$ and  $g(t) = \dfrac{1}{t^2}$
\item  $f(x) = x^2+1$ and $g(t) = \dfrac{1}{t^2+1}$ \label{basicarithonelast}
\end{exenum}

Exercises \ref{arithfromgraphfirst} - \ref{arithfromgraphlast} refer to the functions $f$ and $g$ whose graphs are given in \autoref{fig:yeqfxexeleventotwenty} and \autoref{fig:yeqgtexeleventotwenty} respectively. 

\begin{figure}

\begin{minipage}{0.5\textwidth}
\begin{center}

\begin{mfpic}[15]{-5}{5}{-5}{5}
\tlabel[cc](-3,0.5){\small $\left( -2, 0 \right)$}
\tlabel[cc](2.5,0.5){\small $\left(2, 0 \right)$}
\tlabel[cc](4,-3.5){\small $\left( 4, -3 \right)$}
\tlabel[cc](-4,-3.5){\small $\left(-4, -3 \right)$}
\tlabel[cc](1,3.5){\small $\left(0, 3 \right)$}
\axes
\tlabel[cc](5,-0.5){\scriptsize $x$}
\tlabel[cc](0.5,5){\scriptsize $y$}
\xmarks{-4,-3,-2,-1,1,2,3,4}
\ymarks{-4,-3,-2,-1,1,2,3,4}
\tlpointsep{5pt}
\scriptsize
\axislabels {x}{{$-4 \hspace{7pt}$} -4, {$-3 \hspace{7pt}$} -3, {$-2 \hspace{7pt}$} -2, {$-1 \hspace{7pt}$} -1, {$1$} 1, {$2$} 2, {$3$} 3, {$4$} 4}
\axislabels {y}{{$-4$} -4, {$-3$} -3, {$-2$} -2, {$-1$} -1, {$1$} 1, {$2$} 2, {$4$} 4}
\normalsize
\point[4pt]{(-2,0), (2,0), (4,-3), (-4,-3), (0,3)}
\penwd{1.25pt}
\function{-4,4,.1}{3*cos(3.14159265*x/4)}
\end{mfpic}

\caption{$y = f(x)$}
\label{fig:yeqfxexeleventotwenty}
\end{center}
\end{minipage}
\begin{minipage}{0.5\textwidth}
\begin{center}

\begin{mfpic}[15]{-5}{5}{-5}{5}
\tlabel[cc](-4,-2.5){\small $\left( -4, -2 \right)$}
\tlabel[cc](-2,0.5){\small $\left(-1, 0 \right)$}
\tlabel[cc](-1,2){\small $\left( 0,2 \right)$}
\axes
\tlabel[cc](5,-0.5){\scriptsize $t$}
\tlabel[cc](0.5,5){\scriptsize $y$}
\xmarks{-4,-3,-2,-1,1,2,3,4}
\ymarks{-4,-3,-2,-1,1,2,3,4}
\tlpointsep{5pt}
\scriptsize
\axislabels {x}{{$-4 \hspace{7pt}$} -4, {$-3 \hspace{7pt}$} -3, {$-2 \hspace{7pt}$} -2, {$-1 \hspace{7pt}$} -1, {$1$} 1, {$2$} 2, {$3$} 3, {$4$} 4}
\axislabels {y}{{$-4$} -4, {$-3$} -3, {$-2$} -2, {$-1$} -1, {$1$} 1, {$4$} 4, {$3$} 3}
\normalsize
\point[4pt]{(-4,-2), (-1,0), (0,2)}
\penwd{1.25pt}
\polyline{(-4,-2), (-1,0), (0,2)}
\arrow \polyline{(0,2), (5,2)}
\tcaption{}
\end{mfpic}

\caption{$y = g(t)$}
\label{fig:yeqgtexeleventotwenty}
\end{center}
\end{minipage}

\end{figure}

\begin{multicols}{3}
\begin{enumerate}
\setcounter{enumi}{\value{HW}}

\item $(f + g)(-4)$ \label{arithfromgraphfirst}
\item $(f + g)(0)$
\item $(f- g)(4)$

\setcounter{HW}{\value{enumi}}
\end{enumerate}
\end{multicols}


\begin{multicols}{3}
\begin{enumerate}
\setcounter{enumi}{\value{HW}}

\item $(fg)(-4)$ 
\item $(fg)(-2)$
\item $(fg)(4)$

\setcounter{HW}{\value{enumi}}
\end{enumerate}
\end{multicols}

\enlargethispage{0.5in}

\begin{multicols}{3}
\begin{enumerate}
\setcounter{enumi}{\value{HW}}

\item $\left(\dfrac{f}{g}\right)(0)$
\item $\left(\dfrac{f}{g}\right)(2)$
\item $\left(\dfrac{g}{f}\right)(-1)$ 

\setcounter{HW}{\value{enumi}}
\end{enumerate}
\end{multicols}

\begin{enumerate}
\setcounter{enumi}{\value{HW}}

\item Find the domains of $f+g$, $f-g$,  $fg$, $\dfrac{f}{g}$ and $\dfrac{g}{f}$.  \label{arithfromgraphlast}

\setcounter{HW}{\value{enumi}}
\end{enumerate}

In Exercises \ref{reformarithfirst} - \ref{reformarithlast}, let $f$ be the function defined by \[f = \{(-3, 4), (-2, 2), (-1, 0), (0, 1), (1, 3), (2, 4), (3, -1)\}\] and let $g$ be the function defined by \[g = \{(-3, -2), (-2, 0), (-1, -4), (0, 0), (1, -3), (2, 1), (3, 2)\}\] Compute the indicated value if it exists.


\begin{multicols}{3}
\begin{enumerate}
\setcounter{enumi}{\value{HW}}

\item $(f + g)(-3)$ \label{reformarithfirst}
\item $(f - g)(2)$
\item $(fg)(-1)$

\setcounter{HW}{\value{enumi}}
\end{enumerate}
\end{multicols}

\begin{multicols}{3}
\begin{enumerate}
\setcounter{enumi}{\value{HW}}

\item $(g + f)(1)$
\item $(g - f)(3)$
\item $(gf)(-3)$

\setcounter{HW}{\value{enumi}}
\end{enumerate}
\end{multicols}

\begin{multicols}{3}
\begin{enumerate}
\setcounter{enumi}{\value{HW}}

\item $\left(\frac{f}{g}\right)(-2)$
\item $\left(\frac{f}{g}\right)(-1)$
\item $\left(\frac{f}{g}\right)(2)$

\setcounter{HW}{\value{enumi}}
\end{enumerate}
\end{multicols}

\begin{multicols}{3}
\begin{enumerate}
\setcounter{enumi}{\value{HW}}

\item $\left(\frac{g}{f}\right)(-1)$
\item $\left(\frac{g}{f}\right)(3)$
\item $\left(\frac{g}{f}\right)(-3)$ \label{reformarithlast}

\setcounter{HW}{\value{enumi}}
\end{enumerate}
\end{multicols}

In Exercises \ref{basicarithtwofirst} - \ref{basicarithtwolast}, use the pair of functions $f$ and $g$ to find the domain of the indicated function then find and simplify an expression for it.

\begin{multicols}{4}
\begin{itemize}

\item  $(f+g)(x)$
\item  $(f-g)(x)$
\item  $(fg)(x)$
\item  $\left(\frac{f}{g}\right)(x)$

\end{itemize}
\end{multicols}

\begin{exenum}
\item $f(x) = 2x+1$ and $g(x) = x-2$ \label{basicarithtwofirst}
\item $f(x) = 1-4x$ and $g(x) = 2x-1$
\item $f(x) = x^2$ and $g(x) = 3x-1$
\item $f(x) = x^2-x$ and $g(x) = 7x$
\item $f(x) = x^2-4$ and $g(x) = 3x+6$
\item $f(x) = -x^2+x+6$ and $g(x) = x^2-9$
\item $f(x) = \dfrac{x}{2}$ and $g(x) = \dfrac{2}{x}$
\item $f(x) =x-1$ and $g(x) = \dfrac{1}{x-1}$
\item $f(x) = x$ and $g(x) = \sqrt{x+1}$
\item $f(x) =\sqrt{x-5}$ and $g(x) = f(x) = \sqrt{x-5}$ \label{basicarithtwolast}
\end{exenum}

In Exercises \ref{decomposebasicfirst} - \ref{decomposebasiclast}, write the given function as a nontrivial decomposition of functions as directed.

\begin{enumerate}
\setcounter{enumi}{\value{HW}}

\item  For $p(z) = 4z-z^3$, find functions $f$ and $g$ so that $p=f-g$. \label{decomposebasicfirst}
\item  For $p(z) = 4z-z^3$, find functions $f$ and $g$ so that $p=f+g$.
\item  For $g(t) = 3t|2t-1|$, find functions $f$ and $h$  so that $g = fh$.
\item  For $r(x) = \dfrac{3-x}{x+1}$, find functions $f$ and $g$ so $r = \dfrac{f}{g}$.
\item  For $r(x) = \dfrac{3-x}{x+1}$, find functions $f$ and $g$ so $r = fg$. \label{decomposebasiclast}

\setcounter{HW}{\value{enumi}}
\end{enumerate}

\begin{enumerate}
\setcounter{enumi}{\value{HW}}

\item    Can $f(x) = x$ be decomposed as $f = g-h$ where $g(x) = x+\dfrac{1}{x}$ and $h(x) = \dfrac{1}{x}$?

\item   Discuss with your classmates how to phrase the quantities revenue and profit in Definition \ref{revenueprofitdefns} terms of function arithmetic as defined in Definition \ref{functionarithmeticdefn}.
 
\setcounter{HW}{\value{enumi}}
\end{enumerate}

In Exercises \ref{diffquotexerfirsta} - \ref{diffquotexerlasta}, find and simplify the difference quotients:

\begin{multicols}{2}

\begin{itemize}

\item  $\dfrac{f(2+h) - f(2)}{h}$

\item  $\dfrac{f(x+h) - f(x)}{h}$

\end{itemize}

\end{multicols}


\begin{multicols}{2}

\begin{enumerate}
\setcounter{enumi}{\value{HW}}

\item $f(x) = 2x - 5$ \label{diffquotexerfirsta}
\item $f(x) = -3x + 5$

\setcounter{HW}{\value{enumi}}
\end{enumerate}
\end{multicols}

\begin{multicols}{2}
\begin{enumerate}
\setcounter{enumi}{\value{HW}}

\item $f(x) = 6$
\item $f(x) = 3x^2 - x$

\setcounter{HW}{\value{enumi}}
\end{enumerate}
\end{multicols}

\begin{multicols}{2}
\begin{enumerate}
\setcounter{enumi}{\value{HW}}

\item $f(x) = -x^2 + 2x - 1$
\item  $f(x) = 4x^2$ 

\setcounter{HW}{\value{enumi}}
\end{enumerate}
\end{multicols}

\begin{multicols}{2}
\begin{enumerate}
\setcounter{enumi}{\value{HW}}

\item  $f(x) = x-x^2$ 
\item $f(x) = x^{3} + 1$

\setcounter{HW}{\value{enumi}}
\end{enumerate}
\end{multicols}

\begin{multicols}{2}
\begin{enumerate}
\setcounter{enumi}{\value{HW}}

\item $f(x) = mx + b\;$ where $m \neq 0$
\item $f(x) = ax^{2} + bx + c\;$ where $a \neq 0$  \label{diffquotexerlasta}

\setcounter{HW}{\value{enumi}}
\end{enumerate}
\end{multicols}


In Exercises \ref{diffquotexerfirstb} - \ref{diffquotexerlastb}, find and simplify the difference quotients:

\begin{multicols}{2}

\begin{itemize}

\item  $\dfrac{f(-1+\Delta x) - f(-1)}{\Delta x}$

\item  $\dfrac{f(x+\Delta x) - f(x)}{\Delta x}$

\end{itemize}

\end{multicols}

\begin{multicols}{2}
\begin{enumerate}
\setcounter{enumi}{\value{HW}}

\item $f(x) = \dfrac{2}{x}$  \label{diffquotexerfirstb}
\item $f(x) = \dfrac{3}{1-x}$

\setcounter{HW}{\value{enumi}}
\end{enumerate}
\end{multicols}

\begin{multicols}{2}
\begin{enumerate}
\setcounter{enumi}{\value{HW}}

\item  $f(x) = \dfrac{1}{x^2}$
\item  $f(x) = \dfrac{2}{x+5}$

\setcounter{HW}{\value{enumi}}
\end{enumerate}
\end{multicols}

\begin{multicols}{2}
\begin{enumerate}
\setcounter{enumi}{\value{HW}}

\item $f(x) = \dfrac{1}{4x-3}$ 
\item $f(x) = \dfrac{3x}{x+2}$ 

\setcounter{HW}{\value{enumi}}
\end{enumerate}
\end{multicols}

\begin{multicols}{2}
\begin{enumerate}
\setcounter{enumi}{\value{HW}}

\item $f(x) = \dfrac{x}{x - 9}$
\item $f(x) = \dfrac{x^2}{2x+1}$  \label{diffquotexerlastb}

\setcounter{HW}{\value{enumi}}
\end{enumerate}
\end{multicols}

In Exercises \ref{diffquotexerfirstc} - \ref{diffquotexerlastc}, find and simplify the difference quotients:

\begin{multicols}{2}

\begin{itemize}

\item  $\dfrac{g(\Delta t) - g(0)}{\Delta t}$

\item  $\dfrac{g(t+\Delta t) - g(t)}{\Delta t}$

\end{itemize}

\end{multicols}


\begin{multicols}{2}
\begin{enumerate}
\setcounter{enumi}{\value{HW}}

\item  $g(t) = \sqrt{9-t}$  \label{diffquotexerfirstc}
\item  $g(t) = \sqrt{2t+1}$

\setcounter{HW}{\value{enumi}}
\end{enumerate}
\end{multicols}

\begin{multicols}{2}
\begin{enumerate}
\setcounter{enumi}{\value{HW}}

\item  $g(t) = \sqrt{-4t+5}$
\item  $g(t) = \sqrt{4-t}$

\setcounter{HW}{\value{enumi}}
\end{enumerate}
\end{multicols}

\begin{multicols}{2}
\begin{enumerate}
\setcounter{enumi}{\value{HW}}

\item  $g(t) = \sqrt{at+b}$, where $a \neq 0$.
\item  $g(t) = t \sqrt{t}$ 

\setcounter{HW}{\value{enumi}}
\end{enumerate}
\end{multicols}

\begin{enumerate}
\setcounter{enumi}{\value{HW}}

\item  $g(t) = \sqrt[3]{t}$.  \textbf{HINT:}  $(a-b)\left(a^2+ab+b^2\right) = a^3 - b^3$  \label{diffquotexerlastc}
\setcounter{HW}{\value{enumi}}
\end{enumerate}


\begin{enumerate}
\setcounter{enumi}{\value{HW}}

\item \label{posnegdecompexercise}  In this exercise, we explore decomposing a function into its positive and negative parts.  Given a function $f$, we define the \index{positive part of a function}\textbf{positive part} of $f$, denoted $f_{+}$ and \index{negative part of a function}\textbf{negative part} of $f$, denoted $f_{-}$ by:

\[ f_{+}(x) = \dfrac{f(x) + |f(x)|}{2}, \qquad \text{and} \qquad f_{-}(x) = \dfrac{f(x) - |f(x)|}{2}. \]

\begin{enumerate}

\item Using a graphing utility, graph each of the functions $f$ below along with $f_{+}$ and $f_{-}$.

\begin{shortitemize}
\item  $f(x) = x-3$
\item  $f(x) = x^2-x-6$
\item  $f(x) = 4x-x^3$
\end{shortitemize}

Why is $f_{+}$ called the `positive part' of $f$ and $f_{-}$ called the `negative part' of $f$?

\item Show that $f = f_{+} + f_{-}$.

\item Use Definition \ref{absolutevaluepiecewise} to rewrite the expressions for $f_{+}(x)$ and $f_{-}(x)$ as piecewise defined functions.

\end{enumerate}  

\setcounter{HW}{\value{enumi}}
\end{enumerate}

\clearpage

\subsection{Answers}

\startexenum

\begin{enumerate}

\item For  $f(x) = 3x+1$ and $g(x) = 4-x$

\begin{shortitemize}[MMMMMMMMMM]
\item  $(f+g)(2) = 9$
\item  $(f-g)(-1) = -7$
\item  $(g-f)(1) = -1$
\item  $(fg)\left(\frac{1}{2}\right) = \frac{35}{4}$
\item  $\left(\frac{f}{g}\right)(0) = \frac{1}{4}$
\item  $\left(\frac{g}{f}\right)\left(-2\right) = -\frac{6}{5}$
\end{shortitemize}

\item For  $f(x) = x^2$ and $g(x) = -2x+1$

\begin{shortitemize}[MMMMMMMMMM]
\item  $(f+g)(2) = 1$
\item  $(f-g)(-1) = -2$
\item  $(g-f)(1) = -2$
\item  $(fg)\left(\frac{1}{2}\right) = 0$
\item  $\left(\frac{f}{g}\right)(0) = 0$
\item  $\left(\frac{g}{f}\right)\left(-2\right) = \frac{5}{4}$
\end{shortitemize}

\item For  $f(x) = x^2 - x$ and  $g(x) = 12-x^2$

\begin{shortitemize}[MMMMMMMMMM]
\item  $(f+g)(2) = 10$
\item  $(f-g)(-1) = -9$
\item  $(g-f)(1) = 11$
\item  $(fg)\left(\frac{1}{2}\right) = -\frac{47}{16}$
\item  $\left(\frac{f}{g}\right)(0) = 0$
\item  $\left(\frac{g}{f}\right)\left(-2\right) = \frac{4}{3}$
\end{shortitemize}

\item For $f(x) = 2x^3$ and  $g(x) = -x^2-2x-3$

\begin{multicols}{3}
\begin{itemize}

\item  $(f+g)(2) = 5$
\item  $(f-g)(-1) = 0$
\item  $(g-f)(1) = -8$

\end{itemize}
\end{multicols}

\begin{multicols}{3}
\begin{itemize}

\item  $(fg)\left(\frac{1}{2}\right) = -\frac{17}{16}$
\item  $\left(\frac{f}{g}\right)(0) = 0$
\item  $\left(\frac{g}{f}\right)\left(-2\right) = \frac{3}{16}$

\end{itemize}
\end{multicols}

\item For $f(x) = \sqrt{x+3}$ and  $g(x) = 2x-1$

\begin{shortitemize}[MMMMMMM]
\item  $(f+g)(2) = 3+\sqrt{5}$
\item  $(f-g)(-1) = 3+\sqrt{2}$
\item  $(g-f)(1) = -1$
\item  $(fg)\left(\frac{1}{2}\right) = 0$
\item  $\left(\frac{f}{g}\right)(0) = -\sqrt{3}$
\item  $\left(\frac{g}{f}\right)\left(-2\right) = -5$
\end{shortitemize}

\item For $f(x) = \sqrt{4-x}$ and $g(x) = \sqrt{x+2}$

\begin{shortitemize}[MMMMM]
\item  $(f+g)(2) = 2+\sqrt{2}$
\item  $(f-g)(-1) = -1+\sqrt{5}$
\item  $(g-f)(1) = 0$
\item  $(fg)\left(\frac{1}{2}\right) = \frac{\sqrt{35}}{2}$
\item  $\left(\frac{f}{g}\right)(0) = \sqrt{2}$
\item  $\left(\frac{g}{f}\right)\left(-2\right) = 0$
\end{shortitemize}

\item For  $f(x) = 2x$ and  $g(x) = \frac{1}{2x+1}$

\begin{shortitemize}[MMMMMMMMMMM]
\item  $(f+g)(2) = \frac{21}{5}$
\item  $(f-g)(-1) = -1$
\item  $(g-f)(1) = -\frac{5}{3}$
\item  $(fg)\left(\frac{1}{2}\right) = \frac{1}{2}$
\item  $\left(\frac{f}{g}\right)(0) = 0$
\item  $\left(\frac{g}{f}\right)\left(-2\right) = \frac{1}{12}$
\end{shortitemize}

\item For  $f(x) = x^2$ and $g(x) = \frac{3}{2x-3}$

\begin{shortitemize}[MMMMMMMMMMM]
\item  $(f+g)(2) = 7$
\item  $(f-g)(-1) = \frac{8}{5}$
\item  $(g-f)(1) = -4$
\item  $(fg)\left(\frac{1}{2}\right) = -\frac{3}{8}$
\item  $\left(\frac{f}{g}\right)(0) = 0$
\item  $\left(\frac{g}{f}\right)\left(-2\right) = -\frac{3}{28}$
\end{shortitemize}

\item For  $f(x) = x^2$ and $g(x) = \frac{1}{x^2}$

\begin{shortitemize}[MMMMMMMMMMMM]
\item  $(f+g)(2) =\frac{17}{4}$
\item  $(f-g)(-1) = 0$
\item  $(g-f)(1) = 0$
\item  $(fg)\left(\frac{1}{2}\right) =1$
\item  $\left(\frac{f}{g}\right)(0)$ is undefined.
\item  $\left(\frac{g}{f}\right)\left(-2\right) = \frac{1}{16}$
\end{shortitemize}

\item For  $f(x) = x^2+1$ and $g(x) = \frac{1}{x^2+1}$

\begin{multicols}{3}
\begin{itemize}

\item  $(f+g)(2) =\frac{26}{5}$
\item  $(f-g)(-1) = \frac{3}{2}$
\item  $(g-f)(1) = -\frac{3}{2}$

\end{itemize}
\end{multicols}

\begin{multicols}{3}
\begin{itemize}

\item  $(fg)\left(\frac{1}{2}\right) =1$
\item  $\left(\frac{f}{g}\right)(0) = 1$
\item  $\left(\frac{g}{f}\right)\left(-2\right) = \frac{1}{25}$

\end{itemize}
\end{multicols}

\setcounter{HW}{\value{enumi}}
\end{enumerate}

\begin{multicols}{3}
\begin{enumerate}
\setcounter{enumi}{\value{HW}}

\item $(f + g)(-4) = -5$   
\item $(f + g)(0) = 5$
\item $(f-g)(4) = -5$

\setcounter{HW}{\value{enumi}}
\end{enumerate}
\end{multicols}


\begin{multicols}{3}
\begin{enumerate}
\setcounter{enumi}{\value{HW}}

\item $(fg)(-4) = 6$ 
\item $(fg)(-2) = 0$
\item $(fg)(4) = -6$

\setcounter{HW}{\value{enumi}}
\end{enumerate}
\end{multicols}

\enlargethispage{0.5in}

\begin{multicols}{3}
\begin{enumerate}
\setcounter{enumi}{\value{HW}}

\item $\left(\dfrac{f}{g}\right)(0) = \dfrac{3}{2}$
\item $\left(\dfrac{f}{g}\right)(2) =  0$
\item $\left(\dfrac{g}{f}\right)(-1) = 0$ 

\setcounter{HW}{\value{enumi}}
\end{enumerate}
\end{multicols}

\begin{enumerate}
\setcounter{enumi}{\value{HW}}

\item The domains of $f+g$, $f-g$ and $fg$ are all $[-4,4]$.  The domain of $\frac{f}{g}$ is $[-4, -1) \cup (-1,4]$ and the domain of $\frac{g}{f}$ is $[-4, -2) \cup (-2,2) \cup (2, 4]$.

\setcounter{HW}{\value{enumi}}
\end{enumerate}


\begin{multicols}{3}
\begin{enumerate}
\setcounter{enumi}{\value{HW}}

\item $(f + g)(-3) = 2$
\item $(f - g)(2) = 3$
\item $(fg)(-1) = 0$

\setcounter{HW}{\value{enumi}}
\end{enumerate}
\end{multicols}

\begin{multicols}{3}
\begin{enumerate}
\setcounter{enumi}{\value{HW}}

\item $(g + f)(1) = 0$
\item $(g - f)(3) = 3$
\item $(gf)(-3) = -8$

\setcounter{HW}{\value{enumi}}
\end{enumerate}
\end{multicols}

\begin{shortexenum}[MMMMMMMMMMMMMMM]
\item $\left(\frac{f}{g}\right)(-2)$ does not exist
\item $\left(\frac{f}{g}\right)(-1) = 0$
\item $\left(\frac{f}{g}\right)(2) = 4$
\item $\left(\frac{g}{f}\right)(-1)$ does not exist
\item $\left(\frac{g}{f}\right)(3) = -2$ 
\item $\left(\frac{g}{f}\right)(-3) = -\frac{1}{2}$ 
\end{shortexenum}

\begin{enumerate}
\setcounter{enumi}{\value{HW}}

\item For $f(x) = 2x+1$ and $g(x) = x-2$

\begin{multicols}{2}

\begin{itemize}

\item $(f+g)(x) = 3x-1$ \\
      Domain: $(-\infty, \infty)$
      
      \vfill
      
      \columnbreak
      
\item $(f-g)(x) = x+3$ \\
      Domain:  $(-\infty, \infty)$


\end{itemize}

\end{multicols}

\begin{multicols}{2}

\begin{itemize}

\item $(fg)(x) = 2x^2-3x-2$ \\
      Domain: $(-\infty, \infty)$
      
      \vfill
      
      \columnbreak
      
\item $\left(\frac{f}{g}\right)(x) = \frac{2x+1}{x-2}$ \\
      Domain:  $(-\infty, 2) \cup (2, \infty)$


\end{itemize}

\end{multicols}

\item For $f(x) = 1-4x$ and $g(x) = 2x-1$

\begin{multicols}{2}

\begin{itemize}

\item $(f+g)(x) = -2x$ \\
      Domain: $(-\infty, \infty)$
      
      \vfill
      
      \columnbreak
      
\item $(f-g)(x) = 2-6x$ \\
      Domain:  $(-\infty, \infty)$


\end{itemize}

\end{multicols}

\begin{multicols}{2}

\begin{itemize}

\item $(fg)(x) = -8x^2+6x-1$ \\
      Domain: $(-\infty, \infty)$
      
      \vfill
      
      \columnbreak
      
\item $\left(\frac{f}{g}\right)(x) = \frac{1-4x}{2x-1}$ \\
      Domain:  $\left(-\infty, \frac{1}{2} \right) \cup \left(\frac{1}{2}, \infty \right)$


\end{itemize}

\end{multicols}


\item For $f(x) = x^2$ and $g(x) = 3x-1$

\begin{multicols}{2}

\begin{itemize}

\item $(f+g)(x) = x^2+3x-1$ \\
      Domain: $(-\infty, \infty)$
      
      \vfill
      
      \columnbreak
      
\item $(f-g)(x) = x^2-3x+1$ \\
      Domain:  $(-\infty, \infty)$


\end{itemize}

\end{multicols}

\begin{multicols}{2}

\begin{itemize}

\item $(fg)(x) = 3x^3-x^2$ \\
      Domain: $(-\infty, \infty)$
      
      \vfill
      
      \columnbreak
      
\item $\left(\frac{f}{g}\right)(x) = \frac{x^2}{3x-1}$ \\
      Domain:  $\left(-\infty, \frac{1}{3} \right) \cup \left(\frac{1}{3}, \infty \right)$


\end{itemize}

\end{multicols}

\item For $f(x) = x^2-x$ and $g(x) = 7x$

\begin{multicols}{2}

\begin{itemize}

\item $(f+g)(x) = x^2+6x$ \\
      Domain: $(-\infty, \infty)$
      
      \vfill
      
      \columnbreak
      
\item $(f-g)(x) = x^2-8x$ \\
      Domain:  $(-\infty, \infty)$


\end{itemize}

\end{multicols}

\begin{multicols}{2}

\begin{itemize}

\item $(fg)(x) = 7x^3-7x^2$ \\
      Domain: $(-\infty, \infty)$
      
      \vfill
      
      \columnbreak
      
\item $\left(\frac{f}{g}\right)(x) = \frac{x-1}{7}$ \\
      Domain:  $\left(-\infty, 0 \right) \cup \left(0, \infty \right)$


\end{itemize}

\end{multicols}


\item For $f(x) = x^2-4$ and $g(x) = 3x+6$

\begin{multicols}{2}

\begin{itemize}

\item $(f+g)(x) = x^2+3x+2$ \\
      Domain: $(-\infty, \infty)$
      
      \vfill
      
      \columnbreak
      
\item $(f-g)(x) = x^2-3x-10$ \\
      Domain:  $(-\infty, \infty)$


\end{itemize}

\end{multicols}

\begin{multicols}{2}

\begin{itemize}

\item $(fg)(x) = 3x^3+6x^2-12x-24$ \\
      Domain: $(-\infty, \infty)$
      
      \vfill
      
      \columnbreak
      
\item $\left(\frac{f}{g}\right)(x) = \frac{x-2}{3}$ \\
      Domain:  $\left(-\infty, -2 \right) \cup \left(-2, \infty \right)$


\end{itemize}

\end{multicols}

\item For $f(x) = -x^2+x+6$ and $g(x) = x^2-9$

\begin{multicols}{2}

\begin{itemize}

\item $(f+g)(x) = x-3$ \\
      Domain: $(-\infty, \infty)$
      
      \vfill
      
      \columnbreak
      
\item $(f-g)(x) = -2x^2+x+15$ \\
      Domain:  $(-\infty, \infty)$


\end{itemize}

\end{multicols}

\begin{multicols}{2}

\begin{itemize}

\item $(fg)(x) = -x^4+x^3+15x^2-9x-54$ \\
      Domain: $(-\infty, \infty)$
      
      \vfill
      
      \columnbreak
      
\item $\left(\frac{f}{g}\right)(x) = -\frac{x+2}{x+3}$ \\
      Domain:  $\left(-\infty, -3 \right) \cup \left(-3, 3 \right) \cup (3, \infty)$


\end{itemize}

\end{multicols}


\item For  $f(x) = \frac{x}{2}$ and $g(x) = \frac{2}{x}$

\begin{multicols}{2}

\begin{itemize}

\item $(f+g)(x) = \frac{x^2+4}{2x}$ \\
      Domain: $(-\infty, 0) \cup (0, \infty)$
      
      \vfill
      
      \columnbreak
      
\item $(f-g)(x) = \frac{x^2-4}{2x}$ \\
      Domain:  $(-\infty,0) \cup (0, \infty)$


\end{itemize}

\end{multicols}

\begin{multicols}{2}

\begin{itemize}

\item $(fg)(x) = 1$ \\
      Domain: $(-\infty,0) \cup (0, \infty)$
      
      \vfill
      
      \columnbreak
      
\item $\left(\frac{f}{g}\right)(x) = \frac{x^2}{4}$ \\
      Domain: $(-\infty,0) \cup (0, \infty)$


\end{itemize}

\end{multicols}



\item For   $f(x) =x-1$ and $g(x) = \frac{1}{x-1}$

\begin{multicols}{2}

\begin{itemize}

\item $(f+g)(x) = \frac{x^2-2x+2}{x-1}$ \\
      Domain: $(-\infty, 1) \cup (1, \infty)$
      
      \vfill
      
      \columnbreak
      
\item $(f-g)(x) = \frac{x^2-2x}{x-1}$ \\
      Domain:  $(-\infty,1) \cup (1, \infty)$


\end{itemize}

\end{multicols}

\begin{multicols}{2}

\begin{itemize}

\item $(fg)(x) = 1$ \\
      Domain: $(-\infty,1) \cup (1, \infty)$
      
      \vfill
      
      \columnbreak
      
\item $\left(\frac{f}{g}\right)(x) =x^2-2x+1$ \\
      Domain: $(-\infty,1) \cup (1, \infty)$


\end{itemize}

\end{multicols}


\item For   $f(x) =x$ and $g(x) = \sqrt{x+1}$

\begin{multicols}{2}

\begin{itemize}

\item $(f+g)(x) = x+\sqrt{x+1}$ \\
      Domain: $[-1,\infty)$
      
      \vfill
      
      \columnbreak
      
\item $(f-g)(x) = x-\sqrt{x+1}$ \\
       Domain: $[-1,\infty)$


\end{itemize}

\end{multicols}

\begin{multicols}{2}

\begin{itemize}

\item $(fg)(x) = x\sqrt{x+1}$ \\
       Domain: $[-1,\infty)$
      
      \vfill
      
      \columnbreak
      
\item $\left(\frac{f}{g}\right)(x) =\frac{x}{\sqrt{x+1}}$ \\
       Domain: $(-1,\infty)$


\end{itemize}

\end{multicols}

\item For   $f(x) = \sqrt{x-5}$ and $g(x) = f(x) = \sqrt{x-5}$

\begin{multicols}{2}

\begin{itemize}

\item $(f+g)(x) = 2\sqrt{x-5}$ \\
      Domain: $[5,\infty)$
      
      \vfill
      
      \columnbreak
      
\item $(f-g)(x) =0$ \\
       Domain: $[5,\infty)$


\end{itemize}

\end{multicols}

\begin{multicols}{2}

\begin{itemize}

\item $(fg)(x) =x-5$ \\
       Domain: $[5,\infty)$
      
      \vfill
      
      \columnbreak
      
\item $\left(\frac{f}{g}\right)(x) =1$ \\
       Domain: $(5,\infty)$


\end{itemize}

\end{multicols}

\setcounter{HW}{\value{enumi}}
\end{enumerate}

\newpage

\begin{enumerate}
\setcounter{enumi}{\value{HW}}

\item One solution is $f(z) = 4z$ and $g(z) = z^3$. 
\item One solution is $f(z) = 4z$ and $g(z) = - z^3$. 
\item One solution is  $f(t) = 3t$ and $h(t) = |2t-1|$ 
\item One solution is $f(x) = 3-x$ and $g(x) = x+1$.  
\item  One solution is $f(x) = 3-x$ and $g(x) = (x+1)^{-1}$.  

\setcounter{HW}{\value{enumi}}
\end{enumerate}

\begin{enumerate}
\setcounter{enumi}{\value{HW}}

\item No.  The equivalence does not hold when $x = 0$.

\setcounter{HW}{\value{enumi}}
\end{enumerate}



\begin{multicols}{2}
\begin{enumerate}
\setcounter{enumi}{\value{HW}}
\addtocounter{enumi}{1}

\item $2$, $2$.
\item $-3$, $-3$.

\setcounter{HW}{\value{enumi}}
\end{enumerate}
\end{multicols}

\begin{multicols}{2}
\begin{enumerate}
\setcounter{enumi}{\value{HW}}

\item $0$, $0$
\item $3h+11$, $6x+3h-1$

\setcounter{HW}{\value{enumi}}
\end{enumerate}
\end{multicols}

\begin{multicols}{2}
\begin{enumerate}
\setcounter{enumi}{\value{HW}}

\item $-h-2$, $-2x-h+2$
\item  $4h+16$, $8x+4h$

\setcounter{HW}{\value{enumi}}
\end{enumerate}
\end{multicols}

\begin{multicols}{2}
\begin{enumerate}
\setcounter{enumi}{\value{HW}}

\item $-h-3$, $-2x-h+1$
\item $h^2+6h+12$, $3x^{2} + 3xh + h^{2}$

\setcounter{HW}{\value{enumi}}
\end{enumerate}
\end{multicols}

\begin{multicols}{2}
\begin{enumerate}
\setcounter{enumi}{\value{HW}}

\item $m$, $m$
\item $ah + 4a+b$, $2ax + ah + b$

\setcounter{HW}{\value{enumi}}
\end{enumerate}
\end{multicols}


\begin{multicols}{2}
\begin{enumerate}
\setcounter{enumi}{\value{HW}}

\item  $\dfrac{2}{\Delta x-1}$, $\dfrac{-2}{x(x+\Delta x)}$
\item $\dfrac{-3}{2(\Delta x - 2)}$, $\dfrac{3}{(x+\Delta x-1)(x-1)}$

\setcounter{HW}{\value{enumi}}
\end{enumerate}
\end{multicols}

\begin{multicols}{2}
\begin{enumerate}
\setcounter{enumi}{\value{HW}}

\item  $\dfrac{2-\Delta x}{(\Delta x - 1)^2}$, $\dfrac{-(2x+\Delta x)}{x^2(x+\Delta x)^2}$
\item  $\dfrac{-1}{2(\Delta x+4)}$, $\dfrac{-2}{(x+5)(x+\Delta x+5)}$

\setcounter{HW}{\value{enumi}}
\end{enumerate}
\end{multicols}

\begin{exenum}

\item $\dfrac{4}{7(4 \Delta x - 7)}$, $\dfrac{-4}{(4x-3)(4x+4\Delta x-3)}$
\item $\dfrac{6}{\Delta x + 1}$, $\dfrac{6}{(x+2)(x+\Delta x+2)}$

\item $\dfrac{9}{10(\Delta x - 10)}$, $\dfrac{-9}{(x - 9)(x + \Delta x - 9)}$
\item $\dfrac{\Delta x}{2 \Delta x - 1}$, $\dfrac{2x^2+2x\Delta x+2x+\Delta x}{(2x+1)(2x+2\Delta x+1)}$

\item $\dfrac{-1}{\sqrt{9-\Delta t} +3}$,   $\dfrac{-1}{\sqrt{9-t-\Delta t} + \sqrt{9-t}}$
\item $\dfrac{2}{\sqrt{2\Delta t+1} + 1}$, $\dfrac{2}{\sqrt{2t+2\Delta t+1} + \sqrt{2t+1}}$

\item $\dfrac{-4}{\sqrt{5-4\Delta t} + \sqrt{5}}$, $\dfrac{-4}{\sqrt{-4t-4\Delta t+5} + \sqrt{-4t+5}}$

\item $\dfrac{-1}{\sqrt{4-\Delta t} + 2}$, $\dfrac{-1}{\sqrt{4-t-\Delta t} + \sqrt{4-t}}$

\item $\dfrac{a}{\sqrt{a\Delta t+b} + \sqrt{b}}$, $\dfrac{a}{\sqrt{at+a\Delta t+b} + \sqrt{at+b}}$

\item   $(\Delta t)^{\frac{1}{2}} $, $\dfrac{3t^2+3t\Delta t+(\Delta t)^2}{(t+\Delta t)^{3/2} + t^{3/2}} $

\item $\dfrac{1}{(\Delta t)^{2/3}}$,  $\dfrac{1}{(t+\Delta t)^{2/3} + (t+\Delta t)^{1/3} t^{1/3} + t^{2/3}}$

\item  \begin{enumerate}  

\addtocounter{enumii}{1}

\item $(f_{+} + f_{-})(x) =  f_{+}(x) + f_{-}(x) = \dfrac{f(x) + |f(x)|}{2} + \dfrac{f(x) - |f(x)|}{2} = \dfrac{2f(x)}{2} = f(x)$.

\item   \[ f_{+}(x)  =  \begin{mycases} 
    0 &  \text{if $f(x) < 0$} \\
      f(x) & \text{if $f(x)  \geq 0$} \\
   \end{mycases},  \qquad    f_{-}(x)  =  \begin{mycases} 
    f(x) &  \text{if $f(x) < 0$} \\
      0 & \text{if $f(x)  \geq 0$} \\
   \end{mycases} \]

\end{enumerate}

\end{exenum}





\closegraphsfile
