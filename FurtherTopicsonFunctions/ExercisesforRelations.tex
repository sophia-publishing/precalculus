\startexenum

\label{ExercisesforRelations}

\begin{exenum}

\mexinstr{%
In Exercises \ref{relationfirst} - \ref{relationlast}, graph the given relation in the $xy$-plane.
}

\item \{$(-3, 9)$, $\;(-2, 4)$, $\;(-1, 1)$, $\;(0, 0)$, $\;(1, 1)$, $\;(2, 4)$, $\;(3, 9)\}$ \label{relationfirst}
\item \{$(-2, 0)$, $\;(-1, 1)$, $\;(-1, -1)$, $\;(0, 2)$, $\;(0, -2)$, $\;(1, 3)$, $\;(1, -3)\}$
\item  $\left\{ \left(m, 2m \right) \, | \, m = 0, \pm 1, \pm 2 \right\}$
\item  $\left\{ \left(\frac{6}{k}, k \right) \, | \, k = \pm 1, \pm 2, \pm 3, \pm 4, \pm 5, \pm 6 \right\}$
\item  $\left\{ \left(n, 4 - n^2\right) \, | \, n = 0, \pm 1, \pm 2 \right\}$
\item  $\left\{ \left(\sqrt{j}, j \right) \, | \, j = 0, 1, 4, 9 \right\}$
\item  $\left\{ \left(x, -2 \right) \, | \, x > -4 \right\}$
\item  $\left\{ \left(x, 3 \right) \, | \, x \leq 4 \right\}$
\item  $\left\{ \left(-1, y \right) \, | \, y > 1 \right\}$
\item  $\left\{ \left(2, y \right) \, | \, y \leq 5 \right\}$
\item $\{ (-2, y) \, | \, -3 < y \leq 4\}$
\item  $\left\{ \left(3,y \right) \, | \, -4 \leq y < 3 \right\}$
\item $\{ (x, 2) \, | \, -2 \leq x < 3 \}$
\item  $\left\{ \left(x,-3 \right) \, | \, -4 < x \leq 4 \right\}$
\item $\{ (x, y) \, | \, x > -2 \}$
\item  $\left\{ \left(x,y \right) \, | \, x \leq 3 \right\}$
\item  $\left\{ \left(x,y \right) \, | \, y < 4 \right\}$
\item  $\left\{ \left(x,y \right) \, | \, x \leq 3, \, y < 2 \right\}$
\item  $\left\{ \left(x,y \right) \, | \, x > 0, \, y < 4 \right\}$
\item $\{ (x, y) \, | \, -\sqrt{2} \leq x \leq \frac{2}{3}, \; \pi < y \leq \frac{9}{2} \}$ \label{relationlast}

\mexinstr{%
In Exercises \ref{relationsetfirst} - \ref{relationsetlast}, describe the given relation using either the roster or set-builder method.
}

\item \label{relationsetfirst}
See \autoref{fig:exrelationa}.

\begin{mfigure}
  
\begin{mfpic}[15]{-5}{2}{-2}{5}
\point[4pt]{(-4, -1),  (-2, 1),  (0, 3), (1, 4)}
\axes
\tlabel[cc](2,-0.5){\scriptsize $x$}
\tlabel[cc](0.5,5){\scriptsize $y$}
\xmarks{-4,-3,-2,-1,1}
\ymarks{-1,1,2,3,4}
\tlpointsep{5pt}
\scriptsize
\axislabels {x}{{$-4 \hspace{7pt}$} -4, {$-3 \hspace{7pt}$} -3, {$-2 \hspace{7pt}$} -2, {$-1 \hspace{7pt}$} -1, {$1$} 1}
\axislabels {y}{{$-1$} -1, {$1$} 1, {$2$} 2, {$3$} 3, {$4$} 4}
\normalsize
\end{mfpic}

\caption{Relation $A$}
\label{fig:exrelationa}
\end{mfigure}

\item See \autoref{fig:exrelationb}.

\begin{mfigure}

\begin{mfpic}[12]{-5}{5}{-1}{4}
\axes
\tlabel[cc](5,-0.5){\scriptsize $x$}
\tlabel[cc](0.5,4){\scriptsize $y$}
\xmarks{-4,-3,-2,-1,1,2,3,4}
\ymarks{1,2,3}
\tlpointsep{5pt}
\scriptsize
\axislabels {x}{{$-1 \hspace{7pt}$} -1, {$-2 \hspace{7pt}$} -2, {$-3 \hspace{7pt}$} -3, {$-4 \hspace{7pt}$} -4, {$1$} 1, {$2$} 2, {$3$} 3, {$4$} 4}
\axislabels {y}{{$1$} 1, {$2$} 2, {$3$} 3}
\normalsize
\penwd{1.25pt}
\arrow \polyline{(-3,3), (5,3)}
\point[4pt]{(-3,3)}
\end{mfpic} 

\caption{Relation $B$}
\label{fig:exrelationb}
\end{mfigure}

\item See \autoref{fig:exrelationc}.

\begin{mfigure}

\begin{mfpic}[15]{-1}{4}{-4}{6}
\axes
\tlabel[cc](4,-0.5){\scriptsize $x$}
\tlabel[cc](0.5,6){\scriptsize $y$}
\xmarks{1,2,3}
\ymarks{-3,-2,-1,1,2,3,4,5}
\tlpointsep{5pt}
\scriptsize
\axislabels {x}{{$1$} 1, {$2$} 2, {$3$} 3}
\axislabels {y}{ {$-3$} -3,{$-2$} -2, {$-1$} -1, {$1$} 1, {$2$} 2, {$3$} 3, {$4$} 4, {$5$} 5}
\normalsize
\penwd{1.25pt}
\arrow \polyline{(2,-3), (2,5)}
\pointfillfalse
\point[4pt]{(2,-3)}
\end{mfpic} 

\caption{Relation $C$}
\label{fig:exrelationc}
\end{mfigure}

\item See \autoref{fig:exrelationd}.

\begin{mfigure}

\begin{mfpic}[15]{-4}{1}{-5}{4}
\axes
\tlabel[cc](1,-0.5){\scriptsize $x$}
\tlabel[cc](0.5,4){\scriptsize $y$}
\xmarks{-3,-2,-1}
\ymarks{-4,-3,-2,-1,1,2,3}
\tlpointsep{5pt}
\scriptsize
\axislabels {x}{{$-3 \hspace{7pt}$} -3, {$-2 \hspace{7pt}$} -2, {$-1 \hspace{7pt}$} -1}
\axislabels {y}{{$-4$} -4,{$-3$} -3, {$-2$} -2, {$-1$} -1, {$1$} 1, {$2$} 2, {$3$} 3}
\normalsize
\penwd{1.25pt}
\polyline{(-2,-4), (-2,3)}
\point[4pt]{(-2,-4)}
\pointfillfalse
\point[4pt]{(-2,3)}
\end{mfpic}

\caption{Relation $D$}
\label{fig:exrelationd}
\end{mfigure}

\item See \autoref{fig:exrelatione}.

\begin{mfigure}

\begin{mfpic}[12]{-5}{5}{-1}{4}
\axes
\tlabel[cc](5,-0.5){\scriptsize $t$}
\tlabel[cc](0.5,4){\scriptsize $s$}
\xmarks{-4,-3,-2,-1,1,2,3,4}
\ymarks{1,2,3}
\tlpointsep{5pt}
\scriptsize
\axislabels {x}{{$-4 \hspace{7pt}$} -4,{$-3 \hspace{7pt}$} -3, {$-2 \hspace{7pt}$} -2, {$-1 \hspace{7pt}$} -1, {$1$} 1, {$2$} 2, {$3$} 3, {$4$} 4}
\axislabels {y}{{$1$} 1, {$2$} 2, {$3$} 3}
\normalsize
\penwd{1.25pt}
\polyline{(-4,2), (3,2)}
\point[4pt]{(-4,2)}
\pointfillfalse
\point[4pt]{(3,2)}
\end{mfpic}

\caption{Relation $E$}
\label{fig:exrelatione}
\end{mfigure}

\item See \autoref{fig:exrelationf}.

\begin{mfigure}

\begin{mfpic}[15]{-4}{4}{-1}{5}
\fillcolor[gray]{.7}
\gfill \rect{(-4,0), (3.75,4.75)}
\axes
\tlabel[cc](4,-0.5){\scriptsize $t$}
\tlabel[cc](0.5,5){\scriptsize $s$}
\xmarks{-3,-2,-1,1,2,3}
\ymarks{1,2,3,4}
\tlpointsep{5pt}
\scriptsize
\axislabels {x}{{$-3 \hspace{7pt}$} -3,{$-2 \hspace{7pt}$} -2, {$-1 \hspace{7pt}$} -1, {$1$} 1, {$2$} 2, {$3$} 3}
\axislabels {y}{ {$1$} 1, {$2$} 2, {$3$} 3, , {$4$} 4}
\normalsize
\end{mfpic} 

\caption{Relation $F$}
\label{fig:exrelationf}
\end{mfigure}

\item See \autoref{fig:exrelationg}.

\begin{mfigure}

\begin{mfpic}[15]{-4}{4}{-4}{4}
\fillcolor[gray]{.7}
\gfill \rect{(-1.97,-3.75), (3.75,3.75)}
\arrow \reverse \arrow \dashed \polyline{(-2,-4), (-2,4)}
\axes
\tlabel[cc](4,-0.5){\scriptsize $v$}
\tlabel[cc](0.5,4){\scriptsize $w$}
\xmarks{-3,-2,-1,1,2,3}
\ymarks{-3,-2,-1,1,2,3}
\tlpointsep{5pt}
\scriptsize
\axislabels {x}{{$-3 \hspace{7pt}$} -3,{$-2 \hspace{7pt}$} -2,{$-1 \hspace{7pt}$} -1,{$1$} 1,{$2$} 2,{$3$} 3}
\axislabels {y}{ {$-3$} -3,{$-2$} -2, {$-1$} -1, {$1$} 1, {$2$} 2, {$3$} 3}
\normalsize
\end{mfpic} 

\caption{Relation $G$}
\label{fig:exrelationg}
\end{mfigure}

\item See \autoref{fig:exrelationh}

\begin{mfigure}

\begin{mfpic}[15]{-4.5}{4}{-4}{4}

\fillcolor[gray]{.7}
\gfill \rect{(-2.97,-3.75), (1.97,3.75)}
\arrow \reverse \arrow \dashed \polyline{(-3,-4), (-3,4)}
\axes
\tlabel[cc](4,-0.5){\scriptsize $v$}
\tlabel[cc](0.5,4){\scriptsize $w$}
\xmarks{-4,-3,-2,-1,1,2,3}
\ymarks{-3,-2,-1,1,2,3}
\tlpointsep{5pt}
\scriptsize
\axislabels {x}{{$-4 \hspace{7pt}$} -4,{$-3 \hspace{7pt}$} -3,{$-2 \hspace{7pt}$} -2,{$-1 \hspace{7pt}$} -1,{$1$} 1,{$2$} 2,{$3$} 3}
\axislabels {y}{ {$-3$} -3,{$-2$} -2, {$-1$} -1, {$1$} 1, {$2$} 2, {$3$} 3}
\normalsize
\penwd{1.25pt}
\arrow \reverse \arrow \polyline{(2,-4), (2,4)}
\end{mfpic}

\caption{Relation $H$}
\label{fig:exrelationh}
\end{mfigure}

\item See \autoref{fig:exrelationi}

\begin{mfigure}

\begin{mfpic}[15]{-1.5}{6}{-1.5}{6}
\fillcolor[gray]{.7}
\gfill \rect{(0,0), (5.75,5.75)}
\axes
\tlabel[cc](6,-0.5){\scriptsize $u$}
\tlabel[cc](0.5,6){\scriptsize $v$}
\xmarks{-1,1,2,3,4,5}
\ymarks{-1,1,2,3,4,5}
\tlpointsep{5pt}
\scriptsize
\axislabels {x}{ {$-1 \hspace{7pt}$} -1, {$1$} 1, {$2$} 2, {$3$} 3, {$4$} 4, {$5$} 5}
\axislabels {y}{ {$-1$} -1, {$1$} 1, {$2$} 2, {$3$} 3, {$4$} 4, {$5$} 5}
\normalsize
\end{mfpic} 

\caption{Relation $I$}
\label{fig:exrelationi}
\end{mfigure}

\item \label{relationsetlast}
See \autoref{fig:exrelationj}

\begin{mfigure}
  
\begin{mfpic}[13]{-4.5}{5.5}{-4}{3}
\fillcolor[gray]{.7}
\gfill \rect{(-3.97, -2.97), (4.97, 1.97)}
\dashed \polyline{(-4, -3), (-4, 2)}
\dashed \polyline{(-4, 2), (5, 2)}
\dashed \polyline{(5, 2), (5, -3)}
\dashed \polyline{(5, -3), (-4, -3)}
\axes
\tlabel[cc](5.5,-0.5){\scriptsize $u$}
\tlabel[cc](0.5,3){\scriptsize $v$}
\xmarks{-4,-3,-2,-1,1,2,3,4,5}
\ymarks{-3,-2,-1,1,2}
\tlpointsep{5pt}
\scriptsize
\axislabels {x}{{$-4 \hspace{7pt}$} -4, {$-3 \hspace{7pt}$} -3, {$-2 \hspace{7pt}$} -2, {$-1 \hspace{7pt}$} -1, {$1$} 1, {$2$} 2, {$3$} 3, {$4$} 4, {$5$} 5}
\axislabels {y}{{$-3$} -3, {$-2$} -2, {$-1$} -1, {$1$} 1, {$2$} 2}
\normalsize
\end{mfpic}

\caption{Relation $J$}
\label{fig:exrelationj}
\end{mfigure}

\iexinstr{%
Some relations are fairly easy to describe in words or with the roster method but are rather difficult, if not impossible, to graph. Discuss with your classmates how you might graph the relations given in Exercises \ref{cannotgraphfirst} - \ref{cannotgraphlast}.  Note that in the notation below we are using the ellipsis, `\ldots,' to denote that the list does not end, but rather, continues to follow the established pattern indefinitely.  

For the relations in Exercises \ref{cannotgraphfirst} and \ref{cannotgraphsecond}, give two examples of points which belong to the relation and two points which do not belong to the relation.
}

\item $\{(x, y) \, | \, x \mbox{ is an odd integer, and } y \mbox{ is an even integer.}\}$ \label{cannotgraphfirst}
\item $\{(x, 1) \, | \, x \mbox{ is an irrational number }\}$ \label{cannotgraphsecond}
\item $\{(1, 0), (2, 1), (4, 2), (8, 3), (16, 4), (32, 5), \ldots \}$
\item $\{\ldots, (-3, 9), (-2, 4), (-1, 1), (0, 0), (1, 1), (2, 4), (3, 9), \ldots \}$ \label{cannotgraphlast}

\iexinstr{%
For each equation given in Exercises \ref{oldonethreefirst} - \ref{oldonethreelast}:

\begin{itemize}

\item   Graph the equation in the $xy$-plane by creating a table of points. 

\item  Find the axis intercepts, if they exist.

\item  Test the equation for symmetry.  If the equation fails a symmetry test, find a point on the graph of the equation whose symmetric point is not on the graph of the equation.

\item  Determine if the equation describes $y$ as a function of $x$.  If not, describe the graph of the equation using two or more explicit functions of $x$.  Check your answers using a graphing utility.

\end{itemize}
}

\item  $(x+2)^2+y^2 = 16$  \label{oldonethreefirst}

\item $x^{2} - y^{2} = 1$

\item  $4y^2 - 9x^2 = 36$
\item $x^{3}y = -4$  \label{oldonethreelast}

\iexinstr{%
For each equation given in Exercises \ref{vwfirstrelation} - \ref{vwlastrelation}:

\begin{itemize}

\item   Graph the equation in the $vw$-plane by creating a table of points. 

\item  Find the axis intercepts, if they exist.

\item  Test the equation for symmetry.  If the equation fails a symmetry test, find a point on the graph of the equation whose symmetric point is not on the graph of the equation.

\item  Determine if the equation describes $w$ as a function of $v$.  If not, describe the graph of the equation using two or more explicit functions of $v$.  Check your answers using a graphing utility.

\end{itemize}
}

\item  $v+w^2 = 4$   \label{vwfirstrelation}
\item $v^{3}+w^3 =8$ 

\item  $v^2w^3 = 8$
\item \hspace{-.1in}\sidenote{HINT: $v^4 - 2v^2 w + w^2 = \left(v^2 - w \right)^2$ \ldots} $v^4 - 2v^2 w + w^2 = 16$  \label{vwlastrelation}

\iexinstr{%
The procedures which we have outlined in the Examples of this section and used in Exercises \ref{oldonethreefirst} -  \ref{oldonethreelast} all rely on the fact that the equations were ``well-behaved''.  Not everything in Mathematics is quite so tame, as the following equations will show you.  Discuss with your classmates how you might approach graphing the equations given in Exercises \ref{listofcurvesfirst} - \ref{listofcurveslast}.  What difficulties arise when trying to apply the various tests and procedures given in this section?  For more information, including pictures of the curves, each curve name is a link to its page at www.wikipedia.org.  For a much longer list of fascinating curves, click \href{http://en.wikipedia.org/wiki/List_of_curves}{\underline{here}}.
}

\item \label{listofcurvesfirst} $x^{3} + y^{3} - 3xy = 0\;$ \href{http://en.wikipedia.org/wiki/Folium_of_descartes}{\underline{Folium of Descartes}}
\item $x^{4} = x^{2} + y^{2}\;$ \href{http://en.wikipedia.org/wiki/Kampyle_of_Eudoxus}{\underline{Kampyle of Eudoxus}}

\item $y^{2} = x^{3} + 3x^{2}\;$ \href{http://en.wikipedia.org/wiki/Tschirnhausen_cubic}{\underline{Tschirnhausen cubic}}
\item \label{listofcurveslast} $(x^{2} + y^{2})^{2} = x^{3} + y^{3}\;$ \href{https://en.wikipedia.org/wiki/File:Crooked_egg_curve.svg}{\underline{Crooked egg}} 

\item  With the help of your classmates, find examples of equations whose graphs possess 

\begin{itemize}

\item  symmetry about the $x$-axis only

\item  symmetry about the $y$-axis only

\item  symmetry about the origin only

\item  symmetry about the $x$-axis, $y$-axis, and origin

\end{itemize}

Can you find an example of an equation whose graph possesses exactly \textit{two} of the symmetries listed above?  Why or why not?

\end{exenum}

\clearpage
\startexenum

\subsection{Answers}

\begin{exenum}

\item See \autoref{fig:ansexone}.

\begin{mfigure}

\begin{mfpic}[10]{-4}{4}{-1}{10}
\point[4pt]{(-3, 9), (-2, 4), (-1, 1), (0, 0), (1, 1), (2, 4), (3, 9)}
\axes
\tlabel[cc](4,-0.5){\scriptsize $x$}
\tlabel[cc](0.5,10){\scriptsize $y$}
\xmarks{-3,-2,-1,1,2,3}
\ymarks{1,2,3,4,5,6,7,8,9}
\tlpointsep{5pt}
\scriptsize
\axislabels {x}{{$-3 \hspace{7pt}$} -3, {$-2 \hspace{7pt}$} -2, {$-1 \hspace{7pt}$} -1, {$1$} 1, {$2$} 2, {$3$} 3}
\axislabels {y}{{$1$} 1, {$2$} 2, {$3$} 3, {$4$} 4, {$5$} 5, {$6$} 6, {$7$} 7, {$8$} 8, {$9$} 9}
\normalsize
\end{mfpic}

\caption{}
\label{fig:ansexone}
\end{mfigure}

\item See \autoref{fig:ansextwo}.

\begin{mfigure}

\begin{mfpic}[13]{-3}{3}{-4}{4}
\point[4pt]{(-2, 0), (-1, -1), (-1,1), (0,2), (0,-2), (1,3), (1,-3)}
\axes
\tlabel[cc](3,-0.5){\scriptsize $x$}
\tlabel[cc](0.5,4){\scriptsize $y$}
\xmarks{-2,-1,1,2}
\ymarks{-3,-2,-1,1,2,3}
\tlpointsep{5pt}
\scriptsize
\axislabels {x}{{$-2 \hspace{7pt}$} -2, {$-1 \hspace{7pt}$} -1, {$1$} 1, {$2$} 2}
\axislabels {y}{{$-3$} -3, {$-2$} -2, {$-1$} -1, {$1$} 1, {$2$} 2, {$3$} 3}
\normalsize
\end{mfpic}

\caption{}
\label{fig:ansextwo}
\end{mfigure}

\item See \autoref{fig:ansexthree}.

\begin{mfigure}

\begin{mfpic}[13]{-3}{3}{-5}{5}
\point[4pt]{(-2, -4), (-1, -2), (0, 0), (1, 2), (2,4)}
\axes
\tlabel[cc](3,-0.5){\scriptsize $x$}
\tlabel[cc](0.5,5){\scriptsize $y$}
\xmarks{-2,-1,1,2}
\ymarks{-4,-3,-2,-1,1,2,3,4}
\tlpointsep{5pt}
\scriptsize
\axislabels {x}{{$-2 \hspace{7pt}$} -2, {$-1 \hspace{7pt}$} -1, {$1$} 1, {$2$} 2}
\axislabels {y}{  {$-1$} -1, {$-2$} -2, {$-3$} -3, {$-4$} -4, {$1$} 1, {$2$} 2, {$3$} 3, {$4$} 4}
\normalsize
\end{mfpic} 

\caption{}
\label{fig:ansexthree}
\end{mfigure}

\item See \autoref{fig:ansexfour}

\begin{ifigure}
  
\begin{mfpic}[9]{-7}{7}{-7}{7}
\point[4pt]{(6, 1), (-6, -1), (3, 2), (-3, -2), (2,3), (-2, -3), (-1.5,-4), (1.5,4), (-1.2,-5), (1.2,5), (-1,-6), (1,6) }
\axes
\tlabel[cc](7,-0.5){\scriptsize $x$}
\tlabel[cc](0.5,7){\scriptsize $y$}
\xmarks{-6,-5,-4,-3,-2,-1,1,2,3,4,5,6}
\ymarks{-6,-5,-4,-3,-2,-1,1,2,3,4,5,6}
\tlpointsep{5pt}
\scriptsize
\axislabels {x}{{$-6 \hspace{7pt}$} -6, {$-5 \hspace{7pt}$} -5, {$-4 \hspace{7pt}$} -4, {$-3 \hspace{7pt}$} -3, {$-2 \hspace{7pt}$} -2, {$-1 \hspace{7pt}$} -1, {$1$} 1, {$2$} 2, {$3$} 3, {$4$} 4, {$5$} 5, {$6$} 6}
\axislabels {y}{{$-6$} -6, {$-5$} -5,{$-4$} -4, {$-3$} -3,{$-2$} -2, {$-1$} -1, {$1$} 1, {$2$} 2, {$3$} 3, {$4$} 4, {$5$} 5, {$6$} 6}
\normalsize
\end{mfpic} 

\caption{}
\label{fig:ansexfour}
\end{ifigure}

\item See \autoref{fig:ansexfive}

\begin{ifigure}

\begin{mfpic}[15]{-3}{3}{-1}{5}
\point[4pt]{(0, 4), (1, 3), (-1, 3), (2, 0), (-2,0)}
\axes
\tlabel[cc](3,-0.5){\scriptsize $x$}
\tlabel[cc](0.5,5){\scriptsize $y$}
\xmarks{-2,-1,1,2}
\ymarks{1,2,3,4}
\tlpointsep{5pt}
\scriptsize
\axislabels {x}{{$-2 \hspace{7pt}$} -2, {$-1 \hspace{7pt}$} -1, {$1$} 1, {$2$} 2}
\axislabels {y}{ {$1$} 1, {$2$} 2, {$3$} 3, {$4$} 4}
\normalsize
\end{mfpic} 

\caption{}
\label{fig:ansexfive}
\end{ifigure}

\item See \autoref{fig:ansexsix}

\begin{mfigure}

\begin{mfpic}[10]{-1}{4}{-1}{10}
\point[4pt]{(0,0), (1,1), (2,4), (3,9)}
\axes
\tlabel[cc](4,-0.5){\scriptsize $x$}
\tlabel[cc](0.5,10){\scriptsize $y$}
\xmarks{1,2,3}
\ymarks{1,2,3,4,5,6,7,8,9}
\tlpointsep{5pt}
\scriptsize
\axislabels {x}{{$1$} 1, {$2$} 2, {$3$} 3}
\axislabels {y}{{$1$} 1, {$2$} 2, {$3$} 3, {$4$} 4, {$5$} 5, {$6$} 6, {$7$} 7, {$8$} 8, {$9$} 9}
\normalsize
\end{mfpic} 

\caption{}
\label{fig:ansexsix}
\end{mfigure}

\item See \autoref{fig:ansexseven}

\begin{mfigure}

\begin{mfpic}[11]{-5}{5}{-4}{1}
\axes
\xmarks{-4,-3,-2,-1,1,2,3,4}
\ymarks{-3,-2,-1}
\tlpointsep{5pt}
\scriptsize
\tlabel[cc](5,-0.5){\scriptsize $x$}
\tlabel[cc](0.5,1){\scriptsize $y$}
\axislabels {x}{{$-4 \hspace{7pt}$} -4, {$-3 \hspace{7pt}$} -3,{$-2 \hspace{7pt}$} -2, {$-1 \hspace{7pt}$} -1, {$1$} 1, {$2$} 2, {$3$} 3, {$4$} 4}
\axislabels {y}{ {$-3$} -3,  {$-1$} -1,}
\normalsize
\penwd{1.25pt}
\arrow \polyline{(-4,-2), (5,-2)}
\pointfillfalse
\point[4pt]{(-4,-2)}
\end{mfpic} 

\caption{}
\label{fig:ansexseven}
\end{mfigure}

\item See \autoref{fig:ansexeight}

\begin{mfigure}

\begin{mfpic}[12]{-5}{5}{-1}{4}
\axes
\tlabel[cc](5,-0.5){\scriptsize $x$}
\tlabel[cc](0.5,4){\scriptsize $y$}
\xmarks{-4,-3,-2,-1,1,2,3,4}
\ymarks{1,2,3}
\tlpointsep{5pt}
\scriptsize
\axislabels {x}{{$-1 \hspace{7pt}$} -1, {$-2 \hspace{7pt}$} -2, {$-3 \hspace{7pt}$} -3, {$-4 \hspace{7pt}$} -4, {$1$} 1, {$2$} 2, {$3$} 3, {$4$} 4}
\axislabels {y}{{$1$} 1, {$2$} 2, {$3$} 3}
\normalsize
\penwd{1.25pt}
\arrow \polyline{(4,3), (-5,3)}
\point[4pt]{(4,3)}
\end{mfpic} 

\caption{}
\label{fig:ansexeight}
\end{mfigure}

\item See \autoref{fig:ansexnine}

\begin{mfigure}

\begin{mfpic}[13]{-2}{3}{-1}{9}
\axes
\xmarks{-1,1,2}
\ymarks{1,2,3,4,5,6,7,8}
\tlpointsep{5pt}
\scriptsize
\tlabel[cc](3,-0.5){\scriptsize $x$}
\tlabel[cc](0.5,9){\scriptsize $y$}
\axislabels {x}{{$-1 \hspace{7pt}$} -1, {$1$} 1, {$2$} 2}
\axislabels {y}{{$1$} 1, {$2$} 2, {$3$} 3, {$4$} 4, {$5$} 5, {$6$} 6, {$7$} 7, {$8$} 8}
\normalsize
\penwd{1.25pt}
\arrow \polyline{(-1,1), (-1,9)}
\pointfillfalse
\point[4pt]{(-1,1)}
\end{mfpic} 

\caption{}
\label{fig:ansexnine}
\end{mfigure}

\item See \autoref{fig:ansexten}

\begin{ifigure}

\begin{mfpic}[13]{-1}{4}{-4}{6}
\axes
\tlabel[cc](4,-0.5){\scriptsize $x$}
\tlabel[cc](0.5,6){\scriptsize $y$}
\xmarks{1,2,3}
\ymarks{-3,-2,-1,1,2,3,4,5}
\tlpointsep{5pt}
\scriptsize
\axislabels {x}{{$1$} 1, {$2$} 2, {$3$} 3}
\axislabels {y}{ {$-3$} -3,{$-2$} -2, {$-1$} -1, {$1$} 1, {$2$} 2, {$3$} 3, {$4$} 4, {$5$} 5}
\normalsize
\penwd{1.25pt}
\arrow \polyline{(2,5), (2,-4)}
\point[4pt]{(2,5)}
\end{mfpic} 

\caption{}
\label{fig:ansexten}
\end{ifigure}

\item See \autoref{fig:ansexeleven}

\begin{ifigure}

\begin{mfpic}[13]{-4}{1}{-4}{5}
\axes
\tlabel[cc](1,-0.5){\scriptsize $x$}
\tlabel[cc](0.5,5){\scriptsize $y$}
\xmarks{-3,-2,-1}
\ymarks{-3,-2,-1,1,2,3,4}
\tlpointsep{5pt}
\scriptsize
\axislabels {x}{{$-3 \hspace{7pt}$} -3, {$-2 \hspace{7pt}$} -2, {$-1 \hspace{7pt}$} -1}
\axislabels {y}{{$-3$} -3, {$-2$} -2, {$-1$} -1, {$1$} 1, {$2$} 2, {$3$} 3, {$4$} 4}
\normalsize
\penwd{1.25pt}
\polyline{(-2,-3), (-2,4)}
\point[4pt]{(-2,4)}
\pointfillfalse
\point[4pt]{(-2,-3)}
\end{mfpic}

\caption{}
\label{fig:ansexeleven}
\end{ifigure}

\item See \autoref{fig:ansextwelve}

\begin{ifigure}

\begin{mfpic}[15]{-1}{4}{-5}{4}
\axes
\tlabel[cc](4,-0.5){\scriptsize $x$}
\tlabel[cc](0.5,4){\scriptsize $y$}
\xmarks{1,2,3}
\ymarks{-4,-3,-2,-1,1,2,3}
\tlpointsep{5pt}
\scriptsize
\axislabels {x}{{$1$} 1, {$2$} 2, {$3$} 3}
\axislabels {y}{{$-4$} -4,{$-3$} -3, {$-2$} -2, {$-1$} -1, {$1$} 1, {$2$} 2, {$3$} 3}
\normalsize
\penwd{1.25pt}
\polyline{(3,-4), (3,3)}
\point[4pt]{(3,-4)}
\pointfillfalse
\point[4pt]{(3,3)}
\end{mfpic}

\caption{}
\label{fig:ansextwelve}
\end{ifigure}

\item See \autoref{fig:ansexthirteen}

\begin{ifigure}

\begin{mfpic}[13]{-5}{5}{-1}{4}
\axes
\tlabel[cc](5,-0.5){\scriptsize $x$}
\tlabel[cc](0.5,4){\scriptsize $y$}
\xmarks{-4,-3,-2,-1,1,2,3,4}
\ymarks{1,2,3}
\tlpointsep{5pt}
\scriptsize
\axislabels {x}{{$-4 \hspace{7pt}$} -4,{$-3 \hspace{7pt}$} -3, {$-2 \hspace{7pt}$} -2, {$-1 \hspace{7pt}$} -1, {$1$} 1, {$2$} 2, {$3$} 3, {$4$} 4}
\axislabels {y}{{$1$} 1, {$2$} 2, {$3$} 3}
\normalsize
\penwd{1.25pt}
\polyline{(-2,2), (3,2)}
\point[4pt]{(-2,2)}
\pointfillfalse
\point[4pt]{(3,2)}
\end{mfpic}

\caption{}
\label{fig:ansexthirteen}
\end{ifigure}

\item See \autoref{fig:ansexfourteen}

\begin{ifigure}

\begin{mfpic}[13]{-5}{5}{-4}{1}
\axes
\tlabel[cc](5,-0.5){\scriptsize $x$}
\tlabel[cc](0.5,1){\scriptsize $y$}
\xmarks{-4,-3,-2,-1,1,2,3,4}
\ymarks{-1,-2,-3}
\tlpointsep{5pt}
\scriptsize
\axislabels {x}{{$-4 \hspace{7pt}$} -4,{$-3 \hspace{7pt}$} -3, {$-2 \hspace{7pt}$} -2, {$-1 \hspace{7pt}$} -1, {$1$} 1, {$2$} 2, {$3$} 3, {$4$} 4}
\axislabels {y}{{$-1$} -1, {$-2$} -2, {$-3$} -3}
\normalsize
\penwd{1.25pt}
\polyline{(-4,-3), (4,-3)}
\point[4pt]{(4,-3)}
\pointfillfalse
\point[4pt]{(-4,-3)}
\end{mfpic}

\caption{}
\label{fig:ansexfourteen}
\end{ifigure}

\item See \autoref{fig:ansexfifteen}

\begin{ifigure}

\begin{mfpic}[13]{-3}{2}{-4}{4}
\fillcolor[gray]{.7}
\gfill \rect{(-1.97,-3.75), (0.75,3.75)}
\dashed \arrow \reverse \arrow \polyline{(-2,4), (-2,-4)}
\axes
\tlabel[cc](2,-0.5){\scriptsize $x$}
\tlabel[cc](0.5,4){\scriptsize $y$}
\xmarks{-2,-1,1}
\ymarks{-3,-2,-1,1,2,3}
\tlpointsep{5pt}
\scriptsize
\axislabels {x}{{$1$} 1, {$-2 \hspace{6pt}$} -2, {$-1 \hspace{6pt}$} -1}
\axislabels {y}{ {$-3$} -3,{$-2$} -2, {$-1$} -1, {$1$} 1, {$2$} 2, {$3$} 3}
\normalsize
\end{mfpic} 

\caption{}
\label{fig:ansexfifteen}
\end{ifigure}

\item See \autoref{fig:ansexsixteen}

\begin{ifigure}

\begin{mfpic}[13]{-1}{4}{-4}{4}
\fillcolor[gray]{.7}
\gfill \rect{(-1,-3.75), (3,3.75)}
\axes
\tlabel[cc](4,-0.5){\scriptsize $x$}
\tlabel[cc](0.5,4){\scriptsize $y$}
\xmarks{1,2,3}
\ymarks{-3,-2,-1,1,2,3}
\tlpointsep{5pt}
\scriptsize
\axislabels {x}{{$1$} 1, {$2$} 2, {$3$} 3}
\axislabels {y}{ {$-3$} -3,{$-2$} -2, {$-1$} -1, {$1$} 1, {$2$} 2, {$3$} 3}
\normalsize
\penwd{1.25pt}
\arrow \reverse \arrow \polyline{(3,4), (3,-4)}
\end{mfpic} 

\caption{}
\label{fig:ansexsixteen}
\end{ifigure}

\item See \autoref{fig:ansexseventeen}

\begin{ifigure}

\begin{mfpic}[13]{-4}{4}{-1}{5}

\fillcolor[gray]{.7}
\gfill \rect{(-3.75,-1), (3.75,3.97)}
\arrow \reverse \arrow \dashed \polyline{(-4,4), (4,4)}
\axes
\tlabel[cc](4,-0.5){\scriptsize $x$}
\tlabel[cc](0.5,5){\scriptsize $y$}
\xmarks{-3,-2,-1,1,2,3}
\ymarks{1,2,3,4}
\tlpointsep{5pt}
\scriptsize
\axislabels {x}{{$-3 \hspace{7pt}$} -3,{$-2 \hspace{7pt}$} -2, {$-1 \hspace{7pt}$} -1, {$1$} 1, {$2$} 2, {$3$} 3}
\axislabels {y}{ {$1$} 1, {$2$} 2, {$3$} 3, {$4$} 4}
\normalsize
\end{mfpic} 

\caption{}
\label{fig:ansexseventeen}
\end{ifigure}

\item See \autoref{fig:ansexeighteen}.

\begin{mfigure}

\begin{mfpic}[15]{-1}{4}{-4}{4}
\fillcolor[gray]{.7}
\gfill \rect{(-0.75,-3.75), (3,1.97)}
\arrow \reverse \dashed \polyline{(-1,2), (3,2)}
\axes
\tlabel[cc](4,-0.5){\scriptsize $x$}
\tlabel[cc](0.5,4){\scriptsize $y$}
\xmarks{1,2,3}
\ymarks{-3,-2,-1,1,2,3}
\tlpointsep{5pt}
\scriptsize
\axislabels {x}{{$1$} 1, {$2$} 2, {$3$} 3}
\axislabels {y}{ {$-3$} -3,{$-2$} -2, {$-1$} -1, {$1$} 1, {$2$} 2, {$3$} 3}
\normalsize
\penwd{1.25pt}
\arrow \polyline{(3,2), (3,-4)}
\pointfillfalse
\point[4pt]{(3,2)}
\end{mfpic} 

\caption{}
\label{fig:ansexeighteen}
\end{mfigure}

\item See \autoref{fig:ansexnineteen}

\begin{mfigure}

\begin{mfpic}[15]{-2}{4}{-1}{5}
\fillcolor[gray]{.7}
\gfill \rect{(0.03,-0.75), (3.75,3.97)}
\arrow  \dashed \polyline{(0,4), (4,4)}
\arrow \dashed \polyline{(0,4), (0,-1)}
\arrow \polyline{(0,4), (0,5)}
\arrow \reverse \arrow \polyline{(-2,0), (4,0)}
\tlabel[cc](4,-0.5){\scriptsize $x$}
\tlabel[cc](0.5,5){\scriptsize $y$}
\xmarks{-1,1,2,3}
\ymarks{1,2,3,4}
\tlpointsep{5pt}
\scriptsize
\axislabels {x}{{$-1 \hspace{7pt}$} -1, {$1$} 1, {$2$} 2, {$3$} 3}
\axislabels {y}{ {$1$} 1, {$2$} 2, {$3$} 3, {$4$} 4}
\normalsize
\pointfillfalse
\point[4pt]{(0,4)}
\end{mfpic} 

\caption{}
\label{fig:ansexnineteen}
\end{mfigure}

\item See \autoref{fig:ansextwenty}

\begin{mfigure}

\begin{mfpic}[13]{-3}{2}{-1}{6}
\fillcolor[gray]{.7}
\gfill \rect{(-1.38, 3.17), (0.63, 4.47)}
\dashed \polyline{(0.6667, 3.1415), (-1.414, 3.1415)}
\axes
\tlabel[cc](2,-0.5){\scriptsize $x$}
\tlabel[cc](0.5,6){\scriptsize $y$}
\xmarks{-2,-1,1}
\ymarks{1,2,3,4,5}
\tlpointsep{5pt}
\scriptsize
\axislabels {x}{{$-2 \hspace{7pt}$} -2, {$-1 \hspace{7pt}$} -1, {$1$} 1}
\axislabels {y}{{$1$} 1, {$2$} 2, {$3$} 3, {$4$} 4, {$5$} 5}
\normalsize
\penwd{1.25pt}
\polyline{(-1.414, 3.1415), (-1.414, 4.5)}
\polyline{(-1.414, 4.5), (0.6667, 4.5)}
\polyline{(0.6667, 4.5), (0.6667, 3.1415)}
\pointfillfalse
\point[4pt]{(-1.414, 3.1415), (0.6667, 3.1415)}
\end{mfpic}

\caption{}
\label{fig:ansextwenty}
\end{mfigure}

\item $A = \{(-4, -1),  (-2, 1),  (0, 3), (1, 4)\}$
\item $B = \left\{ \left(x,3 \right) \, | \, x \geq -3 \right\}$
\item $C = \{ \left(2,y) \, | \, y > -3 \right\}$
\item $D = \{ \left(-2,y) \, | \, -4 \leq y < 3 \right\}$
\item $E = \left\{ \left(t,2 \right) \, | \, -4 < t \leq 3 \right\}$
\item $F = \{ \left(t,s) \, | \, s \geq 0 \right\}$
\item $G = \left\{ \left(v,w \right) \, | \, v > -2 \right\}$
\item $H = \left\{ \left(v,w\right) \, | \, -3 < v \leq 2 \right\}$
\item $I = \{ \left(u,v) \, | \, u \geq 0, \! v \geq 0\right\}$
\item $J = \{(u, v) \, | \, -4 < u < 5, \; -3 < v < 2\}$

\addtocounter{enumi}{4}

\item

$(x+2)^2+y^2=16$ \\ Re-write as $y = \pm \sqrt{16-(x+2)^2}$.\\
$x$-intercepts: $(-6, 0)$, $(2,0)$\\
$y$-intercepts: $\left(0, \pm 2\sqrt{3}\right)$\\
See \autoref{tab:anstable}.\\
See \autoref{fig:ansexcircle}.\\
The graph is symmetric about the $x$-axis.\\
The graph is not symmetric about the $y$-axis:  $(-6, 0)$ is on the graph but $(6, 0)$ is not.\\
The graph is not symmetric about the origin:  $(-6, 0)$ is on the graph but $(6, 0)$ is not.\\
The equation does not describe $y$ as a function of $x$.\\
The graph of the equation is the graphs of $f_{1}(x) = \sqrt{16-(x+2)^2}$ together with $f_{2}(x) = -\sqrt{16-(x+2)^2}$.

\begin{mtable}

$\begin{array}{|r||c|c|}  

\hline
 x &   y & (x,y) \\ \hline
-6 & 0 & (-6,0) \\  \hline
-4 & \pm 2 \sqrt{3} & \left(-4,\pm 2 \sqrt{3}\right) \\ \hline
 -2 &  \pm 4 & (-2, \pm 4) \\ \hline
0 &  \pm 2 \sqrt{3} & \left(0,\pm 2 \sqrt{3}\right) \\ \hline
 2 &  0 & (2, 0) \\ \hline
 
\end{array} $

\caption{}
\label{tab:anstable}
\end{mtable}

\begin{mfigure}

\begin{mfpic}[10]{-8}{4}{-6}{6}
\point[4pt]{(-6,0), (-4, 3.4641), (-4, -3.4641), (-2,4), (-2,-4), (0, 3.4641), (0, -3.4641), (2,0) }
\axes
\tlabel[cc](4,-0.5){\scriptsize $x$}
\tlabel[cc](0.5,6){\scriptsize $y$}
\xmarks{-7,-6,-5,-4,-3,-2,-1,1,2,3}
\ymarks{-5,-4,-3,-2,-1,1,2,3,4,5}
\tlpointsep{4pt}
\axislabels {x}{{\tiny $-7 \hspace{6pt}$} -7,{\tiny $-6 \hspace{6pt}$} -6, {\tiny $-5 \hspace{6pt}$} -5,{\tiny $-4 \hspace{6pt}$} -4, {\tiny $-3 \hspace{6pt}$} -3,{\tiny $-2 \hspace{6pt}$} -2, {\tiny $-1 \hspace{6pt}$} -1, {\tiny $1$} 1, {\tiny $2$} 2, {\tiny $3$} 3}
\axislabels {y}{{\tiny $-5$} -5, {\tiny $-4$} -4, {\tiny $-3$} -3, {\tiny $-2$} -2, {\tiny $-1$} -1, {\tiny $1$} 1, {\tiny $2$} 2, {\tiny $3$} 3, {\tiny $4$} 4, {\tiny $5$} 5}
\penwd{1.25pt}
\circle{(-2,0),4}
\end{mfpic}

\caption{}
\label{fig:ansexcircle}
\end{mfigure}

\item $x^{2} - y^{2} = 1$ \\
Re-write as: $y = \pm \sqrt{x^{2} - 1}$.\\
$x$-intercepts: $(-1, 0), (1, 0)$\\
The graph has no $y$-intercepts\\
See \autoref{tab:xyxy}.\\
See \autoref{fig:anssideparabola}.\\
The graph is symmetric about the $x$-axis.\\
The graph is symmetric about the $y$-axis.\\
The graph is symmetric about the origin.\\
The equation does not describe $y$ as a function of $x$.\\
The graph of the equation is the graphs of $f_{1}(x) = \sqrt{x^2-1}$ together with $f_{2}(x) = -\sqrt{x^2-1}$.\\

\begin{mtable}

$\begin{array}{|r||c|c|}  

\hline
 x &            y & (x,y) \\ \hline
-3 & \pm \sqrt{8} & (-3, \pm \sqrt{8}) \\ \hline
-2 & \pm \sqrt{3} & (-2, \pm \sqrt{3}) \\  \hline
-1 &            0 & (-1, 0) \\ \hline
 1 &            0 & (1, 0) \\ \hline
 2 & \pm \sqrt{3} & (2, \pm \sqrt{3}) \\ \hline
 3 & \pm \sqrt{8} & (3, \pm \sqrt{8}) \\ \hline
 
\end{array} $

\caption{}
\label{tab:xyxy}
\end{mtable}

\begin{mfigure}

\begin{mfpic}[10]{-4}{4}{-4}{4}
\point[4pt]{(-3,2.828), (-3,-2.828),(-2,1.732),(-2,-1.732),(-1,0),(1, 0),(3,2.828),(3,-2.828),(2,1.732),(2, -1.732)}
\axes
\tlabel[cc](4,-0.5){\scriptsize $x$}
\tlabel[cc](0.5,4){\scriptsize $y$}
\xmarks{-3,-2,-1,1,2,3}
\ymarks{-3,-2,-1,1,2,3}
\tlpointsep{4pt}
\axislabels {x}{{\tiny $-3 \hspace{6pt}$} -3, {\tiny $-2 \hspace{6pt}$} -2, {\tiny $-1 \hspace{6pt}$} -1, {\tiny $1$} 1, {\tiny $2$} 2, {\tiny $3$} 3}
\axislabels {y}{{\tiny $-3$} -3, {\tiny $-2$} -2, {\tiny $-1$} -1, {\tiny $1$} 1, {\tiny $2$} 2, {\tiny $3$} 3}
\penwd{1.25pt}
\arrow \reverse \arrow \parafcn{-2,2,0.1}{(cosh(t),sinh(t))}
\arrow \reverse \arrow \parafcn{-2,2,0.1}{(-cosh(t),sinh(t))}
\end{mfpic}

\caption{}
\label{fig:anssideparabola}
\end{mfigure}

\item $4y^2-9x^2 = 36$ \\
Re-write as: $y = \pm \dfrac{\sqrt{9x^2+36}}{2}$.\\
The graph has no $x$-intercepts\\
$y$-intercepts:  $(0, \pm 3)$\\
See \autoref{tab:fourysquaredetc}\\
See \autoref{fig:uprightparabola}\\
The graph is symmetric about the $x$-axis.\\
The graph is symmetric about the $y$-axis.\\
The graph is symmetric about the origin.\\
The equation does not describe $y$ as a function of $x$.\\
The graph of the equation is the graphs of $f_{1}(x) =  \dfrac{\sqrt{9x^2+36}}{2}$ together with $f_{2}(x) = - \dfrac{\sqrt{9x^2+36}}{2}$.\\
\begin{mtable}
  
$\begin{array}{|r||c|c|} 

\hline
 x &   y & (x,y) \\ \hline
-4 & \pm 3 \sqrt{5} &  \left(-4,\pm 3 \sqrt{5}\right) \\  \hline
-2 & \pm 3 \sqrt{2} & \left(-2,\pm 3 \sqrt{2}\right) \\ \hline
0 &  \pm 3 & (0, \pm 3) \\ \hline
2 & \pm 3 \sqrt{2} & \left(2,\pm 3 \sqrt{2}\right) \\ \hline
4 & \pm 3 \sqrt{5} &  \left(4,\pm 3 \sqrt{5}\right) \\  \hline
 
\end{array}$

\caption{}
\label{tab:fourysquaredetc}
\end{mtable}
\begin{mfigure}

\begin{mfpic}[10]{-5}{5}{-8}{8}
\point[4pt]{(-4, 6.708), (4, 6.708), (-2, 4.243), (2, 4.243), (0,3), (0,-3),(-4, -6.708), (4, -6.708), (-2, -4.243), (2, -4.243) }
\axes
\tlabel[cc](5,-0.5){\scriptsize $x$}
\tlabel[cc](0.5,8){\scriptsize $y$}
\xmarks{-4,-3,-2,-1, 1, 2, 3, 4}
\ymarks{-7,-6,-5,-4,-3,-2,-1,1,2,3,4,5,6,7}
\tlpointsep{4pt}
\axislabels {x}{{\tiny $-4 \hspace{6pt}$} -4,{\tiny $-3 \hspace{6pt}$} -3,{\tiny $-2 \hspace{6pt}$} -2, {\tiny $-1 \hspace{6pt}$} -1, {\tiny $1$} 1, {\tiny $2$} 2, {\tiny $3$} 3, {\tiny $4$} 4}
\axislabels {y}{{\tiny $-7$} -7, {\tiny $-6$} -6,{\tiny $-5$} -5,{\tiny $-4$} -4,{\tiny $-3$} -3,{\tiny $-2$} -2,{\tiny $-1$} -1,{\tiny $1$} 1,{\tiny $2$} 2,{\tiny $3$} 3,{\tiny $4$} 4,{\tiny $5$} 5,{\tiny $6$} 6,{\tiny $7$} 7 }
\penwd{1.25pt}
\arrow \reverse \arrow \parafcn{-1.6,1.6,0.1}{(2*sinh(t), 3*cosh(t))}
\arrow \reverse \arrow \parafcn{-1.6,1.6,0.1}{(2*sinh(t), 0-3*cosh(t))}
\end{mfpic}

\caption{}
\label{fig:uprightparabola}
\end{mfigure}

\item $x^{3}y = -4$ \\ Re-write as: $y = -\dfrac{4}{x^{3}} = -4x^{-3}$.\\
The graph has no $x$-intercepts\\
The graph has no $y$-intercepts\\
See \autoref{tab:xcubeyetc}\\
See \autoref{fig:xcubeyetc}\\
The graph is not symmetric about the $x$-axis: $(1, -4)$ is on the graph but $(1, 4)$ is not. \\
The graph is not symmetric about the $y$-axis:  $(1, -4)$ is on the graph but $(-1, -4)$ is not. \\
The graph is symmetric about the origin. \\
The equation does  describe $y$ as a function of $x$, namely $y=f(x) = - 4x^{-3}$.

\begin{mtable}
  
$\begin{array}{|r||c|c|}  

\hline
           x &            y & (x,y) \\ \hline
          -2 &  \frac{1}{2} & (-2, \frac{1}{2}) \\  \hline
          -1 &            4 & (-1, 4) \\ \hline
-\frac{1}{2} &           32 & (-\frac{1}{2}, 32) \\ \hline
 \frac{1}{2} &          -32 & (\frac{1}{2}, -32)\\ \hline
           1 &           -4 & (1, -4) \\ \hline
           2 & -\frac{1}{2} & (2, -\frac{1}{2}) \\ \hline
 
\end{array} $ 

\caption{}
\label{tab:xcubeyetc}
\end{mtable}

\begin{mfigure}

\begin{mfpic}[8]{-5}{5}{-9}{9}
\point[4pt]{(-4,0.125), (-2,1), (-1, 8), (1, -8), (2, -1), (4, -0.125)}
\axes
\tlabel[cc](5,-0.5){\scriptsize $x$}
\tlabel[cc](0.5,9){\scriptsize $y$}
\xmarks{-4,-2,2,4}
\ymarks{-8,-1,1,8}
\tlpointsep{4pt}
\axislabels {x}{{\tiny $-2 \hspace{6pt}$} -4, {\tiny $-1 \hspace{6pt}$} -2, {\tiny $1$} 2, {\tiny $2$} 4}
\axislabels {y}{{\tiny $-32$} -8, {\tiny $-4$} -1, {\tiny $4$} 1, {\tiny $32$} 8}
\penwd{1.25pt}
\arrow \reverse \arrow \function{-4.5, -0.95, 0.1}{-8/(x**3)}
\arrow \reverse \arrow \function{0.95, 4.5, 0.1}{-8/(x**3)}
\end{mfpic}

\caption{}
\label{fig:xcubeyetc}
\end{mfigure}

\item $v+w^2 = 4$ \\ Re-write as $w = \pm \sqrt{4-v}$.\\
$v$-intercept: $(4,0)$ \\
$w$-intercepts: $\left(0, \pm 2 \right)$ \\
See \autoref{tab:vpluswsquaredeqfour}\\
See \autoref{fig:vpluswsquaredeqfour}\\
The graph is symmetric about the $v$-axis\\
The graph is not symmetric about the $w$-axis: $(4, 0)$ is on the graph but $(-4, 0)$ is not. \\
The graph is not symmetric about the origin: $(4, 0)$ is on the graph but $(-4, 0)$ is not.\\
The equation does not describe $w$ as a function of $v$.\\
The graph of the equation is the graphs of $f_{1}(v) = \sqrt{4-v}$ together with $f_{2}(v) = -\sqrt{4-v}$.\\
The graph is not symmetric about the $v$-axis:  $(0,2)$ is on the graph but $(0,-2)$ is not. \\
The graph is not symmetric about the $w$-axis: $(2, 0)$ is on the graph but $(-2, 0)$ is not.\\
The graph is not symmetric about the origin: $(0, 2)$ is on the graph but $(0, -2)$ is not. \\
The equation does  describe $w$ as a function of $v$, namely $w=f(v) = \sqrt[3]{8-v^3}$.  \\

\begin{mtable}
  
$\begin{array}{|r||c|c|}  

\hline
 v &   w & (x,y) \\ \hline
-5 & \pm 3 & (-5,\pm 3) \\  \hline
-2 & \pm  \sqrt{6} & \left(-2,\pm  \sqrt{6}\right) \\ \hline
 0 &  \pm 2 & (0, \pm 2) \\ \hline
2 &  \pm \sqrt{2} & \left(1,\pm  \sqrt{3}\right) \\ \hline
 4 &  0 & (4, 0) \\ \hline
 
 
\end{array} $ 

\caption{}
\label{tab:vpluswsquaredeqfour}
\end{mtable}

\begin{mfigure}
  
\begin{mfpic}[10]{-6}{6}{-4}{4}
\point[4pt]{(-5,3), (-5,-3), (-2, 2.45), (-2, -2.45), (0,2), (0,-2), (2, 1.414), (2, -1.414), (4,0) }
\axes
\tlabel[cc](6,-0.5){\scriptsize $v$}
\tlabel[cc](0.5,4){\scriptsize $w$}
\xmarks{-5,-4,-3,-2,-1,1,2,3,4,5}
\ymarks{-3,-2,-1,1,2,3}
\tlpointsep{4pt}
\axislabels {x}{ {\tiny $-5 \hspace{6pt}$} -5,{\tiny $-4 \hspace{6pt}$} -4, {\tiny $-3 \hspace{6pt}$} -3,{\tiny $-2 \hspace{6pt}$} -2, {\tiny $-1 \hspace{6pt}$} -1, {\tiny $1$} 1, {\tiny $2$} 2, {\tiny $3$} 3, {\tiny $5$} 5}
\axislabels {y}{{\tiny $-3$} -3,  {\tiny $-1$} -1, {\tiny $1$} 1, {\tiny $3$} 3}
\penwd{1.25pt}
\arrow \reverse \arrow \parafcn{-3.2, 3.2, 0.1}{(4-t**2, t)}
\end{mfpic}

\caption{}
\label{fig:vpluswsquaredeqfour}
\end{mfigure}

\item $v^{3}+w^3 =8$ \\ Re-write as: $w = \sqrt[3]{8-v^3}$.\\
$v$-intercept: $(2,0)$  \\
$w$-intercept: $(0,2)$ \\
See \autoref{tab:vcubepluswcubeeqeight}\\
See \autoref{fig:vcubepluswcubeeqeight}\\
\begin{mtable}

$\begin{array}{|r||c|c|}  

\hline
 v &            w & (v,w) \\ \hline
-3 &  \sqrt[3]{35} & (-3, \sqrt[3]{35}) \\  \hline
-1 &    \sqrt[3]{9}  & (-1, \sqrt[3]{9}) \\ \hline
 0 &            2 & (0, 2) \\ \hline
 1 &  \sqrt[3]{7} & (1, \sqrt[3]{7}) \\ \hline
 2 & 0 & (2, 0) \\ \hline
 3 & -\sqrt[3]{19} & (3, -\sqrt[3]{19}) \\ \hline
 
\end{array} $ 

\caption{}
\label{tab:vcubepluswcubeeqeight}
\end{mtable}

\begin{mfigure}

\begin{mfpic}[12]{-4}{4}{-4}{4}
\point[4pt]{ (-3, 3.271), (-1, 2.08), (0,2), (1, 1.91), (2,0), (3, -2.67)}
\axes
\tlabel[cc](4,-0.5){\scriptsize $v$}
\tlabel[cc](0.5,4){\scriptsize $w$}
\xmarks{-3,-2,-1,1,2,3}
\ymarks{-3,-2,-1,1,2,3}
\tlpointsep{4pt}
\axislabels {x}{{\tiny $-3 \hspace{6pt}$} -3, {\tiny $-2 \hspace{6pt}$} -2, {\tiny $-1 \hspace{6pt}$} -1, {\tiny $1$} 1, {\tiny $3$} 3}
\axislabels {y}{{\tiny $-3$} -3, {\tiny $-2$} -2, {\tiny $-1$} -1, {\tiny $1$} 1, {\tiny $3$} 3}
\penwd{1.25pt}
\arrow  \reverse \function{-4,2,0.1}{( 8-(x**3) )**(0.3333)}
\arrow  \function{2,4,0.1}{(-1)*(((x**3)-8)**(0.3333))}
\end{mfpic}

\caption{}
\label{fig:vcubepluswcubeeqeight}
\end{mfigure}

\item $v^2w^3 = 8$ \\ Re-write as $w =\dfrac{2}{\sqrt[3]{v^2}} = 2 v^{-\frac{2}{3}}$.\\
The graph has no $v$-intercepts.  \\
The graph has no $w$-intercepts. \\
See \autoref{tab:vsquarewcubeeqeight}\\
See \autoref{fig:vsquarewcubeeqeight}\\
The graph is not symmetric about the $v$-axis:  $(-1,2)$ is on the graph but $(-1,-2)$ is not. \\
The graph is  symmetric about the $w$-axis.  \\
The graph is not symmetric about the origin: $(-1,2)$ is on the graph but $(-1,-2)$ is not.\\
The equation does describe $w$ as a function of $v$, namely $w=f(v) = 2 v^{-\frac{2}{3}}$. \\

\begin{itable}
  
$\begin{array}{|r||c|c|}  

\hline
 v &   w & (x,y) \\ \hline
-8 & \frac{1}{2} & \left(-8, \frac{1}{2} \right) \\  \hline 
-1 & 2 & \left(-1, 2 \right) \\    \hline 
 -\frac{1}{8} &  8 &  \left(-\frac{1}{8}, 8 \right) \\ \hline  
 \frac{1}{8} &  8 &  \left(\frac{1}{8}, 8 \right) \\ \hline  
1 & 2 & \left(1, 2 \right) \\  \hline  
8 & \frac{1}{2} & \left(8, \frac{1}{2} \right) \\  \hline
\end{array} $

\caption{}
\label{tab:vsquarewcubeeqeight}
\end{itable}

\begin{ifigure}

\begin{mfpic}[7][15]{-9}{9}{-1}{10}
\point[4pt]{(-8,0.5), (-1,2), (-0.125, 8), (0.125, 8), (1,2), (8,0.5)}
\axes
\tlabel[cc](9,-0.5){\scriptsize $v$}
\tlabel[cc](0.5,10){\scriptsize $w$}
\xmarks{-8 step 1 until 8}
\ymarks{1 step 1 until 9}
\tlpointsep{4pt}
\axislabels {x}{{\tiny $-8 \hspace{6pt}$} -8, {\tiny $-1 \hspace{6pt}$} -1, {\tiny $1$} 1,  {\tiny $8$} 8}
\axislabels {y}{ {\tiny $1$} 1,  {\tiny $2$} 2,  {\tiny $5$} 5,  {\tiny $6$} 6, {\tiny $7$} 7, {\tiny $8$} 8}
\penwd{1.25pt}
\arrow \reverse \arrow \function{-9, -.1, 0.1}{2* ( (x**2)**(-0.3333) )}
\arrow \reverse \arrow \function{0.1, 9, 0.1}{2*( (x**2)**(-0.3333)  )}
\end{mfpic}

\caption{}
\label{fig:vsquarewcubeeqeight}
\end{ifigure}

\item  $v^4 - 2v^2w + w^2 = 16$ \\ Re-write as:  $\left(v^2-w\right)^2 = 16$ \\  Extracting square roots gives: \\ $w = v^2 + 4$ and $w = v^2-4$\\
$v$-intercepts: $(-2,0), (2,0)$. \\
$w$-intercepts: $(0,-4), (0,4)$ \\
See \autoref{tab:vpowfouretc}\\
See \autoref{fig:vpowfouretc}\\
The graph is not symmetric about the $v$-axis:  $(1,5)$ is on the graph but $(1,-5)$ is not.\\
The graph is  symmetric about the $w$-axis. \\
The graph is not symmetric about the origin: $(1,5)$ is on the graph but $(-1, -5)$ is not.  \\
The equation does not describe $w$ as a function of $v$.  \\
The graph of the equation is the graphs of $f_{1}(v) = v^2+4$ together with $f_{2}(v) = v^2-4$.\\

\begin{mtable}
  
$\begin{array}{|r||c|c|}  

\hline
 v &            w & (v,w) \\ \hline
 -2 &  8 & (-2,8) \\  \hline
-2 &  0 & (-2,0) \\  \hline
-1 &    5  & (-1, 5) \\ \hline
-1 &   -3  & (-1, -3) \\ \hline
 0 &    \pm 4 & (0, \pm4) \\ \hline
  1 &    5  & (1, 5) \\ \hline
  1 &   -3  & (1, -3) \\ \hline
  2 &  8 & (2,8) \\  \hline
  2 &  0 & (2,0) \\  \hline

\end{array} $ 

\caption{}
\label{tab:vpowfouretc}
\end{mtable}

\begin{mfigure}

\begin{mfpic}[10]{-4}{4}{-4.5}{9}
\axes
\tlabel[cc](4,-0.5){\scriptsize $v$}
\tlabel[cc](0.5,9){\scriptsize $w$}
\xmarks{-3,-2,-1,1,2,3}
\ymarks{-3,-2,-1,1,2,3,4,5,6,7,8}
\tlpointsep{4pt}
\axislabels {x}{{\tiny $-3 \hspace{6pt}$} -3,  {\tiny $-1 \hspace{6pt}$} -1, {\tiny $1$} 1, {\tiny $3$} 3}
\axislabels {y}{{\tiny $-3$} -3, {\tiny $-2$} -2, {\tiny $-1$} -1, {\tiny $1$} 1,{\tiny $2$} 2, {\tiny $3$} 3, {\tiny $6$} 6, {\tiny $7$} 7, {\tiny $8$} 8}
\penwd{1.25pt}
\arrow \reverse \arrow \function{-2.2, 2.2, 0.1}{(x**2)+4}
\arrow \reverse \arrow \function{-3, 3, 0.1}{(x**2) - 4}
\point[4pt]{(-2,8), (-2,0), (-1,5), (-1,-3), (0,4), (0,-4), (1,5), (1,-3), (2,0), (2,8)}
\end{mfpic}

\caption{}
\label{fig:vpowfouretc}
\end{mfigure}

\end{exenum}

