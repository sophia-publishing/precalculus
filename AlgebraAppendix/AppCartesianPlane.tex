\mfpicnumber{1}

\opengraphsfile{AppCartesianPlane}

\setcounter{footnote}{0}

\label{AppCartesianPlane}

\subsection{The Cartesian Coordinate Plane}

In order to visualize the pure excitement that is Precalculus, we need to unite Algebra and Geometry.  Simply put, we must find a way to draw algebraic things.  Let's start with possibly the greatest mathematical achievement of all time: the \index{Cartesian coordinate plane} \textbf{Cartesian Coordinate Plane}.\footnote{So named in honor of \href{http://en.wikipedia.org/wiki/Descartes}{\underline{Ren\'{e} Descartes}}.}  Imagine two real number lines crossing at a right angle at $0$ as drawn in \autoref{fig:cartesianplane}.

\begin{figure}[h]
\begin{center}

\begin{mfpic}[20]{-5}{5}{-5}{5}
\axes
\tlabel[cc](5,-0.5){\scriptsize $x$}
\tlabel[cc](0.5,5){\scriptsize $y$}
\xmarks{-4,-3,-2,-1,1,2,3,4}
\ymarks{-4,-3,-2,-1,1,2,3,4}
\tlpointsep{5pt}
\scriptsize
\axislabels {x}{{$-4 \hspace{7pt}$} -4, {$-3 \hspace{7pt}$} -3, {$-2 \hspace{7pt}$} -2, {$-1 \hspace{7pt}$} -1, {$1$} 1, {$2$} 2, {$3$} 3, {$4$} 4}
\axislabels {y}{{$-4$} -4, {$-3$} -3, {$-2$} -2, {$-1$} -1, {$1$} 1, {$2$} 2, {$3$} 3, {$4$} 4}
\normalsize
\end{mfpic}
\caption{Cartesian Coordinate Plane}
\label{fig:cartesianplane}
\end{center}
\end{figure}

The horizontal number line is usually called the \index{$x$-axis} \textbf{\boldmath $x$-axis} while the vertical number line is usually called the \index{$y$-axis} \textbf{\boldmath $y$-axis}. As with things in the `real' world, however, it's best not to get too caught up with labels. Think of $x$ and $y$ as generic label placeholders, in much the same way as the variables $x$ and $y$ are placeholders for real numbers.  The letters we choose to identify with the axes depend on the context.  For example, if we were plotting the relationship between time and the number of Sasquatch sightings, we might label the horizontal axis as the $t$-axis (for `time') and the vertical axis the $N$-axis (for `number' of sightings.)  As with the usual number line, we imagine these axes extending off indefinitely in both directions.\footnote{Usually extending off  towards infinity is indicated by arrows, but here, the arrows are used to indicate the \textit{direction} of increasing values of $x$ and $y$.}
  Having two number lines allows us to locate the positions of points off of the number lines as well as points on the lines themselves.  

For example, consider the point $P$ in \autoref{fig:plot1}.  To use the numbers on the axes to label this point, we imagine dropping a vertical line from the $x$-axis to $P$ and extending a horizontal line from the $y$-axis to $P$.  This process is sometimes called `projecting' the point $P$ to the $x$- (respectively $y$-) axis.  We then describe the point $P$ using the \index{ordered pair} \textbf{ordered pair} $(2,-4)$.  The first number in the ordered pair is called the \index{abscissa} \textbf{abscissa} or \index{$x$-coordinate} \textbf{\boldmath $x$-coordinate} and the second is called the \index{ordinate} \textbf{ordinate} or \index{$y$-coordinate} \textbf{\boldmath $y$-coordinate}.  Again, the names of the coordinates can vary depending on the context of the application. If, as in the previous paragraph, the horizontal axis represented time and the vertical axis represented the number of Sasquatch sightings, the first coordinate would be called the $t$-coordinate and the second coordinate would be the $N$-coordinate. What's important is that we maintain the convention that the abscissa (first coordinate) always corresponds to the horizontal position, while the ordinate (second coordinate) always corresponds to the vertical position.  Taken together, the ordered pair $(2,-4)$ comprise the \index{coordinates ! Cartesian}\index{Cartesian coordinates}\textbf{Cartesian coordinates}\footnote{Also called the `rectangular coordinates' of $P$ -- see Section \ref{PolarCoordinates} for more details.} of the point $P$. 

In practice, the distinction between a point and its coordinates is blurred; for example, we often speak of `the point $(2,-4)$'.  We can think of $(2,-4)$ as instructions on how to reach $P$ from the \index{origin} {\bf origin} $(0, 0)$ by moving $2$ units to the right and $4$ units downwards.  Notice that the order in the \underline{ordered} pair is important, as are the signs of the numbers in the pair.

\begin{figure}
\begin{center}
  
\begin{mfpic}[20]{-5}{5}{-5}{5}
\axes
\tlabel[cc](5,-0.5){\scriptsize $x$}
\tlabel[cc](0.5,5){\scriptsize $y$}
\xmarks{-4,-3,-2,-1,1,2,3,4}
\ymarks{-4,-3,-2,-1,1,2,3,4}
\gfill \circle{(2,-4),0.1}
\tlabel[cc](2.5,-4){\scriptsize $P$}
\dashed \polyline{(2,0),(2,-4),(0,-4)}
\tlpointsep{5pt}
\scriptsize
\axislabels {x}{{$-4 \hspace{7pt}$} -4, {$-3 \hspace{7pt} $} -3, {$-2\hspace{7pt} $} -2, {$-1 \hspace{7pt}$} -1, {$1$} 1, {$2$} 2, {$3$} 3, {$4$} 4}
\axislabels {y}{{$-4$} -4, {$-3$} -3, {$-2$} -2, {$-1$} -1, {$1$} 1, {$2$} 2, {$3$} 3, {$4$} 4}
\normalsize
\end{mfpic}
\caption{Plotting $(2, -4)$}
\label{fig:plot1}

\end{center}
\end{figure}

 If we wish to plot the point $(-4,2)$, we would move to the left $4$ units from the origin and then move upwards $2$ units, as shown in \autoref{fig:plot2}.

\begin{figure}
\begin{center}
  
\begin{mfpic}[20]{-5}{5}{-5}{5}
\axes
\tlabel[cc](5,-0.5){\scriptsize $x$}
\tlabel[cc](0.5,5){\scriptsize $y$}
\xmarks{-4,-3,-2,-1,1,2,3,4}
\ymarks{-4,-3,-2,-1,1,2,3,4}
\gfill \circle{(2,-4),0.1}
\tlabel[cc](3.5,-4){\scriptsize $P(2, -4)$}
\dashed \polyline{(2,0),(2,-4),(0,-4)}
\gfill \circle{(-4,2),0.1}
\tlabel[cc](-4,2.5){\scriptsize $(-4,2)$}
\dashed \polyline{(-4,0),(-4,2),(0,2)}
\tlpointsep{5pt}
\scriptsize
\axislabels {x}{{$-4 \hspace{7pt}$} -4, {$-3 \hspace{7pt}$} -3, {$-2 \hspace{7pt}$} -2, {$-1 \hspace{7pt}$} -1, {$1$} 1, {$2$} 2, {$3$} 3, {$4$} 4}
\axislabels {y}{{$-4$} -4, {$-3$} -3, {$-2$} -2, {$-1$} -1, {$1$} 1, {$2$} 2, {$3$} 3, {$4$} 4}
\end{mfpic}

\caption{Plotting $(-4, 2)$}
\label{fig:plot2}
\end{center}
\end{figure}

When we speak of the Cartesian Coordinate Plane, we mean the set of all possible ordered pairs $(x,y)$ as $x$ and $y$ take values from the real numbers.  \autoref{box:importantfactscartesianplane} shows a summary of some basic, but nonetheless important, facts about Cartesian coordinates.

\begin{floatbox}[label=box:importantfactscartesianplane]{Important Facts about the Cartesian Coordinate Plane}

\begin{itemize}[leftmargin=*]

\item $(a,b)$ and $(c,d)$ represent the same point in the plane if and only if $a = c$ and $b = d$.
\item  $(x,y)$ lies on the $x$-axis if and only if $y = 0$.
\item  $(x,y)$ lies on the $y$-axis if and only if $x=0$.
\item The origin is the point $(0,0)$.  It is the only point common to both axes.

\end{itemize}

\end{floatbox}

\begin{ex} Plot the following points: $A(5,8)$, $B\left(-\frac{5}{2}, 3\right)$, $C(-5.8, -3)$, $D(4.5, -1)$, $E(5,0)$, $F(0,5)$, $G(-7,0)$, $H(0, -9)$, $O(0,0)$.(The letter $O$ is almost always reserved for the origin.)

{\bf Solution.} Refer \autoref{fig:plottingpoints1}. To plot these points, we start at the origin and move to the right if the $x$-coordinate is positive; to the left if it is negative.   Next, we move up if the $y$-coordinate is positive or down if it is negative.  If the $x$-coordinate is $0$, we start at the origin and move along the $y$-axis only.  If the  $y$-coordinate is $0$ we move along the $x$-axis only.


\begin{figure}
\begin{center}

\begin{mfpic}[14]{-10}{10}{-10}{10}
\axes
\tlabel[cc](10,-0.5){\scriptsize $x$}
\tlabel[cc](0.5,10){\scriptsize $y$}
\xmarks{-9,-8,-7,-6,-5,-4,-3,-2,-1,1,2,3,4,5,6,7,8,9}
\ymarks{-9,-8,-7,-6,-5,-4,-3,-2,-1,1,2,3,4,5,6,7,8,9}
\gfill \circle{(5,8),0.1}
\tlabel[cc](5,7.25){$A(5,8)$}
\gfill \circle{(-2.5,3),0.1}
\tlabel[cc](-2.5,2.25){$B\left(-\frac{5}{2},3\right)$}
\gfill \circle{(-5.8,-3),0.1}
\tlabel[cc](-5.8,-3.75){$C(-5.8,-3)$}
\gfill \circle{(4.5,-1),0.1}
\tlabel[cc](4.5,-1.75){$D(4.5,-1)$}
\gfill \circle{(5,0),0.1}
\tlabel[cc](5,0.5){$E(5,0)$}
\gfill \circle{(0,5),0.1}
\tlabel[cc](1.35,5){$F(0,5)$}
\gfill \circle{(-7,0),0.1}
\tlabel[cc](-7,0.5){$G(-7,0)$}
\gfill \circle{(0,-9),0.1}
\tlabel[cc](1.5,-9){$H(0,-9)$}
\gfill \circle{(0,0),0.1}
\tlabel[cc](1.2,0.5){$O(0,0)$}
\tlpointsep{5pt}
\scriptsize
\axislabels {x}{{$-9 \hspace{7pt}$} -9, {$-8 \hspace{7pt}$} -8, {$-7 \hspace{7pt}$} -7, {$-6 \hspace{7pt}$} -6, {$-5 \hspace{7pt}$} -5, {$-4 \hspace{7pt}$} -4, {$-3 \hspace{7pt}$} -3, {$-2 \hspace{7pt}$} -2, {$-1 \hspace{7pt}$} -1, {$1$} 1, {$2$} 2, {$3$} 3, {$4$} 4, {$5$} 5, {$6$} 6, {$7$} 7, {$8$} 8, {$9$} 9}
\axislabels {y}{{$-9$} -9, {$-8$} -8, {$-7$} -7, {$-6$} -6, {$-5$} -5, {$-4$} -4, {$-3$} -3, {$-2$} -2, {$-1$} -1, {$1$} 1, {$2$} 2, {$3$} 3, {$4$} 4, {$5$} 5, {$6$} 6, {$7$} 7, {$8$} 8, {$9$} 9}
\normalsize
\end{mfpic}

\caption{Plotting points}
\label{fig:plottingpoints1}
\end{center}
\end{figure}

\vspace*{-.4in}

\qed

\end{ex}

\vspace*{.1in}

The axes divide the plane into four regions called \index{quadrants} \textbf{quadrants}.  They are labeled with Roman numerals and proceed counterclockwise around the plane as shown in \autoref{fig:quadrant}.

\begin{figure}
\begin{center}

\begin{mfpic}[16]{-5}{5}{-5}{5}
\axes
\tlabel[cc](5,-0.5){\scriptsize $x$}
\tlabel[cc](0.5,5){\scriptsize $y$}
\tlabel[cc](3,3.5){Quadrant I}
\tlabel[cc](3,2.5){ $x > 0$, $y > 0$}
\tlabel[cc](-3,3.5){Quadrant II}
\tlabel[cc](-3,2.5){ $x < 0$, $y > 0$}
\tlabel[cc](-3,-2.5){Quadrant III}
\tlabel[cc](-3,-3.5){ $x < 0$, $y < 0$}
\tlabel[cc](3,-2.5){Quadrant IV}
\tlabel[cc](3,-3.5){ $x > 0$, $y < 0$}
\xmarks{-4,-3,-2,-1,1,2,3,4}
\ymarks{-4,-3,-2,-1,1,2,3,4}
\tlpointsep{5pt}
\scriptsize
\axislabels {x}{{$-4 \hspace{7pt}$} -4, {$-3 \hspace{7pt}$} -3, {$-2 \hspace{7pt}$} -2, {$-1 \hspace{7pt}$} -1, {$1$} 1, {$2$} 2, {$3$} 3, {$4$} 4}
\axislabels {y}{{$-4$} -4, {$-3$} -3, {$-2$} -2, {$-1$} -1, {$1$} 1, {$2$} 2, {$3$} 3, {$4$} 4}
\normalsize
\end{mfpic}

\end{center}
\caption{Quadrants}
\label{fig:quadrant}
\end{figure}

For example, $(1,2)$ lies in Quadrant I, $(-1,2)$ in Quadrant II, $(-1,-2)$ in Quadrant III and $(1,-2)$ in Quadrant IV.  If a point other than the origin happens to lie on the axes, we typically refer to that point as lying on the positive or negative $x$-axis (if $y = 0$) or on the positive or negative $y$-axis (if $x = 0$).  For example, $(0,4)$ lies on the positive $y$-axis whereas $(-117,0)$ lies on the negative $x$-axis.  Such points do not belong to any of the four quadrants.

One of the most important concepts in all of Mathematics is \textbf{symmetry}.\footnote{According to Carl.  Jeff thinks symmetry is overrated.}  There are many types of symmetry in Mathematics, but three of them can be discussed easily using Cartesian Coordinates.

\begin{tcolorbox}
  
\begin{defn}

\label{symmetrydefn}

Two points $(a,b)$ and $(c,d)$ in the plane are said to be

\begin{itemize}

\item \index{symmetry ! about the $x$-axis} \textbf{symmetric about the \boldmath $x$-axis} if $a = c$ and $b = -d$

\item \index{symmetry ! about the $y$-axis} \textbf{symmetric about the \boldmath $y$-axis} if $a = -c$ and $b = d$

\item \index{symmetry ! about the origin} \textbf{symmetric about the origin} if $a = -c$ and $b = -d$

\end{itemize}

\end{defn} 

\end{tcolorbox}


\begin{figure}
\begin{center}

\begin{mfpic}[15]{-5}{5}{-5}{5}
\axes
\tlabel[cc](0.25,-0.35){$0$}
\tlabel[cc](5,-0.5){\scriptsize $x$}
\tlabel[cc](0.5,5){\scriptsize $y$}
\gfill \circle{(4,2),0.1}
\tlabel[cc](4,3){$P(x,y)$}
\gfill \circle{(-4,2),0.1}
\tlabel[cc](-4,3){$Q(-x,y)$}
\gfill \circle{(4,-2),0.1}
\tlabel[cc](4,-3){$S(x,-y)$}
\gfill \circle{(-4,-2),0.1}
\tlabel[cc](-4,-3){$R(-x,-y)$}
\end{mfpic}

\caption{Symmetry of points}
\label{fig:symmetryofpoints}
\end{center}
\end{figure}

Schematically, as shown in \autoref{fig:symmetryofpoints}, $P$ and $S$ are symmetric about the $x$-axis, as are $Q$ and $R$;  $P$ and $Q$ are symmetric about the $y$-axis, as are $R$ and $S$;  and $P$ and $R$ are symmetric about the origin, as are $Q$ and $S$.

\begin{ex}  Let $P$ be the point $(-2,3)$.  Find the points which are symmetric to $P$ about the:

\begin{tasks}(3)
\task  $x$-axis
\task  $y$-axis
\task  origin
\end{tasks}

Check your answer by plotting the points.

{\bf Solution.} The figure after Definition \ref{symmetrydefn} gives us a good way to think about finding symmetric points in terms of taking the opposites of the $x$- and/or $y$-coordinates of $P(-2,3)$.

\begin{tasks}(1)
\task  To find the point symmetric about the $x$-axis, we replace the $y$-coordinate of $3$ with its opposite $-3$ to get  $(-2,-3)$.

\task  To find the point symmetric about the $y$-axis, we replace the $x$-coordinate of $-2$ with its opposite $-(-2) = 2$ to get $(2,3)$.

\task  To find the point symmetric about the origin, we replace both the $x$- and $y$-coordinates with their opposites to get $(2,-3)$.

\end{tasks}

Refer \autoref{fig:pointsofsymmetry2}

\begin{figure}
\begin{center}

\begin{mfpic}[20]{-4}{4}{-4}{4}
\axes
\tlabel[cc](4.1,-0.5){\scriptsize $x$}
\tlabel[cc](0.5,4.1){\scriptsize $y$}
\gfill \circle{(-2,3),0.1}
\tlabel[cc](-2.5,2){\scriptsize $P(-2,3)$}
\gfill \circle{(-2,-3),0.1}
\tlabel[cc](-2.5,-3.7){\scriptsize $(-2,-3)$}
\gfill \circle{(2,3),0.1}
\tlabel[cc](2,2){\scriptsize $(2,3)$}
\gfill \circle{(2,-3),0.1}
\tlabel[cc](2,-3.7){\scriptsize $(2,-3)$}
\xmarks{-3,-2,-1,1,2,3}
\ymarks{-3,-2,-1,1,2,3}
\tlpointsep{5pt}
\scriptsize
\axislabels {x}{{$-3 \hspace{7pt}$} -3, {$-2 \hspace{7pt}$} -2, {$-1 \hspace{7pt}$} -1, {$1$} 1, {$2$} 2, {$3$} 3}
\axislabels {y}{{$-3$} -3, {$-2$} -2, {$-1$} -1, {$1$} 1, {$2$} 2, {$3$} 3}
\normalsize

\end{mfpic}

\caption{Points of symmetry for $(-2, 3)$}
\label{fig:pointsofsymmetry2}
\end{center}
\end{figure}

\vspace{-.4in}

\qed

\end{ex}

One way to visualize the processes in the previous example is with the concept of a \index{reflection ! of a point} \textbf{reflection}.  If we start with our point $(-2,3)$ and pretend that the $x$-axis is a mirror, then the reflection of $(-2,3)$ across the $x$-axis would lie at $(-2,-3)$.  If we pretend that the $y$-axis is a mirror, the reflection of $(-2,3)$ across that axis would be $(2,3)$.  If we reflect across the $x$-axis and then the $y$-axis, we would go from $(-2,3)$ to $(-2,-3)$ then to $(2,-3)$, and so we would end up at the point symmetric to $(-2,3)$ about the origin.  We summarize and generalize this process as shown in \autoref{box:reflectionsinabox}.

\begin{floatbox}[label=box:reflectionsinabox]{Reflections}

To reflect a point $(x,y)$ about the:

\begin{itemize}

\item  $x$-axis, replace $y$ with $-y$.

\item  $y$-axis, replace $x$ with $-x$.

\item  origin, replace $x$ with $-x$ and $y$ with $-y$.

\end{itemize}

\end{floatbox}


\subsection{Distance in the Plane}
\label{Distance}

Another fundamental concept in Geometry is the notion of length.  If we are going to unite Algebra and Geometry using the Cartesian Plane, then we need to develop an algebraic understanding of what distance in the plane means.  Before we can do that, we need to state what we believe is the most important theorem in all of Geometry: \href{http://en.wikipedia.org/wiki/Pythagorean_Theorem}{\underline{The Pythagorean Theorem.}}

\begin{tcolorbox}

\begin{thm} \label{PythagoreanStatement}\index{Pythagorean Theorem}\textbf{The Pythagorean Theorem:}  The triangle $ABC$ shown below is a right triangle if and only if $a^{2} + b^{2} = c^{2}$

\begin{center}

\begin{mfpic}[10]{-5}{5}{-5}{5}
\polyline{(-4.330,0), (4.330,0), (4.330,5), (-4.330,0)}
\tlabel(-5.2,-0.4){$A$}
\tlabel(4,5.1){$B$}
\tlabel(4.5,-0.4){$C$}
\tlabel(0,-1){$b$}
\tlabel(4.7,2.25){$a$}
\tlabel(-1.2,2.9){$c$}
\end{mfpic}

\end{center}

\end{thm}

\end{tcolorbox}

A proof of this theorem will be given in Section \ref{QDensity}.  The theorem actually says two different things.  If we know that $a^{2} + b^{2} = c^{2}$ then the angle $C$ must be a right angle.  If we know geometrically that $C$ is already a right angle then we have that $a^{2} + b^{2} = c^{2}$.  We need the latter statement in the discussion which follows.  

Suppose we have two points, $P\left(x_{\mbox{\tiny$0$}}, y_{\mbox{\tiny$0$}}\right)$ and $Q\left(x_{\mbox{\tiny$1$}}, y_{\mbox{\tiny$1$}}\right),$ in the plane. By the \index{distance ! definition} \textbf{distance} $d$  between $P$ and $Q$, we mean the length of the line segment joining $P$ with $Q$.  (Remember, given any two distinct points in the plane, there is a unique line containing both points.)  Our goal now is to create an algebraic formula to compute the distance between these two points. Consider the generic situation shown first in \autoref{fig:distbetweenpoints}.

\begin{figure}

\begin{center}
\begin{mfpic}[20]{-1}{5}{-1}{4}
\gfill \circle{(0,0),0.1}
\tlabel[c](-1,-1){$P\left(x_{\mbox{\tiny$0$}}, y_{\mbox{\tiny$0$}}\right)$}
\gfill \circle{(4,3),0.1}
\tlabel[c](4.25,3){$Q\left(x_{\mbox{\tiny$1$}}, y_{\mbox{\tiny$1$}}\right)$}
\arrow\reverse\arrow \polyline{(0.1,0.075), (3.9,2.925)}
\tlabel[c](1.25,2.25){$d$}
\end{mfpic}
\end{center}

\begin{center}
\begin{mfpic}[20]{-1}{5}{-1}{4}
\gfill \circle{(0,0),0.1}
\tlabel[c](-1,-1){$P\left(x_{\mbox{\tiny$0$}}, y_{\mbox{\tiny$0$}}\right)$}
\gfill \circle{(4,3),0.1}
\tlabel[c](4.25,3){$Q\left(x_{\mbox{\tiny$1$}}, y_{\mbox{\tiny$1$}}\right)$}
\arrow\reverse\arrow \polyline{(0.1,0.075), (3.9,2.925)}
\tlabel[c](1.25,2.25){$d$}
\dashed \polyline{(0,0), (4,0), (4,3)}
\gfill \circle{(4,0),0.1}
\tlabel[c](4,-1){$\left(x_{\mbox{\tiny$1$}}, y_{\mbox{\tiny$0$}}\right)$}
\polyline{(3.5, 0), (3.5, 0.5), (4, 0.5)}
\end{mfpic}
\end{center}

\caption{Distance between two points}
\label{fig:distbetweenpoints}
\end{figure}

With a little more imagination, we can envision a right triangle whose hypotenuse has length $d$ as drawn above on the right.  From the latter figure, we see that the lengths of the legs of the triangle are $\left|x_{\mbox{\tiny$1$}} - x_{\mbox{\tiny$0$}}\right|$ and $\left|y_{\mbox{\tiny$1$}} - y_{\mbox{\tiny$0$}}\right|$ so the Pythagorean Theorem gives us
 
 \[ \left|x_{\mbox{\tiny$1$}} - x_{\mbox{\tiny$0$}}\right|^2 + \left|y_{\mbox{\tiny$1$}} - y_{\mbox{\tiny$0$}}\right|^2 = d^2\]
 \[ \left(x_{\mbox{\tiny$1$}} - x_{\mbox{\tiny$0$}}\right)^2 + \left(y_{\mbox{\tiny$1$}} - y_{\mbox{\tiny$0$}}\right)^2 = d^2\]
 
(Do you remember why we can replace the absolute value notation with parentheses?)  By extracting the square root of both sides of the second equation and using the fact that distance is never negative, we get
 

\begin{tcolorbox}
  
\begin{eqn} \label{distanceformula}\index{distance ! distance formula}\textbf{The Distance Formula:}  The distance $d$ between the points $P\left(x_{\mbox{\tiny$0$}}, y_{\mbox{\tiny$0$}}\right)$ and $Q\left(x_{\mbox{\tiny$1$}}, y_{\mbox{\tiny$1$}}\right)$ is:
 
\[d = \sqrt{ \left(x_{\mbox{\tiny$1$}} - x_{\mbox{\tiny$0$}}\right)^2 + \left(y_{\mbox{\tiny$1$}} - y_{\mbox{\tiny$0$}}\right)^2} \]

\end{eqn}

\end{tcolorbox}

A couple of remarks about Equation \ref{distanceformula} are in order.  First, it is not always the case that the points $P$ and $Q$ lend themselves to constructing such a triangle.  If the points $P$ and $Q$ are arranged vertically or horizontally, or describe the exact same point, we cannot use the above geometric argument to derive the distance formula.  It is left to the reader in Exercise \ref{distanceothercases} to verify Equation \ref{distanceformula} for these cases.  Second, distance is a `length'.  So, technically, the number we obtain from the distance formula has some attached units of length. In this text, we'll adopt the convention that the phrase `units' refers to some generic units of length.\footnote{As a result, we'll measure area with `square units,' or units$^{2}$ and volume with `cubic units,' or units$^{3}$.}   Our next example gives us an opportunity to test drive the distance formula as well as brush up on some arithmetic and prerequisite algebra.

\begin{ex} \label{distanceexample1} Find and simplify the distance between the following sets of points:

\begin{tasks}(2)
\task $P(-2,3)$ and  $Q(1,-3)$ \vphantom{$R\left( \frac{1}{2}, \frac{2}{3}\right)$ and $S\left( \frac{3}{4}, \frac{1}{5}\right)$}

\task $R\left( \frac{1}{2}, \frac{2}{3}\right)$ and $S\left( \frac{3}{4}, \frac{1}{5}\right)$ 

\task*  $T(\sqrt{3}, -\sqrt{20})$ and $V(\sqrt{12}, \sqrt{5})$

\task   $O(0,0)$ and $P(x,y)$. \vphantom{$T(\sqrt{3}, -\sqrt{20})$ and $V(\sqrt{12}, \sqrt{5})$}

\end{tasks}

\flushleft {\bf Solution.}  In each case, we apply the distance formula, Equation \ref{distanceformula} with the first point listed taken as  $\left(x_{\mbox{\tiny$0$}}, y_{\mbox{\tiny$0$}}\right)$ and the second point taken as $\left(x_{\mbox{\tiny$1$}}, y_{\mbox{\tiny$1$}}\right)$.\footnote{This choice is completely arbitrary.  The reader is encouraged to work these examples taking the first point listed as $\left(x_{\mbox{\tiny$1$}}, y_{\mbox{\tiny$1$}}\right)$ and the second point listed as $\left(x_{\mbox{\tiny$0$}}, y_{\mbox{\tiny$0$}}\right)$ and verifying the distance works out to be the same.  Can you see why the order of the subtraction in Equation \ref{distanceformula} ultimately doesn't matter?}

\begin{enumerate}

\item With $(-2,3) =  \left(x_{\mbox{\tiny$0$}}, y_{\mbox{\tiny$0$}}\right)$ and  $(1,-3) = \left(x_{\mbox{\tiny$1$}}, y_{\mbox{\tiny$1$}}\right)$, we get

\setlength{\extrarowheight}{3pt}

\begin{align*}
 d & = \sqrt{\left(x_{\mbox{\tiny$1$}} - x_{\mbox{\tiny$0$}} \right)^2 + \left(y_{\mbox{\tiny$1$}} - y_{\mbox{\tiny$0$}} \right)^2} \\
   & = \sqrt{ (1-(-2))^2 + (-3-3)^2} \\
   & = \sqrt{9 + 36} \\
   & = \sqrt{45} \\
   & = \sqrt{9 \cdot 5} \\
   & = \begin{array}{lr} \sqrt{9} \sqrt{5} & \text{(For nonnegative numbers, $\sqrt{ab} = \sqrt{a} \sqrt{b}$).} \end{array} \\
   & = 3 \sqrt{5}
\end{align*}

\setlength{\extrarowheight}{2pt}

\medskip

So the distance is $3 \sqrt{5}$ units.

\item With $\left( \frac{1}{2}, \frac{2}{3}\right) =  \left(x_{\mbox{\tiny$0$}}, y_{\mbox{\tiny$0$}}\right)$ and  $\left( \frac{3}{4}, \frac{1}{5}\right) = \left(x_{\mbox{\tiny$1$}}, y_{\mbox{\tiny$1$}}\right)$, we get

\setlength{\extrarowheight}{3pt}

\begin{align*}
 d & = \sqrt{\left(x_{\mbox{\tiny$1$}} - x_{\mbox{\tiny$0$}} \right)^2 + \left(y_{\mbox{\tiny$1$}} - y_{\mbox{\tiny$0$}} \right)^2} \\
   & = \sqrt{ \left(\frac{3}{4}-\frac{1}{2} \right)^2 + \left(\frac{1}{5} - \frac{2}{3} \right)^2} \tag{Get common denominators to add and subtract fractions.}\\
   & = \sqrt{\left(\frac{1}{4} \right)^2 + \left(-\frac{7}{15} \right)^2} \\
   & = \sqrt{\frac{1}{16} + \frac{49}{225}} &&  \tag{ Since $\left(\frac{a}{b}\right)^2 = \frac{a^2}{b^2}$, $b \neq 0$.}\\
   & = \sqrt{\frac{1009}{3600}} \\
   & = \frac{\sqrt{1009}}{\sqrt{3600}} \\
   & = \frac{\sqrt{1009}}{60} \tag{For nonnegative numbers, $\sqrt{\frac{a}{b}} = \frac{\sqrt{a}}{\sqrt{b}}$, $b \neq 0$.}
\end{align*}

\setlength{\extrarowheight}{2pt}

\medskip

So the distance is $\frac{\sqrt{1009}}{60}$ units.

\item With $(\sqrt{3}, -\sqrt{20}) =  \left(x_{\mbox{\tiny$0$}}, y_{\mbox{\tiny$0$}}\right)$ and  $(\sqrt{12}, \sqrt{5}) = \left(x_{\mbox{\tiny$1$}}, y_{\mbox{\tiny$1$}}\right)$, we get

\setlength{\extrarowheight}{3pt}

\begin{align*}
 d & = \sqrt{\left(x_{\mbox{\tiny$1$}} - x_{\mbox{\tiny$0$}} \right)^2 + \left(y_{\mbox{\tiny$1$}} - y_{\mbox{\tiny$0$}} \right)^2} \\
   & = \sqrt{ \left(\sqrt{12}- \sqrt{3} \right)^2 + \left(\sqrt{5} - (-\sqrt{20})\right)^2} \\
   & = \sqrt{\left(2\sqrt{3} - \sqrt{3} \right)^2 + \left(\sqrt{5}+2\sqrt{5} \right)^2} \tag{Simplify the radicals to get like terms.}\\
   & = \sqrt{\left(\sqrt{3}\right)^2 + \left(3 \sqrt{5}\right)^2} \\
   & = \sqrt{3 + 9 \cdot 5} & \tag{Since $(\sqrt{a})^2 = a$ and $(b \sqrt{a})^2 = b^2 (\sqrt{a})^2$.} \\
   & = \sqrt{48} \\
   & = 4\sqrt{3}
\end{align*}

\setlength{\extrarowheight}{2pt}

\medskip

So the distance is $4\sqrt{3}$ units.

\item With $(0, 0) =  \left(x_{\mbox{\tiny$0$}}, y_{\mbox{\tiny$0$}}\right)$ and  $(x,y) = \left(x_{\mbox{\tiny$1$}}, y_{\mbox{\tiny$1$}}\right)$, we get

\setlength{\extrarowheight}{3pt}

\[ \begin{array}{rclr}

 d & = & \sqrt{\left(x_{\mbox{\tiny$1$}} - x_{\mbox{\tiny$0$}} \right)^2 + \left(y_{\mbox{\tiny$1$}} - y_{\mbox{\tiny$0$}} \right)^2} & \\
   & = & \sqrt{ \left(x- 0\right)^2 + \left(y - 0\right)^2} & \\
   & = & \sqrt{x^2+y^2} & \end{array} \]

\setlength{\extrarowheight}{2pt}

As tempting as it may look, $\sqrt{x^2+y^2}$ does not, in general, reduce to $x + y$ or even $|x| + |y|$.  So, in this case, the best we can do is state that the distance is $\sqrt{x^2+y^2}$ units. \qed


\end{enumerate} 

\end{ex}

Related to finding the distance between two points is the problem of finding the \index{midpoint ! definition of} \textbf{midpoint} of the line segment connecting two points.  Given two points, $P\left(x_{\mbox{\tiny$0$}}, y_{\mbox{\tiny$0$}}\right)$ and $Q\left(x_{\mbox{\tiny$1$}}, y_{\mbox{\tiny$1$}}\right)$, the \textbf{midpoint} $M$  of $P$ and $Q$ is defined to be the point on the line segment connecting $P$ and $Q$ whose distance from $P$ is equal to its distance from  $Q$.  

\begin{center}

\begin{mfpic}[15]{-1}{5}{-1}{4}
\gfill \circle{(0,0),0.1}
\tlabel[c](-1,-0.9){$P\left(x_{\mbox{\tiny$0$}}, y_{\mbox{\tiny$0$}}\right)$}
\gfill \circle{(4,3),0.1}
\tlabel[c](4.25,3){$Q\left(x_{\mbox{\tiny$1$}}, y_{\mbox{\tiny$1$}}\right)$}
\polyline{(0,0), (4,3)}
\gfill \circle{(2,1.5),0.1}
\tlabel[c](2,1){$M$}
\end{mfpic}

\end{center}

If we think of reaching $M$ by going `halfway over' and `halfway up' we get the following formula. 

\medskip

\begin{tcolorbox}

\begin{eqn} \index{midpoint ! midpoint formula}\label{midpointformula}\textbf{The Midpoint Formula:}  The midpoint $M$ of the line segment connecting $P\left(x_{\mbox{\tiny$0$}}, y_{\mbox{\tiny$0$}}\right)$ and $Q\left(x_{\mbox{\tiny$1$}}, y_{\mbox{\tiny$1$}}\right)$ is:

\[ M = \left( \dfrac{x_{\mbox{\tiny$0$}} + x_{\mbox{\tiny$1$}}}{2} , \dfrac{y_{\mbox{\tiny$0$}} + y_{\mbox{\tiny$1$}}}{2} \right)\]

\end{eqn}

\end{tcolorbox}

If we let $d$ denote the distance between $P$ and $Q$, we leave it as Exercise \ref{verifymidpointformula} to show that the distance between $P$ and $M$ is $d/2$ which is the same as the distance between $M$ and $Q$.  This suffices to show that Equation \ref{midpointformula} gives the coordinates of the midpoint.

\begin{ex} 

Find the midpoint of the line segment connecting the following pairs of points:  


\begin{tasks}(2)
\task $P(-2,3)$ and  $Q(1,-3)$ \vphantom{$R\left( \frac{1}{2}, \frac{2}{3}\right)$ and $S\left( \frac{3}{4}, \frac{1}{5}\right)$}

\task $R\left( \frac{1}{2}, \frac{2}{3}\right)$ and $S\left( \frac{3}{4}, \frac{1}{5}\right)$ 

\task*  $T(\sqrt{3}, -\sqrt{20})$ and $V(\sqrt{12}, \sqrt{5})$

\task   $O(0,0)$ and $P(x,y)$. \vphantom{$T(\sqrt{3}, -\sqrt{20})$ and $V(\sqrt{12}, \sqrt{5})$}

\end{tasks}

\medskip

{\bf Solution.}  As with Example \ref{distanceexample1}, in each case, we apply the midpoint formula, Equation \ref{midpointformula} with the first point listed taken as  $\left(x_{\mbox{\tiny$0$}}, y_{\mbox{\tiny$0$}}\right)$ and the second point taken as $\left(x_{\mbox{\tiny$1$}}, y_{\mbox{\tiny$1$}}\right)$.\footnote{As in Example \ref{distanceexample1}, this choice is also completely arbitrary.  The reader is encouraged to work these examples taking the first point listed as $\left(x_{\mbox{\tiny$1$}}, y_{\mbox{\tiny$1$}}\right)$ and the second point listed as $\left(x_{\mbox{\tiny$0$}}, y_{\mbox{\tiny$0$}}\right)$ and verifying the midpoint works out to be the same.  Can you see why the order of the points in Equation \ref{midpointformula} doesn't matter?}  We also note that midpoints are \textit{points}, which means all of our answers should be \textit{ordered pairs}.

\begin{enumerate}

\item  With $(-2,3) =  \left(x_{\mbox{\tiny$0$}}, y_{\mbox{\tiny$0$}}\right)$ and  $(1,-3) = \left(x_{\mbox{\tiny$1$}}, y_{\mbox{\tiny$1$}}\right)$, we get

\setlength{\extrarowheight}{10pt}

\[ \begin{array}{rcl}
 M & = & \left( \dfrac{x_{\mbox{\tiny$0$}}+x_{\mbox{\tiny$1$}}}{2},  \dfrac{y_{\mbox{\tiny$0$}}+y_{\mbox{\tiny$1$}}}{2} \right) \\
   & = & \left( \dfrac{(-2)+1}{2},  \dfrac{3+(-3)}{2} \right)  = \left(- \dfrac{1}{2}, \dfrac{0}{2} \right) \\
   & = & \left(- \dfrac{1}{2}, 0 \right) 
   \end{array} \]
   
The midpoint is  $\left(- \frac{1}{2}, 0 \right)$.  

\item With $\left( \frac{1}{2}, \frac{2}{3}\right) =  \left(x_{\mbox{\tiny$0$}}, y_{\mbox{\tiny$0$}}\right)$ and  $\left( \frac{3}{4}, \frac{1}{5}\right) = \left(x_{\mbox{\tiny$1$}}, y_{\mbox{\tiny$1$}}\right)$, we get

\setlength{\extrarowheight}{10pt}

\begin{align*}
 M & = \left( \dfrac{x_{\mbox{\tiny$0$}}+x_{\mbox{\tiny$1$}}}{2},  \dfrac{y_{\mbox{\tiny$0$}}+y_{\mbox{\tiny$1$}}}{2} \right) \\
   & = \left( \dfrac{\frac{1}{2} + \frac{3}{4}}{2},  \dfrac{\frac{2}{3} + \frac{1}{5}}{2} \right)   \\
   & =  \left( \dfrac{\left(\frac{1}{2} + \frac{3}{4}\right) \cdot 4}{2 \cdot 4},  \dfrac{\left(\frac{2}{3} + \frac{1}{5}\right) \cdot 15}{2 \cdot 15} \right) \tag{Simplify compound fractions.} \\ 
   & = \left(\dfrac{5}{8}, \dfrac{13}{30} \right) 
\end{align*}
   
   

The midpoint is $\left(\frac{5}{8}, \frac{13}{30} \right)$.

\item   With $(\sqrt{3}, -\sqrt{20}) =  \left(x_{\mbox{\tiny$0$}}, y_{\mbox{\tiny$0$}}\right)$ and  $(\sqrt{12}, \sqrt{5}) = \left(x_{\mbox{\tiny$1$}}, y_{\mbox{\tiny$1$}}\right)$, we get

\setlength{\extrarowheight}{10pt}

\begin{align*}
 M & = \left( \dfrac{x_{\mbox{\tiny$0$}}+x_{\mbox{\tiny$1$}}}{2},  \dfrac{y_{\mbox{\tiny$0$}}+y_{\mbox{\tiny$1$}}}{2} \right) \\
   & = \left( \dfrac{\sqrt{3} + \sqrt{12}}{2},  \dfrac{-\sqrt{20}+\sqrt{5}}{2} \right) \\
   & =  \left( \dfrac{\sqrt{3}+2\sqrt{3}}{2},  \dfrac{-2\sqrt{5} + \sqrt{5}}{2} \right) \tag{Simplify radicals to get like terms.} \\ 
   & = \left(\dfrac{3\sqrt{3}}{2}, -\dfrac{\sqrt{5}}{2} \right)
\end{align*}
   
   The midpoint is $\left(\frac{3\sqrt{3}}{2}, -\frac{\sqrt{5}}{2} \right)$.

\item With $(0,0) =  \left(x_{\mbox{\tiny$0$}}, y_{\mbox{\tiny$0$}}\right)$ and  $(x,y) = \left(x_{\mbox{\tiny$1$}}, y_{\mbox{\tiny$1$}}\right)$, we get

\setlength{\extrarowheight}{10pt}

\[ \begin{array}{rclr}
 M & = & \left( \dfrac{x_{\mbox{\tiny$0$}}+x_{\mbox{\tiny$1$}}}{2},  \dfrac{y_{\mbox{\tiny$0$}}+y_{\mbox{\tiny$1$}}}{2} \right) & \\
   & = & \left( \dfrac{x+0}{2},  \dfrac{y+0}{2} \right) &   \\
   & = &  \left( \dfrac{x}{2},  \dfrac{y}{2} \right) &  \end{array} \]
   
 The midpoint is $\left(\frac{x}{2}, \frac{y}{2} \right)$. \qed

\end{enumerate}

\end{ex}

\phantomsection

\label{inversemidpoint}

We close with a more abstract application of the Midpoint Formula.  We will expand upon this example in Example \ref{inversemidpointex2} in Section \ref{AppLines}.  

\begin{ex} \label{inversemidpointex1} If $a \neq b$, show that the line $y = x$ equally divides the line segment with endpoints $(a,b)$ and $(b,a)$.

\medskip

{\bf Solution.}  To prove the claim, we use Equation \ref{midpointformula} to find the midpoint  

\setlength{\extrarowheight}{10pt}

\[ \begin{array}{rcl}

 M & = & \left( \dfrac{a+b}{2},  \dfrac{b+a}{2} \right) \\
   & = & \left( \dfrac{a+b}{2},  \dfrac{a+b}{2} \right)  \\ \end{array} \]

Since the $x$ and $y$ coordinates of this point are the same, we find that the midpoint lies on the line $y=x$, as required. \qed

\end{ex}

\setlength{\extrarowheight}{2pt}

\clearpage

\subsection{Exercises}

\label{ExercisesforAppCartesianPlane}

\begin{enumerate}

\item Plot and label the points $\;A(-3, -7)$,  $\;B(1.3, -2)$,  $C(\pi, \sqrt{10})$,  $\;D(0, 8)$,  $\;E(-5.5, 0)$,  $\;F(-8, 4)$, $\;G(9.2, -7.8)$ and $H(7, 5)$ in the Cartesian Coordinate Plane given below. 
 
\label{cartexerciseone}

\begin{center}

\begin{mfpic}[12]{-10}{10}{-10}{10}
\axes
\tlabel[cc](10,-0.5){\scriptsize $x$}
\tlabel[cc](0.5,10){\scriptsize $y$}
\xmarks{-9,-8,-7,-6,-5,-4,-3,-2,-1,1,2,3,4,5,6,7,8,9}
\ymarks{-9,-8,-7,-6,-5,-4,-3,-2,-1,1,2,3,4,5,6,7,8,9}
\tlpointsep{5pt}
\scriptsize
\axislabels {x}{{$-9 \hspace{7pt}$} -9, {$-8 \hspace{7pt}$} -8, {$-7 \hspace{7pt}$} -7, {$-6 \hspace{7pt}$} -6, {$-5 \hspace{7pt}$} -5, {$-4 \hspace{7pt}$} -4, {$-3 \hspace{7pt}$} -3, {$-2 \hspace{7pt}$} -2, {$-1 \hspace{7pt}$} -1, {$1$} 1, {$2$} 2, {$3$} 3, {$4$} 4, {$5$} 5, {$6$} 6, {$7$} 7, {$8$} 8, {$9$} 9}
\axislabels {y}{{$-9$} -9, {$-8$} -8, {$-7$} -7, {$-6$} -6, {$-5$} -5, {$-4$} -4, {$-3$} -3, {$-2$} -2, {$-1$} -1, {$1$} 1, {$2$} 2, {$3$} 3, {$4$} 4, {$5$} 5, {$6$} 6, {$7$} 7, {$8$} 8, {$9$} 9}
\normalsize
\end{mfpic}

\end{center}

\item \label{quadsymmpointexercise} For each point given in Exercise \ref{cartexerciseone} above

\begin{itemize}
\item Identify the quadrant or axis in/on which the point lies.
\item Find the point symmetric to the given point about the $x$-axis.
\item Find the point symmetric to the given point about the $y$-axis.
\item Find the point symmetric to the given point about the origin.

\end{itemize}

\setcounter{HW}{\value{enumi}}

\end{enumerate}

In Exercises \ref{distmidfirst} - \ref{distmidlast}, find the distance $d$ between the points and the midpoint $M$ of the line segment which connects them.


\begin{multicols}{2}
\begin{enumerate}
\setcounter{enumi}{\value{HW}}

\item $(1,2)$, $(-3,5)$ \label{distmidfirst}
\item $(3, -10)$, $(-1, 2)$ 

\setcounter{HW}{\value{enumi}}
\end{enumerate}
\end{multicols}

\begin{multicols}{2}
\begin{enumerate}
\setcounter{enumi}{\value{HW}}

\item $\left( \dfrac{1}{2}, 4\right)$, $\left(\dfrac{3}{2}, -1\right)$ 
\item $\left(- \dfrac{2}{3}, \dfrac{3}{2} \right)$, $\left(\dfrac{7}{3}, 2\right)$ 

\setcounter{HW}{\value{enumi}}
\end{enumerate}
\end{multicols}


\begin{multicols}{2}
\begin{enumerate}
\setcounter{enumi}{\value{HW}}

\item  $\left( \dfrac{24}{5}, \dfrac{6}{5} \right)$, $\left( -\dfrac{11}{5}, -\dfrac{19}{5} \right)$.
\item $\left(\sqrt{2}, \sqrt{3}\right)$, $\left(-\sqrt{8}, -\sqrt{12}\right)$ \vphantom{$\left( \dfrac{6}{5} \right)$}

\setcounter{HW}{\value{enumi}}
\end{enumerate}
\end{multicols}

\begin{enumerate}
\setcounter{enumi}{\value{HW}}

\item  $\left(2 \sqrt{45}, \sqrt{12} \right)$, $\left(\sqrt{20}, \sqrt{27} \right)$. \vphantom{$\left(-\dfrac{\sqrt{3}}{2}, \dfrac{1}{2} \right)$, $\left(\dfrac{\sqrt{3}}{2}, -\dfrac{1}{2} \right)$ }
\item $\left(-\dfrac{\sqrt{3}}{2}, \dfrac{1}{2} \right)$, $\left(\dfrac{\sqrt{3}}{2}, -\dfrac{1}{2} \right)$ \label{distmidlast}

\setcounter{HW}{\value{enumi}}
\end{enumerate}

\begin{enumerate}
\setcounter{enumi}{\value{HW}}

\item Let's assume that we are standing at the origin and the positive $y$-axis points due North while the positive $x$-axis points due East.  Our Sasquatch-o-meter tells us that Sasquatch is 3 miles West and 4 miles South of our current position.  What are the coordinates of his position?  How far away is he from us?  If he runs 7 miles due East what would his new position be?

\item \label{distanceothercases} Verify the Distance Formula \ref{distanceformula} for the cases when:

\begin{enumerate}

\item The points are arranged vertically.  (Hint: Use $P(a, y_{\mbox{\tiny$0$}})$ and $Q(a, y_{\mbox{\tiny$1$}})$.)
\item The points are arranged horizontally. (Hint: Use $P(x_{\mbox{\tiny$0$}}, b)$ and $Q(x_{\mbox{\tiny$1$}}, b)$.)
\item The points are actually the same point. (You shouldn't need a hint for this one.)

\end{enumerate}

\item \label{verifymidpointformula} Verify the Midpoint Formula by showing the distance between $P(x_{\mbox{\tiny$1$}}, y_{\mbox{\tiny$1$}})$ and $M$ and the distance between $M$ and $Q(x_{\mbox{\tiny$2$}}, y_{\mbox{\tiny$2$}})$ are both half of the distance between $P$ and $Q$. 

\item Show that the points $A$, $\;B$ and $C$ below are the vertices of a right triangle.

\begin{multicols}{2}

\begin{enumerate}

\item  $A(-3,2)$, $\;B(-6,4)$, and $C(1,8)$

\item   $A(-3, 1)$, $\;B(4, 0)$ and $C(0, -3)$


\end{enumerate}
\end{multicols}

\item Find a point $D(x, y)$ such that the points $A(-3, 1). \, B(4, 0), \, C(0, -3)$ and $D$ are the corners of a square.  Justify your answer.

\item  Suppose the distance between  $C(h,k)$ and $P(x,y)$ is $r$.  Use the distance formula to show \[(x-h)^2 + (y-k)^2 = r^2\]

We will see this formula (and its cousins) in Chapter \ref{TheConicSections}.


\item  Let $P(x,y)$ be a point in the plane and let $Q$ be the result of reflecting $P$ about the $x$-axis, $y$-axis, or origin.  Show the distance from the origin to $P$ is the same as the distance from the origin to $Q$.  


\item \label{scalingdistance} Let $O(0,0)$ (that is, $O$ is the origin),  $P(-2,1)$, $Q(-4,2)$, and $R(6,-3)$.  

\begin{enumerate}

\item  Find the distance from $O$ to $P$ and from $O$ to $Q$.  What do you notice?

\item  Find the distance from $O$ to $P$ and from $O$ to $R$.  What do you notice?

\item  For a generic point $P(x,y)$, let $Q(kx, ky)$ be the point obtained from $P$ by multiplying both the $x$ and $y$ coordinates of $P$ by the same number, $k$.  Show the distance from $O$ to $Q$ is exactly $|k|$ times the distance from $O$ to $P$.  Explain what these results mean geometrically. (We'll revisit this in Theorem \ref{magdirprops} in Section \ref{Vectors}.)

\end{enumerate}


\item \label{distancemetricprops} In this exercise, we explore some of the properties of distance.  For brevity, we'll adopt the notation `$d(P,Q)$' to denote the distance between points $P$ and $Q$.
\begin{enumerate}  

\item  (Non-negative Property) Explain why $d(P,Q) \geq 0$ for any two points in the plane.

\item  (Symmetric Property) Explain why $d(P,Q) = d(Q,P)$ for any two points in the plane.

\item  (Identity Property) Show that $d(P,Q) = 0$ \underline{if and only if} $P$ and $Q$ are the same point.

\textbf{NOTE:}  The phrase `if and only if' means you need to show two things:

\begin{itemize}

\item  If $P$ and $Q$ are the same point, then $d(P,Q) = 0$.
\item  If $d(P,Q) = 0$, then $P$ and $Q$ are the same point.


\end{itemize}




\item  (Triangle Inequality) The \href{http://en.wikipedia.org/wiki/Triangle_inequality}{\underline{Triangle Inequality}} says that for any triangle, the sum of the lengths of two sides of a triangle always exceeds the length of the third.  Use the Triangle Inequality to show that for any three points $P$, $Q$, and $R$, \[ d(P,R) \leq d(P,Q) + d(Q,R) \]

Under what conditions does $d(P,R) = d(P,Q) + d(Q,R)$?

\end{enumerate} 



\item \label{taxidistance} (Another way to measure distance.) In this text, we defined the distance between two points as the length of the line segment connecting the two points.  Depending on the situation, however, there may be better ways to describe how far one location is from another.  Consider the situation below on the left.  Suppose $P$ and $Q$ are locations on a city grid, and a taxi is hailed at point $P$ to travel to point $Q$.  In this situation, diagonal movement is impossible,\footnote{Maybe `discouraged' or `difficult' would be better word choices.} so the taxi is limited to traveling horizontally and vertically.  

\begin{center}
\begin{tabular}{cc}

\begin{mfpic}[20]{-1}{5}{-1}{4}

\drawcolor[gray]{0.7}

\drawcolor[gray]{0.7}
\polyline{(0,0.75), (0,3.25)}
\polyline{(1,0.75), (1,3.25)}
\polyline{(2,0.75), (2,3.25)}
\polyline{(3,0.75), (3,3.25)}
\polyline{(0,1), (3.25,1)}
\polyline{(0,2), (3.25,2)}
\polyline{(0,3), (3.25,3)}
\point[3pt]{(0,1), (3,3)}
\tlabel[cc](0,0.25){\scriptsize $P\left(x_{\mbox{\tiny$0$}}, y_{\mbox{\tiny$0$}}\right)$}
\tlabel[cc](3,3.75){\scriptsize $Q\left(x_{\mbox{\tiny$1$}}, y_{\mbox{\tiny$1$}}\right)$}

\end{mfpic} & 
\begin{mfpic}[20]{-1}{5}{-1}{4}

\drawcolor[gray]{0.7}
\polyline{(0,0.75), (0,3.25)}
\polyline{(1,0.75), (1,3.25)}
\polyline{(2,0.75), (2,3.25)}
\polyline{(3,0.75), (3,3.25)}
\polyline{(0,1), (3.25,1)}
\polyline{(0,2), (3.25,2)}
\polyline{(0,3), (3.25,3)}
\point[3pt]{(0,1), (3,3), (3,1)}
\tlabel[cc](1.5,0.25){\scriptsize $\left|x_{\mbox{\tiny$1$}} - x_{\mbox{\tiny$0$}}\right|$}
\tlabel[cc](4.25,2){\scriptsize $\left|y_{\mbox{\tiny$1$}} - y_{\mbox{\tiny$0$}}\right|$}
\tlabel[cc](0,0.5){\scriptsize $P$}
\tlabel[cc](3,3.5){\scriptsize $Q$}
\drawcolor[gray]{0.0}
\arrow \reverse \arrow \polyline{(0.1,1), (2.9,1)}
\arrow \reverse \arrow \polyline{(3,1.1), (3,2.9)}

\end{mfpic} \\ 

\end{tabular}
\end{center}

\medskip

From the diagram, we see the horizontal distance  is $\left|x_{\mbox{\tiny$1$}} - x_{\mbox{\tiny$0$}}\right|$ and the vertical distance is $\left|y_{\mbox{\tiny$1$}} - y_{\mbox{\tiny$0$}}\right|$, so the total distance the taxi needs to travel to get from $P$ to $Q$ is given by:

\[ d_{T} =  \left|x_{\mbox{\tiny$1$}} - x_{\mbox{\tiny$0$}}\right| + \left|y_{\mbox{\tiny$1$}} - y_{\mbox{\tiny$0$}}\right| \]

We call $d_{T}$ the `taxi distance' from $P$ to $Q$.  

\begin{enumerate}

\item  Let $P(-2,3)$ and $Q(4,2)$.  Find the distance, $d$ from $P$ to $Q$ and the taxi distance, $d_{T}$ from $P$ to $Q$.  Repeat this exercise with several points of your own choosing.  Which is larger, $d$ or $d_{T}$? 

\item Using the notation of Exercise \ref{distancemetricprops}, show that $d(P,Q) \leq d_{T}(P,Q)$ for any two points $P$ and $Q$ in the plane.  (The \href{http://en.wikipedia.org/wiki/Triangle_inequality}{\underline{Triangle Inequality}} is useful once again here.)  Under what conditions is $d(P,Q) = d_{T}(P,Q)$? 
 
\item  Repeat Exercise \ref{distancemetricprops} with the taxi distance, $d_{T}$.  (You may need to skip ahead to Exercise \ref{triangleinequalityreals} in Section \ref{AbsoluteValueFunctions} to verify the Triangle Inequality piece.)

\item Think about ways to define a `midpoint' using the taxi distance.  What would your formula be?  To help you get started, play around with the origin $(0,0)$ as one point and the point $(4,2)$ as the other.

\end{enumerate}


\item \label{orderedtripleexercise} The world is not flat.\footnote{There are those who disagree with this statement.  Look them up on the Internet some time when you're bored.}  Thus the Cartesian Plane cannot possibly be the end of the story.  Discuss with your classmates how you would extend Cartesian Coordinates to represent the three dimensional world.  What would the Distance and Midpoint formulas look like, assuming those concepts make sense at all?

\end{enumerate}


\clearpage

\subsection{Answers}

\begin{enumerate} 
\item The required points $\;A(-3, -7)$, $\;B(1.3, -2)$, $\;C(\pi, \sqrt{10})$, $\;D(0, 8)$, $\;E(-5.5, 0)$, $\;F(-8, 4)$, $\;G(9.2, -7.8)$, and $H(7, 5)$ are plotted in the Cartesian Coordinate Plane shown in \autoref{fig:plottingpoints2}. 

\begin{figure}
\begin{center}

\begin{mfpic}[12]{-10}{10}{-10}{10}
\axes
\tlabel[cc](10,-0.5){\scriptsize $x$}
\tlabel[cc](0.5,10){\scriptsize $y$}
\xmarks{-9,-8,-7,-6,-5,-4,-3,-2,-1,1,2,3,4,5,6,7,8,9}
\ymarks{-9,-8,-7,-6,-5,-4,-3,-2,-1,1,2,3,4,5,6,7,8,9}
\gfill \circle{(-3, -7),0.1}
\tlabel[cc](-3, -7.75){$A(-3,-7)$}
\gfill \circle{(1.3,-2),0.1}
\tlabel[cc](1.5, -2.5){$B(1.3, -2)$}
\gfill \circle{(3.14159, 3.16228),0.1}
\tlabel[cc](3.14, 2.7){$C(\pi, \sqrt{10})$}
\gfill \circle{(0, 8),0.1}
\tlabel[cc](1.25, 8){$D(0, 8)$}
\gfill \circle{(-5.5,0),0.1}
\tlabel[cc](-5.5, 0.5){$E(-5.5,0)$}
\gfill \circle{(-8,4),0.1}
\tlabel[cc](-8, 3.5){$F(-8, 4)$}
\gfill \circle{(9.2,-7.8),0.1}
\tlabel[cc](9.2, -8.3){$G(9.2, -7.8)$}
\gfill \circle{(7 ,5),0.1}
\tlabel[cc](7, 5.5){$H(7, 5)$}
\tlpointsep{5pt}
\scriptsize
\axislabels {x}{{$-9 \hspace{7pt}$} -9, {$-8 \hspace{7pt}$} -8, {$-7 \hspace{7pt}$} -7, {$-6 \hspace{7pt}$} -6, {$-5 \hspace{7pt}$} -5, {$-4 \hspace{7pt}$} -4, {$-3 \hspace{7pt}$} -3, {$-2 \hspace{7pt}$} -2, {$-1 \hspace{7pt}$} -1, {$1$} 1, {$2$} 2, {$3$} 3, {$4$} 4, {$5$} 5, {$6$} 6, {$7$} 7, {$8$} 8, {$9$} 9}
\axislabels {y}{{$-9$} -9, {$-8$} -8, {$-7$} -7, {$-6$} -6, {$-5$} -5, {$-4$} -4, {$-3$} -3, {$-2$} -2, {$-1$} -1, {$1$} 1, {$2$} 2, {$3$} 3, {$4$} 4, {$5$} 5, {$6$} 6, {$7$} 7, {$8$} 8, {$9$} 9}
\normalsize
\end{mfpic}

\caption{Plotting points}
\label{fig:plottingpoints2}
\end{center}
\end{figure}

\item \begin{multicols}{2}

\begin{enumerate}

\item The point $A(-3, -7)$ is 

\begin{itemize}

\item in Quadrant III
\item symmetric about $x$-axis with $(-3, 7)$
\item symmetric about $y$-axis with $(3, -7)$
\item symmetric about origin with $(3, 7)$

\end{itemize}

\item The point $B(1.3, -2)$ is 

\begin{itemize}

\item in Quadrant IV
\item symmetric about $x$-axis with $(1.3, 2)$
\item symmetric about $y$-axis with $(-1.3, -2)$
\item symmetric about origin with $(-1.3, 2)$

\end{itemize}

\setcounter{HWindent}{\value{enumii}}
\end{enumerate}
\end{multicols}

\begin{multicols}{2}
\begin{enumerate}
\setcounter{enumii}{\value{HWindent}}

\item The point $C(\pi, \sqrt{10})$ is 

\begin{itemize}

\item in Quadrant I
\item symmetric about $x$-axis with {\small $(\pi, -\sqrt{10})$}
\item symmetric about $y$-axis with {\small $(-\pi, \sqrt{10})$}
\item symmetric about origin with {\scriptsize $(-\pi, -\sqrt{10})$}

\end{itemize}

\item The point $D(0, 8)$ is 

\begin{itemize}

\item on the positive $y$-axis
\item symmetric about $x$-axis with $(0, -8)$
\item symmetric about $y$-axis with $(0, 8)$
\item symmetric about origin with $(0, -8)$

\end{itemize}


\setcounter{HWindent}{\value{enumii}}
\end{enumerate}
\end{multicols}

\begin{multicols}{2}
\begin{enumerate}
\setcounter{enumii}{\value{HWindent}}

\item The point $E(-5.5, 0)$ is 

\begin{itemize}

\item on the negative $x$-axis
\item symmetric about $x$-axis with $(-5.5, 0)$
\item symmetric about $y$-axis with $(5.5, 0)$
\item symmetric about origin with $(5.5, 0)$

\end{itemize}

\item The point $F(-8, 4)$ is 

\begin{itemize}

\item in Quadrant II
\item symmetric about $x$-axis with $(-8, -4)$
\item symmetric about $y$-axis with $(8, 4)$
\item symmetric about origin with $(8, -4)$

\end{itemize}

\setcounter{HWindent}{\value{enumii}}
\end{enumerate}
\end{multicols}

\begin{multicols}{2}
\begin{enumerate}
\setcounter{enumii}{\value{HWindent}}

\item The point $G(9.2, -7.8)$ is 

\begin{itemize}

\item in Quadrant IV
\item symmetric about $x$-axis with $(9.2, 7.8)$
\item symmetric about $y$-axis with {\scriptsize $(-9.2, -7.8)$}
\item symmetric about origin with $(-9.2, 7.8)$

\end{itemize}

\item The point $H(7, 5)$ is 

\begin{itemize}

\item in Quadrant I
\item symmetric about $x$-axis with $(7, -5)$
\item symmetric about $y$-axis with $(-7, 5)$
\item symmetric about origin with $(-7, -5)$

\end{itemize}

\end{enumerate}
\end{multicols}
\setcounter{HW}{\value{enumi}}
\end{enumerate}


\begin{multicols}{2}
\begin{enumerate}
\setcounter{enumi}{\value{HW}}

\item $d = 5$ units, $M = \left(-1, \frac{7}{2} \right)$
\item $d = 4 \sqrt{10}$ units, $M = \left(1, -4 \right)$

\setcounter{HW}{\value{enumi}}
\end{enumerate}
\end{multicols}

\begin{multicols}{2}
\begin{enumerate}
\setcounter{enumi}{\value{HW}}

\item $d = \sqrt{26}$ units, $M = \left(1, \frac{3}{2} \right)$
\item $d= \frac{\sqrt{37}}{2}$ units, $M = \left(\frac{5}{6}, \frac{7}{4} \right)$

\setcounter{HW}{\value{enumi}}
\end{enumerate}
\end{multicols}

\begin{multicols}{2}
\begin{enumerate}
\setcounter{enumi}{\value{HW}}

\item  $d = \sqrt{74}$ units, $M = \left(\frac{13}{10}, -\frac{13}{10} \right)$ \vphantom{$\left( \frac{\sqrt{3}}{2} \right)$}
\item $d= 3\sqrt{5}$ units, $M = \left(-\frac{\sqrt{2}}{2}, -\frac{\sqrt{3}}{2} \right)$

\setcounter{HW}{\value{enumi}}
\end{enumerate}
\end{multicols}

\begin{multicols}{2}
\begin{enumerate}
\setcounter{enumi}{\value{HW}}

\item  $d = \sqrt{83}$ units, $M = \left(4 \sqrt{5}, \frac{5 \sqrt{3}}{2} \right)$
\item $d = 2$ units, $M = \left( 0, 0\right)$ \vphantom{$\left( \frac{\sqrt{3}}{2} \right)$}

\setcounter{HW}{\value{enumi}}
\end{enumerate}
\end{multicols}


\begin{enumerate}
\setcounter{enumi}{\value{HW}}

\item $(-3, -4)$, $5$ miles, $(4, -4)$


\addtocounter{enumi}{2}

\item  \begin{enumerate}  

\item  The distance from $A$ to $B$ is $|AB| = \sqrt{13}$, the distance from $A$ to $C$ is $|AC| = \sqrt{52}$, and the distance from $B$ to $C$ is $|BC| = \sqrt{65}$.  Since $\left(\sqrt{13}\right)^2 + \left( \sqrt{52} \right)^2 = \left( \sqrt{65} \right)^2$, we are guaranteed by the \href{http://en.wikipedia.org/wiki/Pythagorean_theorem\#Converse}{\uline{converse of the Pythagorean Theorem}} that the triangle is a right triangle.
\item Show that $|AC|^{2} + |BC|^{2} = |AB|^{2}$

\end{enumerate}

\end{enumerate}

\normalsize

\closegraphsfile