\label{ExercisesforAppVariation}

In Exercises \ref{varexercisefirst} - \ref{varexerciselast},  translate the following into mathematical equations.

\begin{enumerate}
%\setcounter{enumi}{\value{HW}}

\item  At a constant pressure, the temperature $T$ of an ideal gas is directly proportional to its volume $V$.  (This is \href{http://en.wikipedia.org/wiki/Charles's_law}{\underline{Charles's Law}}) \index{Charles's Law} \label{varexercisefirst}

\item  The frequency of a wave $f$ is inversely proportional to the wavelength of the wave $\lambda$.

\item  The density $d$ of a material is directly proportional to the mass of the object $m$ and inversely proportional to its volume $V$.

\item  The square of the orbital period of a planet $P$ is directly proportional to the cube of the semi-major axis of its orbit $a$. (This is \href{http://en.wikipedia.org/wiki/Kepler}{\underline{Kepler's Third Law of Planetary Motion }}) \index{Kepler's Third Law of Planetary Motion}

\item  The drag of an object traveling through a fluid $D$ varies jointly with the density of the fluid $\rho$ and the square of the velocity of the object $\nu$.

\item Suppose two electric point charges, one with charge $q$ and one with charge $Q$, are positioned $r$ units apart. The electrostatic force $F$ exerted on the charges varies directly with the product of the two charges and inversely with the square of the distance between the charges. (This is \href{http://en.wikipedia.org/wiki/Electrostatic\#Coulomb.27s_law}{\underline{Coulomb's Law}}) \index{Coulomb's Law} \label{varexerciselast}

\setcounter{HW}{\value{enumi}}
\end{enumerate}

\begin{enumerate}
\setcounter{enumi}{\value{HW}}

\item According to \href{http://en.wikipedia.org/wiki/Vibrating_string}{\underline{this webpage}}, the frequency $f$ of a vibrating string is given by $f = \dfrac{1}{2L} \sqrt{\dfrac{T}{\mu}}$ where $T$ is the tension, $\mu$ is the linear mass\footnote{Also known as the linear density.  It is simply a measure of mass per unit length.} of the string and $L$ is the length of the vibrating part of the string.  Express this relationship using the language of variation.

\item According to the Centers for Disease Control and Prevention \href{http://www.cdc.gov}{\underline{www.cdc.gov}}, a person's Body Mass Index $B$ is directly proportional to his weight $W$ in pounds and inversely proportional to the square of his height $h$ in inches. \index{BMI, body mass index}

\begin{enumerate}

\item Express this relationship as a mathematical equation. \label{BMIfirst} 
\item If a person who was $5$ feet, $10$ inches tall weighed 235 pounds had a Body Mass Index of 33.7, what is the value of the constant of proportionality? \label{BMIsecond}
\item Rewrite the mathematical equation found in part \ref{BMIfirst} to include the value of the constant found in part \ref{BMIsecond} and then find your Body Mass Index.

\end{enumerate}

\item This exercise refers back to the volume of a right circular cone formula found in Example \ref{variationexample}.  

\begin{enumerate}

\item \label{coneexercisenounits} First assume that $V$, $h$ and $r$ are all measured using the same unit of length.  Work with your classmates to show that in this case, the $k$ needed for the volume formula $V = k h r^{2}$ has no units on it.

\item \label{coneexercisebadunits} Now assume that $V$ is measured in milliliters, $h$ is measured in meters and $r$ is measured in yards.  Work with your classmates to find the units on $k$ so that the volume formula $V = k h r^{2}$ makes sense.

\end{enumerate}

\item We know that the circumference of a circle varies directly with its radius with $2\pi$ as the constant of proportionality. (That is, we know $C = 2\pi r.$)  With the help of your classmates, compile a list of other basic geometric relationships which can be seen as variations.

\item \label{idealgasexercise} Research the Ideal Gas Law $PV = nRT$ to see what sorts of units are used for the constant $R$.  What other formulations of this law did you find in your research?

\end{enumerate}

\newpage

\subsection{Answers}

\begin{multicols}{3}
\begin{enumerate}
%\setcounter{enumi}{\value{HW}}

\item $T = k V$

\item \hspace{-.1in} \footnote{The character $\lambda$ is the lower case Greek letter `lambda.'} $f = \dfrac{k}{\lambda}$

\item $d = \dfrac{k m}{V}$ 

\setcounter{HW}{\value{enumi}}
\end{enumerate}
\end{multicols}


\begin{multicols}{3}
\begin{enumerate}
\setcounter{enumi}{\value{HW}}

\item $P^2 = k a^3$

\item \hspace{-.1in} \footnote{The characters $\rho$ and $\nu$ are the lower case Greek letters `rho' and `nu,' respectively.} $D = k \rho \nu^2$

\item \hspace{-.1in} \footnote{Note the similarity to this formula and Newton's Law of Universal Gravitation as discussed in Example \ref{gravitylaw}.}  $F = \dfrac{kqQ}{r^2}$   

\setcounter{HW}{\value{enumi}}
\end{enumerate}
\end{multicols}

\begin{enumerate}
\setcounter{enumi}{\value{HW}}

\item Rewriting $f = \dfrac{1}{2L} \sqrt{\dfrac{T}{\mu}}$ as $f = \dfrac{\frac{1}{2} \sqrt{T}}{L \sqrt{\mu}}$ we see that the frequency $f$ varies directly with the square root of the tension and varies inversely with the length and the square root of the linear mass.

\item \begin{multicols}{3} 
\begin{enumerate}
\item $B = \dfrac{kW}{h^{2}}$
\item \hspace{-.1in} \footnote{The CDC uses 703.} $k = 702.68$ 
\item $B = \dfrac{702.68W}{h^{2}}$
\end{enumerate}
\end{multicols}

\end{enumerate}