\mfpicnumber{1}

\opengraphsfile{AppCmpNums}

\setcounter{footnote}{0}

\label{AppCmpNums}

The results of Section \ref{AppQuadEqus} tell us that the equation $x^{2} + 1 = 0$ has no real number solutions. However, it \emph{would} have solutions if we could make sense of $\sqrt{-1}$. The \textbf{Complex Numbers}\index{numbers ! complex}\index{complex numbers} do just that - they give us a mechanism for working with $\sqrt{-1}$.  As such, the set of complex numbers fill in an algebraic gap left by the set of real numbers.  

Here's the basic plan.  There is no real number $x$ with $x^2 = -1$, since for any real number $x^2 \geq 0$.  However, we could formally extract square roots and write $x = \pm \sqrt{-1}$.  We build the complex numbers by relabeling the quantity $\sqrt{-1}$ as $i$, the unfortunately misnamed \index{imaginary unit, $i$}\index{complex number ! imaginary unit, $i$}\textbf{imaginary unit}.\footnote{Some Technical Mathematics textbooks label it `$j$'.  While it carries the adjective `imaginary', these numbers have essential real-world implications.  For example, every electronic device owes its existence to the study of `imaginary' numbers.}  The number $i$, while not a real number, is defined so that it plays along well with real numbers and acts very much like any other radical expression.  For instance, $3(2i) = 6i$, $7i-3i = 4i$, $(2-7i) + (3 + 4i) = 5-3i$, and so forth.  The key properties which distinguish $i$ from the real numbers are listed below.

\begin{tcolorbox}

\begin{defn} \label{idefn} The imaginary unit $i$ satisfies the two following properties:

\begin{enumerate}

\item  $i^2 = -1$

\item  If $c$ is a real number with $c \geq 0$ then $\sqrt{-c} = i \sqrt{c}$

\end{enumerate}

\end{defn}

\end{tcolorbox}

Property 1 in Definition \ref{idefn} establishes that $i$ does act as a square root\footnote{Note the use of the indefinite article `a'.  Whatever beast is chosen to be $i$, $-i$ is the other square root of $-1$.} of $-1$, and property 2 establishes what we mean by the `principal square root' of a negative real number.  In property 2, it is important to remember the restriction on $c$.  For example, it is perfectly acceptable to say  $\sqrt{-4} = i \sqrt{4} = i(2) = 2i$. However, $\sqrt{-(-4)} \neq i \sqrt{-4}$, otherwise, we'd get\[ 2 = \sqrt{4} = \sqrt{-(-4)} = i \sqrt{-4} = i (2i) = 2i^2 = 2(-1) = -2,\] which is unacceptable. The moral of this story is that the general properties of radicals do not apply for even roots of negative quantities.  With Definition \ref{idefn} in place, we can define the set of \index{complex number ! definition of}\textbf{complex numbers}.

\begin{tcolorbox}

\begin{defn} \label{complexdefn} A \textbf{complex number} is a number of the form $a+bi$, where $a$ and $b$ are real numbers and $i$ is the imaginary unit.  The set of complex numbers is denoted $\C$.
\end{defn}

\end{tcolorbox}

Complex numbers include things you'd normally expect, like $3+2i$ and $\frac{2}{5} - i\sqrt{3}$.  However, don't forget that $a$ or $b$ could be zero, which means numbers like $3i$ and $6$ are also complex numbers.  In other words, don't forget that the complex numbers \textit{include} the real numbers,\footnote{To use the language  of Section \ref{SetsofRealNumbers}, $\R \subseteq \C$.} so $0$ and $\pi - \sqrt{21}$ are both considered complex numbers.   The arithmetic of complex numbers is as you would expect.  The only things you need to remember are the two properties in Definition \ref{idefn}.  The next example should help recall how these animals behave.

\begin{ex} \label{complexzeroex1} Perform the indicated operations.
\label{complexnumberarithmetic}

\begin{tasks}(2)
\task  $(1-2i) - (3+4i)$ \vphantom{$\dfrac{1-2i}{3-4i}$}
\task  $(1-2i)(3+4i)$ \vphantom{$\dfrac{1-2i}{3-4i}$}
\task  $\dfrac{1-2i}{3-4i}$
\task  $\sqrt{-3} \sqrt{-12}$
\task  $\sqrt{(-3)(-12)}$
\task  $(x-[1+2i])(x-[1-2i])$
\end{tasks}

{\bf Solution.} 

\begin{enumerate}

\item  As mentioned earlier, we treat expressions involving $i$ as we would any other radical. We distribute and combine like terms:

\begin{align*}
(1-2i) - (3+4i) & = 1-2i-3-4i \tag{Distribute} \\
& = -2 - 6i \tag{Gather like terms} \\
\end{align*}

Technically, we'd have to rewrite our answer  $-2-6i$ as $(-2) + (-6)i$ to be (in the strictest sense) `in the form $a+bi$'. That being said, even pedants have their limits, so $-2-6i$ is good enough.

\item  Using the Distributive Property (a.k.a. F.O.I.L.), we get
\begin{align*}
 (1-2i)(3+4i)  & = (1)(3) + (1)(4i) - (2i)(3) - (2i)(4i) \tag{F.O.I.L.} \\
& = 3+4i-6i-8i^2 \\			
& = 3 - 2i - 8(-1) \tag{$i^2=-1$} \\
& = 3 - 2i + 8 \\
& = 11 - 2i \\ 
\end{align*}

\item  How in the world are we supposed to simplify $\frac{1-2i}{3-4i}$?  Well, we deal with the denominator $3-4i$ as we would any other denominator containing two terms, one of which is a square root.\footnote{See subsection \ref{rationalizingdenomandnumer} for a more thorough treatment of this type of maneuver.} We multiply both numerator and denominator by $3+4i$, the (complex) conjugate of $3 - 4i$.  Doing so produces

\begin{align*}
 \dfrac{1-2i}{3-4i} & = \dfrac{(1-2i)(3+4i)}{(3-4i)(3+4i)} \tag{Equivalent Fractions} \\
& = \dfrac{3 + 4i - 6i - 8i^2}{9 - 16i^2} \tag{F.O.I.L.}\\
& = \dfrac{3 - 2i - 8(-1)}{9  - 16(-1)} \tag{$i^2 = -1$}\\
& = \dfrac{11 - 2i}{25} \\
& = \dfrac{11}{25} - \dfrac{2}{25} \, i \\ 
\end{align*}
										
\item  We use property 2 of Definition \ref{idefn} first, then apply the rules of radicals applicable to real numbers to get $\sqrt{-3} \sqrt{-12} = \left(i \sqrt{3}\right) \left(i \sqrt{12}\right) = i^2 \sqrt{3\cdot 12} = -\sqrt{36} = -6$.

\item  We adhere to the order of operations here and perform the multiplication before the radical to get  $\sqrt{(-3)(-12)} = \sqrt{36} = 6$. 

\item  We brute force multiply using the distributive property and find that

\begin{multline*}
(x-[1+2i])(x-[1-2i])\\
\end{multline*}
\begin{align*}
& =  x^2 -x[1-2i]-x[1+2i]+[1-2i][1+2i] \tag{F.O.I.L.}\\
&= x^2-x+2ix-x-2ix+1-2i+2i-4i^2 \tag{Distribute} \\
&= x^2-2x + 1-4(-1) \tag{Gather like terms; $i^2=-1$}\\
&= x^2 -2x +5 \\
\end{align*}

This type of factoring will be revisited in Section \ref{ComplexZeros}. \qed

\end{enumerate}

\end{ex}

In the previous example, we used the `conjugate' idea from Section \ref{AppRadEqus} to divide two complex numbers.  More generally, the \index{complex number ! complex conjugate ! definition of}\index{conjugate ! complex conjugate ! definition of}\textbf{complex conjugate} of a complex number $a+bi$ is the number $a-bi$.  The notation commonly used for complex conjugation is a `bar':  $\overline{a+bi} = a-bi$. For example, $\overline{3+2i} = 3-2i$ and $\overline{3-2i} = 3+2i$. To find $\overline{6}$, we note that $\overline{6} = \overline{6+0i}= 6 - 0i = 6$, so $\overline{6} = 6$. Similarly, $\overline{4i} = -4i$, since $\overline{4i} = \overline{0 + 4i} = 0 - 4i =  -4i$.  Note that $\overline{3+\sqrt{5}} = 3 + \sqrt{5}$, not $3 - \sqrt{5}$, since  $\overline{3+\sqrt{5}} = \overline{3+\sqrt{5} + 0i} = 3+\sqrt{5}  - 0i = 3+\sqrt{5}$. Here, the conjugation specified by the `bar' notation involves reversing the sign before $i = \sqrt{-1}$, not before  $\sqrt{5}$.  The properties of the conjugate are summarized in the following theorem.


\begin{tcolorbox}
	
\begin{thm}  \label{conjugateprops}\index{complex number ! conjugate ! properties of}\index{conjugate of a complex number ! properties of}{\bf Properties of the Complex Conjugate:} Let $z$ and $w$ be complex numbers. 

\begin{itemize}

\item  $\overline{\overline{z}} = z$

\item  $ \overline{z+w} = \overline{z} + \overline{w}$

\item  $ \overline{zw} = \overline{z} \, \overline{w}$

\item  $\overline{z^{n}} = \left(\overline{z}\right)^n$, for any natural number $n$

\item  $z$ is a real number if and only if $\overline{z} = z$.

\end{itemize}

\end{thm}

\end{tcolorbox}

\medskip

Theorem \ref{conjugateprops} says in part that complex conjugation works well with addition, multiplication and powers.  The proofs of these properties can best be achieved by writing out $z = a+bi$ and $w = c+di$ for real numbers $a$, $b$, $c$ and $d$.   Next, we compute the left and right sides of each equation and verify that they are the same.  

\smallskip

The proof of the first property is a very quick exercise.\footnote{Trust us on this.}  To prove the second property, we compare $\overline{z+w}$ with $\overline{z} + \overline{w}$.  We have $\overline{z} + \overline{w} = \overline{a+bi} + \overline{c+di}  = a-bi + c-di$.  To find $\overline{z+w}$, we first compute \[z+w = (a+bi) + (c+di) = (a+c)+(b+d)i\] so \[\overline{z+w} = \overline{(a+c)+(b+d)i} = (a+c) - (b+d)i = a+c - bi - di = a - bi + c - di = \overline{z} + \overline{w}\]  As such, we have established  $\overline{z+w} = \overline{z}+\overline{w}$. The proof for multiplication works similarly.  The proof that the conjugate works well with powers can be viewed as a repeated application of the product rule, and is best proved using a technique called Mathematical Induction.\footnote{See Section \ref{Induction}.}  The last property is a characterization of real numbers.  If $z$ is real, then $z = a + 0i$, so $\overline{z} = a - 0i = a = z$.  On the other hand, if $z=\overline{z}$, then $a+bi = a - bi$ which means $b=-b$ so $b=0$.  Hence, $z = a +0i = a$ and is real.

\medskip

We now return to the business of solving quadratic equations. Consider  $x^2-2x+5 = 0$. The discriminant $b^2 - 4ac = -16$ is negative, so we know by Theorem \ref{discriminanttheoremrealversion} there are no \textit{real} solutions, since the Quadratic Formula would involve the term $\sqrt{-16}$.  Complex numbers, however, are built just for such situations, so we can go ahead and apply the Quadratic Formula to get: \[ x = \dfrac{-(-2) \pm \sqrt{(-2)^2-4(1)(5)}}{2(1)} = \dfrac{2 \pm \sqrt{-16}}{2} = \dfrac{2 \pm 4i}{2} = 1 \pm 2i.\]  

\begin{ex} \label{complexsolnsreviewex}  Find the complex solutions to the following equations.\footnote{Remember, all real numbers are complex numbers, so `complex solutions' means both real and non-real answers.} 

\begin{multicols}{3}
\begin{enumerate}


\item  $\dfrac{2x}{x+1} = x+3$

\item $2t^4 = 9t^2 + 5$

\item  $z^3 + 1 = 0$

\end{enumerate}
\end{multicols}

{\bf Solution.}

\begin{enumerate}

\enlargethispage{20pt}

\item  Clearing fractions yields a quadratic equation so we then proceed as in Section \ref{AppQuadEqus}.

\begin{align*}
\dfrac{2x}{x+1} & = x+3 \\
2x & = (x+3)(x+1) \tag{Multiply by $(x+1)$ to clear denominators} \\
2x & = x^2 + x + 3x + 3 \tag{F.O.I.L.} \\
2x & = x^2 + 4x + 3 \tag{Gather like terms} \\
0 & = x^2 + 2x + 3 \tag{Subtract $2x$} \\
\end{align*}

From here, we apply the Quadratic Formula

\begin{align*}
x  & = \dfrac{-2 \pm \sqrt{2^2 - 4(1)(3)}}{2(1)} \tag{Quadratic Formula}\\
& = \dfrac{-2 \pm \sqrt{-8}}{2} \tag{Simplify}\\
& = \dfrac{-2 \pm i \sqrt{8}}{2} \tag{Definition of $i$}\\
& = \dfrac{-2 \pm i 2\sqrt{2}}{2} \tag{Product Rule for Radicals}\\
& = \dfrac{\cancel{2}(-1 \pm i\sqrt{2})}{\cancel{2}} \tag{Factor and reduce}\\
& = -1 \pm i \sqrt{2} \\
\end{align*}
		
We get two answers: $x = -1 + i\sqrt{2}$ and its conjugate $x = -1 - i\sqrt{2}$.  Checking both of these answers reviews all of the salient points about complex number arithmetic and is therefore strongly encouraged.

\item  Since we have three terms, and the exponent on one term (`$4$' on $t^4$) is exactly twice the exponent on the other (`$2$' on $t^2$), we have a Quadratic in Disguise.  We proceed accordingly.

\begin{align*}{rclr}
2t^4 & = 9t^2 + 5 & \\
2t^4 - 9t^2 - 5 & = 0 & \tag{Subtract $9t^2$ and $5$} \\
(2t^2 + 1)(t^2 - 5) & = 0 & \tag{Factor} \\
2t^2 + 1 = 0 & \text{ or } t^2 = 5 \tag{Zero Product Property} \\
\end{align*}

From $2t^2 + 1 = 0$ we get $2t^2 = -1$, or $t^2 = -\frac{1}{2}$.  We extract square roots as follows: \[ t = \pm \sqrt{-\dfrac{1}{2}} = \pm i \sqrt{\dfrac{1}{2}} = \pm i \dfrac{\sqrt{1}}{\sqrt{2}} = \pm i \dfrac{1}{\sqrt{2}} = \pm \dfrac{i \sqrt{2}}{2},\]
where we have rationalized the denominator per convention.  From $t^2 = 5$, we get $t = \pm \sqrt{5}$. In total, we have four complex solutions - two real: $t = \pm \sqrt{5}$ and two non-real: $t = \pm \frac{i \sqrt{2}}{2}$.

\item To find  the \textit{real} solutions to  $z^3 + 1 = 0$, we can subtract the $1$ from both sides and extract cube roots: $z^3 = -1$, so $z  = \sqrt[3]{-1} = -1$.  It turns out there are two more non-real complex number solutions to this equation.  To get at these, we factor:

\begin{align*}
z ^ 3 + 1 & = 0 \\
(z + 1)(z^2 - z + 1) & = 0 \tag{Factor\textemdash Sum of Two Cubes} \\
z + 1 = 0 & \text{ or } z^2 - z + 1 = 0 \\
\end{align*}

From $z+1 = 0$, we get our real solution $z = -1$.  From $z^2 -z + 1 = 0$, we apply the Quadratic Formula to get: \[z = \dfrac{-(-1) \pm \sqrt{(-1)^2 - 4(1)(1)}}{2(1)} = \dfrac{1 \pm \sqrt{-3}}{2} = \dfrac{1 \pm i\sqrt{3}}{2} \]
Thus we get \textit{three} solutions to $z^3 + 1 = 0$ - one real: $z = -1$ and two non-real: $z =  \frac{1 \pm i\sqrt{3}}{2}$.  As always, the reader is encouraged to test their algebraic mettle and check these solutions. \qed
		
\end{enumerate}


\end{ex}

It is no coincidence that the non-real solutions to the equations in Example \ref{complexsolnsreviewex} appear in  complex conjugate pairs. Any time we use the Quadratic Formula to solve an equation with \underline{real} coefficients, the answers will form a  complex conjugate pair owing to the $\pm$ in the Quadratic Formula.  This leads us to a generalization of Theorem \ref{discriminanttheoremrealversion} which we state below.

\begin{tcolorbox}
	
\begin{thm} \textbf{Discriminant Theorem:} \label{discriminanttheoremcomplexversion} Given a Quadratic Equation $ax^2 + bx + c = 0$, where $a$, $b$ and $c$ are real numbers, let $D = b^2 - 4ac$ be the discriminant.

\begin{itemize}

\item  If $D > 0$, there are two distinct real number solutions to the equation. 

\item  If $D = 0$, there is one (repeated) real number solution.  

\textbf{Note:}  `Repeated' here comes from the fact that `both' solutions $\frac{-b \pm 0}{2a}$ reduce to $-\frac{b}{2a}$.

\item  If $D < 0$, there are two non-real solutions which form a complex conjugate pair.

\end{itemize}

\end{thm}

\end{tcolorbox}

We will have much more to say about complex solutions to equations in Section \ref{ComplexZeros} and we will revisit Theorem \ref{discriminanttheoremcomplexversion} then.

\clearpage

\subsection{Exercises}

\label{ExercisesforAppCmpNums}

In Exercises \ref{compnumbasicfirst} - \ref{compnumbasiclast}, use the given complex numbers $z$ and $w$ to find and simplify the following. 

\begin{multicols}{3}

\begin{itemize}

\item $z+w$
\item $zw$
\item $z^2$

\end{itemize}

\end{multicols}

\begin{multicols}{3}

\begin{itemize}

\item $\dfrac{1}{z}$
\item $\dfrac{z}{w}$
\item $\dfrac{w}{z}$

\end{itemize}

\end{multicols}

\begin{multicols}{3}

\begin{itemize}

\item $\overline{z}$
\item $z\overline{z}$
\item $(\overline{z})^2$

\end{itemize}

\end{multicols}

\begin{tasks}(2)

\task  $z = 2+3i$, $w = 4i$ \label{compnumbasicfirst}
\task  $z = 1+i$, $w = -i$

\task  $z = i$, $w = -1+2i$
\task  $z = 4i$, $w = 2-2i$

\task  $z = 3-5i$, $w = 2+7i$
\task  $z = -5+i$, $w = 4+2i$

\task  $z = \sqrt{2} - i\sqrt{2}$, $w = \sqrt{2} + i\sqrt{2}$
\task  $z = 1 - i\sqrt{3}$, $w = -1 - i\sqrt{3}$

\task  $z = \dfrac{1}{2} + \dfrac{\sqrt{3}}{2} \, i$, $w = -\dfrac{1}{2} + \dfrac{\sqrt{3}}{2} \,i$
\task! $z = -\dfrac{\sqrt{2}}{2} + \dfrac{\sqrt{2}}{2} \, i$, $w = -\dfrac{\sqrt{2}}{2} - \dfrac{\sqrt{2}}{2} \, i$ \label{compnumbasiclast}

\end{tasks}

In Exercises \ref{rootsofnegfirst} - \ref{rootsofneglast}, simplify the quantity.

\begin{tasks}[resume](3)

\task $\sqrt{-49}$ \label{rootsofnegfirst}
\task $\sqrt{-9}$
\task $\sqrt{-25}\sqrt{-4}$
\task $\sqrt{(-25)(-4)}$

\task $\sqrt{-9}\sqrt{-16}$
\task $\sqrt{(-9)(-16)}$
\task $\sqrt{-(-9)}$
\task $-\sqrt{(-9)}$ \label{rootsofneglast}

\end{tasks}

We know that $i^{2} = -1$ which means $i^{3} = i^{2} \cdot i = (-1) \cdot i = -i$ and $i^{4} = i^{2} \cdot i^{2} = (-1)(-1) = 1$. In Exercises \ref{powerofifirst} - \ref{powerofilast}, use this information to simplify the given power of $i$.

\begin{tasks}[resume](4)
    
\task $i^{5}$ \label{powerofifirst}
\task $i ^{6}$
\task $i^{7}$
\task $i^{8}$


\task $i^{15}$
\task $i^{26}$
\task $i^{117}$
\task $i^{304}$ \label{powerofilast}

\end{tasks}

In Exercises \ref{solvecomplexfirst} - \ref{solvecomplexlast}, find all complex solutions.

\begin{tasks}[resume](2)

\task $3x^2 + 6 = 4x$ \label{solvecomplexfirst}
\task  $15t^2+2t+5= 3t(t^2+1)$
\task  $3y^2 + 4 = y^4$

\task $\dfrac{2}{1-w}= w$ \vphantom{$\dfrac{x^3}{2x-1} = \dfrac{x}{3}$ 
}
\task $\dfrac{y}{3} - \dfrac{3}{y} = y$ \vphantom{$\dfrac{x^3}{2x-1} = \dfrac{x}{3}$ 
}
\task $\dfrac{x^3}{2x-1} = \dfrac{x}{3}$ 

\task  $x =\dfrac{2}{\sqrt{5} - x}$\vphantom{$\dfrac{5z^4 + 1}{z^2-1} = 3z^2$}
\task  $\dfrac{5y^4 + 1}{y^2-1} = 3y^2$
\task  $z^{4} = 16$ \vphantom{$\dfrac{5z^4 + 1}{z^2-1} = 3z^2$} \label{solvecomplexlast}


\task!  Multiply and simplify:  $\left(x - [3 - i\sqrt{23}]\right)\left(x - [3+i\sqrt{23}]\right)$

\end{tasks}


\clearpage

\subsection{Answers}

\begin{enumerate}

\item  For $z = 2+3i$ and $w = 4i$

\begin{multicols}{3}

\begin{itemize}

\item $z+w = 2+7i$

\item $zw = -12+8i$

\item $z^2 = -5 + 12i$

\end{itemize}

\end{multicols}

\begin{multicols}{3}

\begin{itemize}

\item $\frac{1}{z} = \frac{2}{13} - \frac{3}{13} \, i$

\item $\frac{z}{w} = \frac{3}{4} - \frac{1}{2} \, i$

\item $\frac{w}{z} = \frac{12}{13} + \frac{8}{13} \,i$

\end{itemize}

\end{multicols}

\begin{multicols}{3}

\begin{itemize}

\item $\overline{z} = 2-3i$

\item $z\overline{z} = 13$

\item $(\overline{z})^2 = -5-12i$

\end{itemize}

\end{multicols}

\item  For $z = 1+i$ and $w = -i$

\begin{multicols}{3}

\begin{itemize}

\item $z+w = 1$

\item $zw = 1-i$

\item $z^2 = 2i$

\end{itemize}

\end{multicols}

\begin{multicols}{3}

\begin{itemize}

\item $\frac{1}{z} = \frac{1}{2} - \frac{1}{2} \, i$

\item $\frac{z}{w} = -1+i$

\item $\frac{w}{z} = -\frac{1}{2} - \frac{1}{2} \, i$

\end{itemize}

\end{multicols}

\begin{multicols}{3}

\begin{itemize}

\item $\overline{z} = 1-i$

\item $z\overline{z} = 2$

\item $(\overline{z})^2 = -2i$

\end{itemize}

\end{multicols}



\item  For  $z = i$ and $w = -1+2i$

\begin{multicols}{3}

\begin{itemize}

\item $z+w = -1+3i$

\item $zw = -2-i$

\item $z^2 = -1$

\end{itemize}

\end{multicols}

\begin{multicols}{3}

\begin{itemize}

\item $\frac{1}{z} = -i$

\item $\frac{z}{w} = \frac{2}{5} - \frac{1}{5} \, i$

\item $\frac{w}{z} = 2+i$

\end{itemize}

\end{multicols}

\begin{multicols}{3}

\begin{itemize}

\item $\overline{z} = -i$

\item $z\overline{z} = 1$

\item $(\overline{z})^2 = -1$

\end{itemize}

\end{multicols}

\item  For  $z = 4i$ and $w = 2-2i$

\begin{multicols}{3}

\begin{itemize}

\item $z+w = 2+2i$

\item $zw = 8+8i$

\item $z^2 = -16$

\end{itemize}

\end{multicols}

\begin{multicols}{3}

\begin{itemize}

\item $\frac{1}{z} = -\frac{1}{4} \,i$

\item $\frac{z}{w} = -1+i$

\item $\frac{w}{z} = -\frac{1}{2} - \frac{1}{2} \,i$

\end{itemize}

\end{multicols}

\begin{multicols}{3}

\begin{itemize}

\item $\overline{z} = -4i$

\item $z\overline{z} = 16$

\item $(\overline{z})^2 = -16$

\end{itemize}

\end{multicols}

\item  For  $z = 3-5i$ and $w = 2+7i$

\begin{multicols}{3}

\begin{itemize}

\item $z+w = 5+2i$

\item $zw = 41+11i$

\item $z^2 = -16-30i$

\end{itemize}

\end{multicols}

\begin{multicols}{3}

\begin{itemize}

\item $\frac{1}{z} = \frac{3}{34} + \frac{5}{34} \,i$

\item $\frac{z}{w} = -\frac{29}{53} - \frac{31}{53} \, i$

\item $\frac{w}{z} = -\frac{29}{34} + \frac{31}{34} \,i$

\end{itemize}

\end{multicols}

\begin{multicols}{3}

\begin{itemize}

\item $\overline{z} = 3+5i$

\item $z\overline{z} = 34$

\item $(\overline{z})^2 = -16+30i$

\end{itemize}

\end{multicols}

\item  For  $z = -5+i$ and  $w = 4+2i$

\begin{multicols}{3}

\begin{itemize}

\item $z+w = -1+3i$

\item $zw = -22-6i$

\item $z^2 = 24-10i$

\end{itemize}

\end{multicols}

\begin{multicols}{3}

\begin{itemize}

\item $\frac{1}{z} = -\frac{5}{26} - \frac{1}{26} \,i$

\item $\frac{z}{w} = -\frac{9}{10} + \frac{7}{10} \, i$

\item $\frac{w}{z} = -\frac{9}{13} - \frac{7}{13} \,i$

\end{itemize}

\end{multicols}

\begin{multicols}{3}

\begin{itemize}

\item $\overline{z} = -5-i$

\item $z\overline{z} = 26$

\item $(\overline{z})^2 = 24+10i$

\end{itemize}

\end{multicols}


\item  For  $z = \sqrt{2} - i\sqrt{2}$ and $w = \sqrt{2} + i\sqrt{2}$

\begin{multicols}{3}

\begin{itemize}

\item $z+w = 2\sqrt{2}$

\item $zw = 4$

\item $z^2 = -4i$

\end{itemize}

\end{multicols}

\begin{multicols}{3}

\begin{itemize}

\item $\frac{1}{z} = \frac{\sqrt{2}}{4} + \frac{\sqrt{2}}{4} \,i$

\item $\frac{z}{w} = -i$

\item $\frac{w}{z} = i$

\end{itemize}

\end{multicols}

\begin{multicols}{3}

\begin{itemize}

\item $\overline{z} = \sqrt{2}+i\sqrt{2}$

\item $z\overline{z} = 4$

\item $(\overline{z})^2 = 4i$

\end{itemize}

\end{multicols}

\item  For   $z = 1 - i\sqrt{3}$ and $w = -1-i\sqrt{3}$

\begin{multicols}{3}

\begin{itemize}

\item $z+w = -2i\sqrt{3}$

\item $zw = -4$

\item $z^2 = -2-2i\sqrt{3}$

\end{itemize}

\end{multicols}

\begin{multicols}{3}

\begin{itemize}

\item $\frac{1}{z} = \frac{1}{4} + \frac{\sqrt{3}}{4} \,i$

\item $\frac{z}{w} = \frac{1}{2} + \frac{\sqrt{3}}{2} \,i$

\item $\frac{w}{z} = \frac{1}{2} - \frac{\sqrt{3}}{2} \,i$

\end{itemize}

\end{multicols}

\begin{multicols}{3}

\begin{itemize}

\item $\overline{z} = 1+i\sqrt{3}$

\item $z\overline{z} = 4$

\item $(\overline{z})^2 = -2+2i\sqrt{3}$

\end{itemize}

\end{multicols}

\item  For   $z = \frac{1}{2} + \frac{\sqrt{3}}{2} \, i$ and $w = -\frac{1}{2} + \frac{\sqrt{3}}{2} \,i$

\begin{multicols}{3}

\begin{itemize}

\item $z+w = i\sqrt{3}$

\item $zw = -1$

\item $z^2 = -\frac{1}{2} + \frac{\sqrt{3}}{2} \,i$

\end{itemize}

\end{multicols}

\begin{multicols}{3}

\begin{itemize}

\item $\frac{1}{z} = \frac{1}{2} - \frac{\sqrt{3}}{2} \, i$

\item $\frac{z}{w} = \frac{1}{2} - \frac{\sqrt{3}}{2} \, i$

\item $\frac{w}{z} = \frac{1}{2} + \frac{\sqrt{3}}{2} \, i$

\end{itemize}

\end{multicols}

\begin{multicols}{3}

\begin{itemize}

\item $\overline{z} = \frac{1}{2} - \frac{\sqrt{3}}{2} \, i$

\item $z\overline{z} = 1$

\item $(\overline{z})^2 = -\frac{1}{2} - \frac{\sqrt{3}}{2} \, i$

\end{itemize}

\end{multicols}

\item  For   $z = -\frac{\sqrt{2}}{2} + \frac{\sqrt{2}}{2} \, i$ and $w = -\frac{\sqrt{2}}{2} - \frac{\sqrt{2}}{2} \, i$

\begin{multicols}{3}

\begin{itemize}

\item $z + w = -\sqrt{2}$

\item $zw = 1$

\item $z^2 =-i$

\end{itemize}

\end{multicols}

\begin{multicols}{3}

\begin{itemize}

\item $\frac{1}{z} = -\frac{\sqrt{2}}{2} - \frac{\sqrt{2}}{2} \, i$

\item $\frac{z}{w} = -i$

\item $\frac{w}{z} = i$

\end{itemize}

\end{multicols}

\begin{multicols}{3}

\begin{itemize}

\item $\overline{z} = -\frac{\sqrt{2}}{2} - \frac{\sqrt{2}}{2} \, i$

\item $z\overline{z} = 1$

\item $(\overline{z})^2 = i$

\end{itemize}
\end{multicols}
\setcounter{HW}{\value{enumi}}
\end{enumerate}

\begin{multicols}{4}
\begin{enumerate}
\setcounter{enumi}{\value{HW}}

\item $7i$
\item $3i$

\item $-10$
\item $10$

\setcounter{HW}{\value{enumi}}
\end{enumerate}
\end{multicols}

\begin{multicols}{4}
\begin{enumerate}
\setcounter{enumi}{\value{HW}}

\item $-12$
\item $12$

\item $3$
\item $-3i$

\setcounter{HW}{\value{enumi}}
\end{enumerate}
\end{multicols}

\begin{multicols}{2}
\begin{enumerate}
\setcounter{enumi}{\value{HW}}

\item $i^{5} = i^{4} \cdot i = 1 \cdot i = i$
\item $i ^{6} =  i^{4} \cdot i^{2} = 1 \cdot (-1) = -1$
\setcounter{HW}{\value{enumi}}
\end{enumerate}
\end{multicols}

\begin{multicols}{2}
\begin{enumerate}
\setcounter{enumi}{\value{HW}}


\item $i^{7} = i^{4} \cdot i^{3} = 1 \cdot (-i) = -i$
\item $i^{8} = i^{4} \cdot i^{4} = \left(i^{4}\right)^{2} = (1)^{2} =1$

\setcounter{HW}{\value{enumi}}
\end{enumerate}
\end{multicols}

\begin{multicols}{2}
\begin{enumerate}
\setcounter{enumi}{\value{HW}}

\item $i^{15} = \left(i^{4}\right)^{3} \cdot i^{3} = 1 \cdot (-i) = -i$
\item $i ^{26} = \left(i^{4}\right)^{6} \cdot i^{2} = 1\cdot (-1) = -1$

\setcounter{HW}{\value{enumi}}
\end{enumerate}
\end{multicols}

\begin{multicols}{2}
\begin{enumerate}
\setcounter{enumi}{\value{HW}}
\item $i^{117} = \left(i^{4}\right)^{29} \cdot i = 1 \cdot i = i$
\item $i ^{304} = \left(i^{4}\right)^{76} = 1^{76} = 1$

\setcounter{HW}{\value{enumi}}
\end{enumerate}
\end{multicols}

\begin{multicols}{3}
\begin{enumerate}
\setcounter{enumi}{\value{HW}}

\item $x = \dfrac{2 \pm i\sqrt{14}}{3}$ 
\item  $t = 5, \pm \dfrac{i \sqrt{3}}{3}$ \vphantom{$x = \dfrac{2 \pm i\sqrt{14}}{3}$}
\item  $y = \pm 2, \pm i$ \vphantom{$x = \dfrac{2 \pm i\sqrt{14}}{3}$}

\setcounter{HW}{\value{enumi}}
\end{enumerate}
\end{multicols}
\begin{multicols}{3}
\begin{enumerate}
\setcounter{enumi}{\value{HW}}

\item $w = \dfrac{1 \pm i \sqrt{7}}{2}$ \vphantom{$x= 0, \dfrac{1 \pm i\sqrt{2}}{3}$ }
\item $y = \pm \dfrac{3i \sqrt{2}}{2}$ \vphantom{$x= 0, \dfrac{1 \pm i\sqrt{2}}{3}$ }
\item $x= 0, \dfrac{1 \pm i\sqrt{2}}{3}$ 

\setcounter{HW}{\value{enumi}}
\end{enumerate}
\end{multicols}

\begin{multicols}{3}
\begin{enumerate}
\setcounter{enumi}{\value{HW}}

\item  $x = \dfrac{\sqrt{5} \pm i\sqrt{3}}{2}$
\item  $y = \pm i, \pm \dfrac{i\sqrt{2}}{2}$ \vphantom{$x = \dfrac{\sqrt{5} \pm i\sqrt{3}}{2}$}
\item  $z = \pm 2, \pm 2i$  \vphantom{$x = \dfrac{\sqrt{5} \pm \sqrt{3}}{2}$}
\setcounter{HW}{\value{enumi}}

\end{enumerate}
\end{multicols}


\begin{enumerate}
\setcounter{enumi}{\value{HW}}

\item $x^2 - 6x + 32$

\end{enumerate}




\closegraphsfile