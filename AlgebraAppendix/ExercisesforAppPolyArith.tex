\label{ExercisesforAppPolyArith}

In Exercises \ref{polyarithexfirst} - \ref{polyarithexlast}, perform the indicated operations and simplify.

\begin{tasks}(2)
\task  $(4-3x) + (3x^2 + 2x + 7)$ \label{polyarithexfirst}
\task $t^2 + 4t - 2(3-t)$
\task $q(200-3q) - (5q + 500)$
\task $(3y-1)(2y+1)$\vphantom{$\left(3-\dfrac{x}{2}\right)(2x+5)$}
\task $\left(3-\dfrac{x}{2}\right)(2x+5)$
\task $-(4t+3)(t^2-2)$\vphantom{$\left(3-\dfrac{x}{2}\right)(2x+5)$}

\task $2w(w^3-5)(w^3+5)$
\task $(5a^2 - 3)(25a^4 + 15a^2 + 9)$
\task $(x^2-2x+3)(x^2+2x+3)$

\task $(\sqrt{7} - z)(\sqrt{7} + z)$
\task $(x - \sqrt[3]{5})^3$
\task $(x - \sqrt[3]{5})(x^2 + x\sqrt[3]{5} + \sqrt[3]{25})$

\task $(w-3)^2 - (w^2 + 9)$
\task $(x+h)^2 - 2(x+h) - (x^2 - 2x)$
\task $(x-[2+\sqrt{5}])(x-[2-\sqrt{5}])$ \label{polyarithexlast}

\end{tasks}

In Exercises \ref{polydivexfirst} - \ref{polydivexlast}, perform the indicated division.  Check your answer by showing \[\text{dividend} = (\text{divisor})( \text{quotient}) + \text{remainder}\]

\begin{tasks}[resume](2)
\task $(5x^2 - 3x + 1) \div (x + 1)$ \label{polydivexfirst}
\task $(3y^2 + 6y - 7) \div (y-3)$
\task $(6w - 3) \div (2w+5)$
\task $(2x+1) \div (3x-4)$
\task $(t^2 - 4) \div (2t + 1)$

\task $(w^3 - 8) \div (5w-10)$
\task $(2x^2 - x + 1) \div (3x^2 + 1)$

\task $(4y^4+3y^2+1) \div (2y^2-y+1)$

\task $w^4 \div (w^3 - 2)$
\task $(5t^3 - t + 1) \div (t^2 + 4)$

\task $(t^3 - 4) \div (t - \sqrt[3]{4})$

\task $(x^2-2x-1) \div (x-[1-\sqrt{2}])$  \label{polydivexlast}

\end{tasks}

In Exercises \ref{specialformexfirst} - \ref{specialformexlast} verify the given formula by showing the left hand side of the equation simplifies to the right hand side of the equation.

\begin{tasks}[resume]
\task \textbf{Perfect Cube:} $(a+b)^3 = a^3 + 3a^2b + 3ab^2 + b^3$ \label{specialformexfirst}

\task \textbf{Difference of Cubes:} $(a - b)(a^2 + ab + b^2) = a^3 - b^3$

\task \textbf{Sum of Cubes:} $(a + b)(a^2 - ab + b^2) = a^3 + b^3$

\task \textbf{Perfect Quartic:} $(a+b)^4 = a^4 + 4a^3b + 6a^2b^2 + 4ab^3 + b^4$

\task \textbf{Difference of Quartics:} $(a-b)(a+b)(a^2+b^2) = a^4 - b^4$

\task \textbf{Sum of Quartics:}  $(a^2 + ab \sqrt{2} + b^2)(a^2 - ab \sqrt{2} + b^2) = a^4 + b^4$ \label{specialformexlast}

\task With help from your classmates, determine under what conditions $(a+b)^2 = a^2 + b^2$.  What about $(a+b)^3 = a^3 + b^3$? In general, when does $(a+b)^n = a^n + b^n$ for a natural number $n \geq 2$?

\end{tasks}

\clearpage

\subsection{Answers}

\begin{tasks}(2)
\task $3x^2 - x + 11$

\task $t^2 + 6t-6$

\task $-3q^2+195q-500$

\task $6y^2+y-1$\vphantom{$-x^2 + \dfrac{7}{2} x + 15$}

\task $-x^2 + \dfrac{7}{2} x + 15$

\task $-4t^3-3t^2+8t+6$\vphantom{$-x^2 + \dfrac{7}{2} x + 15$}

\task $2w^7 - 50w$

\task $125a^6 - 27$

\task $x^4+2x^2+9$

\task $7-z^2$

\task $x^3 - 3x^2\sqrt[3]{5} + 3x\sqrt[3]{25} - 5$

\task $x^3 - 5$

\task $-6w$

\task $h^2 + 2xh - 2h$

\task $x^2 - 4x - 1$ 

\end{tasks}

\begin{tasks}[resume]
\task quotient: $5x-8$, remainder: $9$ 

\task quotient: $3y+15$, remainder: $38$

\task quotient: $3$, remainder: $18$ \vphantom{$\dfrac{11}{3}$}

\task quotient: $\dfrac{2}{3}$, remainder: $\dfrac{11}{3}$

\task quotient: $\dfrac{t}{2} - \dfrac{1}{4}$, remainder: $-\dfrac{15}{4}$ \vphantom{ $\dfrac{w^2}{5} + \dfrac{2w}{5} + \dfrac{4}{5}$, remainder: $0$}

\task quotient: $\dfrac{w^2}{5} + \dfrac{2w}{5} + \dfrac{4}{5}$, remainder: $0$

\task quotient: $\dfrac{2}{3}$, remainder: $-x + \dfrac{1}{3}$

\task quotient:  $2y^2+y+1$, remainder: $0$ \vphantom{$\dfrac{2}{3}$, remainder: $-x + \dfrac{1}{3}$}

\task quotient: $w$, remainder: $2w$

\task quotient: $5t$, remainder: $-21t + 1$

\task quotient:\footnote{Note: $\sqrt[3]{16} = 2\sqrt[3]{2}$.} $t^2 + t \sqrt[3]{4} + 2\sqrt[3]{2}$, remainder: $0$

\task quotient: $x -1 - \sqrt{2}$, remainder: 0  

\end{tasks}
