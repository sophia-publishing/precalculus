\label{ExercisesforAppLines}

In Exercises \ref{pointslopegivenlinefirst} - \ref{pointslopegivenlinelast}, find both the point-slope form and the slope-intercept form of the line with the given slope which passes through the given point.

\begin{tasks}(2)
\task $m = 3, \;\; P(3, -1)$ \label{pointslopegivenlinefirst}
\task $m = -2, \;\; P(-5, 8)$
\task $m = -1, \;\; P(-7, -1)$
\task $m = \frac{2}{3}, \;\; P(-2, 1)$
\task $m = -\frac{1}{5}, \;\; P(10, 4)$
\task $m = \frac{1}{7}, \;\; P(-1, 4)$
\task $m = 0, \;\; P(3, 117)$
\task $m = -\sqrt{2}, \;\; P(0, -3)$
\task $m = -5, \;\; P(\sqrt{3}, 2\sqrt{3})$
\task $m = 678, \;\; P(-1, -12)$ \label{pointslopegivenlinelast}
\end{tasks}

In Exercises \ref{twopointsgivenlinefirst} - \ref{twopointsgivenlinelast}, find the slope-intercept form of the line which passes through the given points.

\begin{tasks}[resume](2)
\task $P(0, 0), \; Q(-3, 5)$ \label{twopointsgivenlinefirst}
\task $P(-1, -2), \; Q(3, -2)$
\task $P(5, 0), \; Q(0, -8)$
\task $P(3, -5), \; Q(7, 4)$
\task $P(-1,5), \; Q(7, 5)$
\task $P(4, -8), \; Q(5, -8)$
\task $P\left(\frac{1}{2}, \frac{3}{4} \right), \; Q\left(\frac{5}{2}, -\frac{7}{4} \right)$
\task $P\left(\frac{2}{3}, \frac{7}{2} \right), \; Q\left(-\frac{1}{3}, \frac{3}{2} \right)$
\task! $P\left(\sqrt{2}, -\sqrt{2} \right), \; Q\left(-\sqrt{2}, \sqrt{2} \right)$
\task $P\left(-\sqrt{3}, -1 \right), \; Q\left(\sqrt{3}, 1 \right)$ \label{twopointsgivenlinelast}
\end{tasks}

In Exercises \ref{graphlineexerfirst} - \ref{graphlineexerlast}, graph the line.  Find the slope, $y$-intercept and $x$-intercept, if any exist.

\begin{tasks}[resume](2)
\task $y = 2x - 1$ \label{graphlineexerfirst}
\task $y = 3 - x$
\task $y = 3$
\task $y = 0$
\task $y = \frac{2}{3} x + \frac{1}{3}$ \vphantom{$\dfrac{1-x}{2}$}
\task $y = \dfrac{1-x}{2}$ \label{graphlineexerlast}
\task!  Graph $3v + 2w = 6$ on both the $vw$- and $wv$-axes.  What characteristics to both graphs share?  What's different?
\task!  Find all of the points on the line $y=2x+1$ which are $4$ units from the point $(-1,3)$.
\end{tasks}

In Exercises \ref{parallelfirst} - \ref{parallellast}, you are given a line and a point which is not on that line.  Find the line parallel to the given line which passes through the given point.

\begin{tasks}[resume](2)
\task $y = 3x + 2, \; P(0, 0)$ \label{parallelfirst}
\task $y = -6x + 5, \; P(3, 2)$
\task $y = \frac{2}{3} x - 7, \; P(6, 0)$
\task $y = \dfrac{4-x}{3}, \; P(1, -1)$
\task $y = 6, \; P(3, -2)$
\task $x=1, \; P(-5,0)$ \label{parallellast}
\end{tasks}

\phantomsection
\label{perpendicularlines}

In Exercises \ref{perpendlinefirst} - \ref{perpendlinelast}, you are given a line and a point which is not on that line.  Find the line perpendicular to the given line which passes through the given point.

\begin{tasks}[resume](2)
\task $y = \frac{1}{3}x + 2, \; P(0, 0)$ \label{perpendlinefirst}
\task $y = -6x + 5, \; P(3, 2)$
\task $y = \frac{2}{3} x - 7, \; P(6, 0)$
\task $y = \dfrac{4-x}{3}, \; P(1, -1)$
\task $y = 6, \; P(3, -2)$
\task $x=1, \; P(-5,0)$ \label{perpendlinelast}
\task! We shall now prove that $y = m_{\mbox{\tiny$1$}}x + b_{\mbox{\tiny$1$}}$ is perpendicular to $y = m_{\mbox{\tiny$2$}}x + b_{\mbox{\tiny$2$}}$ if and only if $m_{\mbox{\tiny$1$}} \cdot m_{\mbox{\tiny$2$}} = -1$.  To make our lives easier we shall assume that $m_{\mbox{\tiny$1$}} > 0$ and $m_{\mbox{\tiny$2$}} < 0$.  We can also ``move'' the lines so that their point of intersection is the origin without messing things up, so we'll assume $b_{\mbox{\tiny$1$}} = b_{\mbox{\tiny$2$}} = 0.$  (Take a moment with your classmates to discuss why this is okay.)  Graphing the lines and plotting the points $O(0, 0)\;$, $P(1, m_{\mbox{\tiny$1$}})\;$ and $Q(1, m_{\mbox{\tiny$2$}})$ gives us the set up shown in \autoref{fig:slopesofperplines}. \label{perpendicularlineproof}

The line $y = m_{\mbox{\tiny$1$}}x$ will be perpendicular to the line $y = m_{\mbox{\tiny$2$}}x$ if and only if $\bigtriangleup OPQ$ is a right triangle.  Let $d_{\mbox{\tiny$1$}}$ be the distance from $O$ to $P$, let $d_{\mbox{\tiny$2$}}$ be the distance from $O$ to $Q$ and let $d_{\mbox{\tiny$3$}}$ be the distance from $P$ to $Q$.  Use the Pythagorean Theorem to show that $\bigtriangleup OPQ$ is a right triangle if and only if $m_{\mbox{\tiny$1$}} \cdot m_{\mbox{\tiny$2$}} = -1$ by showing $d_{\mbox{\tiny$1$}}^{2} + d_{\mbox{\tiny$2$}}^{2} = d_{\mbox{\tiny$3$}}^2$ if and only if $m_{\mbox{\tiny$1$}} \cdot m_{\mbox{\tiny$2$}} = -1$.  

\end{tasks}

\begin{figure}
\begin{center}

\begin{mfpic}[18]{-5}{5}{-5}{5}
\point[3pt]{(0, 0), (1.5, 0.75), (1.5, -3)}
\arrow \reverse \arrow \polyline{( -4, -2), (4, 2)}
\arrow \reverse \arrow \polyline{( -2, 4), (2, -4)}
\polyline{(1.5, 0.75), (1.5, -3)}
\tlabel(1.2, 1){\scriptsize $P$}
\tlabel(-.5,-.6){\scriptsize $O$}
\tlabel(1.2,-3.55){\scriptsize $Q$}
\axes
\tlabel[cc](5,-0.5){\scriptsize $x$}
\tlabel[cc](0.5,5){\scriptsize $y$}
\end{mfpic}

\caption{Product of the slops of perpendicular lines}
\label{fig:slopesofperplines}
\end{center}
\end{figure}

\clearpage

\subsection{Answers}

\begin{tasks}(2)
\task $y+1 = 3(x-3)$ \\ $y = 3x-10$
\task $y-8 = -2(x+5)$ \\ $y = -2x-2$
\task $y + 1 = -(x+7)$ \\ $y = -x-8$
\task $y - 1 = \frac{2}{3} (x+2)$ \\ $y = \frac{2}{3} x + \frac{7}{3}$
\task $y - 4 = -\frac{1}{5} (x-10)$ \\ $y = -\frac{1}{5} x + 6$
\task $y - 4 = \frac{1}{7}(x + 1)$ \\ $y = \frac{1}{7}x + \frac{29}{7}$
\task $y - 117 = 0$ \\ $y = 117$
\task $y + 3 = -\sqrt{2}(x - 0)$ \\ $y = -\sqrt{2}x - 3$
\task $y - 2\sqrt{3} = -5(x - \sqrt{3})$ \\ $y = -5x + 7\sqrt{3}$ 
\task $y + 12 = 678(x + 1)$ \\ $y = 678x + 666$
\task $y = -\frac{5}{3}x$
\task $y = -2$
\task $y = \frac{8}{5}x - 8$ 
\task $y = \frac{9}{4}x - \frac{47}{4}$
\task $y = 5$
\task $y = -8$
\task $y = -\frac{5}{4} x + \frac{11}{8}$ 
\task $y = 2x + \frac{13}{6}$ 
\task $y = -x$
\task $y = \frac{\sqrt{3}}{3} x$
\task! \begin{multicols}{2} \raggedcolumns
$y =2x-1$

slope: $m = 2$ 

$y$-intercept:  $(0,-1)$

$x$-intercept: $\left(\frac{1}{2}, 0 \right)$

\columnbreak

\begin{mfpic}[15]{-3}{3}{-4}{4}
\point[3pt]{(0,-1), (0.5,0)}
\axes
\tlabel[cc](3,-0.5){\scriptsize $x$}
\tlabel[cc](0.5,4){\scriptsize $y$}
\xmarks{-2,-1,1,2}
\ymarks{-3,-2,-1,1,2,3}
\tlpointsep{4pt}
\tiny 
\axislabels {x}{{$-2 \hspace{6pt}$} -2,{$-1 \hspace{6pt}$} -1, {$1$} 1, {$2$} 2}
\axislabels {y}{{$-3$} -3,{$-2$} -2,{$-1$} -1, {$1$} 1, {$2$} 2, {$3$} 3}
\normalsize
\arrow \reverse \arrow \function{-1,2, 0.1}{2*x-1}
\end{mfpic} 
\end{multicols}

\task! \begin{multicols}{2} \raggedcolumns
$y =3-x$

slope: $m = -1$ 

$y$-intercept:  $(0,3)$

$x$-intercept: $(3, 0)$ 

\columnbreak

\begin{mfpic}[15]{-2}{5}{-2}{5}
\point[3pt]{(0,3), (3,0)}
\axes
\tlabel[cc](5,-0.5){\scriptsize $x$}
\tlabel[cc](0.5,5){\scriptsize $y$}
\xmarks{-1,1,2,3,4}
\ymarks{-1,1,2,3,4}
\tlpointsep{4pt}
\tiny 
\axislabels {x}{{$-1 \hspace{6pt}$} -1, {$1$} 1, {$2$} 2, {$3$} 3, {$4$} 4}
\axislabels {y}{{$-1$} -1, {$1$} 1, {$2$} 2, {$3$} 3, {$4$} 4}
\normalsize
\arrow \reverse \arrow \function{-1,4, 0.1}{3-x}
\end{mfpic}
\end{multicols}

\task! \begin{multicols}{2} \raggedcolumns
$y = 3$

slope: $m =0$ 

$y$-intercept:  $(0,3)$

$x$-intercept: none

\columnbreak
\begin{mfpic}[15]{-3}{3}{-1}{5}
\point[3pt]{(0,3)}
\axes
\tlabel[cc](3,-0.5){\scriptsize $x$}
\tlabel[cc](0.5,5){\scriptsize $y$}
\xmarks{-2,-1,1,2}
\ymarks{1,2,3,4}
\tlpointsep{4pt}
\tiny 
\axislabels {x}{{$-2 \hspace{6pt}$} -2,{$-1 \hspace{6pt}$} -1, {$1$} 1, {$2$} 2}
\axislabels {y}{{$1$} 1, {$2$} 2, {$3$} 3, {$4$} 4}
\normalsize
\arrow \reverse \arrow \function{-3,3, 0.1}{3}
\end{mfpic}
\end{multicols}

\task! \begin{multicols}{2} \raggedcolumns
$y = 0$

slope: $m =0$ 

$y$-intercept:  $(0,0)$

$x$-intercept:

$\{ (x,0) \, | \, \text{$x$ is a real number} \}$

\columnbreak
\begin{mfpic}[15]{-3}{3}{-2}{2}
\arrow \polyline{(0,-2), (0,2)}
\tlabel[cc](3,-0.5){\scriptsize $x$}
\tlabel[cc](0.5,2){\scriptsize $y$}
\xmarks{-2,-1,1,2}
\ymarks{-1,1}
\tlpointsep{4pt}
\tiny 
\axislabels {x}{{$-2 \hspace{6pt}$} -2,{$-1 \hspace{6pt}$} -1, {$1$} 1, {$2$} 2}
\axislabels {y}{{$-1$} -1,{$1$} 1}
\normalsize
\penwd{1.15pt}
\arrow \reverse \arrow \function{-3,3, 0.1}{0}
\end{mfpic}
\end{multicols}

\task! \begin{multicols}{2} \raggedcolumns
$y = \frac{2}{3} x + \frac{1}{3}$

slope: $m = \frac{2}{3}$ 

$y$-intercept:  $\left(0, \frac{1}{3}\right)$

$x$-intercept:  $\left(-\frac{1}{2}, 0\right)$

\columnbreak
\begin{mfpic}[15]{-3}{3}{-2}{3}
\point[3pt]{(0,0.33333), (-0.5,0)}
\axes
\tlabel[cc](3,-0.5){\scriptsize $x$}
\tlabel[cc](0.5,3){\scriptsize $y$}
\xmarks{-2,-1,1,2}
\ymarks{-1,1,2}
\tlpointsep{4pt}
\tiny 
\axislabels {x}{{$-2 \hspace{6pt}$} -2, {$1$} 1, {$2$} 2}
\axislabels {y}{{$-1$} -1, {$1$} 1, {$2$} 2}
\normalsize
\arrow \reverse \arrow \function{-3,3, 0.1}{0.66667*x+0.33333}
\end{mfpic}
\end{multicols}

\task! \begin{multicols}{2} \raggedcolumns
$y = \dfrac{1-x}{2}$

slope: $m = -\frac{1}{2}$ 

$y$-intercept:  $\left(0, \frac{1}{2}\right)$

$x$-intercept:  $\left(1, 0\right)$

\columnbreak
\begin{mfpic}[15]{-3}{3}{-2}{3}
\point[3pt]{(0,0.5), (1,0)}
\axes
\tlabel[cc](3,-0.5){\scriptsize $x$}
\tlabel[cc](0.5,3){\scriptsize $y$}
\xmarks{-2,-1,1,2}
\ymarks{-1,1,2}
\tlpointsep{4pt}
\tiny 
\axislabels {x}{{$-2 \hspace{6pt}$} -2,{$-1 \hspace{6pt}$} -1, {$1$} 1, {$2$} 2}
\axislabels {y}{{$-1$} -1, {$1$} 1, {$2$} 2}
\normalsize
\arrow \reverse \arrow \function{-3,3, 0.1}{0.5-0.5*x}
\end{mfpic}
\end{multicols}

\task! \begin{multicols}{2} \raggedcolumns
$w = -\frac{3}{2} v + 3$

slope: $m = -\frac{3}{2}$ 

$w$-intercept:  $\left(0, 3\right)$

$v$-intercept:  $\left(2, 0\right)$

\columnbreak
\begin{mfpic}[15]{-1}{4}{-1}{4}
\point[3pt]{(0,3), (2,0)}
\axes
\tlabel[cc](4,-0.5){\scriptsize $v$}
\tlabel[cc](0.5,4){\scriptsize $w$}
\xmarks{1,2,3}
\ymarks{1,2,3}
\tlpointsep{4pt}
\tiny 
\axislabels {x}{{$1$} 1, {$2$} 2, {$3$} 3}
\axislabels {y}{{$1$} 1, {$2$} 2, , {$3$} 3}
\normalsize
\arrow \reverse \arrow \polyline{(-0.5, 3.75), (2.5, -0.75)}
\end{mfpic}
\end{multicols}

\begin{multicols}{2} \raggedcolumns
$v = -\frac{2}{3} w + 2$

slope: $m = -\frac{2}{3}$ 

$v$-intercept:  $\left(0,2 \right)$

$w$-intercept:  $\left(3,0\right)$

\columnbreak
\begin{mfpic}[15]{-1}{4}{-1}{4}
\point[3pt]{(0,2), (3,0)}
\axes
\tlabel[cc](4,-0.5){\scriptsize $w$}
\tlabel[cc](0.5,4){\scriptsize $v$}
\xmarks{1,2,3}
\ymarks{1,2,3}
\tlpointsep{4pt}
\tiny 
\axislabels {x}{{$1$} 1, {$2$} 2, {$3$} 3}
\axislabels {y}{{$1$} 1, {$2$} 2, , {$3$} 3}
\normalsize
\arrow \reverse \arrow \polyline{(3.75, -0.5), (-0.75, 2.5)}
\end{mfpic}
\end{multicols} 

\task $(-1,-1)$ and $\left(\frac{11}{5}, \frac{27}{5}\right)$
\task $y = 3x$
\task $y = -6x + 20$
\task $y = \frac{2}{3} x - 4$
\task $y = -\frac{1}{3} x - \frac{2}{3}$
\task $y=-2$
\task $x=-5$
\task $y = -3x$
\task $y = \frac{1}{6}x + \frac{3}{2}$
\task $y = -\frac{3}{2} x +9$
\task $y = 3x-4$
\task $x=3$
\task $y=0$
\end{tasks}