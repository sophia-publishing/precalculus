\mfpicnumber{1}

\opengraphsfile{AppRadEqus}

\setcounter{footnote}{0}

\label{AppRadEqus}

In this section we review simplifying expressions and solving equations involving radicals.  In addition to the product, quotient and power rules stated in Theorem \ref{radicalprops} in Section \ref{AppRealNumberArithmetic},  we present the following result which states that $\text{n}^{\text{th}}$ roots and $\text{n}^{\text{th}}$ powers more or less `undo' each other.\footnote{See Sections \ref{OtherFunctionsinvolvingRadicals} and \ref{InverseFunctions} for a more precise understanding of what we mean here.} 

\begin{tcolorbox}
	
\begin{thm}\label{simplifyradicals} \textbf{Simplifying $\text{n}^{\text{th}}$ powers of $\text{n}^{\text{th}}$ roots and $\text{n}^{\text{th}}$ roots of $\text{n}^{\text{th}}$ powers}:  Suppose $n$ is a natural number, $a$ is a real number and $\sqrt[n]{a}$ is a real number.  Then

\begin{itemize}

\item $(\sqrt[n]{a})^{n} = a$

\item  if $n$ is odd, $\sqrt[n]{a^{n}} = a$; if $n$ is even, $\sqrt[n]{a^{n}} = |a|$.

\end{itemize}

\end{thm}

\end{tcolorbox}

Since $\sqrt[n]{a}$ is \textit{defined} so that $(\sqrt[n]{a})^n = a$,  the first claim in the theorem is just a re-wording of  Definition \ref{principalnthrootdefn}.  The second part of the theorem breaks down along odd/even exponent lines due to how exponents affect negatives. To see this, consider the specific cases of $\sqrt[3]{(-2)^3}$ and $\sqrt[4]{(-2)^{4}}$.  

In the first case,  $\sqrt[3]{(-2)^3} =\sqrt[3]{-8} = -2$, so we have an instance of when $\sqrt[n]{a^{n}} = a$.  The reason that the cube root `undoes' the third power in $\sqrt[3]{(-2)^3} = -2$ is because the negative is preserved when raised to the third (odd) power.  In  $\sqrt[4]{(-2)^{4}}$,  the negative `goes away' when raised to the fourth (even) power:$\sqrt[4]{(-2)^{4}} = \sqrt[4]{16}$.  According to Definition \ref{principalnthrootdefn}, the fourth root is defined to give only \textit{non-negative} numbers, so $\sqrt[4]{16} = 2$.  Here we have a case where $\sqrt[4]{(-2)^{4}} = 2 = |-2|$, not $-2$. 

In general, we need the absolute values to simplify $\sqrt[n]{a^{n}}$ only when $n$ is even because a negative to an even power is always positive.  In particular, $\sqrt{x^2} = |x|$, not just `$x$' (unless we \textit{know} $x \geq 0$.)\footnote{This discussion should sound familiar - see the discussion following Definition \ref{rationalexponentdefn} and the discussion following `Extracting the Square Root' on page \pageref{extractingthesquareroot}.}  We practice these formulas in the following example.

\begin{ex}\label{simplifyradexpressions}  Perform the indicated operations and simplify.

\begin{tasks}(2)
\task  $\sqrt{x^{2} + 1}$\vphantom{$\sqrt[4]{\dfrac{\pi r^{4}}{L^{8}}}$}

\task  $\sqrt{t^2-10t+25}$\vphantom{$\sqrt[4]{\dfrac{\pi r^{4}}{L^{8}}}$}

\task  $\sqrt[3]{48x^{14}}$\vphantom{$\sqrt[4]{\dfrac{\pi r^{4}}{L^{8}}}$}

\task  $\sqrt[4]{\dfrac{\pi r^{4}}{L^{8}}}$

\task! $2x \sqrt[3]{x^2-4} + 2\left(\dfrac{1}{2(\sqrt[3]{x^2-4})^2}\right)  (2x)$ 

\task!  $\sqrt{(\sqrt{18y} - \sqrt{8y})^2 + (\sqrt{20} - \sqrt{80})^2}$ \vphantom{$2x \sqrt[3]{x^2-4} + 2\left(\dfrac{1}{2(\sqrt[3]{x^2-4})^2}\right)  (2x)$ }

\end{tasks}

{\bf Solution.}

\begin{enumerate}

\item We told you back on page \pageref{donotdistributeexponents} that roots do not `distribute' across addition and since $x^{2} + 1$ cannot be factored over the real numbers, $\sqrt{x^{2} + 1}$ cannot be simplified.  It may seem silly to start with this example but it is extremely important that you understand what maneuvers are legal and which ones are not.\footnote{You really do need to understand this otherwise horrible evil will plague your future studies in Math.  If you say something totally wrong like $\sqrt{x^{2} + 1} = x + 1$ then you may never pass Calculus.  PLEASE be careful!}

\item Again we note that $\sqrt{t^2-10t+25}  \neq \sqrt{t^2} - \sqrt{10t} + \sqrt{25}$, since radicals do \textit{not} distribute across addition and subtraction.\footnote{Let $t = 1$ and see what happens to $\sqrt{t^2-10t+25}$  versus $\sqrt{t^2} - \sqrt{10t} + \sqrt{25}$.}  In this case, however, we can factor the radicand and simplify as \[ \sqrt{t^2 - 10t + 25} = \sqrt{(t-5)^2} = |t-5| \]
Without knowing more about the value of $t$, we have no idea if $t-5$ is positive or negative so $|t-5|$ is our final answer.\footnote{In general,  $|t-5| \neq |t| - |5|$ and  $|t-5| \neq t + 5$ so watch what you're doing!}

\item  To simplify $\sqrt[3]{48x^{14}}$, we need to look for perfect cubes in the radicand.  For the cofficient, we have $48 = 8 \cdot 6 = 2^3 \cdot 6$.  To find the largest perfect cube factor in $x^{14}$, we divide $14$ (the exponent on $x$) by $3$ (since we are looking for a perfect \textit{cube}).  We get $4$ with a remainder of $2$.  This means $14 = 4 \cdot 3 + 2$, so $x^{14} = x^{4 \cdot 3 + 2} = x^{4 \cdot 3} x^2 = (x^4)^3 x^2$.  Putting this altogether gives:

\begin{align*}
\sqrt[3]{48x^{14}} & = \sqrt[3]{2^3 \cdot 6 \cdot(x^4)^3 x^2} \tag{Factor out perfect cubes} \\
& = \sqrt[3]{2^3} \sqrt[3]{(x^4)^3} \sqrt[3]{6x^2} \tag{Rearrange factors,  Product Rule of Radicals} \\
& = 2x^4\sqrt[3]{6x^2} & \\
\end{align*}

\item  In this example, we are looking for perfect fourth powers in the radicand.  In the numerator $r^4$ is clearly a perfect fourth power.  For the denominator, we take the power on the $L$, namely $12$, and divide by $4$ to get $3$.  This means $L^{8} = L^{2\cdot 4} = (L^2)^{4}$.  We get 

\begin{align*}
\sqrt[4]{\dfrac{\pi r^{4}}{L^{12}}} & = \dfrac{\sqrt[4]{\pi r^{4}}}{\sqrt[4]{L^{12}}} \tag{Quotient Rule of Radicals} \\ 
& = \dfrac{\sqrt[4]{\pi}\sqrt[4]{r^{4}}}{\sqrt[4]{(L^2)^{4}}} \tag{Product Rule of Radicals} \\ 
& = \dfrac{\sqrt[4]{\pi}|r|}{|L^2|} \tag{Simplify} \\
\end{align*}

Without more information about $r$, we cannot simplify $|r|$ any further.  However, we can simplify $|L^2|$.  Regardless of the choice of $L$, $L^2 \geq 0$. Actually, $L^2 > 0$ because $L$ is in the denominator which means $L \neq 0$. Hence, $|L^2| = L^2$.  Our answer simplifies to: \[ \dfrac{\sqrt[4]{\pi}|r|}{|L^2|} = \dfrac{|r|\sqrt[4]{\pi}}{L^2} \]

\item After a quick cancellation (two of the $2$'s in the second term) we need to obtain a common denominator.  Since we can view the first term as having a denominator of $1$,  the common denominator is precisely the denominator of the second term, namely $(\sqrt[3]{x^2-4})^2$.  With common denominators, we proceed to add the two fractions.  Our last step is to factor the numerator to see if there are any cancellation opportunities with the denominator.

\begin{multline*}
2x \sqrt[3]{x^2-4} + 2\left(\dfrac{1}{2(\sqrt[3]{x^2-4})^2}\right)  (2x) \\
\end{multline*}
\begin{align*}
& = 2x \sqrt[3]{x^2-4} + \cancel{2}\left(\dfrac{1}{\cancel{2}(\sqrt[3]{x^2-4})^2}\right)  (2x) \tag{Reduce}\\ 
& = 2x \sqrt[3]{x^2-4} + \dfrac{2x}{(\sqrt[3]{x^2-4})^2} \tag{Mutiply} \\
& = (2x \sqrt[3]{x^2-4}) \cdot \dfrac{(\sqrt[3]{x^2-4})^2}{(\sqrt[3]{x^2-4})^2} + \dfrac{2x}{(\sqrt[3]{x^2-4})^2} \tag{Equivalent fractions} \\
& = \dfrac{2x(\sqrt[3]{x^2-4})^3}{(\sqrt[3]{x^2-4})^2} + \dfrac{2x}{(\sqrt[3]{x^2-4})^2} \tag{Multiply}\\
& = \dfrac{2x(x^2-4)}{(\sqrt[3]{x^2-4})^2} + \dfrac{2x}{(\sqrt[3]{x^2-4})^2} \tag{Simplify}\\ 
& = \dfrac{2x(x^2-4) + 2x}{(\sqrt[3]{x^2-4})^2} \tag{Add} \\ 
& = \dfrac{2x(x^2-4 +1)}{(\sqrt[3]{x^2-4})^2} \tag{Factor}\\ 
& = \dfrac{2x(x^2-3)}{(\sqrt[3]{x^2-4})^2} & \\
\end{align*}

We cannot reduce this any further because $x^2 - 3$ is irreducible over the rational numbers. 


\item  We begin by working inside each set of parentheses, using the product rule for radicals and combining like terms.

\begin{multline*}
 \sqrt{(\sqrt{18y} - \sqrt{8y})^2 + (\sqrt{20} - \sqrt{80})^2} \\
\end{multline*}
\begin{align*}
& = \sqrt{(\sqrt{9\cdot 2y} - \sqrt{4 \cdot 2y})^2 + (\sqrt{4\cdot 5} - \sqrt{16 \cdot 5})^2} & \\
& = \sqrt{(\sqrt{9} \sqrt{2y} - \sqrt{4}\sqrt{2y})^2 + (\sqrt{4}\sqrt{5} - \sqrt{16}\sqrt{5})^2} & \\
& = \sqrt{(3\sqrt{2y} - 2\sqrt{2y})^2 + (2\sqrt{5} - 4\sqrt{5})^2} & \\
& = \sqrt{(\sqrt{2y})^2 + (-2\sqrt{5})^2} & \\
& = \sqrt{2y + (-2)^2(\sqrt{5})^2} & \\
& = \sqrt{2y + 4\cdot 5} & \\
& = \sqrt{2y + 20} & \\ 
\end{align*}

To see if this simplifies any further, we factor the radicand:  $\sqrt{2y+20} = \sqrt{2(y+10)}$.  Finding no perfect square factors, we are done. \qed

\end{enumerate}

\end{ex}

Theorem \ref{simplifyradicals} allows us to generalize the process of `Extracting Square Roots' to `Extracting $\text{n}^{\text{th}}$ Roots' which in turn allows us to solve equations\footnote{Well, not entirely.  The equation $x^{7} = 1$ has seven answers: $x = 1$ and six complex number solutions which we'll find using techniques in Section \ref{PolarComplex}.} of the form $X^n  = c$.

\begin{floatbox}[label=box:extractingnthroots]{Extracting $\text{n}^{\text{th}}$ roots:}

\begin{itemize}[leftmargin=*]

\item If $c$ is a real number and $n$ is odd then the real number solution to $X^{n} = c$ is $X = \sqrt[n]{c}$.

\item  If $c \geq 0$ and $n$ is even then the real number solutions to $X^{n} = c$ are $X = \pm \sqrt[n]{c}$.

\textbf{Note:} If $c < 0$ and $n$ is even then $X^{n} = c$ has no real number solutions.

\end{itemize}

\end{floatbox}

Essentially, we solve $X^{n} = c$ by `taking the $\text{n}^{\text{th}}$ root' of both sides:  $\sqrt[n]{X^{n}} = \sqrt[n]{c}$. Simplifying the left side gives us just $X$ if $n$ is odd or $|X|$ if $n$ is even.  In the first case,  $X =  \sqrt[n]{c}$, and in the second, $X = \pm \sqrt[n]{c}$.  Putting this together with the other part of Theorem \ref{simplifyradicals}, namely $(\sqrt[n]{a})^n = a$, gives us a strategy for solving equations which involve $\text{n}^{\text{th}}$ powers and $n^{\text{th}}$ roots. 

\begin{floatbox}[label=box:solvepowerandradicaleqns]{Strategies for Solving Power and Radical Equations}

\begin{itemize}[leftmargin=*]

\item  If the equation involves an $\text{n}^{\text{th}}$ power and the variable appears in only one term, isolate the term with the $\text{n}^{\text{th}}$ power and extract $\text{n}^{\text{th}}$ roots.

\item  If the equation involves an $\text{n}^{\text{th}}$ root and the variable appears in that $\text{n}^{\text{th}}$ root, isolate the $\text{n}^{\text{th}}$ root and raise both sides of the equation to the $\text{n}^{\text{th}}$ power.

\textbf{Note:}  When raising both sides of an equation to an \textit{even} power, be sure to check for extraneous solutions.

\end{itemize}

\end{floatbox}

The note about `extraneous solutions' can be demonstrated by the basic equation: $\sqrt{x} = -2$.  This equation has no solution since, by definition, $\sqrt{x} \geq 0$ for all real numbers $x$.  However, if we square both sides of this equation, we get $(\sqrt{x})^2 = (-2)^2$ or $x = 4$.  However, $x = 4$ doesn't check in the original equation, since $\sqrt{4} = 2$, not $-2$.  Once again, the root\footnote{Pun intended!} of all of our problems lies in the fact that a \textit{negative} number to an \textit{even} power results in a \textit{positive} number. In other words, raising both sides of an equation to an even power does \textit{not} produce an equivalent equation, but rather, an equation which may possess \textit{more} solutions than the original.  Hence the cautionary remark above about extraneous solutions.

\begin{ex}\label{radicaleqnreview}  Solve the following equations.


\begin{tasks}(2)
\task  $(5x +3)^{4} = 16$\vphantom{$1 - \dfrac{(5-2w)^3}{7} = 9$}

\task  $1 - \dfrac{(5-2w)^3}{7} = 9$\vphantom{$1 - \dfrac{(5-2w)^3}{7} = 9$}

\task  $t + \sqrt{2t+3} = 6$\vphantom{$1 - \dfrac{(5-2w)^3}{7} = 9$}

\task $\sqrt{2} - 3\sqrt[3]{2y+1} = 0$ 

\task  $\sqrt{4x-1}  + 2\sqrt{1 - 2x} = 1$

\task  $\sqrt[4]{n^2 + 2} + n = 0$

\end{tasks}

For the remaining problems, assume that all of the variables represent positive real numbers.\footnote{That is, you needn't worry that you're multiplying or dividing by $0$ or that you're forgetting absolute value symbols.}

\begin{tasks}[resume](2)
\task  Solve for $r$:  $V = \dfrac{4\pi}{3}(R^3 - r^3)$.\vphantom{$\dfrac{r_{\text{\tiny$1$}}}{r_{\text{\tiny$2$}}} = \sqrt{\dfrac{M_{\text{\tiny$2$}}}{M_{\text{\tiny$1$}}}}$}

\task  Solve for $M_{\text{\tiny$1$}}$:  $\dfrac{r_{\text{\tiny$1$}}}{r_{\text{\tiny$2$}}} = \sqrt{\dfrac{M_{\text{\tiny$2$}}}{M_{\text{\tiny$1$}}}}$\vphantom{$\dfrac{r_{\text{\tiny$1$}}}{r_{\text{\tiny$2$}}} = \sqrt{\dfrac{M_{\text{\tiny$2$}}}{M_{\text{\tiny$1$}}}}$}

\task!  Solve for $v$:  $m = \dfrac{m_{\text{\tiny$0$}}}{\sqrt{1 - \dfrac{v^2}{c^2}}}$.  Again, assume that no arithmetic rules are violated.

\end{tasks}


{\bf Solution.}

\begin{enumerate}

\item  In our first equation, the quantity containing $x$ is already isolated, so we extract fourth roots. The exponent is even, so when the roots are extracted we need both the positive and negative roots. 

\begin{align*}
(5x +3)^{4} & = 16 \\ 
5x+3 & = \pm \sqrt[4]{16} \tag{Extract fourth roots} \\ 
5x + 3 & = \pm 2 \\ 
5x+3 = 2 & \text{ or } 5x+3 = -2 \\
x = -\dfrac{1}{5} & \text{ or } x = -1 \\
\end{align*}

We leave it to the reader to verify that both of these solutions satisfy the original equation.
\item  In this example, we first need to isolate the quantity containing the variable $w$.  Here, third (cube) roots are required and since the exponent (index) is odd, we do not need the $\pm$:

\begin{align*}
1 - \dfrac{(5-2w)^3}{7} & =  9 & \\ 
- \dfrac{(5-2w)^3}{7} & = 8 \tag{Subtract $1$} \\
(5-2w) ^ 3 & = -56 \tag{Multiply by $-7$} \\
5 - 2w & = \sqrt[3]{-56} \tag{Extract cube root} \\
5 - 2w & = \sqrt[3]{(-8)(7)} & \\
5 - 2w & = \sqrt[3]{-8} \sqrt[3]{7} \tag{Product Rule}\\
5 - 2w & = -2\sqrt[3]{7} & \\
-2w & = -5-2 \sqrt[3]{7} \tag{Subtract $5$} \\
w & = \dfrac{-5 - 2\sqrt[3]{7}}{-2} \tag{Divide by $-2$} \\
w & = \dfrac{5 + 2\sqrt[3]{7}}{2} \tag{Properties of Negatives} \\
\end{align*}

The reader should check the answer because it provides a hearty review of arithmetic.

\item  To solve  $t + \sqrt{2t+3} = 6$, we first isolate the square root, then proceed to square both sides of the equation.  In doing so, we run the risk of introducing extraneous solutions so checking our answers here is a necessity.

\begin{align*}
t + \sqrt{2t+3}  & = 6 & \\ 
\sqrt{2t+3} & = 6 - t \tag{Subtract $t$} \\ 
(\sqrt{2t+3})^2 & = (6-t)^2 \tag{Square both sides} \\ 
2t + 3 & = 36-12t + t^2 \tag{F.O.I.L. / Perfect Square Trinomial} \\ 
0 & = t^2 - 14t + 33 \tag{Subtract $2t$ and $3$} \\ 
0 & = (t-3)(t-11) \tag{Factor} \\
\end{align*}

From the Zero Product Property, we know either $t - 3 = 0$ (which gives $t=3$) or $t-11 = 0$ (which gives $t=11$).  When checking our answers, we find $t = 3$ satisfies the original equation, but $t = 11$ does not.\footnote{It is worth noting that when $t=11$ is substituted into the original equation, we get $11 + \sqrt{25} = 6$.  If the $+\sqrt{25}$ were $-\sqrt{25}$, the solution would check. Once again, when squaring both sides of an equation, we lose track of $\pm$, which is what lets extraneous solutions in the door.}  So our final answer is $t = 3$ only.

\item  In our next example, we locate the variable (in this case $y$) beneath a cube root, so we first isolate that root and cube both sides.

\begin{align*}
\sqrt{2} - 3\sqrt[3]{2y+1} &  =  0 \\
- 3\sqrt[3]{2y+1} &  =  - \sqrt{2} \tag{Subtract $\sqrt{2}$} \\
\sqrt[3]{2y+1} & = \dfrac{-\sqrt{2}}{-3} \tag{Divide by $-3$} \\
\sqrt[3]{2y+1} & = \dfrac{\sqrt{2}}{3}  \tag{Properties of Negatives} \\
(\sqrt[3]{2y+1})^3 & = \left( \dfrac{\sqrt{2}}{3} \right)^{3} \tag{Cube both sides} \\
2y + 1 & = \dfrac{(\sqrt{2})^3}{3^3} \\
2y + 1 & = \dfrac{2\sqrt{2}}{27} \\ 
2y  & = \dfrac{2 \sqrt{2}}{27}  - 1 \tag{Subtract $1$} \\
2y  & = \dfrac{2 \sqrt{2}}{27}  - \dfrac{27}{27} \tag{Common denominators} \\ 
2y  & = \dfrac{2 \sqrt{2} - 27}{27} \tag{Subtract fractions} \\
y  & = \dfrac{2 \sqrt{2} - 27}{54} \tag{Divide by $2$ $\left(\text{multiply by $\frac{1}{2}$} \right)$} \\
\end{align*}

Since we raised both sides to an \textit{odd} power, we don't need to worry about extraneous solutions but we encourage the reader to check the solution just for the fun of it.

\item In the equation $\sqrt{4x-1}  + 2\sqrt{1 - 2x} = 1$, we have not one but two square roots.  We begin by isolating one of the square roots and squaring both sides.

\begin{align*}
\sqrt{4x-1}  + 2\sqrt{1 - 2x} & = 1 \\ 
\sqrt{4x-1} & = 1 - 2\sqrt{1-2x} \tag{Subtract $2\sqrt{1 - 2x}$ from both sides} \\
(\sqrt{4x-1})^2 & = (1 - 2\sqrt{1-2x})^2 \tag{Square both sides} \\
4x - 1 & = 1 - 4\sqrt{1-2x} + (2\sqrt{1-2x})^2 \tag{F.O.I.L. / Perfect Square Trinomial} \\
4x - 1 & = 1 - 4\sqrt{1-2x} + 4(1-2x) & \\
4x - 1 & = 1 - 4\sqrt{1-2x} + 4 - 8x \tag{Distribute} \\ 
4x - 1 & = 5 - 8x - 4\sqrt{1-2x} \tag{Gather like terms} \\
\end{align*}

At this point, we have just one square root so we proceed to isolate it and square both sides a second time.\footnote{To avoid complications with fractions, we'll forego dividing by the coefficient of $\sqrt{1-2x}$, namely $-4$. This is perfectly fine so long as we don't forget to square it when we square both sides of the equation.}

\begin{align*}
4x - 1 & = 5 - 8x - 4\sqrt{1-2x} &  \\ 
12x - 6 & = -4\sqrt{1-2x} \tag{Subtract $5$, add $8x$}\\ 
(12x-6)^2 & = (-4\sqrt{1-2x})^2 \tag{Square both sides} \\
144x^2 - 144x + 36 & = 16(1-2x) & \\ 
144x^2 -  144x + 36 & = 16 - 32x & \\
144x^2 - 112x + 20 & = 0 \tag{Subtract $16$, add $32x$} \\
4(36x^2 - 28x + 5) & = 0 \tag{Factor} \\
4(2x-1)(18x - 5) & = 0 \tag{Factor some more} \\
\end{align*}

From the Zero Product Property, we know either $2x-1 = 0$ or $18x - 5 = 0$.  The former gives $x = \frac{1}{2}$ while the latter gives us $x = \frac{5}{18}$.  Since we squared both sides of the equation (twice!), we need to check for extraneous solutions.  We find $x = \frac{5}{18}$ to be extraneous, so our only solution is $x = \frac{1}{2}$.

\item As usual, our first step in solving $\sqrt[4]{n^2 + 2} + n = 0$ is to isolate the radical.  We then proceed to raise both sides to the fourth power to eliminate the fourth root:

\begin{align*}
\sqrt[4]{n^2 + 2} + n & = 0 &  \\
\sqrt[4]{n^2 + 2} & =  -n \tag{Subtract $n$} \\
(\sqrt[4]{n^2 + 2})^4 & = (-n)^4 \tag{Raise both sides to the $4^{\text{th}}$ power} \\
n^2 + 2 & = n^4 \tag{Properties of Negatives}\\
0 & = n^{4} - n^2 - 2 \tag{Subtract $n^2$ and $2$} \\
0 & = (n^2 - 2)(n^2 + 1) \tag{Factor - this is a `Quadratic in Disguise'} \\
\end{align*}

At this point, the Zero Product Property gives either $n^2 - 2 = 0$ or $n^2 + 1 = 0$.  From $n^2 - 2 = 0$, we get $n^2 = 2$, so $n = \pm \sqrt{2}$.  From $n^2 + 1 = 0$, we get $n^2 = -1$, which gives no real solutions.\footnote{Why is that again?}  Since we raised both sides to an even (the fourth) power, we need to check for extraneous solutions. We find that $n = -\sqrt{2}$ works but $n = \sqrt{2}$ is extraneous.

\item In this problem, we are asked to solve for $r$. While there are a lot of letters in this equation\footnote{including a Greek letter, no less!}, $r$ appears in only one term:  $r^3$.  Our strategy is to isolate $r^3$ then extract the cube root.

\begin{align*}
V & = \dfrac{4\pi}{3}(R^3 - r^3) \\ 
3V & = 4\pi (R^3 - r^3) & \tag{Multiply by $3$ to clear fractions}\\
3V & = 4\pi R^3 - 4\pi r^3 \tag{Distribute} \\
3V - 4\pi R^3 & = -4 \pi r^3 \tag{Subtract $4 \pi R^3$} \\
\dfrac{3V - 4\pi R^3}{-4\pi} & = r^3 \tag{Divide by $-4\pi$} \\
\dfrac{4\pi R^3 - 3V}{4\pi} & = r^3 \tag{Properties of Negatives} \\
\sqrt[3]{\dfrac{4\pi R^3 - 3V}{4\pi}} & = r \tag{Extract the cube root} \\
\end{align*}

The check is, as always, left to the reader and highly encouraged.

\item  The equation we are asked to solve in this example is from the world of Chemistry and is none other than \href{http://en.wikipedia.org/wiki/Graham's_law}{\underline{Graham's Law of Effusion}}.  As was mentioned in Example \ref{rateqnreviewex}, subscripts in Mathematics are used to distinguish between variables and have no arithmetic significance.  In this example, $r_{\text{\tiny$1$}}$, $r_{\text{\tiny$2$}}$, $M_{\text{\tiny$1$}}$ and $M_{\text{\tiny$2$}}$ are as different as $x$, $y$, $z$ and $117$.  Since we are asked to solve for $M_{\text{\tiny$1$}}$, we locate $M_{\text{\tiny$1$}}$ and see it is in the denominator of a fraction which is inside of a square root.  We eliminate the square root by squaring both sides and proceed from there.

\begin{align*}
 \dfrac{r_{\text{\tiny$1$}}}{r_{\text{\tiny$2$}}} & =  \sqrt{\dfrac{M_{\text{\tiny$2$}}}{M_{\text{\tiny$1$}}}} & \\
\left(\dfrac{r_{\text{\tiny$1$}}}{r_{\text{\tiny$2$}}}\right)^2 & = \left(\sqrt{\dfrac{M_{\text{\tiny$2$}}}{M_{\text{\tiny$1$}}}}\right)^2 \tag{Square both sides} \\
\dfrac{r_{\text{\tiny$1$}}^2}{r_{\text{\tiny$2$}}^2} & = \dfrac{M_{\text{\tiny$2$}}}{M_{\text{\tiny$1$}}} & \\
r_{\text{\tiny$1$}}^2 M_{\text{\tiny$1$}} & = M_{\text{\tiny$2$}}r_{\text{\tiny$2$}}^2 \tag{Multiply by $r_{\text{\tiny$2$}}^2 M_{\text{\tiny$1$}}$ to clear fractions, assume $r_{\text{\tiny$2$}}$,  $M_{\text{\tiny$1$}} \neq 0$ } \\
M_{\text{\tiny$1$}} & = \dfrac{M_{\text{\tiny$2$}}r_{\text{\tiny$2$}}^2}{r_{\text{\tiny$1$}}^2} \tag{Divide by $r_{\text{\tiny$1$}}^2$, assume $r_{\text{\tiny$1$}} \neq 0$} \\
\end{align*}

As the reader may expect, checking the answer amounts to a good exercise in simplifying rational and radical expressions.  The fact that we are assuming all of the variables represent positive real numbers comes in to play, as well.

\item  Our last equation to solve comes from Einstein's Special Theory of Relativity and relates the mass of an object to its velocity as it moves.\footnote{See this article on the \href{http://en.wikipedia.org/wiki/Lorentz_factor}{\underline{Lorentz Factor}}.} We are asked to solve for $v$ which is located in just one term, namely $v^2$, which happens to lie in a fraction underneath a square root which is itself a denominator. We have quite a lot of work ahead of us!

\begin{align*}
m &  = \dfrac{m_{\text{\tiny$0$}}}{\sqrt{1 - \dfrac{v^2}{c^2}}} & \\
m \sqrt{1 - \dfrac{v^2}{c^2}} & = m_{\text{\tiny$0$}} \tag{Multiply by $\sqrt{1 - \dfrac{v^2}{c^2}}$ to clear fractions}\\ 
\left(m \sqrt{1 - \dfrac{v^2}{c^2}}\right)^{2} & = m_{\text{\tiny$0$}}^{2} \tag{Square both sides}\\ 
m^2 \left(1 - \dfrac{v^2}{c^2}\right) & =  m_{\text{\tiny$0$}}^{2} \tag{Properties of Exponents}\\
m^2 - \dfrac{m^2 v^2}{c^2} & = m_{\text{\tiny$0$}}^{2} \tag{Distribute} \\
- \dfrac{m^2 v^2}{c^2} & = m_{\text{\tiny$0$}}^{2} - m^2 \tag{Subtract $m^2$}  \\ 
m^2 v^2 & = -c^2 (m_{\text{\tiny$0$}}^{2} - m^2)  \tag{Multiply by $-c^2$ ($c^2 \neq 0$)} \\ 
m^2 v^2 & = -c^2m_{\text{\tiny$0$}}^{2} + c^2 m^2 \tag{Distribute} \\
v^2  & = \dfrac{c^2 m^2 -c^2m_{\text{\tiny$0$}}^{2}}{m^2} \tag{Rearrange terms, divide by $m^2$ ($m^2 \neq 0$)} \\
v & = \sqrt{\dfrac{c^2 m^2 -c^2m_{\text{\tiny$0$}}^{2}}{m^2}} \tag{Extract Square Roots, $v > 0$ so no $\pm$} \\
v & = \dfrac{\sqrt{c^2 (m^2 -m_{\text{\tiny$0$}}^{2})}}{\sqrt{m^2}} \tag{Properties of Radicals, factor} \\
v & = \dfrac{|c|\sqrt{m^2 -m_{\text{\tiny$0$}}^{2}}}{|m|} &  \\
v & = \dfrac{c\sqrt{m^2 -m_{\text{\tiny$0$}}^{2}}}{m} \tag{$c > 0$ and $m > 0$ so $|c| = c$ and $|m| = m$} \\
\end{align*}

Checking the answer algebraically would earn the reader great honor and respect on the Algebra battlefield so it is highly recommended.

\end{enumerate}

\end{ex}

\subsection{Rationalizing Denominators and Numerators}
\label{rationalizingdenomandnumer}

In Section \ref{AppQuadEqus}, there were a few instances where we needed to `rationalize' a denominator - that is, take a fraction with radical in the denominator and re-write it as an equivalent fraction without a radical in the denominator.  There are various reasons for wanting to do this,\footnote{Before the advent of the handheld calculator, rationalizing denominators made it easier to get decimal approximations to fractions containing radicals.   However, some (admittedly more abstract) applications remain today --  one of which we'll explore in Section \ref{AppCmpNums}; one you'll see in Calculus.} but the most pressing reason is that rationalizing denominators - and numerators as well - gives us an opportunity for more practice with fractions and radicals. To refresh your memory, we rationalize a denominator and a numerator below: \[ \dfrac{1}{\sqrt{2}} = \dfrac{\sqrt{2}}{\sqrt{2} \sqrt{2}} = \dfrac{\sqrt{2}}{\sqrt{4}} = \dfrac{\sqrt{2}}{2} \quad \text{and} \quad \dfrac{7\sqrt[3]{4}}{3} = \dfrac{7 \sqrt[3]{4}\sqrt[3]{2}}{3\sqrt[3]{2}} = \dfrac{7\sqrt[3]{8}}{3\sqrt[3]{2}} = \dfrac{7 \cdot 2}{3\sqrt[3]{2}} =  \dfrac{14}{3\sqrt[3]{2}} \]

In general, if the fraction contains either a single term numerator or denominator with an undesirable $\text{n}^{\text{th}}$ root, we multiply the numerator and denominator by whatever is required to obtain a perfect $\text{n}^{\text{th}}$ power in the radicand that we want to eliminate. If the fraction contains two terms the situation is somewhat more complicated.  To see why, consider the fraction $\frac{3}{4 - \sqrt{5}}$.  Suppose we wanted to rid the denominator of the $\sqrt{5}$ term.  We could try as above and multiply numerator and denominator by $\sqrt{5}$ but that just yields: \[ \dfrac{3}{4 - \sqrt{5}} = \dfrac{3\sqrt{5}}{(4 - \sqrt{5})\sqrt{5}} = \dfrac{3\sqrt{5}}{4\sqrt{5} - \sqrt{5}\sqrt{5}} = \dfrac{3\sqrt{5}}{4\sqrt{5} - 5}\] We haven't removed $\sqrt{5}$ from the denominator - we've just shuffled it over to the other term in the denominator.  As you may recall, the strategy here is to multiply both the numerator and the denominator by what's called the \textbf{conjugate}\index{conjugate}.  

\begin{tcolorbox}
	
\begin{defn}\label{squarerootconj} \textbf{Congugate of a Square Root Expression:}  If $a$, $b$ and $c$ are real numbers with $c > 0$ then the quantities $(a + b \sqrt{c})$ and $(a - b\sqrt{c})$ are \textbf{conjugates} of one another.\footnote{As are $(b\sqrt{c} -a)$ and $(b\sqrt{c} + a)$ because $(b\sqrt{c} -a)(b\sqrt{c} + a) = b^2c - a^2$.}  Conjugates multiply according to the Difference of Squares Formula:  \[ (a + b \sqrt{c})(a - b\sqrt{c}) = a^2 - (b \sqrt{c})^2 = a^2 - b^2c\]
\end{defn}

\end{tcolorbox}

That is, to get the conjugate of a two-term expression involving a square root, you change the `$-$' to a `$+$,' or vice-versa.  For example, the conjugate of $4 - \sqrt{5}$ is $4 + \sqrt{5}$, and when we multiply these two factors together, we get $(4 - \sqrt{5})(4 + \sqrt{5}) = 4^2 - (\sqrt{5})^2 = 16 - 5 = 11$.  Hence, to eliminate the $\sqrt{5}$ from the denominator of our original fraction, we multiply both the numerator and the denominator by the \textit{conjugate} of $4-\sqrt{5}$ to get: \[\dfrac{3}{4 - \sqrt{5}} = \dfrac{3 (4 + \sqrt{5})}{(4 - \sqrt{5})(4 + \sqrt{5})} = \dfrac{3 (4 + \sqrt{5})}{4^2 - (\sqrt{5})^2} = \dfrac{3(4 + \sqrt{5})}{16 - 5} = \dfrac{12 + 3\sqrt{5}}{11}\] 

What if we had $\sqrt[3]{5}$ instead of $\sqrt{5}$?  We could try multiplying $4 - \sqrt[3]{5}$ by $4 + \sqrt[3]{5}$ to get  \[(4 - \sqrt[3]{5})(4 + \sqrt[3]{5}) = 4^2 - (\sqrt[3]{5})^2 = 16 - \sqrt[3]{25},\] which leaves us with a cube root.  What we need to undo the cube root is a perfect cube, which means we look to the Difference of Cubes Formula for inspiration:  $a^3 - b^3 = (a-b)(a^2+ab+b^2)$.  If we take $a = 4$ and $b = \sqrt[3]{5}$, we multiply 

\begin{align*}
(4 - \sqrt[3]{5})(4^2 + 4\sqrt[3]{5} + (\sqrt[3]{5})^2) &= 4^3 + 4^2\sqrt[3]{5} + 4 \sqrt[3]{5} - 4^2\sqrt[3]{5}-4(\sqrt[3]{5})^2 - (\sqrt[3]{5})^3 \\
&= 64 - 5 \\
&= 59
\end{align*}

So if we were charged with rationalizing the denominator of $\frac{3}{4 - \sqrt[3]{5}}$, we'd have:

\[ \dfrac{3}{4 - \sqrt[3]{5}} = \dfrac{3(4^2 + 4\sqrt[3]{5} + (\sqrt[3]{5})^2)}{(4 - \sqrt[3]{5})(4^2 + 4\sqrt[3]{5} + (\sqrt[3]{5})^2)} = \dfrac{48 + 12\sqrt[3]{5}+ 3\sqrt[3]{25}}{59}\]

This sort of thing extends to $\text{n}^{\text{th}}$ roots since $(a-b)$ is a factor of $a^n - b^n$ for all natural numbers $n$, but in practice, we'll stick with square roots with just a few cube roots thrown in for a challenge.\footnote{To see what to do about fourth roots, use long division to find $(a^4 - b^4) \div (a-b)$, and apply this to $4 - \sqrt[4]{5}$.}

\begin{ex} \label{rationalizenumdenom} Rationalize the indicated numerator or denominator:

\begin{tasks}(1)
\task  Rationalize the denominator:  $\dfrac{3}{\sqrt[5]{24x^2}}$

\task  Rationalize the numerator: $\dfrac{\sqrt{9 + h} - 3}{h}$

\end{tasks}

{\bf Solution.}

\begin{enumerate}

\item We are asked to rationalize the denominator, which in this case contains a fifth root.  That means we need to work to create fifth powers of each of the factors of the radicand.  To do so, we first factor the radicand:  $24x^2 = 8 \cdot 3 \cdot x^2 = 2^3 \cdot 3 \cdot x^2$.  To obtain fifth powers, we need to multiply by $2^2 \cdot 3^4 \cdot x^3$ inside the radical.

\begin{align*}
\dfrac{3}{\sqrt[5]{24x^2}} & = \dfrac{3}{\sqrt[5]{2^3 \cdot 3 \cdot x^2}} & \\ 
& = \dfrac{3 \sqrt[5]{2^2 \cdot 3^4 \cdot x^3}}{\sqrt[5]{2^3 \cdot 3 \cdot x^2}\sqrt[5]{2^2 \cdot 3^4 \cdot x^3}} \tag{Equivalent Fractions} \\
& = \dfrac{3 \sqrt[5]{2^2 \cdot 3^4 \cdot x^3}}{\sqrt[5]{2^3 \cdot 3 \cdot x^2 \cdot 2^2 \cdot 3^4 \cdot x^3}} \tag{Product Rule} \\
& = \dfrac{3 \sqrt[5]{2^2 \cdot 3^4 \cdot x^3}}{\sqrt[5]{2^5 \cdot 3^5 \cdot x^5}} \tag{Property of Exponents}\\
& = \dfrac{3 \sqrt[5]{2^2 \cdot 3^4 \cdot x^3}}{\sqrt[5]{2^5} \sqrt[5]{3^5}\sqrt[5]{x^5}} \tag{Product Rule}\\
& = \dfrac{3 \sqrt[5]{2^2 \cdot 3^4\cdot x^3}}{2 \cdot 3 \cdot x} \tag{Product Rule}\\
& = \dfrac{\cancel{3} \sqrt[5]{4 \cdot 81\cdot x^3}}{2 \cdot \cancel{3} \cdot x} \tag{Reduce}\\
& = \dfrac{\sqrt[5]{324x^3}}{2x} \tag{Simplify}\\
\end{align*}
													
\item  Here, we are asked to rationalize the \textit{numerator}.  Since it is a two term numerator involving a square root, we multiply both numerator and denominator by the conjugate of $\sqrt{9 + h} - 3$, namely $\sqrt{9 + h} + 3$.  After simplifying, we find an opportunity to reduce the fraction:

\begin{align*}
\dfrac{\sqrt{9 + h} - 3}{h} & = \dfrac{(\sqrt{9 + h} - 3)(\sqrt{9 + h} + 3)}{h(\sqrt{9 + h} + 3)} \tag{Equivalent Fractions} \\
& = \dfrac{(\sqrt{9+h})^2 - 3^2}{h(\sqrt{9 + h} + 3)} \tag{Difference of Squares} \\
& = \dfrac{(9+h) - 9}{h(\sqrt{9 + h} + 3)} \tag{Simplify} \\
& = \dfrac{h}{h(\sqrt{9 + h} + 3)} \tag{Simplify} \\
& = \dfrac{\cancelto{1}{h}}{\cancel{h}(\sqrt{9 + h} + 3)} \tag{Reduce}\\
& = \dfrac{1}{\sqrt{9 + h} + 3} & \\
\end{align*}

\end{enumerate}

\end{ex}

We close this section with an awesome example from Calculus.%\footnote{Slaying this expert-level Algebra boss is not for the fainthearted.  As an added challenge, we are not giving you the labels on the operations we perform at each step.  You and your brave Precalculus warrior-classmates should fill in those gaps on your own.}

\begin{ex} \label{rationalizenumdenombosslevel}

Simplify the compound fraction $\dfrac{\dfrac{1}{\sqrt{2(x+h)+1}} - \dfrac{1}{\sqrt{2x+1}}}{h}$ then rationalize the numerator of the result.

\medskip

{\bf Solution.} We start by multiplying the top and bottom of the `big' fraction by $\sqrt{2x+2h+1} \sqrt{2x+1}$.

\begin{multline*}
\dfrac{\dfrac{1}{\sqrt{2(x+h)+1}} - \dfrac{1}{\sqrt{2x+1}}}{h} \\
\end{multline*}
\begin{align*}
& = \dfrac{\dfrac{1}{\sqrt{2x+2h+1}} - \dfrac{1}{\sqrt{2x+1}}}{h}\\
& = \dfrac{\left(\dfrac{1}{\sqrt{2x+2h+1}} - \dfrac{1}{\sqrt{2x+1}}\right) \sqrt{2x+2h+1} \sqrt{2x+1}}{h\sqrt{2x+2h+1} \sqrt{2x+1}}\\
& = \dfrac{\dfrac{\cancel{\sqrt{2x+2h+1}}\sqrt{2x+1}}{\cancel{\sqrt{2x+2h+1}}} - \dfrac{\sqrt{2x+2h+1} \cancel{\sqrt{2x+1}}}{\cancel{\sqrt{2x+1}}}}{h\sqrt{2x+2h+1} \sqrt{2x+1}}\\
& = \dfrac{\sqrt{2x + 1}- \sqrt{2x+2h+1}}{h\sqrt{2x+2h+1} \sqrt{2x+1}}\\	
\end{align*}
																															
Next, we multiply the numerator and denominator by the conjugate of $\sqrt{2x+1} - \sqrt{2x+2h+1}$, namely $\sqrt{2x+1} + \sqrt{2x+2h+1}$, simplify and reduce:

\begin{multline*}
\dfrac{\sqrt{2x + 1}- \sqrt{2x+2h+1}}{h\sqrt{2x+2h+1} \sqrt{2x+1}} \\
\end{multline*}
\begin{align*}
& = \dfrac{(\sqrt{2x+1} - \sqrt{2x+2h+1})(\sqrt{2x+1} + \sqrt{2x+2h+1})}{h\sqrt{2x+2h+1} \sqrt{2x+1}(\sqrt{2x+1} + \sqrt{2x+2h+1})} \\ 
& = \dfrac{(\sqrt{2x+1})^2 - (\sqrt{2x+2h+1})^2}{h\sqrt{2x+2h+1} \sqrt{2x+1}(\sqrt{2x+1} + \sqrt{2x+2h+1})} \\
& = \dfrac{(2x+1) - (2x+2h+1)}{h\sqrt{2x+2h+1} \sqrt{2x+1}(\sqrt{2x+1} + \sqrt{2x+2h+1})} \\
& = \dfrac{2x+1 -2x-2h-1}{h\sqrt{2x+2h+1} \sqrt{2x+1}(\sqrt{2x+1} + \sqrt{2x+2h+1})} \\
& = \dfrac{-2\cancel{h}}{\cancel{h}\sqrt{2x+2h+1} \sqrt{2x+1}(\sqrt{2x+1} + \sqrt{2x+2h+1})} \\
& = \dfrac{-2}{\sqrt{2x+2h+1} \sqrt{2x+1}(\sqrt{2x+1} + \sqrt{2x+2h+1})} \\
\end{align*}

While the denominator is quite a bit more complicated than what we started with, we have done what was asked of us.  In the interest of full disclosure, the reason we did all of this was to cancel the original `$h$' from the denominator. That's an awful lot of effort to get rid of just one little $h$, but you'll see the significance of this in Calculus.\qed

\end{ex}

\clearpage

\subsection{Exercises}

\label{ExercisesforAppRadEqus}

In Exercises \ref{simpradfirst} - \ref{simpradlast}, perform the indicated operations and simplify.

\begin{tasks}(2)
\task   $\sqrt{9x^2}$ \label{simpradfirst}

\task   $\sqrt[3]{8t^3}$

\task   $\sqrt{50y^6}$

\task  $\sqrt{4t^2 + 4t + 1}$

\task  $\sqrt{w^2 - 16w + 64}$

\task  $\sqrt{(\sqrt{12x} - \sqrt{3x})^2+1}$

\task  $\sqrt{\dfrac{c^2 - v^2}{c^2}}$\vphantom{$\sqrt[3]{\dfrac{24 \pi r^5}{L^3}}$}

\task  $\sqrt[3]{\dfrac{24 \pi r^5}{L^3}}$

\task   $\sqrt[4]{\dfrac{32 \pi \varepsilon^8}{\rho^{12}}}$ \vphantom{$\sqrt[3]{\dfrac{24 \pi r^5}{L^3}}$}   

\task $\sqrt{x} - \dfrac{x+1}{\sqrt{x}}$\vphantom{$3 \sqrt{1-t^2} + 3t\left(\dfrac{1}{2 \sqrt{1-t^2}}\right)(-2t)$}

\task! $3 \sqrt{1-t^2} + 3t\left(\dfrac{1}{2 \sqrt{1-t^2}}\right)(-2t)$

\task! $2 \sqrt[3]{1-z} + 2z \left(\dfrac{1}{3 \left(\sqrt[3]{1-z}\right)^2}\right)(-1)$\vphantom{$\dfrac{3}{\sqrt[3]{2x-1}} + (3x)\left(-\dfrac{1}{3 \left(\sqrt[3]{2x-1} \right)^4}\right)(2)$}


\task!  $\dfrac{3}{\sqrt[3]{2x-1}} + (3x)\left(-\dfrac{1}{3 \left(\sqrt[3]{2x-1} \right)^4}\right)(2)$  \label{simpradlast}

\end{tasks}



In Exercises \ref{algineqexfirst} - \ref{algineqexlast}, find all real solutions.

\begin{tasks}[resume](2)

\task  $(2x+1)^3 + 8 = 0$ \label{algineqexfirst} \vphantom{ $\dfrac{1}{1 + 2t^3} = 4$}
\task $\dfrac{(1-2y)^{4}}{3} = 27$ \vphantom{ $\dfrac{1}{1 + 2t^3} = 4$}
\task  $\dfrac{1}{1 + 2t^3} = 4$ 

\task $\sqrt{3x+1} = 4$
\task $5 - \sqrt[3]{t^2+1} = 1$
\task $x+1 = \sqrt{3x+7}$ % $x=3$     

\task  $y + \sqrt{3y+10} = -2$ % $y=-3$
\task  $3t+\sqrt{6-9t}=2$ % $x = -\dfrac{1}{3}, \dfrac{2}{3}$
\task $2x - 1 = \sqrt{x + 3}$ % $x = \dfrac{5 + \sqrt{57}}{8}$


\task $w = \sqrt[4]{12-w^2}$
\task $\sqrt{x - 2} + \sqrt{x - 5} = 3$ % $x = 6$
\task $\sqrt{2x+1} = 3 + \sqrt{4-x}$  \label{algineqexlast} % $x = 4$

\end{tasks}

In Exercises \ref{radliteqnfirst} - \ref{radliteqnlast}, solve each equation for the indicated variable.  Assume all quantities represent positive real numbers.

\begin{tasks}[resume](2)

\task Solve for $h$:  $I = \dfrac{bh^3}{12}$. \vphantom{$I_{\text{\tiny $0$}} = \dfrac{5\sqrt{3} a^4}{16}$} \label{radliteqnfirst}

\task Solve for $a$:  $I_{\text{\tiny $0$}} = \dfrac{5\sqrt{3} a^4}{16}$

\task Solve for $g$:  $T = 2\pi \sqrt{\dfrac{L}{g}}$

\task Solve for $v$:   $L = L_{\text{\tiny $0$}} \sqrt{1 - \dfrac{v^2}{c^2}}$. \vphantom{$T = 2\pi \sqrt{\dfrac{L}{g}}$} \label{radliteqnlast}

\end{tasks}

In Exercises \ref{rationalizefirst} - \ref{rationalizelast}, rationalize the numerator or denominator, and simplify.


\begin{tasks}[resume](2)

\task   $\dfrac{4}{3 - \sqrt{2}}$ \vphantom{$\dfrac{7}{\sqrt[3]{12x^7}}$} \label{rationalizefirst}

\task  $\dfrac{7}{\sqrt[3]{12x^7}}$

\task   $\dfrac{\sqrt{x} - \sqrt{c}}{x - c}$ \vphantom{$\dfrac{7}{\sqrt[3]{12x^7}}$}


\task  $\dfrac{\sqrt{2x+2h+1} - \sqrt{2x+1}}{h}$ 

\task  $\dfrac{\sqrt[3]{x+1} - 2}{x- 7}$                                          

\task  $\dfrac{\sqrt[3]{x+h} - \sqrt[3]{x}}{h}$  \label{rationalizelast}

\end{tasks}

\clearpage

\subsection{Answers}

\begin{tasks}(3)

\task   $3|x|$ 

\task   $2t$

\task   $5|y^3|\sqrt{2}$

\task  $|2t+1|$

\task  $|w-8|$

\task  $\sqrt{3x+1}$

\task $\dfrac{\sqrt{c^2-v^2}}{|c|}$ \vphantom{$\dfrac{2 \varepsilon^2 \sqrt[4]{2\pi}}{|\rho^3|}$ }

\task $\dfrac{2r \sqrt[3]{3 \pi r^2}}{L}$ \vphantom{$\dfrac{2 \varepsilon^2 \sqrt[4]{2\pi}}{|\rho^3|}$ }

\task  $\dfrac{2 \varepsilon^2 \sqrt[4]{2\pi}}{|\rho^3|}$ 


\task  $-\dfrac{1}{\sqrt{x}}$ \vphantom{$\dfrac{3-6t^2}{\sqrt{1-t^2}}$}

\task  $\dfrac{3-6t^2}{\sqrt{1-t^2}}$
\task  $\dfrac{6-8z}{3 (\sqrt[3]{1-z})^2}$


\task  $\dfrac{4x-3}{(2x-1)\sqrt[3]{2x-1}}$  


\task  $x = -\dfrac{3}{2}$ \vphantom{$t = -\dfrac{\sqrt[3]{3}}{2}$}
\task $y = -1, 2$ \vphantom{$t = -\dfrac{\sqrt[3]{3}}{2}$}
\task  $t = -\dfrac{\sqrt[3]{3}}{2}$ 

\task $x = 5$
\task $t = \pm 3 \sqrt{7}$
\task $x=3$     


\task  $y=-3$ \vphantom{$x = \dfrac{5 + \sqrt{57}}{8}$}
\task  $t = -\dfrac{1}{3}, \dfrac{2}{3}$ \vphantom{$x = \dfrac{5 + \sqrt{57}}{8}$}
\task $x = \dfrac{5 + \sqrt{57}}{8}$

\task $w = \sqrt{3}$
\task $x = 6$
\task $x = 4$


\task $h = \sqrt[3]{\dfrac{12I}{b}}$ \vphantom{ $a = \dfrac{2 \sqrt[4]{I_{\text{\tiny $0$}}}}{\sqrt[4]{5\sqrt{3}}}$}

\task  $a = \dfrac{2 \sqrt[4]{I_{\text{\tiny $0$}}}}{\sqrt[4]{5\sqrt{3}}}$
\task $g = \dfrac{4 \pi^2 L}{T^2}$ \vphantom{$v = \dfrac{c \sqrt{L_{\text{\tiny $0$}}^2 - L^2}}{L_{\text{\tiny $0$}}}$}

\task $v = \dfrac{c \sqrt{L_{\text{\tiny $0$}}^2 - L^2}}{L_{\text{\tiny $0$}}}$  

\task   $\dfrac{12 + 4\sqrt{2}}{7}$ \vphantom{$\dfrac{7 \sqrt[3]{18x^2}}{6x^3}$} 

\task  $\dfrac{7 \sqrt[3]{18x^2}}{6x^3}$

\task   $\dfrac{1}{\sqrt{x}+ \sqrt{c}}$ \vphantom{$\dfrac{7 \sqrt[3]{18x^2}}{6x^3}$}

\task!  $\dfrac{2}{\sqrt{2x+2h+1} + \sqrt{2x+1}}$ \vphantom{$\dfrac{1}{(\sqrt[3]{x+h})^2 + \sqrt[3]{x+h}\sqrt[3]{x} + (\sqrt[3]{x})^2}$}                                 

\task  $\dfrac{1}{(\sqrt[3]{x+1})^2 + 2\sqrt[3]{x+1} + 4}$   \vphantom{$\dfrac{1}{(\sqrt[3]{x+h})^2 + \sqrt[3]{x+h}\sqrt[3]{x} + (\sqrt[3]{x})^2}$}                                 

\task!  $\dfrac{1}{(\sqrt[3]{x+h})^2 + \sqrt[3]{x+h}\sqrt[3]{x} + (\sqrt[3]{x})^2}$

\end{tasks}


\closegraphsfile