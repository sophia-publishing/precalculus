\usepackage{marginfix}
\usepackage[alerton,per=section,ragged]{sidenotesplus}
\usepackage{tabularx}
\usepackage{amssymb,amsmath,amsthm,fancyhdr,supertabular,longtable,hhline,mathtools}
\usepackage{colortbl}
\usepackage{import, multicol,boxedminipage}
\usepackage{chapterfolder}
\usepackage[metapost,truebbox]{mfpic}
\usepackage[pdflatex]{graphicx}
\usepackage{makeidx}
\usepackage{emptypage}

\usepackage{tikz-cd}

\usepackage{../shortlst}

\usepackage[colorlinks, hyperindex, plainpages=false, linkcolor=blue, urlcolor=blue, pdfpagelabels]{hyperref}
\let\oldhref\href
\renewcommand{\href}[2]{\oldhref{#1}{#2}\footnote{\url{#1}}}

\newcommand{\tabref}[1]{\hyperref[#1]{Table \ref{#1}}}

\usepackage[labelfont=bf]{caption}
\counterwithin{figure}{section}
\counterwithin{table}{section}
\usepackage[all]{hypcap}
\usepackage{bm}
\usepackage{braket}
\usepackage[normalem]{ulem}
\usepackage{enumitem}
\usepackage[skip=\medskipamount]{parskip}
\usepackage{pbox}
\usepackage[most]{tcolorbox}

\usepackage{tasks}
\usepackage{adjustbox}
\settasks{label=\arabic*., label-offset=0.6666em, label-width=14.1pt, counter=HW}

%% custom layout environments

% defn
\newenvironment{mdefn}{%
  \begin{marginfigure}
    \begin{tcolorbox}
      \begin{defn}
        \raggedright
}{%
      \end{defn}
    \end{tcolorbox}
  \end{marginfigure}
}
\newenvironment{idefn}{%
  \begin{figure}[htbp]
    \begin{tcolorbox}
      \begin{defn}
}{%
      \end{defn}
    \end{tcolorbox}
  \end{figure}
}
\newenvironment{fdefn}{%
  \begin{figure*}[tbp]
    \begin{tcolorbox}
      \begin{defn}
}{%
      \end{defn}
    \end{tcolorbox}
  \end{figure*}
}

% floatbox
\newenvironment{mfloatbox}[1]{%
  \begin{marginfigure}
    \begin{tcolorbox}
      \begin{floatbox}
        \begin{center}
          \bf #1
        \end{center}
}{%
      \end{floatbox}
    \end{tcolorbox}
  \end{marginfigure}
}
\newenvironment{ifloatbox}[1]{%
  \begin{figure}
    \begin{tcolorbox}
      \begin{floatbox}
        \begin{center}
          \bf #1
        \end{center}
}{%
      \end{floatbox}
    \end{tcolorbox}
  \end{figure}
}
\newenvironment{ffloatbox}[1]{%
  \begin{figure*}[tbp]
    \begin{tcolorbox}
      \begin{floatbox}
        \begin{center}
          \bf #1
        \end{center}
}{%
      \end{floatbox}
    \end{tcolorbox}
  \end{figure*}
}

% thm
\newenvironment{mthm}{%
  \begin{marginfigure}
    \begin{tcolorbox}
      \begin{thm}
}{%
      \end{thm}
    \end{tcolorbox}
  \end{marginfigure}
}
\newenvironment{ithm}{%
  \begin{tcolorbox}
    \begin{thm}
}{%
    \end{thm}
  \end{tcolorbox}
}
\newenvironment{fthm}{%
  \begin{figure*}
    \begin{tcolorbox}
      \begin{thm}
}{%
      \end{thm}
    \end{tcolorbox}
  \end{figure*}
}

% table
\newenvironment{mtable}{%
  \begin{margintable}
    \captionsetup{labelfont=bf,justification=centering}
    \begin{center}
}{%
    \end{center}
  \end{margintable}
}
\newenvironment{ftable}{%
  \begin{table*}
    \begin{center}
}{%
    \end{center}
  \end{table*}
}
\newenvironment{itable}{%
  \begin{table}[htp]
    \begin{center}
}{%
    \end{center}
  \end{table}
}

% figure
\newbool{ismfigure}
\newenvironment{mfigure}{%
  \begin{marginfigure}
    \booltrue{ismfigure}
    \captionsetup{labelfont=bf,justification=centering}
    \begin{center}
}{%
    \end{center}
    \boolfalse{ismfigure}
  \end{marginfigure}
}
\newenvironment{ifigure}{%
  \begin{figure}[htp]
    \begin{center}
}{%
    \end{center}
  \end{figure}
}
\newenvironment{ffigure}{%
  \begin{figure*}[htp]
    \begin{center}
}{%
    \end{center}
  \end{figure*}
}
\newenvironment{graphtrans}{%
  \begin{tikzcd}
}{%
  \end{tikzcd}
}

\newcommand{\trans}[1]{%
  \ifbool{ismfigure}{%
    \arrow[d, "\text{\parbox{20mm}{#1}}"] \\
  }{%
    \arrow[r, "\text{#1}"]
  }
}

\newcommand{\xincludegraphics}[1]{\includegraphics[width=45mm]{#1}}

% exinstr
\newcommand{\mexinstr}[1]{%
  \begin{marginfigure}
    \begin{tcolorbox}
      {\raggedright #1}
    \end{tcolorbox}
  \end{marginfigure}
}
\newcommand{\iexinstr}[1]{%
    \begin{tcolorbox}
      {#1}
    \end{tcolorbox}
}

\newcommand{\startexenum}{\setcounter{HW}{0}}

\newenvironment{exenum}{%
  \begin{enumerate}
  \setcounter{enumi}{\value{HW}}
}{%
  \setcounter{HW}{\value{enumi}}
  \end{enumerate}
}

\newenvironment{shortexenum}[1][\unskip]{%
  \begin{exenum}
}{%
  \end{exenum}
}


\newcommand{\transgraph}[2]{\stackrel{\text{\parbox{#1}{\centering \scriptsize #2}}}{\xrightarrow{\hspace{#1}}}}

\newcommand{\transgraphtwo}[3]{\stackrel{\transgraph{#1}{#2}}{\parbox{#1}{\centering \scriptsize #3}}}

\newenvironment{halfpage}{%
  \begin{minipage}{0.5\textwidth}
}{%
  \end{minipage}
}

\theoremstyle{definition}  % this prevents the text in definitions, theorems, and corollaries from being italicized
\newtheorem{floatbox}{\bf Box}[section]
\newcommand{\floatboxautorefname}{Box}
\newtheorem{defn}{\bf Definition}[section]
\newcommand{\defnautorefname}{Definition}
\newtheorem{thm}{\bf Theorem}[section]
\newcommand{\thmautorefname}{Theorem}
\newtheorem{cor}[thm]{\bf Corollary}
\newcommand{\corautorefname}{Corollary}
\newtheorem{eqn}{\bf Equation}[section]
\newcommand{\eqnautorefname}{Equation}

\newtheoremstyle{example-style}% 〈name〉
  {\medskipamount}%                      〈Space above〉
  {}%                       〈Space below〉
  {}%                  〈Body font〉
  {}%                          〈Indent amount〉
  {\bfseries}%                 〈Theorem head font〉
  {.}%                         〈Punctuation after theorem head〉
  { }%                         〈Space after theorem head〉
  {}%   
\theoremstyle{example-style}
\newtheorem{ex}{\bf Example}[section]
\newtheorem{fig}{\bf Figure}[section]

\setlength{\parindent}{0in}
\newcommand{\bbm}{\begin{boxedminipage}{\linewidth}}
\newcommand{\ebm}{\end{boxedminipage}}
\usepackage{array}
\setlength{\extrarowheight}{2pt}
\allowdisplaybreaks[2]
\usepackage{cancel}
\usepackage{sectsty}
%\usepackage{appendix}
\usepackage{textcomp}
\usepackage{multirow}
\usepackage[nottoc]{tocbibind}

\DeclareSymbolFont{AMSb}{U}{msb}{m}{n}
\DeclareMathSymbol{\C}{\mathbin}{AMSb}{"43}
\DeclareMathSymbol{\N}{\mathbin}{AMSb}{"4E}
\DeclareMathSymbol{\I}{\mathbin}{AMSb}{"5A}
\DeclareMathSymbol{\Q}{\mathbin}{AMSb}{"51}
\DeclareMathSymbol{\R}{\mathbin}{AMSb}{"52}
\DeclareMathSymbol{\W}{\mathbin}{AMSb}{"57}

\allsectionsfont{\mdseries \scshape}
\makeatletter
\renewcommand\l@section{\@dottedtocline{1}{1.5em}{3em}}
\renewcommand\l@subsection{\@dottedtocline{2}{4.5em}{3.5em}}
\makeatother
\pagestyle{fancy}
\newcounter{HW}
\newcounter{HWindent}

\renewcommand{\textinterrobang}{$! \! \! ?$}

%Below is for Iowna Font
%\renewcommand*\sfdefault{iwona}
%\usepackage[math]{iwona}

%Below is for Helvetica (scaled): 
\usepackage[scaled=.92]{helvet}   
\renewcommand{\familydefault}{\sfdefault}  %makes the text of the book sans serif
\usepackage[helvet]{sfmath}  %makes the math in the book sans serif
\allsectionsfont{\sffamily}  %makes the chapter and section titles sans serif

\makeatletter
\newcases{mycases}{\quad}{%
  \hfil$\m@th\displaystyle{##}$}{$\m@th\displaystyle{##}$\hfil}{\lbrace}{.}
\makeatother

\makeindex

\begin{document}

%removed \sc command from lines below.
\renewcommand{\chaptermark}[1]%
                  {\markboth{#1}{}}
\renewcommand{\sectionmark}[1]%
               {\markright{\thesection\ #1}}
\renewcommand{\headrulewidth}{0pt}
\lhead[\fancyplain{}{\thepage}]%
      {\fancyplain{}{\nouppercase{\rightmark}}}
\rhead[\fancyplain{}{\nouppercase{\leftmark}}]%
      {\fancyplain{}{\thepage}}
\cfoot{}
\fancypagestyle{plain}{\fancyhf{}}

\newgeometry{textwidth=130mm}
\frontmatter

\newcommand\sophia{
    \includegraphics[width=18pt]{logo/logo.pdf}

    $\sigma o \phi \acute{\iota} \alpha$
}

\begin{titlepage}
\begin{center}

\vspace*{0.1\paperheight}

\Huge Precalculus --- Algebra II  \\ \vspace{.1in} \large Version $4 - \epsilon$  \\ \vspace{.25in} \large by

\vspace{0.1\paperheight}

\newcolumntype{Y}{>{\centering\arraybackslash}X}
\begin{tabularx}{0.953\linewidth}{YY} Carl Stitz, Ph.D. &  Jeff Zeager, Ph.D. \\ Lakeland Community College & Lorain County Community College \\\end{tabularx}

\vfill

\begin{center}
    \sophia
\end{center}

\end{center}
\end{titlepage}

% copyright page
\begingroup
\footnotesize
\setlength{\parindent}{0pt}
\setlength{\parskip}{\baselineskip}

\textcopyright{} 2025 Carl Stitz and Jeff Zeager \\
\url{https://stitz-zeager.com/}

\includegraphics[height=20pt]{../by-nc-sa.pdf}

This work is licensed under CC BY-NC-SA 3.0. To view a copy of this licence,
visit:

\url{https://creativecommons.org/licenses/by-nc-sa/3.0/deed.en}

\vfill

Sophia Publishing \\
\textsc{Chennai} \\
\url{https://sophia-publishing.github.io/} \\
\endgroup

%\include{acknowledgements}

\thispagestyle{empty}

\renewcommand{\contentsname}{Table of Contents}

\addtocontents{toc}{\protect\thispagestyle{empty}}

\clearpage

\pdfbookmark[1]{\contentsname}{toc}

\tableofcontents

%\chapter{Preface}
%\label{OldPreface}
%\thispagestyle{empty}
%\input{../OldPreface}

%\chapter{The New Preface}
%\label{NewPreface}
%\thispagestyle{empty}
%\input{NewPreface}

\mainmatter
 

\renewcommand{\chaptername}{Chapter}

\restoregeometry

%\chapter{Introduction to Functions}
%\label{IntroductiontoFunctions}
%\thispagestyle{empty}
%\import{../IntroductiontoFunctions/}{IntroductiontoFunctions}

%\chapter{Polynomial Functions}
%\label{PolynomialFunctions}
%\thispagestyle{empty}
%\import{../PolynomialFunctions/}{PolynomialFunctions}

\chapter{Rational Functions}
\label{RationalFunctions}
\thispagestyle{empty}
\import{../RationalFunctions/}{RationalFunctions}

\chapter{Root, Radical and Power Functions}
\label{RootRadicalPowerFunctions}
\thispagestyle{empty}
\import{../RootRadicalPowerFunctions/}{RootRadicalPowerFunctions}

\chapter{Further Topics on Functions}
\label{FurtherTopicsonFunctions}
\thispagestyle{empty}
\import{../FurtherTopicsonFunctions/}{FurtherTopicsonFunctions}

%\chapter{Exponential and Logarithmic Functions}
%\label{ExponentialandLogarithmicFunctions}
%\thispagestyle{empty}
%\import{../ExponentialandLogarithmicFunctions/}{ExponentialandLogarithmicFunctions}

% \chapter{The Conic Sections}
% \label{TheConicSections}
% \thispagestyle{empty}
% \import{./TheConicSections/}{TheConicSections}

% \chapter{Systems of Equations and Matrices}
% \label{SystemsofEquationsandMatrices}
% \thispagestyle{empty}
% \import{./SystemsofEquationsandMatrices/}{SystemsofEquationsandMatrices}

% \chapter{Sequences and the Binomial Theorem}
% \label{SequencesandtheBinomialTheorem}
% \thispagestyle{empty}
% \import{./SequencesandtheBinomialTheorem/}{SequencesandtheBinomialTheorem}

% \chapter{Foundations of Trigonometry}
% \label{FoundationsofTrigonometry}
% \thispagestyle{empty}
% \import{./FoundationsofTrigonometry/}{FoundationsofTrigonometry}

% \chapter{Analytical Trigonometry}
% \label{AnalyticalTrigonometry}
% \thispagestyle{empty}
% \import{./AnalyticalTrigonometry/}{AnalyticalTrigonometry}

% \chapter{Geometric Applications of Trigonometry}
% \label{GeometricApplicationsofTrigonometry}
% \thispagestyle{empty}
% \import{./GeometricApplicationsofTrigonometry/}{GeometricApplicationsofTrigonometry}

% \chapter{Polar Coordinates and Parametric Equations}
% \label{PolarCoordinatesandParametricEquations}
% \thispagestyle{empty}
% \import{./PolarCoordinatesandParametricEquations/}{PolarCoordinatesandParametricEquations}

\backmatter

\let\originalstyle=\thispagestyle            % Store the command for later reuse.
%\def\thispagestyle#1{\fancyfoot[C]{}}       % This clears footer in the center if fancyhdr is in use.
\def\thispagestyle#1{\originalstyle{empty}} % Use this to get blank header+footer, TeXnically it is only \thispagestyle{empty}.
%\def\thispagestyle#1{}                       % This line completely ignores the content of the \thispagestyle command.
\printindex                                  % Typeset the actual Index.
\let\thispagestyle=\originalstyle            % Let's get back to the original version of the \thispagestyle, if needed later in the document.



\end{document}
